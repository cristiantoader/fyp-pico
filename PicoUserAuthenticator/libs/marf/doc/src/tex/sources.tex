  \title{MARF Source Code}

   \title{Formatting}

   \par
    Source code formatting uses a 4 column tab spacing, currently with
    tabs preserved (i.e. tabs are not expanded to spaces).
   

   \par
    For Emacs, add the following (or something similar)
    to your <filename>~/.emacs</filename>
    initialization file:

\begin{verbatim}
;; check for files with a path containing "marf"
(setq auto-mode-alist
  (cons '("\\(marf\\).*\\.java\\'" . marf-java-mode)
        auto-mode-alist))
(setq auto-mode-alist
  (cons '("\\(marf\\).*\\.java\\'" . marf-java-mode)
        auto-mode-alist))

(defun pgsql-java-mode ()
  ;; sets up formatting for MARF Java code
  (interactive)
  (c-mode)
  (setq-default tab-width 4)
  (c-set-style "bsd")             ; set c-basic-offset to 4, plus other stuff
  (c-set-offset 'case-label '+)   ; tweak case indent to match MARF custom
  (setq indent-tabs-mode t))      ; make sure we keep tabs when indenting
\end{verbatim}
   

   \par
    For <application>vim</application>, your
    <filename>~/.vimrc</filename> or equivalent file should contain
    the following:

\begin{verbatim}
set tabstop=4
\end{verbatim}

    or equivalently from within vim, try

\begin{verbatim}
:set ts=4
\end{verbatim}
   

   \par
    The text browsing tools <application>more</application> and
    <application>less</application> can be invoked as

\begin{verbatim}
more -x4
less -x4
\end{verbatim}
