\section{Experementation Results}\label{sect:results}
\noindent
\rule{7.0in}{.013in}

\subsection{Notes}

Before we get to numbers, few notes and observations first:

\begin{enumerate}

\item We've got more samples since the demo. The obvious: by increasing the number of samples our results got
      better; with few exceptions, however. This can be explained by
      the diversity of the recording equipment, a lot less than uniform
      number of samples per speaker, and absence of noise and silence
      removal. All the samples were recorded in not the same environments.
      The results then start averaging after awhile.

\item Another observation we made from our output, is that
      when the speaker is guessed incorrectly, quite often the second
      guess is correct, so we included this in our results as if we were
      ``guessing'' right from the second attempt.

\item FUN. Interesting to note, that we also tried to take some
      samples of music bands, and feed it to our application
      along with the speakers, and application's performance didn't suffer,
      yet even improved because the samples were treated in
      the same manner. The groups were not mentioned in the table,
      so we name them here: Van Halen [8:1] and Red Hot Chili Peppers [10:1] (where numbers
      represent [training:testing] samples used).

\end{enumerate}

\clearpage

\subsection{Configuration Explained}

Configuration parameters were exracted from the command line
which SpeakerIdentApp was invoked with. They mean the following:

\vspace{15pt}
\hrule
{\small
\begin{verbatim}
Usage:
    java SpeakerIdentApp --train <samples-dir> [options]  -- train mode
                         --ident <sample> [options]       -- identification mode
                         --stats                          -- display stats
                         --reset                          -- reset stats
                         --version                        -- display version info
                         --help                           -- display this help and exit

Options (one or more of the following):

Preprocessing:

  -norm         - use just normalization, no filtering
  -low          - use low pass filter
  -high         - use high pass filter
  -boost        - use high frequency boost filter
  -band         - use bandpass filter

Feature Extraction:

  -lpc          - use LPC
  -fft          - use FFT
  -randfe       - use random feature extraction

Classification:

  -nn           - use Neural Network
  -cheb         - use Chebyshev Distance
  -eucl         - use Euclidean Distance
  -mink         - use Minkowski Distance
  -randcl       - use random classification

Misc:

  -debug        - include verbose debug output
  -spectrogram  - dump spectrogram image after feature extraction
  -graph        - dump wave graph before preprocessing and after feature extraction
  <integer>     - expected speaker ID
\end{verbatim}}
\hrule
\vspace{15pt}

\clearpage

\subsection{Consolidated Results}

Our ultimate results \footnote{as of \input{date}} for all configurations we can have
and samples we've got are in the Table \ref{tab:results}.
Looks like our best results are with
``-norm -fft -cheb'',
``-norm -fft -eucl'',
``-norm -fft -mah'',
``-high -fft -eucl'',
``-high -fft -mah'' and,
``-high -fft -mink''
with the top result being 80\%.

\begin{table}
%\begin{minipage}[b]{\textwidth}
\centering
\begin{tabular}{|c|c|l|c|c|l|} \hline
Run \# & Guess & Configuration & GOOD & BAD & Recogniton Rate,\%\\ \hline\hline
1 & 1st & -band -fft -cheb & 7 & 13 & 35.0\\ \hline
2 & 1st & -band -fft -eucl & 9 & 11 & 45.0\\ \hline
3 & 1st & -band -fft -mah & 9 & 11 & 45.0\\ \hline
4 & 1st & -band -fft -mink & 6 & 14 & 30.0\\ \hline
5 & 1st & -band -fft -randcl & 2 & 18 & 10.0\\ \hline
6 & 1st & -band -lpc -cheb & 10 & 10 & 50.0\\ \hline
7 & 1st & -band -lpc -eucl & 10 & 10 & 50.0\\ \hline
8 & 1st & -band -lpc -mah & 10 & 10 & 50.0\\ \hline
9 & 1st & -band -lpc -mink & 9 & 11 & 45.0\\ \hline
10 & 1st & -band -lpc -nn & 3 & 17 & 15.0\\ \hline
11 & 1st & -band -lpc -randcl & 2 & 18 & 10.0\\ \hline
12 & 1st & -band -randfe -cheb & 2 & 18 & 10.0\\ \hline
13 & 1st & -band -randfe -eucl & 2 & 18 & 10.0\\ \hline
14 & 1st & -band -randfe -mah & 2 & 18 & 10.0\\ \hline
15 & 1st & -band -randfe -mink & 1 & 19 & 5.0\\ \hline
16 & 1st & -band -randfe -randcl & 1 & 19 & 5.0\\ \hline
17 & 1st & -boost -fft -cheb & 12 & 8 & 60.0\\ \hline
18 & 1st & -boost -fft -eucl & 13 & 7 & 65.0\\ \hline
19 & 1st & -boost -fft -mah & 13 & 7 & 65.0\\ \hline
20 & 1st & -boost -fft -mink & 12 & 8 & 60.0\\ \hline
21 & 1st & -boost -fft -randcl & 0 & 20 & 0.0\\ \hline
22 & 1st & -boost -lpc -cheb & 13 & 7 & 65.0\\ \hline
23 & 1st & -boost -lpc -eucl & 13 & 7 & 65.0\\ \hline
24 & 1st & -boost -lpc -mah & 13 & 7 & 65.0\\ \hline
25 & 1st & -boost -lpc -mink & 14 & 6 & 70.0\\ \hline
26 & 1st & -boost -lpc -nn & 2 & 18 & 10.0\\ \hline
27 & 1st & -boost -lpc -randcl & 0 & 20 & 0.0\\ \hline
28 & 1st & -boost -randfe -cheb & 5 & 15 & 25.0\\ \hline
29 & 1st & -boost -randfe -eucl & 5 & 15 & 25.0\\ \hline
\end{tabular}
%\end{minipage}
\caption{Consolidated results, Part 1.}
\label{tab:results}
\end{table}

\begin{table}
%\begin{minipage}[b]{\textwidth}
\centering
\begin{tabular}{|c|c|l|c|c|l|} \hline
Run \# & Guess & Configuration & GOOD & BAD & Recogniton Rate,\%\\ \hline\hline
30 & 1st & -boost -randfe -mah & 5 & 15 & 25.0\\ \hline
31 & 1st & -boost -randfe -mink & 4 & 16 & 20.0\\ \hline
32 & 1st & -boost -randfe -randcl & 1 & 19 & 5.0\\ \hline
33 & 1st & -high -fft -cheb & 15 & 5 & 75.0\\ \hline
34 & 1st & -high -fft -eucl & 16 & 4 & 80.0\\ \hline
35 & 1st & -high -fft -mah & 16 & 4 & 80.0\\ \hline
36 & 1st & -high -fft -mink & 16 & 4 & 80.0\\ \hline
37 & 1st & -high -fft -randcl & 1 & 19 & 5.0\\ \hline
38 & 1st & -high -lpc -cheb & 12 & 8 & 60.0\\ \hline
39 & 1st & -high -lpc -eucl & 11 & 9 & 55.00000000000001\\ \hline
40 & 1st & -high -lpc -mah & 11 & 9 & 55.00000000000001\\ \hline
41 & 1st & -high -lpc -mink & 9 & 11 & 45.0\\ \hline
42 & 1st & -high -lpc -nn & 3 & 17 & 15.0\\ \hline
43 & 1st & -high -lpc -randcl & 0 & 20 & 0.0\\ \hline
44 & 1st & -high -randfe -cheb & 3 & 17 & 15.0\\ \hline
45 & 1st & -high -randfe -eucl & 3 & 17 & 15.0\\ \hline
46 & 1st & -high -randfe -mah & 3 & 17 & 15.0\\ \hline
47 & 1st & -high -randfe -mink & 3 & 17 & 15.0\\ \hline
48 & 1st & -high -randfe -randcl & 2 & 18 & 10.0\\ \hline
49 & 1st & -low -fft -cheb & 15 & 5 & 75.0\\ \hline
50 & 1st & -low -fft -eucl & 14 & 6 & 70.0\\ \hline
51 & 1st & -low -fft -mah & 14 & 6 & 70.0\\ \hline
52 & 1st & -low -fft -mink & 13 & 7 & 65.0\\ \hline
53 & 1st & -low -fft -randcl & 1 & 19 & 5.0\\ \hline
54 & 1st & -low -lpc -cheb & 13 & 7 & 65.0\\ \hline
55 & 1st & -low -lpc -eucl & 11 & 9 & 55.00000000000001\\ \hline
56 & 1st & -low -lpc -mah & 11 & 9 & 55.00000000000001\\ \hline
57 & 1st & -low -lpc -mink & 11 & 9 & 55.00000000000001\\ \hline
58 & 1st & -low -lpc -nn & 5 & 15 & 25.0\\ \hline
59 & 1st & -low -lpc -randcl & 1 & 19 & 5.0\\ \hline
\end{tabular}
%\end{minipage}
\caption{Consolidated results, Part 2.}
\label{tab:results}
\end{table}

\begin{table}
%\begin{minipage}[b]{\textwidth}
\centering
\begin{tabular}{|c|c|l|c|c|l|} \hline
Run \# & Guess & Configuration & GOOD & BAD & Recogniton Rate,\%\\ \hline\hline
60 & 1st & -low -randfe -cheb & 5 & 15 & 25.0\\ \hline
61 & 1st & -low -randfe -eucl & 4 & 16 & 20.0\\ \hline
62 & 1st & -low -randfe -mah & 4 & 16 & 20.0\\ \hline
63 & 1st & -low -randfe -mink & 5 & 15 & 25.0\\ \hline
64 & 1st & -low -randfe -randcl & 2 & 18 & 10.0\\ \hline
65 & 1st & -norm -fft -cheb & 16 & 4 & 80.0\\ \hline
66 & 1st & -norm -fft -eucl & 16 & 4 & 80.0\\ \hline
67 & 1st & -norm -fft -mah & 16 & 4 & 80.0\\ \hline
68 & 1st & -norm -fft -mink & 15 & 5 & 75.0\\ \hline
69 & 1st & -norm -fft -randcl & 1 & 19 & 5.0\\ \hline
70 & 1st & -norm -lpc -cheb & 13 & 7 & 65.0\\ \hline
71 & 1st & -norm -lpc -eucl & 13 & 7 & 65.0\\ \hline
72 & 1st & -norm -lpc -mah & 13 & 7 & 65.0\\ \hline
73 & 1st & -norm -lpc -mink & 14 & 6 & 70.0\\ \hline
74 & 1st & -norm -lpc -nn & 3 & 17 & 15.0\\ \hline
75 & 1st & -norm -lpc -randcl & 1 & 19 & 5.0\\ \hline
76 & 1st & -norm -randfe -cheb & 5 & 15 & 25.0\\ \hline
77 & 1st & -norm -randfe -eucl & 6 & 14 & 30.0\\ \hline
78 & 1st & -norm -randfe -mah & 6 & 14 & 30.0\\ \hline
79 & 1st & -norm -randfe -mink & 5 & 15 & 25.0\\ \hline
80 & 1st & -norm -randfe -randcl & 3 & 17 & 15.0\\ \hline
81 & 2nd & -band -fft -cheb & 13 & 7 & 65.0\\ \hline
82 & 2nd & -band -fft -eucl & 13 & 7 & 65.0\\ \hline
83 & 2nd & -band -fft -mah & 13 & 7 & 65.0\\ \hline
84 & 2nd & -band -fft -mink & 10 & 10 & 50.0\\ \hline
85 & 2nd & -band -fft -randcl & 5 & 15 & 25.0\\ \hline
86 & 2nd & -band -lpc -cheb & 13 & 7 & 65.0\\ \hline
87 & 2nd & -band -lpc -eucl & 12 & 8 & 60.0\\ \hline
88 & 2nd & -band -lpc -mah & 12 & 8 & 60.0\\ \hline
89 & 2nd & -band -lpc -mink & 12 & 8 & 60.0\\ \hline
\end{tabular}
%\end{minipage}
\caption{Consolidated results, Part 3.}
\label{tab:results}
\end{table}

\begin{table}
%\begin{minipage}[b]{\textwidth}
\centering
\begin{tabular}{|c|c|l|c|c|l|} \hline
Run \# & Guess & Configuration & GOOD & BAD & Recogniton Rate,\%\\ \hline\hline
90 & 2nd & -band -lpc -nn & 3 & 17 & 15.0\\ \hline
91 & 2nd & -band -lpc -randcl & 3 & 17 & 15.0\\ \hline
92 & 2nd & -band -randfe -cheb & 3 & 17 & 15.0\\ \hline
93 & 2nd & -band -randfe -eucl & 3 & 17 & 15.0\\ \hline
94 & 2nd & -band -randfe -mah & 3 & 17 & 15.0\\ \hline
95 & 2nd & -band -randfe -mink & 4 & 16 & 20.0\\ \hline
96 & 2nd & -band -randfe -randcl & 3 & 17 & 15.0\\ \hline
97 & 2nd & -boost -fft -cheb & 15 & 5 & 75.0\\ \hline
98 & 2nd & -boost -fft -eucl & 15 & 5 & 75.0\\ \hline
99 & 2nd & -boost -fft -mah & 15 & 5 & 75.0\\ \hline
100 & 2nd & -boost -fft -mink & 17 & 3 & 85.0\\ \hline
101 & 2nd & -boost -fft -randcl & 0 & 20 & 0.0\\ \hline
102 & 2nd & -boost -lpc -cheb & 15 & 5 & 75.0\\ \hline
103 & 2nd & -boost -lpc -eucl & 15 & 5 & 75.0\\ \hline
104 & 2nd & -boost -lpc -mah & 15 & 5 & 75.0\\ \hline
105 & 2nd & -boost -lpc -mink & 16 & 4 & 80.0\\ \hline
106 & 2nd & -boost -lpc -nn & 2 & 18 & 10.0\\ \hline
107 & 2nd & -boost -lpc -randcl & 2 & 18 & 10.0\\ \hline
108 & 2nd & -boost -randfe -cheb & 6 & 14 & 30.0\\ \hline
109 & 2nd & -boost -randfe -eucl & 7 & 13 & 35.0\\ \hline
110 & 2nd & -boost -randfe -mah & 7 & 13 & 35.0\\ \hline
111 & 2nd & -boost -randfe -mink & 7 & 13 & 35.0\\ \hline
112 & 2nd & -boost -randfe -randcl & 4 & 16 & 20.0\\ \hline
113 & 2nd & -high -fft -cheb & 18 & 2 & 90.0\\ \hline
114 & 2nd & -high -fft -eucl & 18 & 2 & 90.0\\ \hline
115 & 2nd & -high -fft -mah & 18 & 2 & 90.0\\ \hline
116 & 2nd & -high -fft -mink & 17 & 3 & 85.0\\ \hline
117 & 2nd & -high -fft -randcl & 1 & 19 & 5.0\\ \hline
118 & 2nd & -high -lpc -cheb & 15 & 5 & 75.0\\ \hline
119 & 2nd & -high -lpc -eucl & 14 & 6 & 70.0\\ \hline
\end{tabular}
%\end{minipage}
\caption{Consolidated results, Part 4.}
\label{tab:results}
\end{table}

\begin{table}
%\begin{minipage}[b]{\textwidth}
\centering
\begin{tabular}{|c|c|l|c|c|l|} \hline
Run \# & Guess & Configuration & GOOD & BAD & Recogniton Rate,\%\\ \hline\hline
120 & 2nd & -high -lpc -mah & 14 & 6 & 70.0\\ \hline
121 & 2nd & -high -lpc -mink & 13 & 7 & 65.0\\ \hline
122 & 2nd & -high -lpc -nn & 3 & 17 & 15.0\\ \hline
123 & 2nd & -high -lpc -randcl & 3 & 17 & 15.0\\ \hline
124 & 2nd & -high -randfe -cheb & 4 & 16 & 20.0\\ \hline
125 & 2nd & -high -randfe -eucl & 3 & 17 & 15.0\\ \hline
126 & 2nd & -high -randfe -mah & 3 & 17 & 15.0\\ \hline
127 & 2nd & -high -randfe -mink & 4 & 16 & 20.0\\ \hline
128 & 2nd & -high -randfe -randcl & 2 & 18 & 10.0\\ \hline
129 & 2nd & -low -fft -cheb & 17 & 3 & 85.0\\ \hline
130 & 2nd & -low -fft -eucl & 17 & 3 & 85.0\\ \hline
131 & 2nd & -low -fft -mah & 17 & 3 & 85.0\\ \hline
132 & 2nd & -low -fft -mink & 17 & 3 & 85.0\\ \hline
133 & 2nd & -low -fft -randcl & 2 & 18 & 10.0\\ \hline
134 & 2nd & -low -lpc -cheb & 17 & 3 & 85.0\\ \hline
135 & 2nd & -low -lpc -eucl & 16 & 4 & 80.0\\ \hline
136 & 2nd & -low -lpc -mah & 16 & 4 & 80.0\\ \hline
137 & 2nd & -low -lpc -mink & 15 & 5 & 75.0\\ \hline
138 & 2nd & -low -lpc -nn & 5 & 15 & 25.0\\ \hline
139 & 2nd & -low -lpc -randcl & 3 & 17 & 15.0\\ \hline
140 & 2nd & -low -randfe -cheb & 7 & 13 & 35.0\\ \hline
141 & 2nd & -low -randfe -eucl & 7 & 13 & 35.0\\ \hline
142 & 2nd & -low -randfe -mah & 7 & 13 & 35.0\\ \hline
143 & 2nd & -low -randfe -mink & 7 & 13 & 35.0\\ \hline
144 & 2nd & -low -randfe -randcl & 2 & 18 & 10.0\\ \hline
145 & 2nd & -norm -fft -cheb & 18 & 2 & 90.0\\ \hline
146 & 2nd & -norm -fft -eucl & 18 & 2 & 90.0\\ \hline
147 & 2nd & -norm -fft -mah & 18 & 2 & 90.0\\ \hline
148 & 2nd & -norm -fft -mink & 17 & 3 & 85.0\\ \hline
149 & 2nd & -norm -fft -randcl & 2 & 18 & 10.0\\ \hline
\end{tabular}
%\end{minipage}
\caption{Consolidated results, Part 5.}
\label{tab:results}
\end{table}

\begin{table}
%\begin{minipage}[b]{\textwidth}
\centering
\begin{tabular}{|c|c|l|c|c|l|} \hline
Run \# & Guess & Configuration & GOOD & BAD & Recogniton Rate,\%\\ \hline\hline
150 & 2nd & -norm -lpc -cheb & 16 & 4 & 80.0\\ \hline
151 & 2nd & -norm -lpc -eucl & 15 & 5 & 75.0\\ \hline
152 & 2nd & -norm -lpc -mah & 15 & 5 & 75.0\\ \hline
153 & 2nd & -norm -lpc -mink & 17 & 3 & 85.0\\ \hline
154 & 2nd & -norm -lpc -nn & 3 & 17 & 15.0\\ \hline
155 & 2nd & -norm -lpc -randcl & 3 & 17 & 15.0\\ \hline
156 & 2nd & -norm -randfe -cheb & 8 & 12 & 40.0\\ \hline
157 & 2nd & -norm -randfe -eucl & 9 & 11 & 45.0\\ \hline
158 & 2nd & -norm -randfe -mah & 9 & 11 & 45.0\\ \hline
159 & 2nd & -norm -randfe -mink & 8 & 12 & 40.0\\ \hline
160 & 2nd & -norm -randfe -randcl & 5 & 15 & 25.0\\ \hline
\end{tabular}
%\end{minipage}
\caption{Consolidated results, Part 6.}
\label{tab:results}
\end{table}



%\begin{table}
%\begin{minipage}[b]{\textwidth}
%\centering
%\begin{tabular}{|c|c|l|c|c|l|} \hline
%Run \# & Guess & Configuration & GOOD & BAD & Recogniton Rate,\%\\ \hline\hline
%\begin{table}
%\begin{minipage}[b]{\textwidth}
\centering
\begin{tabular}{|c|c|l|c|c|l|} \hline
Run \# & Guess & Configuration & GOOD & BAD & Recogniton Rate,\%\\ \hline\hline
1 & 1st & -band -fft -cheb & 7 & 13 & 35.0\\ \hline
2 & 1st & -band -fft -eucl & 9 & 11 & 45.0\\ \hline
3 & 1st & -band -fft -mah & 9 & 11 & 45.0\\ \hline
4 & 1st & -band -fft -mink & 6 & 14 & 30.0\\ \hline
5 & 1st & -band -fft -randcl & 2 & 18 & 10.0\\ \hline
6 & 1st & -band -lpc -cheb & 10 & 10 & 50.0\\ \hline
7 & 1st & -band -lpc -eucl & 10 & 10 & 50.0\\ \hline
8 & 1st & -band -lpc -mah & 10 & 10 & 50.0\\ \hline
9 & 1st & -band -lpc -mink & 9 & 11 & 45.0\\ \hline
10 & 1st & -band -lpc -nn & 3 & 17 & 15.0\\ \hline
11 & 1st & -band -lpc -randcl & 2 & 18 & 10.0\\ \hline
12 & 1st & -band -randfe -cheb & 2 & 18 & 10.0\\ \hline
13 & 1st & -band -randfe -eucl & 2 & 18 & 10.0\\ \hline
14 & 1st & -band -randfe -mah & 2 & 18 & 10.0\\ \hline
15 & 1st & -band -randfe -mink & 1 & 19 & 5.0\\ \hline
16 & 1st & -band -randfe -randcl & 1 & 19 & 5.0\\ \hline
17 & 1st & -boost -fft -cheb & 12 & 8 & 60.0\\ \hline
18 & 1st & -boost -fft -eucl & 13 & 7 & 65.0\\ \hline
19 & 1st & -boost -fft -mah & 13 & 7 & 65.0\\ \hline
20 & 1st & -boost -fft -mink & 12 & 8 & 60.0\\ \hline
21 & 1st & -boost -fft -randcl & 0 & 20 & 0.0\\ \hline
22 & 1st & -boost -lpc -cheb & 13 & 7 & 65.0\\ \hline
23 & 1st & -boost -lpc -eucl & 13 & 7 & 65.0\\ \hline
24 & 1st & -boost -lpc -mah & 13 & 7 & 65.0\\ \hline
25 & 1st & -boost -lpc -mink & 14 & 6 & 70.0\\ \hline
26 & 1st & -boost -lpc -nn & 2 & 18 & 10.0\\ \hline
27 & 1st & -boost -lpc -randcl & 0 & 20 & 0.0\\ \hline
28 & 1st & -boost -randfe -cheb & 5 & 15 & 25.0\\ \hline
29 & 1st & -boost -randfe -eucl & 5 & 15 & 25.0\\ \hline
\end{tabular}
%\end{minipage}
\caption{Consolidated results, Part 1.}
\label{tab:results}
\end{table}

\begin{table}
%\begin{minipage}[b]{\textwidth}
\centering
\begin{tabular}{|c|c|l|c|c|l|} \hline
Run \# & Guess & Configuration & GOOD & BAD & Recogniton Rate,\%\\ \hline\hline
30 & 1st & -boost -randfe -mah & 5 & 15 & 25.0\\ \hline
31 & 1st & -boost -randfe -mink & 4 & 16 & 20.0\\ \hline
32 & 1st & -boost -randfe -randcl & 1 & 19 & 5.0\\ \hline
33 & 1st & -high -fft -cheb & 15 & 5 & 75.0\\ \hline
34 & 1st & -high -fft -eucl & 16 & 4 & 80.0\\ \hline
35 & 1st & -high -fft -mah & 16 & 4 & 80.0\\ \hline
36 & 1st & -high -fft -mink & 16 & 4 & 80.0\\ \hline
37 & 1st & -high -fft -randcl & 1 & 19 & 5.0\\ \hline
38 & 1st & -high -lpc -cheb & 12 & 8 & 60.0\\ \hline
39 & 1st & -high -lpc -eucl & 11 & 9 & 55.00000000000001\\ \hline
40 & 1st & -high -lpc -mah & 11 & 9 & 55.00000000000001\\ \hline
41 & 1st & -high -lpc -mink & 9 & 11 & 45.0\\ \hline
42 & 1st & -high -lpc -nn & 3 & 17 & 15.0\\ \hline
43 & 1st & -high -lpc -randcl & 0 & 20 & 0.0\\ \hline
44 & 1st & -high -randfe -cheb & 3 & 17 & 15.0\\ \hline
45 & 1st & -high -randfe -eucl & 3 & 17 & 15.0\\ \hline
46 & 1st & -high -randfe -mah & 3 & 17 & 15.0\\ \hline
47 & 1st & -high -randfe -mink & 3 & 17 & 15.0\\ \hline
48 & 1st & -high -randfe -randcl & 2 & 18 & 10.0\\ \hline
49 & 1st & -low -fft -cheb & 15 & 5 & 75.0\\ \hline
50 & 1st & -low -fft -eucl & 14 & 6 & 70.0\\ \hline
51 & 1st & -low -fft -mah & 14 & 6 & 70.0\\ \hline
52 & 1st & -low -fft -mink & 13 & 7 & 65.0\\ \hline
53 & 1st & -low -fft -randcl & 1 & 19 & 5.0\\ \hline
54 & 1st & -low -lpc -cheb & 13 & 7 & 65.0\\ \hline
55 & 1st & -low -lpc -eucl & 11 & 9 & 55.00000000000001\\ \hline
56 & 1st & -low -lpc -mah & 11 & 9 & 55.00000000000001\\ \hline
57 & 1st & -low -lpc -mink & 11 & 9 & 55.00000000000001\\ \hline
58 & 1st & -low -lpc -nn & 5 & 15 & 25.0\\ \hline
59 & 1st & -low -lpc -randcl & 1 & 19 & 5.0\\ \hline
\end{tabular}
%\end{minipage}
\caption{Consolidated results, Part 2.}
\label{tab:results}
\end{table}

\begin{table}
%\begin{minipage}[b]{\textwidth}
\centering
\begin{tabular}{|c|c|l|c|c|l|} \hline
Run \# & Guess & Configuration & GOOD & BAD & Recogniton Rate,\%\\ \hline\hline
60 & 1st & -low -randfe -cheb & 5 & 15 & 25.0\\ \hline
61 & 1st & -low -randfe -eucl & 4 & 16 & 20.0\\ \hline
62 & 1st & -low -randfe -mah & 4 & 16 & 20.0\\ \hline
63 & 1st & -low -randfe -mink & 5 & 15 & 25.0\\ \hline
64 & 1st & -low -randfe -randcl & 2 & 18 & 10.0\\ \hline
65 & 1st & -norm -fft -cheb & 16 & 4 & 80.0\\ \hline
66 & 1st & -norm -fft -eucl & 16 & 4 & 80.0\\ \hline
67 & 1st & -norm -fft -mah & 16 & 4 & 80.0\\ \hline
68 & 1st & -norm -fft -mink & 15 & 5 & 75.0\\ \hline
69 & 1st & -norm -fft -randcl & 1 & 19 & 5.0\\ \hline
70 & 1st & -norm -lpc -cheb & 13 & 7 & 65.0\\ \hline
71 & 1st & -norm -lpc -eucl & 13 & 7 & 65.0\\ \hline
72 & 1st & -norm -lpc -mah & 13 & 7 & 65.0\\ \hline
73 & 1st & -norm -lpc -mink & 14 & 6 & 70.0\\ \hline
74 & 1st & -norm -lpc -nn & 3 & 17 & 15.0\\ \hline
75 & 1st & -norm -lpc -randcl & 1 & 19 & 5.0\\ \hline
76 & 1st & -norm -randfe -cheb & 5 & 15 & 25.0\\ \hline
77 & 1st & -norm -randfe -eucl & 6 & 14 & 30.0\\ \hline
78 & 1st & -norm -randfe -mah & 6 & 14 & 30.0\\ \hline
79 & 1st & -norm -randfe -mink & 5 & 15 & 25.0\\ \hline
80 & 1st & -norm -randfe -randcl & 3 & 17 & 15.0\\ \hline
81 & 2nd & -band -fft -cheb & 13 & 7 & 65.0\\ \hline
82 & 2nd & -band -fft -eucl & 13 & 7 & 65.0\\ \hline
83 & 2nd & -band -fft -mah & 13 & 7 & 65.0\\ \hline
84 & 2nd & -band -fft -mink & 10 & 10 & 50.0\\ \hline
85 & 2nd & -band -fft -randcl & 5 & 15 & 25.0\\ \hline
86 & 2nd & -band -lpc -cheb & 13 & 7 & 65.0\\ \hline
87 & 2nd & -band -lpc -eucl & 12 & 8 & 60.0\\ \hline
88 & 2nd & -band -lpc -mah & 12 & 8 & 60.0\\ \hline
89 & 2nd & -band -lpc -mink & 12 & 8 & 60.0\\ \hline
\end{tabular}
%\end{minipage}
\caption{Consolidated results, Part 3.}
\label{tab:results}
\end{table}

\begin{table}
%\begin{minipage}[b]{\textwidth}
\centering
\begin{tabular}{|c|c|l|c|c|l|} \hline
Run \# & Guess & Configuration & GOOD & BAD & Recogniton Rate,\%\\ \hline\hline
90 & 2nd & -band -lpc -nn & 3 & 17 & 15.0\\ \hline
91 & 2nd & -band -lpc -randcl & 3 & 17 & 15.0\\ \hline
92 & 2nd & -band -randfe -cheb & 3 & 17 & 15.0\\ \hline
93 & 2nd & -band -randfe -eucl & 3 & 17 & 15.0\\ \hline
94 & 2nd & -band -randfe -mah & 3 & 17 & 15.0\\ \hline
95 & 2nd & -band -randfe -mink & 4 & 16 & 20.0\\ \hline
96 & 2nd & -band -randfe -randcl & 3 & 17 & 15.0\\ \hline
97 & 2nd & -boost -fft -cheb & 15 & 5 & 75.0\\ \hline
98 & 2nd & -boost -fft -eucl & 15 & 5 & 75.0\\ \hline
99 & 2nd & -boost -fft -mah & 15 & 5 & 75.0\\ \hline
100 & 2nd & -boost -fft -mink & 17 & 3 & 85.0\\ \hline
101 & 2nd & -boost -fft -randcl & 0 & 20 & 0.0\\ \hline
102 & 2nd & -boost -lpc -cheb & 15 & 5 & 75.0\\ \hline
103 & 2nd & -boost -lpc -eucl & 15 & 5 & 75.0\\ \hline
104 & 2nd & -boost -lpc -mah & 15 & 5 & 75.0\\ \hline
105 & 2nd & -boost -lpc -mink & 16 & 4 & 80.0\\ \hline
106 & 2nd & -boost -lpc -nn & 2 & 18 & 10.0\\ \hline
107 & 2nd & -boost -lpc -randcl & 2 & 18 & 10.0\\ \hline
108 & 2nd & -boost -randfe -cheb & 6 & 14 & 30.0\\ \hline
109 & 2nd & -boost -randfe -eucl & 7 & 13 & 35.0\\ \hline
110 & 2nd & -boost -randfe -mah & 7 & 13 & 35.0\\ \hline
111 & 2nd & -boost -randfe -mink & 7 & 13 & 35.0\\ \hline
112 & 2nd & -boost -randfe -randcl & 4 & 16 & 20.0\\ \hline
113 & 2nd & -high -fft -cheb & 18 & 2 & 90.0\\ \hline
114 & 2nd & -high -fft -eucl & 18 & 2 & 90.0\\ \hline
115 & 2nd & -high -fft -mah & 18 & 2 & 90.0\\ \hline
116 & 2nd & -high -fft -mink & 17 & 3 & 85.0\\ \hline
117 & 2nd & -high -fft -randcl & 1 & 19 & 5.0\\ \hline
118 & 2nd & -high -lpc -cheb & 15 & 5 & 75.0\\ \hline
119 & 2nd & -high -lpc -eucl & 14 & 6 & 70.0\\ \hline
\end{tabular}
%\end{minipage}
\caption{Consolidated results, Part 4.}
\label{tab:results}
\end{table}

\begin{table}
%\begin{minipage}[b]{\textwidth}
\centering
\begin{tabular}{|c|c|l|c|c|l|} \hline
Run \# & Guess & Configuration & GOOD & BAD & Recogniton Rate,\%\\ \hline\hline
120 & 2nd & -high -lpc -mah & 14 & 6 & 70.0\\ \hline
121 & 2nd & -high -lpc -mink & 13 & 7 & 65.0\\ \hline
122 & 2nd & -high -lpc -nn & 3 & 17 & 15.0\\ \hline
123 & 2nd & -high -lpc -randcl & 3 & 17 & 15.0\\ \hline
124 & 2nd & -high -randfe -cheb & 4 & 16 & 20.0\\ \hline
125 & 2nd & -high -randfe -eucl & 3 & 17 & 15.0\\ \hline
126 & 2nd & -high -randfe -mah & 3 & 17 & 15.0\\ \hline
127 & 2nd & -high -randfe -mink & 4 & 16 & 20.0\\ \hline
128 & 2nd & -high -randfe -randcl & 2 & 18 & 10.0\\ \hline
129 & 2nd & -low -fft -cheb & 17 & 3 & 85.0\\ \hline
130 & 2nd & -low -fft -eucl & 17 & 3 & 85.0\\ \hline
131 & 2nd & -low -fft -mah & 17 & 3 & 85.0\\ \hline
132 & 2nd & -low -fft -mink & 17 & 3 & 85.0\\ \hline
133 & 2nd & -low -fft -randcl & 2 & 18 & 10.0\\ \hline
134 & 2nd & -low -lpc -cheb & 17 & 3 & 85.0\\ \hline
135 & 2nd & -low -lpc -eucl & 16 & 4 & 80.0\\ \hline
136 & 2nd & -low -lpc -mah & 16 & 4 & 80.0\\ \hline
137 & 2nd & -low -lpc -mink & 15 & 5 & 75.0\\ \hline
138 & 2nd & -low -lpc -nn & 5 & 15 & 25.0\\ \hline
139 & 2nd & -low -lpc -randcl & 3 & 17 & 15.0\\ \hline
140 & 2nd & -low -randfe -cheb & 7 & 13 & 35.0\\ \hline
141 & 2nd & -low -randfe -eucl & 7 & 13 & 35.0\\ \hline
142 & 2nd & -low -randfe -mah & 7 & 13 & 35.0\\ \hline
143 & 2nd & -low -randfe -mink & 7 & 13 & 35.0\\ \hline
144 & 2nd & -low -randfe -randcl & 2 & 18 & 10.0\\ \hline
145 & 2nd & -norm -fft -cheb & 18 & 2 & 90.0\\ \hline
146 & 2nd & -norm -fft -eucl & 18 & 2 & 90.0\\ \hline
147 & 2nd & -norm -fft -mah & 18 & 2 & 90.0\\ \hline
148 & 2nd & -norm -fft -mink & 17 & 3 & 85.0\\ \hline
149 & 2nd & -norm -fft -randcl & 2 & 18 & 10.0\\ \hline
\end{tabular}
%\end{minipage}
\caption{Consolidated results, Part 5.}
\label{tab:results}
\end{table}

\begin{table}
%\begin{minipage}[b]{\textwidth}
\centering
\begin{tabular}{|c|c|l|c|c|l|} \hline
Run \# & Guess & Configuration & GOOD & BAD & Recogniton Rate,\%\\ \hline\hline
150 & 2nd & -norm -lpc -cheb & 16 & 4 & 80.0\\ \hline
151 & 2nd & -norm -lpc -eucl & 15 & 5 & 75.0\\ \hline
152 & 2nd & -norm -lpc -mah & 15 & 5 & 75.0\\ \hline
153 & 2nd & -norm -lpc -mink & 17 & 3 & 85.0\\ \hline
154 & 2nd & -norm -lpc -nn & 3 & 17 & 15.0\\ \hline
155 & 2nd & -norm -lpc -randcl & 3 & 17 & 15.0\\ \hline
156 & 2nd & -norm -randfe -cheb & 8 & 12 & 40.0\\ \hline
157 & 2nd & -norm -randfe -eucl & 9 & 11 & 45.0\\ \hline
158 & 2nd & -norm -randfe -mah & 9 & 11 & 45.0\\ \hline
159 & 2nd & -norm -randfe -mink & 8 & 12 & 40.0\\ \hline
160 & 2nd & -norm -randfe -randcl & 5 & 15 & 25.0\\ \hline
\end{tabular}
%\end{minipage}
\caption{Consolidated results, Part 6.}
\label{tab:results}
\end{table}


%1 & {\bf 1st} & {\bf -norm -fft -eucl} &  {\bf 14} & {\bf 6} & {\bf 70.0}\\
% & 2nd & -norm -fft -eucl &  18 & 2 & 90.0\\ \hline
%2 & 1st & -high -lpc -cheb &  12 & 8 & 60.0\\
% & 2nd & -high -lpc -cheb &  14 & 6 & 70.0\\ \hline
%3 & {\bf 1st} & {\bf -low -fft -eucl} &  {\bf 14} & {\bf 6} & {\bf 70.0}\\
% & 2nd & -low -fft -eucl &  17 & 3 & 85.0\\ \hline
%4 & {\bf 1st} & {\bf -boost -fft -cheb} &  {\bf 14} & {\bf 6} & {\bf 70.0}\\
% & 2nd & -boost -fft -cheb &  15 & 5 & 75.0\\ \hline
%5 & 1st & -norm -lpc -cheb &  13 & 7 & 65.0\\
% & 2nd & -norm -lpc -cheb &  15 & 5 & 75.0\\ \hline
%6 & {\bf 1st} & {\bf -high -fft -eucl} &  {\bf 14} & {\bf 6} & {\bf 70.0}\\
% & 2nd & -high -fft -eucl &  18 & 2 & 90.0\\ \hline
%7 & 1st & -low -lpc -eucl &  11 & 9 & 55.00000000000001\\
% & 2nd & -low -lpc -eucl &  13 & 7 & 65.0\\ \hline
%8 & 1st & -boost -lpc -cheb &  13 & 7 & 65.0\\
% & 2nd & -boost -lpc -cheb &  15 & 5 & 75.0\\ \hline
%9 & 1st & -high -lpc -eucl &  11 & 9 & 55.00000000000001\\
% & 2nd & -high -lpc -eucl &  14 & 6 & 70.0\\ \hline
%{\bf 10} & {\sc 1st} & {\sc -norm -fft -cheb} &  {\bf 16} & {\bf 4} & {\bf 80.0}\\
% & 2nd & -norm -fft -cheb &  17 & 3 & 85.0\\ \hline
%11 & 1st & -boost -fft -eucl &  13 & 7 & 65.0\\
% & 2nd & -boost -fft -eucl &  16 & 4 & 80.0\\ \hline
%12 & 1st & -low -fft -cheb &  12 & 8 & 60.0\\
% & 2nd & -low -fft -cheb &  14 & 6 & 70.0\\ \hline
%13 & 1st & -norm -lpc -eucl &  13 & 7 & 65.0\\
% & 2nd & -norm -lpc -eucl &  15 & 5 & 75.0\\ \hline
%14 & {\bf 1st} & {\bf -high -fft -cheb} &  {\bf 14} & {\bf 6} & {\bf 70.0}\\
% & 2nd & -high -fft -cheb &  18 & 2 & 90.0\\ \hline
%15 & 1st & -boost -lpc -eucl &  13 & 7 & 65.0\\
% & 2nd & -boost -lpc -eucl &  15 & 5 & 75.0\\ \hline
%16 & 1st & -low -lpc -cheb &  13 & 7 & 65.0\\
% & 2nd & -low -lpc -cheb &  15 & 5 & 75.0\\ \hline
%\end{tabular}
%\end{minipage}
%\caption{Consolidated results.}
%\label{tab:results}
%\end{table}

%\input{old-results.tex}
