\section{Conclusions}
\noindent
\rule{7.0in}{.013in}

Our best configuration yielded 80\% correctness
of our work when identifying subjects. Having a total
of 20 testing samples (including the two music bands) that means
16 subjects identified correctly out of 20 per run on average.

The main reasons the recognition rate could be that low
is due to ununiform sample taking, lack of good preprocessing
techniques, such as noise/silence removal, lack
of sophisticated classification modules (e.g. Stochastic models),
and lack the samples themselves to train and test on.

Even though for commercial and University-level research
standards 80\% recognition rate is considered to be very
low as opposed to a required minimum of 95\%-97\% and above,
we think it is still reasonably well provided that this was
a school project and is maintained as a hobby now.
That still involved a substantial
amount of research and findings considering our workload
and lack of experience in the area.

We would like to thank Dr. Suen and Mr. Sadri
for the course and help provided.
