\title{The <productname>CVS</productname> Repository}

  The <productname>MARF</productname> source code is stored and managed using the
  <productname>CVS</productname> code management system at SourceForge.

  Anonymous CVS
  is available to pull the <productname>CVS</productname> code tree from the
  <productname>MARF</productname> package to your local machine.

{\bf Getting The Source Via Anonymous <productname>CVS</productname>}

   If you would like to keep up with the current sources on a regular
   basis, you can fetch them from SourceForge <productname>CVS</productname> server
   and then use <productname>CVS</productname> to
   retrieve updates from time to time.

   Anonymous CVS

\item
     You will need a local copy of <productname>CVS</productname>
     (Concurrent Version Control System), which you can get from
     <ulink url="http://www.cyclic.com/">http://www.cyclic.com/</ulink> or
     any GNU software archive site. There is also WinCVS and
     CVS mode built in JBulider if you plan to use these products
     on Win32 platforms.

\item
     Do an initial login to the <productname>CVS</productname> server:

     \begin{verbatim}
$ cvs -d:pserver:anonymous@cvs.marf.sourceforge.org:/cvsroot/marf login
     \end{verbatim}

     You will be prompted for a password; just press <literal>ENTER</literal>.
     You should only need to do this once, since the password will be
     saved in <literal>.cvspass</literal> in your home directory.

\item
     Fetch the <productname>MARF</productname> sources:
     \begin{verbatim}
cvs -z3 -d:pserver:anonymous@cvs.marf.sourceforge.org:/cvsroot/marf co -P marf
     \end{verbatim}

     which installs the <productname>PostgreSQL</productname> sources into a
     subdirectory \verb+marf+
     of the directory you are currently in.

     If you'd like to download sample apps which use <productname>MARF</productname>:
\begin{verbatim}
cvs -z3 -d:pserver:anonymous@cvs.marf.sourceforge.org:/cvsroot/marf co -P apps
\end{verbatim}


\item
     Whenever you want to update to the latest <productname>CVS</productname> sources,
     <command>cd</command> into
     the \verb+marf+ or \verb+apps+ subdirectories, and issue
\begin{verbatim}
$ cvs -z3 update -d -P
\end{verbatim}

     This will fetch only the changes since the last time you updated.

\item
     You can save yourself some typing by making a file \verb+.cvsrc+
     in your home directory that contains

\begin{verbatim}
cvs -z3
update -d -P
\end{verbatim}

     This supplies the <option>-z3</option> option to all cvs commands, and the
     <option>-d</option> and <option>-P</option> options to cvs update.  Then you just have
     to say
\begin{verbatim}
$ cvs update
\end{verbatim}

     to update your files.
