\subsubsection{Fast Fourier Transform (FFT)}\label{sect:fft}

The Fast Fourier Transform (FFT) algorithm is used both for feature extraction and as the basis for the
filter algorithm used in preprocessing.  Although a complete discussion of the
FFT algorithm is beyond the scope of this document, a short description of the
implementation will be provided here.

Essentially the FFT is an optimized version of the Discrete Fourier Transform.
It takes a window of size $2^{k}$ and returns a complex array of coefficients
for the corresponding frequency curve.  For feature extraction, only the
magnitudes of the complex values are used, while the FFT filter operates
directly on the complex results.

The implementation involves two steps: First, shuffling the input positions by a
binary reversion process, and then combining the results via a ``butterfly''
decimation in time to produce the final frequency coefficients.
The first step corresponds to breaking down the time-domain sample of size $n$
into $n$ frequency-domain samples of size 1.  The second step re-combines the $n$
samples of size 1 into 1 n-sized frequency-domain sample.

The code used in MARF has been translated from the C code provided in the book,
``Numeric Recipes in C''.

\paragraph{FFT Feature Extraction}

The frequency-domain view of a window of a time-domain sample gives us the
frequency characteristics of that window.  In feature identification, the
frequency characteristics of a voice can be considered as a list of ``features''
for that voice.  If we combine all windows of a vocal sample by
taking the average between them, we can get the average frequency
characteristics of the sample.  Subsequently, if we average the frequency
characteristics for samples from the same speaker, we are essentially finding
the center of the cluster for the speaker's samples.  Once all speakers have
their cluster centers recorded in the training set, the speaker of an
input sample should be identifiable by comparing its frequency analysis with
each cluster center by some classification method.

Since we are dealing with speech, greater accuracy should be attainable by
comparing corresponding phonemes with each other.  That is, ``th'' in ``the''
should bear greater similarity to ``th'' in ``this'' than will ``the" and ``this'' when
compared as a whole.

The only characteristic of the FFT to worry about is the window used as input.
Using a normal rectangular window can result in glitches in the frequency
analysis because a sudden cutoff of a high frequency may distort the results.
Therefore it is necessary to apply a Hamming window to the input sample, and
to overlap the windows by half.  Since the Hamming window adds up to a constant
when overlapped, no distortion is introduced.

When comparing phonemes, a window size of about 2 or 3 ms is appropriate, but
when comparing whole words, a window size of about 20 ms is more likely to be
useful.  A larger window size produces a higher resolution in the frequency
analysis.
