\subsection{Spectrogram}

Sometimes it is useful to visualize the data we are
playing with. One of the typical thing when dealing with
sounds, specifically voice, people are interested in
spectrograms of frequency distributions. The \verb+Spectrogram+
class was designed to handle that and produce spectrograms
from both FFT and LPC algorithms and simply draw them. We did not
manage to make it a true GUI component yet, but instead we made
it to dump the spectrograms into PPM-format image files to be
looked at using some graphical package. Two examples of
such spectrograms are in the Appendix \ref{appx:spectra}.

We are just taking all the \verb+Output[]+ for the spectrogram. It's
supposed to be only half (\cite{shaughnessy2000}).
We took a hamming window of the waveform at $1/2$ intervals of 128
samples (ie: 8 kHz, 16 ms). By half intervals we mean that the second half
of the window was the first half of the next. O'Shaughnessy in \cite{shaughnessy2000} says this is a
good way to use the window. Thus, any streaming of waveform must consider
this.

What we did for both the FFT spectrogram and LPC determination was to
multiply the signal by the window and do a \verb+DoFFT()+ or a \verb+DoLPC()+ coefficient
determination on the resulting array of $N$ windowed samples. This gave us
an approximation of the stable signal at $s(i \cdot N/2)$.
Or course, we will have to experiment with windows and see which one is better, but
there may be no difinitive best.
