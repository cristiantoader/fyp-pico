 \title{MARF Installation Instructions}

  {\bf Short Version}

make


   The long version is the rest of this

\begin{verbatim}
  <title>Requirements</title>

  <para>
   In general, any modern platform should be able to run
   <productname>MARF</>.
  </para>

  <para>
   The following software packages are required for building
   <productname>MARF</>:

   <itemizedlist>
    <listitem>
     <para>
      <indexterm>
       <primary>make</primary>
      </indexterm>

      <acronym>GNU</> <application>make</> is required; other
      <application>make</> programs will <emphasis>not</> work.
      <acronym>GNU</> <application>make</> is often installed under
      the name <filename>gmake</filename>; this document will always
      refer to it by that name. (On some systems GNU make is the
      default tool with the name <filename>make</>.) To test for
      <acronym>GNU</acronym> <application>make</application> enter
<screen>
<userinput>gmake --version</userinput>
</screen>
      It is recommended to use version 3.76.1 or later.
     </para>
    </listitem>

    <listitem>
     <para>
      You need a Java compiler. Recent
      versions of <productname>javac</> are recommendable.
     </para>
    </listitem>

    <listitem>
     <para>
NeuralNetwork module requires the JAXP XML parser. You can get it
<ulink url="http://java.sun.com/xml/downloads/javaxmlpack.html">here</ulink>.

Click the "Download now" under the heading "Java XML Pack - Summer 02 Update
Release". This should be the right one.
     </para>
    </listitem>


    <listitem>
     <para>
      <indexterm>
       <primary>installation</primary>
       <secondary>on Windows</secondary>
      </indexterm>
     </para>
    </listitem>
   </itemizedlist>
  </para>

  <para>
   The following packages are required in the
   default configuration, as explained below.

   <itemizedlist>
    <listitem>
     <para>
     </para>

     <para>
     </para>
    </listitem>

   </itemizedlist>
  </para>

  <para>
  </para>

  <para>
  </para>
 </sect1>

<![%standalone-ignore;[
 <sect1 id="install-getsource">
  <title>Getting The Source</title>

  <para>
   The <productname>MARF</> sources can be obtained
   from <ulink
   url="http://marf.sf.net"></ulink>.
<screen>
<userinput>gunzip MARF-&version;.tar.gz</userinput>
<userinput>tar xf MARF-&version;.tar</userinput>
</screen>
   This will create a directory
   <filename>MARF-&version;</filename> under the current directory
   with the <productname>MARF</> sources.
   Change into that directory for the rest
   of the installation procedure.
  </para>
 </sect1>
]]>

 <sect1 id="install-upgrading">
  <title>If You Are Upgrading</title>

  <indexterm zone="install-upgrading">
   <primary>upgrading</primary>
  </indexterm>

  <para>
   The internal data storage format changes with new releases of
   <productname>MARF</>. Therefore, .
  </para>

  <procedure>
   <step>
    <para>
    </para>
   </step>

   <step>
    <para>
     <indexterm>
      <primary></primary>
     </indexterm>

    </para>

    <para>
    </para>

    <para>
    </para>
   </step>

   <step>
    <para>
    </para>

    <para>
    </para>
   </step>

   <step>
    <para>
     If you are installing in the same place as the old version then
     it is also a good idea to move the old installation out of the
     way, in case you have trouble and need to revert to it.
     Use a command like this:
<screen>
<userinput>mv</>
</screen>
    </para>
   </step>
  </procedure>

  <para>
  </para>

  <para>
  </para>
 </sect1>


 <sect1 id="install-procedure">
  <title>Installation Procedure</title>

  <procedure>

  <step id="configure">
   <title>Configuration</>

   <indexterm zone="configure">
    <primary></primary>
   </indexterm>

   <para>
   </para>

   <para>
     <note>
      <para>
      </para>
     </note>
    </para>

    </para>
   </step>

  <step>
   <title>Build</title>

   <para>
    To start the build, type
<screen>
<userinput>gmake</userinput>
</screen>
    (Remember to use <acronym>GNU</> <application>make</>.) The build
    may take anywhere from 5 minutes to half an hour depending on your
    hardware. The last line displayed should be
<screen>
(-: MARF build has been successful :-)
</screen>
   </para>
  </step>

  <step>
   <title>Regression Tests</title>

   <indexterm>
    <primary>regression test</primary>
   </indexterm>

   <para>
    If you want to test the newly built server before you install it,
    you can run the regression tests at this point. The regression
    tests are a test suite to verify that <productname>MARF</>
    runs on your machine in the way the developers expected it
    to. Type
<screen>
<userinput>gmake test</userinput>
</screen>
   </para>
  </step>

  <step id="install">
   <title>Installing The Files</title>

   <note>
    <para>
     If you are upgrading an existing system and are going to install
     the new files over the old ones, then you should have backed up
     your data and shut down the old server by now, as explained in
     <xref linkend="install-upgrading"> above.
    </para>
   </note>

   <para>
    To install <productname>MARF</> enter
<screen>
<userinput>gmake install</userinput>
</screen>
    This will install files into the directories that were specified.
   </para>

   <para>
   </para>

   <para>
   </para>

   <para>
   </para>

  </step>
  </procedure>

  <formalpara>
   <title>Uninstall:</title>
   <para>
    To undo the installation use the command <command>gmake
    uninstall</>. However, this will not remove any created directories.
   </para>
  </formalpara>

  <formalpara>
   <title>Cleaning:</title>

   <para>
    After the installation you can make room by removing the built
    files from the source tree with the command <command>gmake
    clean</>. This will preserve the files made by the configure
    program, so that you can rebuild everything with <command>gmake</>
    later on.
   </para>
  </formalpara>

  <para>
   If you perform a build and then discover that your configure
   options were wrong, or if you change anything that configure
   investigates (for example, software upgrades), then it's a good
   idea to do <command>gmake distclean</> before reconfiguring and
   rebuilding.  Without this, your changes in configuration choices
   may not propagate everywhere they need to.
  </para>
 </sect1>

 <sect1 id="install-post">
  <title>Post-Installation Setup</title>

  <sect2>
   <title>Shared Libraries</title>

   <indexterm>
    <primary>shared libraries</primary>
   </indexterm>

   <para>
   </para>

  </sect2>

  <sect2>
   <title>Environment Variables</title>

   <indexterm>
    <primary><envar>PATH</envar></primary>
   </indexterm>

   <para>
   </para>
  </sect2>
 </sect1>


<![%standalone-include;[
 <sect1 id="install-getting-started">
  <title>Getting Started</title>

  <para>
   The following is a quick summary of how to get <productname>MARF</> up and
   running once installed.
  </para>

  <procedure>
   <step>
    <para>
    </para>
   </step>

   <step>
    <para>
    </para>
   </step>

   <step>
    <para>
    </para>

    <para>
    </para>

    <para>
     In order to allow TCP/IP connections (rather than only Unix
     domain socket ones) you need to pass the <option>-i</> option to
     <filename>postmaster</>.
    </para>
   </step>

   <step>
    <para>
     Create a database:
<screen>
<userinput>createdb testdb</>
</screen>
     Then enter
<screen>
<userinput>psql testdb</>
</screen>
     to connect to that database. At the prompt you can enter SQL
     commands and start experimenting.
    </para>
   </step>
  </procedure>

  <title>What Now?</title>


  <title>Supported Platforms</title>
   <productname>MARF</> has been verified by the developers
   to work on the platforms listed below. A supported
   platform generally means that <productname>MARF</> builds and
   installs according to these instructions.
\end{verbatim}
