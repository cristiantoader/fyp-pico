% Chapter Template

\chapter{Threat Model} % Main chapter title

\label{ChapterX} % Change X to a consecutive number; for referencing this chapter elsewhere, use \ref{ChapterX}

\lhead{Chapter X. \emph{Threat Model}} % Change X to a consecutive number; this is for the header on each page - perhaps a shortened title

An accurate threat model on the proposed unlock mechanism must start by analysing the set of assumptions made about the mechanism. From there we can identify available threats and how the scheme can be exploited in order to unlock the Pico without owner permission. Throughout the threat model we will explain how relaxing the initial set of assumptions may change the security outcome.

We will start from the assumption that the unlock mechanism is integrated on the same device with the Pico. Furthermore, the set of available sensors will also be integrated within the device, or that they are peripheral sensors with no way of tampering with the communication to the authenticator. 



- assumptions
  - everything on one device
  - 
  
  
- replay attack
- data tampering
- man-in-the-middle between authenticator and pico
- 