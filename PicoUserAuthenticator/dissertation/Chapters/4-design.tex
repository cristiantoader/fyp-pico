% Chapter Template

\chapter{Design} % Main chapter title

\label{Chapter3} % Change X to a consecutive number; for referencing this chapter elsewhere, use \ref{ChapterX}

\lhead{Chapter 3. \emph{Design}} % Change X to a consecutive number; this is for the header on each page - perhaps a shortened title

\section{Design requirements}
% Motivation for design
The framework evaluation of Picosiblings provides insight as to how the scheme can be improved. We identify as a key downside that it does not guarantee the identity of the owner. This information is mainly inferred from the number of Picosibling shares in the proximity of the Pico. However, anyone may be in possession of the shares, therefore being temporarily granted full authentication privileges. This is reflected in the evaluation by failing to fully offer the `resilient-to-theft'' and ``non-disclosability'' properties. A further improvement can be made by introducing ``multi-level-unlocking'', allowing for multiple levels of authentication depending on the confidence in the owner's presence.

% Pico properties that need to be maintained
The Pico design proposed by Stajano \cite{stajano2011pico} claims two properties that also need to be supported by the token unlocking mechanism: the authentication process is memory effortless; and the unlocking scheme needs to support continuous authentication\footnote{Continuous authentication is defined by the ability to re-authenticate the user without the need for any physical effort.}. 

% TODO: maybe offer more details or what the requirements are
One of the goals of designing the new Pico unlocking mechanism is to fully satisfy the the properties presented in this section.

\section{Proposed solution}
\label{propopsedsol}
% combine multiple authentication mechanisms
The idea explored in this dissertation project is to simultaneously use multiple continuous authentication mechanisms. Each mechanism needs to provide a quantifiable confidence level which will be used in calculating a combined score. This satisfies the memoryless and continuous authentication properties required by Pico. By combining mechanisms we achieve a higher confidence of correctly identifying the owner. Furthermore, given that each individual mechanism supports continuous authentication, using them simultaneously does not create any inconvenience for the owner.

\subsection*{Multi-level unlocking model}
% multi-level unlocking model
The Pico token should no longer enter a general locked or unlocked state. Its most important secret, the ``Pico Master Key'' should be kept in tamper resistant memory, and be accessible at all times. Using the overall score computed by the proposed mechanism, Pico should offer granular user authentication. Each app needs to be associated with a confidence level defined during the registration process. If the overall confidence of the unlocking mechanism exceeds the app's confidence level, then the token becomes ``unlocked'' for that specific app. All authentication sessions between Pico and apps need to be managed independently based on this model.

% examples  of authentication mechanisms
The scheme should achieve continuous authentication, while correctly identifying the owner of the token. For this reason we have decided that authentication mechanisms combined in the scheme need to be based either on biometrics or behavioural analysis. Biometric features that can be used with this scheme include iris, face, voice, and gait. Behavioural sources of data can be obtained from frequent GPS location, travel paths, wireless network connections, and others.

% each mechanism has a weight
Each mechanism of the scheme is assigned a different initial weight that is based on the level of trust it offers in identifying the owner. This doesn't necessarily need to be related to the precision of the mechanism, but it would be a good indicator for choosing the value.

% how it is different than simple biometrics
The solution offered in this project is different from simply stating that Pico is using biometric data as an unlocking mechanism. The novelty in the design is based on how data is combined in order to compute the overall confidence level. 

\subsection*{Decaying weights}
% decaying weights
Data samples captured for the owner authentication process are not always meaningful. For example, accelerometer values for gait recognition are only usable when the user is travelling on foot. Depending on how the sensors are integrated with the Pico, camera input for face recognition may not always capture a valid image. The confidence of each mechanism should therefore decrease in time from the last valid authentication sample. This introduces another original feature of this scheme, which is having a decaying weight. Each mechanism starts with a predefined initial value, reflecting the weight of the mechanism in the overall unlocking process. This value is decreases in time until a valid user data sample is provided to the mechanism for authentication. 

% example of decaying confidence
Let us take for example a voice recognition mechanism which samples data every minute. The current weight of the mechanism is 0 so its output is completely ignored. The next sample is recorded, and the voice recognition mechanism outputs a confidence of $70\%$ that the owner is present. After the successful recording, the mechanism weight is updated to its predefined starting value of 30. For the next 10 minutes the owner will be silently reading a book. Since the mechanism only identifies background noise, the weight value of 30 decreases in time. This will induce a smaller impact of the voice recognition mechanism on the overall score. Each individual confidence level of a mechanism can decrease up to 0, at which point the mechanism is ignored. Computing the overall score will be explained in more detail later in the chapter.

\subsection{Explicit authentication}
% Explicit authentication mechanisms
We need to consider the case where the owner wants to use Pico to authenticate to a high security app, given a low confidence level from the authenticator. As an example, the Pico owner wants to access their bank account after sitting silent in a dark room for the past hour. Let us say the app requires a confidence level of $95\%$. Due to the lack of valid authentication data, the authenticator only outputs a $20\%$ overall confidence that the owner is present. To solve this problem we have introduced the concept of explicit authentication mechanisms. When the confidence score drops below the threshold required by an app, the user is given the chance to authenticate through an explicit request to provide valid sample data to one or more mechanisms.

% Combining explicit authentication mechanisms
Combining explicit and continuous authentication can be performed consistently with the current design. Whenever explicit authentication is required, the only difference for the owner is that they becomes aware of the authentication process. This grants them the chance to produce valid input for the authenticator, which may then proceed to compute an accurate score. Given that prior to the explicit authentication request the unlocking mechanism didn't produce a high enough confidence, it is assumed that this will also happen prior to that. Therefore, explicit authentication requests need to have a slower decay rate. This will enable the continuous authentication process.

\subsection*{Authentication feedback}
\label{authfeedback}
% Bayesian update
Each mechanism outputs a value, which is the probability that the sample data belongs to the owner of the token. Upon each recording, this probability is updated using Bayes' Law. This process is also known as a Bayesian update, and is descried in the following equation:

\begin{equation} 
\label{eq:bayes1}
P(H|E) = \frac{P(H) * P(E|H)}{P(E)}
\end{equation}

In the equation above:
\begin{itemize} 
	\item E: Stands for evidence and in this case represents the data sample.
	\item H: Stands for hypothesis. In this case we refer to the hypothesis that the owner is present.
	\item $P(H|E)$: Represents the probability of hypothesis $H$ after observing evidence $E$. This is the final probability we are trying to compute after each sample. It is also known as the posterior probability. 
	\item $P(H)$: Represents the probability of hypothesis $H$ before observing evidence $E$. This is also known as the prior probability and is the probability computed at the previous step.
	\item $P(E|H)$: Represents the probability that the current evidence belongs to hypothesis $H$. It is the probability outputted by the mechanism given the sample data.
	\item $P(E)$: This is the model evidence, and has a constant value for all hypothesis.
\end{itemize}

Although $P(E)$ is constant we need its value in order to calculate $P(H|E)$. We can compute it using the ``Law of total probability'', which is the following:

\begin{equation} 
\label{eq:lotp}
P(E) = \sum_{n}^{}P(E|H_n) * P(H_n)
\end{equation}

Using equation \ref{eq:lotp} Bayes' Law equation \ref{eq:bayes1} becomes:
\begin{equation} 
\label{eq:bayes2}
P(H|E) = \frac{P(H) * P(E|H)}{\sum_{n}^{}P(E|H_n) * P(H_n)}
\end{equation}

Our model however, contains only two hypothesis\footnote{Arguably there is a third case where the data sample is not a valid recording of an user. This case is ignored and no probability is computed. The only result in this scenario is the resuming of the decay process in the weight of the mechanism.}: the recording of the data either belongs to the owner, or not. We can therefore consider $P(H)$ to be the probability that the data belongs to the owner and $P(\neg H)$ the probability that the data belongs to someone else. This means the value of $P(\neg H)$ is $1 - P(H)$ and $P(E|\neg H) = 1 - P(E|H)$ Introducing this in equation \ref{eq:bayes2}, the rule for updating the mechanism's probability becomes:

\begin{equation} 
\label{eq:final}
P(H|E) = \frac{P(H) * P(E|H)}{P(H) * P(E|H) + P(\neg H) * P(E|\neg H)}
\end{equation}

Equation \ref{eq:final} represents the final probability that the owner is present given the sampled data. All the variables in this equation are known, for reasons explained above.

% Overall confidence
% 	TODO: update this to have wii and wid (decayed and initial), as well as above when describing the decay rate
Thus far we have defined how individual scores are calculated, and that each mechanism has a decaying weight. Using this data we can calculate the overall score of the scheme. This is performed quite trivially by using a weighted sum. The following equation describes the process. 

\begin{equation} 
\label{eq:overall}
P_{Total} = \frac{\sum_{i=1}^{n}(w_ii * P_i(H|E_i))}{\sum_{i=1}^{n}w_id}
\end{equation}

% Pico and confidence
Pico needs to use the overall score to adjust the state of its authentication sessions. If the score required by an app is lower than the overall score provided by the scheme, the user is granted authentication access. To provide continuous authentication, Pico will make periodic requests for confidence level updates. Feedback is provided by the unlocking mechanism based on available sample data, and decaying weights.

\section{Related work}
Clarke et al \cite{clarke2005authentication} present statistics confirming the need for an unlocking scheme different from PINs. They conduct a couple of surveys trying to assess the reliability of a PIN in unlocking a mobile phone. The study involves 297 participants and assess: day to day usage of mobile phone devices, existing authentication mechanisms, and the users' attitude towards further security options. The paper reveals a high number of bad practices involved in PIN authentication: $45\%$ of owners never change the default factory code, $42\%$ only change it once after buying the device, reusing the PIN with other authenticators, forgetting the PIN, and sharing the PIN with someone else.

A promising result showed in the paper is that $83\%$ of users are willing to accept some sort of biometric authentication mechanism in order to unlock their devices. The following biometric mechanisms were included in the study: fingerprint analysis, voice recognition, iris recognition, hand recognition, keystroke analysis \cite{clarke2003using}, and face recognition. The paper also shows that $61\%$ of users would accept a non intrusive biometric continuous authentication mechanism. Using multiple biometrics for continuous authentication is mentioned briefly, but each mechanism is used individually based on what the user is doing. As an example, when the user walks he is authenticated using gait recognition, and while he is speaking on the phone, voice recognition. This is a divergence point from what we are trying to achieve in this dissertation.

In a different paper, Clarke et al \cite{clarke2002acceptance} study PIN alternatives for mobile phone unlocking. The authors conduct a survey with interesting results. A remarkable $11\%$ of participants were not even aware of PIN authentication. An average of $81\%$ of participants agree that PINs should be replaced with a mechanism that provides better security. Although they report the need and desire for a different type of phone unlocking, many of them do not use currently available alternatives.

% TODO: this...
Gregory Williamson \cite{williamson2006enhanced} writes in his PhD dissertation  about the need for an enhanced security authentication mechanism for on-line banking. He proposes a multi-factor authentication model, and presents two interesting options: the traditional one where both mechanisms are required in the multi-factor model (blanket authentication), and one where the second authentication mechanism is only requested from the user if the transaction appears to be risky (risk mode authentication). A risky situation is defined as either an important transaction like withdrawing money, or a transaction made under unusual circumstances such as using an unknown device. 

A similar approach to ``risk mode authentication'' presented by Williamson \cite{williamson2006enhanced} is proposed in this project. Our scheme yields a confidence level which may or may not be sufficient to allow access for different transactions. If the confidence level is not high enough, an explicit authentication mechanism will prompt the user for input. As the dissertation by Williamson shows, $75\%$ of users questioned in his study agree with having biometric authentication as a second factor authentication to passwords. This shows promising results in adopting our scheme for token unlocking purposes.

Elena Vildjiounaite et al describe in their paper \cite{vildjiounaite2007increasing} a similar mechanism of combining biometric authentication data on mobile phone devices. The authors identify the security risk of granting long-term authentication after a single verification challenge. They explore an alternative based on a two stage ``risk mode authentication'' \cite{williamson2006enhanced}. The first stage combines biometric data in order to achieve continuous authentication. This is achieved by training a cascade classifier to a target false acceptance rate (FAR)\footnote{The false acceptance rate is the equivalent of false positive precision. It is the probability of incorrectly granting authentication privileges to an user}. Data from mechanisms is merged using a weighted sum fusion rule. Mechanism weights are chosen based on their error rates. The second stage is only enabled if the cascade classifier does not identify the owner as being present. In low noise scenarios, continuous authentication is achieved without the need for an explicit challenge $80\%$ of the time. In noisy situations (city and car noise), the percentage drops ranging from $40$ to $60\%$. The cascade classifier was trained with a FAR of $1\%$, with results showing a false rejection rate (FRR)\footnote{The false rejection rate is the probability of incorrectly denying access to the rightful owner.} of only $3$ to $7\%$.

The paper by Elena Vildjiounaite et al \cite{vildjiounaite2007increasing} is similar with the solution proposed in this dissertation. It also combines multiple authentication mechanisms, each being assigned different weights. Differences between the two are in the fact that weights are maintained static in time. The weights of the sums are computed differently, and there is no mention of Bayesian updates or probabilities. Furthermore, the authors use a classifier instead of producing a confidence level, which cannot be used for granting different levels of security. The results presented by this paper are however encouraging, showing that continuous authentication presents good results using multiple biometric authentication mechanisms.

\section{Conclusions}
% Conclusion
We have defined a conceptual model for a new Pico unlocking mechanism. The scheme is guaranteed to improve on the existing Picosiblings solution at least by offering a better way of correctly identifying its owner. We have offered reasonable explanation that it supports Pico's claims for continuous and memory effortless authentication. 

% TODO: sounds a bit bad..
An evaluation of the scheme is not yet offered, because mechanisms such as ``Negligible-cost-per-user'' are implementation dependent. The next chapter will present a prototype solution that will be evaluated using the token unlocking assessment framework. The results will be compared with the current Picosiblings implementation allowing for further analysis and conclusions.


