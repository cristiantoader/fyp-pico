% Chapter Template

\chapter{Conclusion} % Main chapter title

\label{Chapter7} % Change X to a consecutive number; for referencing this chapter elsewhere, use \ref{ChapterX}

\lhead{Chapter 7. \emph{Conclusion}} % Change X to a consecutive number; this is for the header on each page - perhaps a shortened title

The purpose of this dissertation was to provide an alternative unlocking scheme for the Pico token \cite{stajano2011pico}. We have started by analysing its design in chapter \ref{Chapter2}. Using this we have identified that a new unlocking mechanism requires to be memorywise effortless, and provide support for continuous authentication.

We have identified and briefly presented the web authentication UDS assessment framework developed by Bonneau et al \cite{bonneau2012quest}. This work provided an initial evaluation of Pico using Picosiblings. Analysing the paper revealed issues of the Pico token, some of which directly related to its unlocking mechanism (e.g. not easy to learn).

To have a better way of evaluating a token unlocking scheme, in section \ref{tokenframework} we have created a derivation of the original framework by Bonneau et al. Some properties were removed, and others were changed in order to fit the context of token unlocking mechanisms. Furthermore, we have extended the framework by adding 4 original properties.

Having a list of requirements, and a way of assessing the new solution, we have designed a scheme based on combining biometric and behavioural analysis mechanisms. Each mechanism generates a probability that the owner is in possession of the token. These probabilities are combined using a modified weighted sum, in order to generate an overall confidence level. Another original contribution is that each mechanism has an initial weight that decays in time from one valid data sample to the other. 

% android prototype
Given a well defined design of the new scheme, we have successfully developed a prototype using a Google Nexus 5 device that runs an Android 4.4.2 operating system. This has proven that the solution can be implemented using existing hardware. Furthermore, the prototype offered useful insight for future implementations of the scheme. We have concluded that an efficient implementation requires authentication mechanisms to run as independent processes. We have presented a basic application design, and showed how different components should interact. As presented, data should be validated prior to analysis, and in the lack of a valid sample, the weight decay process should continue. An unexpected problem was that although the Android platform offers a wide range of sensors for the implementation of multiple mechanisms (details in appendix \ref{AppendixC}), the lack of open source biometric libraries has lead to a low precision of the overall scheme.

% evaluation of scheme
The proposed unlocking mechanism was evaluated using the UDS framework developed by Bonneau et al, and the token unlocking framework developed in section \ref{tokenframework}. The results of the analysis have shown that the new proposed scheme cannot completely outperform Picosiblings due to the ``unlinkable'' property. Otherwise, results have shown an overall improvement in the number of offered properties. In addition, the new unlocking mechanism offers the possibility of having a granular unlocking, where the Pico token could offer individual locked and ulocked states based on the current confidence level and the security level required by the account.

A threat model of the prototype has shown a number of attacks that may be performed on the Android application. Important insight was provided when studying design model attacks in section \ref{secdesignattacks}. This has shown that in scenarios where no valid data can be collected, a compromise needs to be made regarding the time interval between successful explicit authentication requests. Together with the power consumption analysis, this essentially becomes an multivariate optimisation problem where we are trying to minimise power consumption and user inconvenience, while maximising accuracy.

\section*{Future work}
As shown in the previous section, the new token unlocking scheme offers Pico an overall improvement. However, additional work is required in order to improve details of the design, as well as the prototype. Additional details are presented in this section. 

% Experiment with weights and decay factors
The Android prototype was developed as a proof of concept. Further experiments need to be performed using different weights and decay functions. A user study is required in order to determine the acceptable time interval between consecutive explicit authentication requests, and the implementation needs to be adapted accordingly. When performing this analysis, the power consumption results from section \ref{} need to be considered, in order to improve the lifetime of the device.

% Add aditional mechanisms and improve accuracy of current.
The set of individual mechanisms used with the scheme's prototype can be improved. Explicit authentication mechanisms are not currently supported and need to be implemented. Better biometric libraries should be either developed or imported in order to increase accuracy. Furthermore, additional mechanisms should be added for the platform. A number of viable suggestions are made in appendix \ref{AppendixC}.

% Take data samples based on external events
With the current prototype, the voice and face recognition mechanisms sample data at fixed time intervals. This should be change by taking advantage of user behaviour and Android events. Examples for this were given in section \ref{impleoverview}. 

The face recognition mechanism can be improved either by introducing another library that performs face detection, or by using a different face recognition library that offers both features. Cryptographic support needs to be added for this mechanism. It can be performed through additional modifications of the {\tt Javafaces} library that would allow it to use raw binary data during the training process.

% Safer implementation
A safer prototype would be to develop a root system service using the Android NDK C compiler. The binary has to be included in the system partition of the boot image in order to be accessible by the {\tt init} process during start up. The {\tt init.rc} configuration file used by {\tt init} also needs to be configured to start the service. This implementation requires modifications to the {\tt /system} partition. The process does not limit to simply gaining root privileges. The root directory {\tt /} is mounted as ramdisk, and therefore any modifications will be reverted once the device is rebooted. In order to make persistent changes, the user needs to modify the boot image, and re-flash it on the device.
