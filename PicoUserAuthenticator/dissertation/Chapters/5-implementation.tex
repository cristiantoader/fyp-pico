% Chapter Template

\chapter{Implementation Prototype} % Main chapter title
%TODO:
%	- make sure that I emphasize that it is up to the user to allow malware to read data from sensors!

\label{Chapter4} % Change X to a consecutive number; for referencing this chapter elsewhere, use \ref{ChapterX}

\lhead{Chapter 4. \emph{Implementation Prototype}} 
% Introduction for implementation
In this chapter we will develop a prototype for the scheme proposed in section \ref{propopsedsol}. Just as presented by Stajano \cite{stajano2011pico}, smart phones offer a cheap alternative to prototype and test. Therefore, we have chosen as an implementation platform the Android Nexus 5 hand held device. It offers enough sensors to perform biometric and behavioural analysis\footnote{The full range of sensors supported by the Android platform can be found here: http://developer.android.com/guide/topics/sensors/sensors\_overview.html (accessed on 28.05.2014)}. These resources will be used to demonstrate that the scheme can be implemented using similar dedicated hardware that may offer more security features.

% Android
\subsection{Android development and security}
\label{androidsecurity}
% Introduction to android security and dev model
To gain a better understanding of different design decisions and limitations of our implementation, we will present a brief literature review of the Android development platform. Mechanisms and components will be described with an emphasis on security. The prototype developed for this dissertation is a proof of concept. However, we still aim to understand and make proper use of available security mechanisms. 

% First paper: introduction
William Enck et al \cite{enck2009understanding} offer a good introduction to Android application development. They focus on the security aspects of the development platform. It is a relatively old paper (2008), from the same year of the Android initial release. However, the fundamental design principles and security concepts that are discussed did not change considerably. The platform's open standards were made public in November 2007. This allowed researchers such as the authors of this paper to perform a pre-release analysis of the system.

% First paper: OS short description
Android uses as a core operating system a port of the Linux kernel. This introduces to the platform some of the Linux security mechanisms (i.e. file permissions, access control policies). On top of the kernel there is an application middleware layer composed out of the Java Dalvik virtual machine, core Java application libraries, as well as libraries which offer support for storage, sensors, display, and other device features. Applications are supported by the middleware and developed using the Android Java SDK.

% First paper: Android components
The Android development model is based on building an application from multiple components. Based on their purpose, the SDK defines four types: activity, service, content provider, and broadcast receiver. For the purpose of brevity we will not discuss each individual component\footnote{More details on the role of each component can be found on the Android website: http://developer.android.com/guide/components/fundamentals.html}. To allow meaningful interaction, Inter Component Communication (ICC) is enabled using special objects called Intents.

% First paper: binding services
The application we are developing needs to perform most of its processing in the background. It does not require any explicit user interaction. According to the Android model, this should be achieved using Services. To enable convenient component interaction, services may become bound engaging in a client-server communication. An important note made in the paper is that while a Service is bound, it cannot be terminated by an explicit stop action. This provides an useful guarantee regarding its lifetime, which we will use in the prototype.

% First paper: security enforcements, system level
The paper discusses two types of Android security enforcements: ICC, and system level. System level security is based on the Linux permission model. When installed, each app is allocated an UID and GID. This allows internal storage access control restrictions, keeping application data sandboxed from other apps.

% First paper: security enforcements, ICC
% 	TODO: too vague, can expand and cite something
ICC security is the main focus of the paper. Intent communication is based on commands sent to the ``/dev/binder'' device node. The node needs to be world readable and writeable by any application. Therefore, Android cannot mediate ICC using the Linux permissions model. Security relies on a Mandatory Access Control (MAC) framework enforced by a reference monitor. This mechanism validates requests sent to the ``/dev/binder'' node. 

% Manifest file
During development, each application needs to define a manifest file\footnote{Full details regarding the manifest file can be found on the Android website: http://developer.android.com/guide/topics/manifest/manifest-intro.html}. Some of the security configurations defined in this file are: declared components and their capabilities, permissions required by the app, and permissions other apps need to have in order to interact with app components. These entries are used as labels for the MAC framework. 

% First paper, types of components: public/private.
Using the app manifest file, each component can be defined as either public or private. This refinement is configured by the ``exported'' field. It defines whether or not another application may launch or interact with one of its components. When this paper was written, the ``exported'' field was defaulted to ``true''. However, as shown by Steffen and Mathias \cite{liebergeld2013android}  in 2013, starting with Android 4.2  the default of this value was changed to ``false'', and now conforms to the ``principle of least privilege''.

% First paper, Intent filters
Components listening for Intents need to have an intent-filter registered in the application manifest file. This allows them to export only a limited set of intents to other applications. Further restrictions to Intent objects are offered by the SDK using permission labels. This mechanism provides runtime security checks for the application. It is an additional prevention mechanism for data leaks through ICC. An application may broadcast an event throughout the system. By using permission labels, only apps that have the respective permission may process the event. Furthermore, Services may check for permissions when they are bound by another component. This allows them to expose different APIs depending on the binder.

% Paper two!
Steffen and Mathias \cite{liebergeld2013android} focus on deeper issues of the Android platform. They show how problems are solved from one Android version to the other. Unfortunately, OEMs tend not to update the software of their devices once they have shipped, which creates a high security risk.

The starting point of understanding Android security is learning how it is bootstrapped during the five step booting process:
\begin{enumerate}
	\item Initial bootloader (IBL) is loaded from ROM.
	\item IBL checks the signature of the bootloader (BL) and loads it into RAM.
	\item BL checks the signature of the linux kernel (LK) and loads it into RAM.
	\item LK initialises all existing hardware and starts the linux ``init'' process.
	\item The init process reads a configuration file and boots the rest of LA.
\end{enumerate}

The android security model is split by the paper in two categories: system security, and application security.

% keychain encryption and security
Android provides a keychain API used for storing sensitive material such as certificates and other credentials. These are encrypted using a master key, which is stored using AES encryption. Security needs to begin somewhere. An assumption has to be made about a state being secure from which multiple security extensions can be made. In this case, the master key is considered to be that point of security. However, given a rooted device, the master key itself can be retrieved from the system and therefore compromising all other credentials. The Android base system (libraries, app framework, and app runtime) is located in the ``system'' partition. Although it is writeable only by the root user, as mentioned before, exploits which grant this privilege exist. 

% Same author, shared privileges
From the user's perspective, an interesting ``feature'' which may affect the flow of information within Android is the fact that applications from the same author may share private resources. When installing an app the user needs to accept its predefined set of permissions. Due to resource sharing, a situation may present itself where an application that has permissions for the owner's contacts may communicate with an application that has permissions for internet in order to leak confidential data. A developer may therefore construct pairs of legitimate applications in order to mask a data flow attack.

% Android low level security
The Android OS offers a number of memory corruption mitigations in order to avoid buffer overflow attacks, or return oriented programming. The following list 
presents these low level security mechanisms:
\begin{itemize}
	\item Implements mmap\_min\_addr which restricts mmap memory mapping calls. This prevents NULL pointer related attacks.
	\item Implements XN (execute never) bit to mark memory as non-executable. The mechanism prevents attackers from executing remote code passed as data.
	\item Address space layout randomisation(ASLR) was implemented starting with Android 4.0. This is a first step to preventing return oriented programming attacks. The memory location of the binary library itself is however static. After a number of attempts using trial and error, the attacker may succeed using return oriented programming.
	\item Position independent and randomised linker (PIE) is implemented starting with Android 4.1 to support ASLP. This makes the memory location of binary libraries to be randomised.
	\item Read only relocation and immediate binding space (RELro) was implemented starting with Android 4.1. It solves an ASLR issue where an attacker could modify the global offset table (GOT) used when resolving a function from a dynamically linked library. Before this update an attacker may insert his own code to be executed using the GOT table.
\end{itemize}

% On device bouncer
A number of application security mechanisms are in place to make Android a safer environment for its users. A device program also known as the ``Bouncer'' prevents malware to be distributed from the Android App store (Google Play). The purpose of the bouncer is to verify apps prior to installation by checking for malware signatures and patterns. 

% Secure USB debugging
Secure USB debugging was introduced starting with Android 4.4.2. This only allows hosts registered with the device to have USB debugging permissions. The mechanism is circumvented if the user does not have a screen lock.

% The 4 big issues with android and malware
According to the paper, the Android OS is responsible for $96\%$ of mobile phone malware. The authors claim that this is the case due to 4 big issues of the Android platform:
\begin{enumerate}
	\item Security updates are delayed or never deployed. This is due to a number of approvals that an update needs to receive prior to deployment. This introduces an additional cost to the manufacturer (OEM), that does not generate any revenue. The majority of teams working on the Android platform are focusing on current releases. In most cases there are simply not enough resources to merge Google security updates to the OEM repository. Furthermore, the consequences of a failed OS update may cause ``bricking'' of the device, which is a huge risk for the manufacturer. All these issues lead to very few security updates. Therefore, important features such as RELro are never deployed, making older Android releases vulnerable.
	
	\item OEMs weaken the security of Android by introducing custom modifications before they roll out a device.
	
	\item The Android permission model is defective. According to Kelley et al \cite{kelley2012conundrum}, most users do not understand the permission dialogue when installing an application. Furthermore, even if they could understand the dialogue, most of the time it is ignored in order to use the exciting new app. According to the same study, most applications are over-privileged. This is due to developers not understanding what each privilege grants. Furthermore, as previously pointed out, apps developed by the same owner may share resources and implicitly privileges.
	
	\item Google Play has a low barrier for malware. A developer distribution agreement (DDA) and a developer program policy (DPP) need to be agreed to and signed by the developer before submitting the application to the Android market. However, Google Play does not check upfront if an application adheres to DDA and DDP. The application is only reviewed if it becomes suspect of breaking the agreements. Furthermore, according to \cite{percoco2012adventures} there are ways of circumventing the Bouncer program\footnote{An example of such an application is presented in an article written in Tech Republic: http://www.techrepublic.com/blog/google-in-the-enterprise/malware-in-the-google-play-store-enemy-inside-the-gates/\# (visited on 29.05.2014).}. 
\end{enumerate}

% Conclusion of the section, just a summary of what we presented
We have briefly presented the Android development model, existing mechanisms, and the security of the platform. Given this information we may proceed to present the prototype implementation of the token unlocking scheme. 

\section{Authenticator design}
% Introduction with UAService
The Android user authenticator prototype is designed to work as a bound service implemented in the UAService\footnote{The name of the class stands for User Authentication Service} class. The service collects data from each mechanism and computes the final authentication confidence level. The result is provided to clients either after an explicit request or through periodic broadcasts.

% Mention independent services for each mechanism and why
Each authentication mechanism may have a different requirement for sampling and processing data. As an example, voice recognition may gather optimal data during a phone call\footnote{Call events can be intercepted by registering a listener for the PHONE\_STATE event}, while face recognition when the phone screen is unlocked. Therefore, to enable more flexibility in the individual mechanisms' implementation, we have chosen each to be developed as an independent service.

% Management of authentication mechanisms
UAService communicates with the authentication mechanisms by binding their service. This allows message passing using a ServiceConnection object. On predefined time intervals UAService acquires the confidence level and weight of each mechanism. Using this data it then calculates the overall result according to the design in section \ref{authfeedback}. Feedback is sent back to each registered client for interpretation.

\section{Implementation details}
% Introduction (can expand)
This section presents the implementation of the scheme proposed in section \ref{propopsedsol}. The full source code of the prototype can be downloaded from from github: 

``https://github.com/cristiantoader/fyp-pico''.

% TODO: rename section (something with activity and services, or idk)
\subsection{Main application and services}
% UAService
The Android token unlocking scheme is implemented as a bound service in the UAService class. According to the Android documentation a bound service is the server in the client-server interface. It enables other components to send requests and receive responses by binding to it. It is developed as a regular service that implements the ``onBind()'' callback method to return an IBinder. 

% Keep UAService alive via startService().
According to the Android development API guide\footnote{http://developer.android.com/guide/components/bound-services.html} there are two independent scenarios describing the lifetime of a bound service, depending on how the service was started:
\begin{enumerate}
	\item If the service was not previously running, and a ``bindService()'' command is issued by a component, the service is kept alive for as long as clients are still bound. A client becomes unbound by calling ``unbindService()''.
	
	\item If the service is started using ``onStartCommand()'' it can only be stopped if it has no bound clients and an explicit request is made either via ``stopSelf()'' or ``stopService''. Unlike the previous case, its lifetime persists even with no bound components.
\end{enumerate}

% UAService and lifetime
The prototype we have developed takes into account the lifetime of bound services. The ``PicoMainActivity'' class calls ``startService()'' and ``bindService()'' to the ``UAService'' component. When ``PicoMainActivity'' gets sent to background and loses control of the screen, the service is not explicitly unbound. This should prevent other components from killing ``UAService''. 

% UAService, a safer yet riskier implementation
% 	TODO: rephrase the ending a bit
A safer alternative would be to create a root service. This requires modifications to the system partition. The process does not resume to simply gaining root privileges and making the modifications. The root directory is mounted as ramdisk, and therefore any direct changes will be reverted once the device is rebooted. In order to make persistent modifications, the user needs to alter the boot image and re-flash it on the device. The service needs to be compiled using the Android NDK C compiler. The binary has to be included in the system partition of the boot image in order to be accessible by the ``init'' process during start up. The ``init.rc'' configuration file used by ``init'' also needs to be configured to start the service.

% UAService, central node.
% 	TODO: I think I can drop this paragraph if the word count is exceeded
``UAService'' is a central node in the application. It gathers data from the authentication mechanisms, computes the overall confidence data, and sends feedback to the clients. 

In order to receive authentication updates, clients need to bind ``UAService''. Communication is enabled using the ``Messenger'' interface, which is the simplest way to perform Inter Process Communication (IPC). The ``Messenger'' queues all requests on a single thread, and therefore the application does not require to be thread safe. The following commands are exposed to clients such as Pico through the ``what'' parameter of the ``Message'' class:
\begin{description}
  \item[MSG\_REGISTER\_CLIENT] \hfill \\
  Used for registering a client for periodic broadcasts\footnote{An alternative implementation explored in the project was to have each client also register a confidence level using the ``arg1'' parameter of a ``Message''. In this case, the authenticator would only provide each client with a locked/unlocked result. However, this would shift the meaning of client to that of an authentication session, with state managed by the unlocking scheme. A client would therefore have multiple connections, requiring more IPC. Since all ``Messenger'' requests made to ``UAService'' are queued to a single thread, this would slow down the feedback process and possibly even lead to a denial of service attack. Therefore we have chosen to reduce the communication overhead, and have each client manage the status of its authentication sessions based on the confidence level provided by the unlocking scheme.} of the current unlocking confidence level. Feedback is provided at a fixed time interval of 1000ms.

  \item[MSG\_UNREGISTER\_CLIENT] \hfill \\
  Used for any application to unregister as a listener of this service. No additional parameters required.
  
  \item[MSG\_GET\_STATUS] \hfill \\
  Used by any application when an authentication request is needed. Although the service periodically broadcasts the authentication status to its clients, explicit requests may also be performed using this type of ``Message''.
\end{description}

% Introduction tu authentication mechanism services, UAService side
The communication between ``UAService'' and an authentication mechanism service is intermediated by an ``AuthMech'' object. Each ``AuthMech'' is responsible for interfacing the communication with its corresponding service. 

% Authentication mechanism side
From the authentication mechanism's perspective, each service needs to extend the ``AuthMechService'' abstract class. This standardises the communication with the ``AuthMech'' objects, and implicitly ``UAService''. This software engineering approach facilitates adding additional mechanisms with minimal changes to the original code.

Each ``AuthMechService'' is implemented as a bound service. When binded by ``UAService'' through ``AuthMech'' they expose the following message passing interface:
\begin{description}
  \item[AUTH\_MECH\_REGISTER] \hfill \\
  Used for registering the ``UAService'' client to the ``AuthMechService''.
  
  \item[AUTH\_MECH\_UNREGISTER] \hfill \\
  Used for unregistering the ``UAService'' client from the ``AuthMechService''.
  
  % TODO: might want to change this design!
  \item[AUTH\_MECH\_GET\_STATUS] \hfill \\
  Used by the ``UAService'' to request the authentication feedback from the ``AuthMechService''. The Message response has in ``arg1'' the authentication confidence multiplied by the decayed weight, and in ``arg2'' the original weight of the mechanism.
  \end{description}

\subsection{Authentication mechanisms}
\label{implauthmech}
% Introduction for authentication mechanisms
In order to create a functional prototype of the scheme, we have implemented a number user authentication mechanisms. We will not focus on the result quality of the biometric and behavioural mechanisms. Their sole purpose is to demonstrate that the design of the scheme is functional, and can be implemented using only a smart phone.

% Mechanism requirements (they are a bit logical rather than engineering)
When developing an authentication mechanism for the scheme, the following design requirements need to be satisfied: 
\begin{enumerate}
	\item The result needs to be quantifiable in the form of a percentage ranging from 0 to 100, where 100 means that the mechanism has $100\%$ confidence that the user is the owner of the token.
	\item The mechanism needs to support continuous authentication of the user.
	\item The authentication process needs to be effortless and preferably unobtrusive for the user.
\end{enumerate}

% Presenting examples of what we can port on android
% 	TODO: rephrase, make larger a bit
Android provides an extensive sensor API that can support the scheme. This can be used to develop a number of continuous authentication mechanisms.  We have listed the following non-exhaustive set of examples:
\begin{description}
  \item[Face recognition] \hfill \\
  The mechanism is based on capturing an image of the user's face and performing face recognition. Sampling valid face images can be performed without explicit requests by predicting user behaviour. We will use as an example an user that owns a phone with a front-facing camera. When the owner is unlocking the phone, there is a high probability that they will be looking towards the screen. This provides a good opportunity for the face recognition service to capture a valid sample. Using the Android API, this can be achieved by registering a ``BroadcastReceiver'' to listen for the one of the following events: ACTION\_SCREEN\_ON, ACTION\_SCREEN\_OFF, or ACTION\_USER\_PRESENT. The mechanism may continue to perform face recognition based on collected data and a previously recorded sample of the owner. A simple face recognition mechanism was also implemented as part of the prototype.
  
  % TODO: replay attacks are easy if using only features
  \item[Voice recognition] \hfill \\
  A voice recognition mechanism can record data either periodically, or based on Android events. It may then perform voice recognition and provide a confidence level of the owner being present. Voice sampling does not necessarily imply a voice password. An analysis can be performed using feature extraction. This facilitates the sampling process, which may be performed at any time. With a frequent sampling period, the owner of the device is likely to be recorded while speaking, which would provide a valid data sample. For even better confidence the mechanism can be implemented to start recording when a call is either made or received. On Android this can be achieved by listening for a PHONE\_STATE event. A simple voice recognition mechanism was implemented as part of the prototype.
  
  \item[Iris scanning] \hfill \\
  Similar to face recognition, this can be implemented by taking advantage of user behaviour while using the phone. When the phone is unlocked, the user is very likely to face the front camera, allowing for a good capture. The only problem with this mechanism is the quality of pictures offered by most phones. If the sampling quality is not sufficiently good, meaningful features from the iris may not be extracted. This would make the confidence level of the mechanism relatively low, but may change in the future as devices become increasingly performant.
  
  \item[Keystroke analysis] \hfill \\
  This mechanism was inspired from a paper by Clarke et al \cite{clarke2007authenticating}. The principle of keystroke analysis is based on the patterns in which the user types on his mobile phone. Different features can be extracted here, such as: letter sequence timings, words per minute, letters per minute, frequent used words, and others. Using this data a confidence level can be generated. 
  
  This mechanism is harder to implement using solely the Android SDK. A good starting point would be to have a keyboard app developed for the user that also communicates with the authentication mechanism. If the keyboard is disabled by an attacker this should be considered, especially if the authenticator was originally configured to listen for input.
  
  \item[Gait recognition] \hfill \\
  This mechanism is based on analysing individual walking patterns. According to data presented by Derawi et al \cite{derawi2010unobtrusive}, error rates\footnote{The performance indicator used in biometric analysis is the Equal Error Rate (EER).} may vary between $5\%$ to $20\%$.  Android offers native recognition support for walking, driving, or standing still. Applications can register a sensor callback for the TYPE\_STEP\_DETECTOR composite sensor. Whenever such an event is detected, data can be recorded from the accelerometer and validated using an algorithm similar to the one described by Derawi et al \cite{derawi2010unobtrusive}.
  
  \item[Ear shape analysis] \hfill \\
  Research shows (i.e. Burge et al \cite{burge1996ear}, Mu et al \cite{mu2005shape}) that the shape of the human ear contains enough unique features to perform biometric authentication. Taking advantage of user behaviour, valid data can be captured and analysed using a smart phone. We suggest that a picture is taken a few seconds after a phone call event is detected. If no peripherals are attached, the user is likely to move the device towards the ear. Images captured by such a mechanism could then be used to calculate an accurate confidence level of the user's identity. This method was not tested, so therefore we cannot ensure whether the auto-focus of the camera is sufficiently fast to obtain a valid image.
  
  \item[Proximity devices] \hfill \\
  This is an original idea based on providing a confidence level depending on the presence of known devices. The mechanism should connect with other devices that are also running the authenticator. The two owners don't necessarily need to know one another for the acknowledgement to be performed. Whether regular travel schedules, or working in an office, users are constantly being in the presence of other known devices. This should provide a confidence as to whether the device is in the presence of its owner. 
  
  The authentication works by seeking connections with other devices. Whenever a device is identified, its ID is recorded. The mechanism needs to keep track of the number of times it has connected with another device. Some connections may be established for the first time, and should not bring any confidence. Other connections, such as the Pico of a co-worker, would probably have a high number of connections, and therefore the mechanism should output a higher confidence level in its presence. This mechanism is similar to the Picosiblings solution, but with no k-out-of-n secrets. Each Pico is essentially a Picosibling for another Pico, with each device having a different weight based on familiarity.

  As an example, when travelling with your family on holiday most of the devices there are unknown. However, given that a number of frequent IDs are in the proximity of the authenticator, the mechanism should still consider to some extent that it is in the possession of its owner. 
  
  The mechanism can be circumvented in the scenario where co-workers or friends try to unlock the Pico. Due to this downside, it should never have sufficient weight to unlock the token on its own. However, in combination with other mechanisms it would provide a good approximation of whether it is in the possession of its owner. If the device is in good company there is a good chance the owner is also present. 
  
  \item[Location data] \hfill \\
  This mechanism is similar to ``Proximity devices'' and much easier to implement. Based on Android GPS and network location data, the phone may detect whether it is in an usual location or not. Just as ``Proximity devices'' this should not carry a high weight in the scheme, especially since it would not provide accurate results in scenarios such as holidays.
  
   \item[Service utilisation] \hfill \\
  This mechanism exploits patterns in the Android phone's service and app utilisation. Based on current running applications, services, and the time they were started we may create a model where some confidence is given regarding the ownership of the device. This mechanism would only be effective in detecting sudden changes. It would have a low weight in the overall scheme due to its lack in precision. 
  
  \item[Picosiblings]
  The original Picosiblings mechanism may also be used with this scheme. Although not part of the standard set of Android device sensors, if available, a Picosiblings implementation may be included as one of the authentication mechanisms.
\end{description}

% continuous mechanisms for explicit authentication
A number of continuous authentication mechanisms may also be used for explicit authentication. The user can be notified to provide accurate information for the following mechanism: face recognition, voice recognition, iris scanning, keystroke analysis, gait recognition, and ear shape analysis. This creates the opportunity for a valid data sample to be collected.

% explicit authentication mechanisms
A number of explicit authentication mechanisms which do not satisfy the continuous authentication property of Pico may be implemented for the Android platform. It is important to note that additional mechanisms not included in this list need to satisfy the memorywise-effortless property of the token unlocking framework (\ref{tokenframework}). We suggest the following mechanisms for implementation:
\begin{description}
  \item[Fingerprint scanner] \hfill \\
  Devices that incorporate a fingerprint scanner (such as the IPhone 5S) can use the sensor as an explicit authentication mechanism. It cannot be used for continuous authentication because the user doesn't come in contact with the sensors on a regular basis. A mechanism can therefore request explicit fingerprint data, which would then be compared with the owner's biometric model, outputting a confidence for the authentication. The result will be combined in the overall scheme just as any other mechanism. The the only difference will be in terms of weight and decay rate.
    
  \item[Hand writing recognition] \hfill \\
  The user may be prompted to use the touch screen in order to write a word of his choice. This would guarantee the memorywise-effortless property because the user doesn't need to remember any secret. The handwriting would be analysed with a preconfigured set of handwriting samples in order to compute the confidence level that the owner produced the input.
  
  \item[Lip movement analysis] \hfill \\
  According to Faraj and Bigun \cite{faraj2006motion}, analysing lip movement while speaking can be used for authentication. The user would be prompted to provide a data sample such as reading a word provided by the authenticator. Using lip movement authentication, a quantifiable confidence level would be produced. This mechanism can also be implemented as a continuous authentication mechanism. However, data sampling would likely have a low success rate as users tend not to have their mouth within the camera's field of view.
\end{description}

% Introduction for developed mechanisms
In order to have a functional prototype of the scheme, we have developed a number of authentication mechanisms. The following sections will confirm that the Android platform offers sufficient functionality for supporting the scheme. Furthermore, this should stand as proof that the design can be implemented on a dedicated Pico device with a similar set of sensors.

% Listing mechanisms with promises of future details
The following mechanisms have been implemented as part of the prototype: voice recognition, face recognition, location analysis, and a dummy mechanism used for testing. The following sections will provide details regarding their functionality and implementation process.

\subsubsection{Dummy mechanism}
%
% TODO: weight decay!
%============================
% Introduction: what it does, why it was developed
A dummy authentication mechanism was developed for testing the overall scheme. It produces random confidence levels within a predefined range, which provides a good controlled environment for testing purposes.

% General implementation detail
The mechanism was developed consistently with the application model. It is implemented in the ``AuthDummyService'' class, extending the ``AuthMechService'' abstract class. This makes it an independent bound service with a predefined communication interface.

% DAO
All authentication mechanism services have a data access object (DAO) responsible for interfacing with imported libraries and managing authentication data. In this case the DAO only produces random confidence levels within a given range. A thread belonging to the service makes periodic requests to the DAO. This mimics an authentication mechanism that periodically samples for data. The service confidence is updated based on the produced value. 

% sending data back to UAService
When ``UAService'' needs to update its overall confidence, it makes an AUTH\_MECH\_GET\_STATUS request to the service. The reply contains the most recent confidence level multiplied by the current decay factor, and the original weight of the mechanism. 

\subsubsection{Voice recognition}
% Introduction to mechanism
The voice recognition mechanism is implemented in the ``VoiceService'' class. It extends the ``AuthMechService'' abstract class that defines its communication interface. When the service is created, the ``onCreate()'' method is called automatically by the Android platform. The method was developed to start a thread that periodically gathers data from the microphone, performs biometric authentication, and produces a confidence level.

% TODO: decay weight

% Library used for implementation
% 	TODO: library repeats a bit
The library used for voice recognition is called Recognito\footnote{The library can be downloaded using github from the following link: https://github.com/amaurycrickx/recognito} and was developed by Amaury Crickx. It is a text independent speaker recognition library developed in Java. We do not claim that it is the best voice recognition library, but it was best suited for the purpose of this prototype. Porting the library for Android required minimal changes. It claims very good results in scenarios with minimal background noise\footnote{It was tested by the author on TED talks, such as:  https://www.ted.com/talks/browse (visited on 06.01.2014)}.

% Hack to compile with the library
In order for the application to compile the Recognito library, a subset of the rt.jar Java (SE) library was required. This is due to ``javax.sound.'' packages included in Recognito that are not available on Android. Trying to import and use the ``javax.sound.'' package is not possible due to the name collision with the ``javax.'' system library available on Android. Therefore, we had to include ``rt.jar'' as part of the application, but without actually using it. This was purely done to trick the Android Java compiler to package the application. Using ``javax.sound'' features would generate a runtime error. This was avoided by only using Recognito functions which require direct sound input, without any knowledge of sound file formats.

% Recording configuration
In order to gather and manage samples compatible with the Recognito library we have created the ``VoiceRecord'' class. This is responsible of gathering microphone input using the following predefined configuration:
\begin{itemize}
	\item Sample rate: 44100
	\item Channel configuration: AudioFormat.CHANNEL\_IN\_MONO
	\item Audio format: AudioFormat.ENCODING\_PCM\_16BIT
\end{itemize}

% More random recording stuff..
The minimum buffer size required by the ``VoiceRecord'' class is device dependant and pre-calculated in the constructor. The class wraps an ``AudioRecord'' object used for gathering microphone data. Due limitations of the SDK, the recording is saved as a file and loaded into memory when needed. This makes the implementation not efficient, but it does serve the purpose of the prototype.

% DAO
A DAO class is created to facilitate the interface to the Recognito library. When initialised, it loads the owner configuration, and a predefined set of background noises. It then creates a Recognito object and trains it using the data. This is done using the library's ``createVocalPrint'' public method.

% Getting confidence level
Every predefined time interval, the ``VoiceService'' authentication thread records data in ``double[]'' format using the ``VoiceRecord'' class. It then calls the ``recognize'' public method of the Recognito object. This returns the Euclidean distance to the closest match, which is either the owner, or one of the background noises used for training. 

% Getting closest match
To convert the Euclidean distance to a percentage confidence level, we define an acceptable threshold. Any result above the threshold is considered too high and is truncated to its value. Using the following formula we convert the Euclidean distance to a confidence level. The final result is $P(E|H)$ (the probability that the evidence belongs to the hypothesis) used in equation \ref{eq:final}.

$$P(E|H) = 1 - \frac{distance}{THRESHOLD}$$

% Probability computation estimate
Dividing the distance over the threshold yields a confidence value between 0 and 1, where 1 is a very large distance and hence a bad result. By using one minus this value we invert the meaning. Values will range between 0 and 1, and 1 corresponds to a confidence level of $100\%$. 

% Do Bayesian update
Having known $P(E|H)$ we continue to calculate $P(H|E)$ by using the Bayesian update formula defined in equation \ref{eq:final}. When calculating the final confidence level of the mechanism, we multiply $P(H|E)$ with the current decay rate. Due to the message passing API, this value needs to be an integer and is therefore multiplied by 100. The overall result is stored in the service and updated whenever the decaying weight is modified. When a request is made by UAService, the value is returned using the IBinder message passing mechanism.

\subsubsection{Face recognition}
\label{implface}
% Introduction: brief description of the implementation
The face recognition mechanism was implemented in the ``FaceService'' class, which extends the ``AuthMechService'' abstract class. When created, the service starts a thread that periodically collects data from the camera. Each sample is analysed using a face recognition library, and a confidence level is outputted for the current sample. The result is multiplied by the mechanism's weight which is a decaying factor.

% Javafaces library
For biometric face recognition we use a port of the ``Javafaces'' library \footnote{The ``JavaFaces'' library is maintained at the following address: https://code.google.com/p/javafaces/}. This was the closest functional library found that was compatible with the Android API. Javafaces is written entirely using Java (SE). Unfortunately, it makes use of the ``javax.imageio.'' package which is not available in the standard Android SDK. Therefore, a considerable amount of code needed to be ported for the Android platform. The new library is publicly available at the following link:https://github.com/cristiantoader/JavafacesLib. It is currently not optimised for public use.

% Javafaces changes for porting to Android
We will briefly present the changes made when porting the ``Javafaces'' library. The ``BufferedImage'' class had to be replaced by its closest Android equivalent, which is ``Bitmap''. All ``BufferedImage'' references in the original project had to be adapted. Furthermore, The API was modified to support direct ``Bitmap'' input in order to add more flexibility and lighten the main code of the authenticator. 

In the original ``Javafaces'' library, data formats for black and white images were assumed to have a single colour channel representing the grey value. This had to be changed to reflect the Bitmap convention, where all 3 colour channels are present but have the same value. Additional modifications were required due to data type mismatches, as well as other related issues.

% Authentication process
%	TODO: include that if no faces are found, the process is not performed
Every predefined time interval, the authentication thread running within the ``FaceService'' object samples data from the camera. This is performed by using a ``CameraUtil'' object. The ``CameraUtil'' class was developed as a mediator to simplify the interface to the Android ``Camera''. For example, it performs additional checks such as the orientation of the phone. 

A DAO class called ``FaceDAO'' was developed to mediate calls to the Android ``Javafaces'' library. Images captured using ``CameraUtil'' are validated using the DAO. The value returned from the ``Javafaces'' library is the Euclidean distance between collected data and the registered owner. This distance is handled in the exact same way as the voice recognition mechanism \ref{implface}.

% Gather picture with no notification: shutter sound
By default, the Android API does not easily allow for a Camera picture to be taken without any sort of notification to the user. Both a shutter sound and a visual preview display should be present. The shutter sound can be disabled easily by not providing a shutter callback function when calling the Camera.takePicture() method. 

% Gather picture with no notification: shutter sound
Disabling the user preview of the camera was more difficult to achieve. The solution used with this prototype was to exploit an Android feature that allows to render the preview in a ``SurfaceTexture'' object. This satisfies the API's requirement to have a visual display preview for the camera, while the ``SurfaceTexture'' itself does not need to be displayed on screen. Therefore a picture can be taken from a background service without any interruption to the user.

% Issue of image too large
Another problem encountered by the face recognition service is data sizes. When the ``Javafaces'' library performed its algorithm, the device was running out of memory. This caused the app to be closed by the Android OS. To fix this issue, Bitmaps collected from the camera are scaled to $50\%$ before they are processed by the library.

% Results
Unfortunately, the library combined with the Android SDK does not provide accurate results. The reason is that it requires as input a bitmap perfectly containing the face of an individual. Unfortunately, although the Android SDK offers face detection, it only provides the location of the midway coordinate between the eyes, and the distance between the eyes. Using this data alone, an accurate crop cannot be made. As a solution, yet another library would need to be used in order to properly detect face regions. This would provide better input data and would increase the precision of the mechanism.

\subsubsection{Location analysis}
% Introduction: short description of the mechanism
The mechanism is based on gathering location data and using it to generate a probability that the owner is present. This is implemented in the ``LocationService'' class that extends the ``AuthMechService'' abstract class. Data is collected periodically by using the ``LocationManager'' provided by the Android API.

% DAO object used for collecting data
A DAO object is used to mediate calls to the Android API and manage the existing owner configuration. It is implemented in the ``LocationDAO'' class. It offers functionality for gathering and saving location updates. It is developed to use the most accurate data provider. The Android API offers the following sources of collecting ``Location'' data:
\begin{itemize}
	\item GPS\_PROVIDER: Collects data from the GPS.
	\item NETWORK\_PROVIDER: Collects data from cell tower and WiFi access points.
	\item PASSIVE\_PROVIDER: Passively collects data from other applications which receive ``Location'' updates.
\end{itemize}

% Describe the algorithm class
External libraries were not used for the authentication process. We have developed a primitive location analysis algorithm in the ``LocationAnalyser'' class. During the configuration stage, which is a process managed by ``LocationActivity'', location data is sampled every 5 minutes and saved in internal storage. After the process has ended, each time a ``Location'' is sent for authentication it is compared with all the locations saved during the configuration process. The final result is the minimum Euclidean distance between the current ``Location'' and any other saved ``Location''. 

% Describe how authenticator thread works
When the service is started by ``UAService'', its ``onCreate()'' method spawns an authentication thread. This thread periodically requests the current location using the DAO. Data is returned in a ``Location'' object and is provided as input to the ``LocationAnalyser''. The result of this operation is an Euclidean distance which gets converted to a percentage using a threshold, just as in the previous mechanisms. The value is stored by the service for future requests from ``UAService''.

% Conclusion
Just as mentioned in section \ref{implauthmech}, the mechanism has a lower confidence level. Although being in a known location provides some confidence that the token was not stolen, it does not offer any guarantees that the device is still with its owner.

\subsection{Owner configuration}
% Owner configuration activities exist
There are a number of Activity components that are used in the configuration of the prototype. Each authentication mechanism has a corresponding Activity that can be started from the main Activity called ``PicoUserAuthenticator''. These are used to register owner biometrics needed by the mechanisms.

% They use DAO objects to store data in internal storage
Each configuration Activity uses the same DAO class as the mechanism Service. The DAO is used for storing the owner data once it was collected. Given that the overall size of the data is relatively small, the files are kept in internal storage. The Linux Android permissions mechanism guarantees its confidentiality and integrity, and therefore further encryption is not necessary.

\section{Conclusion}
% Small overall
We have described the Activity and Service components developed for the prototype, as well as their communication flow. We have ported two biometric libraries and developed a location analysis mechanism. DAO objects facilitate accessing owner configuration files and the interface with auxiliary mechanisms. An overview of the app design can be seen in figure \ref{TODO}.

% TODO: insertgraphics() overall design of the app.

% Limitations and solutions
One of the limitations of the prototype is the lack of explicit authentication mechanisms. Another issue is the precision of the biometric mechanisms, in the lack of better libraries. However, due to the modular design of the application, existing mechanisms can be improved simply by importing a new library and modifying its DAO. The existing set of mechanisms can easily be increased by creating a new class that extends ``AuthMechService'', and implementing the algorithm logic. In order to be managed by ``UAService'', the new mechanism needs to be included in the ``UserAuthenticator.initAvailableDevices()'' method.

% TODO: continue from here
\section{Related work}
% Sensor sniffing
Liang Cai et al \cite{cai2009defending} analyse ways of protecting users from mobile phone sensor sniffing attacks. The authors design a framework used for protecting sensor data from being leaked. From a security perspective the user should not to be trusted with granting permissions to different applications. A solution provided in the paper is for sensors to become locked once they are used. A downside to this is that malware may deny service to legitimate applications (such as our prototype) by creating a race condition for acquiring a sensor lock. This can be solved by using a user notification, allowing for the owner to decide which application acquires the lock. A suggestion to this approach would be to allow for different priority levels, such that malware applications would not acquire the lock in a race condition, or even more, would lose it when a high priority application such as the Pico authenticator would require sensor data.

% Gait recognition
The paper by Derawi et al \cite{derawi2010unobtrusive} presents the feasibility of implementing gait authentication on Android as an unobtrusive unlocking mechanism. According to the definition offered by the authors ``gait recognition describes a biometric method which allows an automatic verification of the identity of a person by the way he walks''. The Android implementation developed by the authors has an equal error rate (EER) of $20\%$. Dedicated devices have an EER of only $12.9\%$, and the main cause for this is the sampling rate available at that time (2010). They have used a Google G1 phone with approximately 40-50 samples per second. This is much inferior to dedicated accelerometers that sample data at 100 samples per second. However, by conducting personal experiments with the accelerometer of a Google Nexus 5 phone, using the highest sampling setting (SENSOR\_DELAY\_FASTEST) the rates go above 100 samples per second. Therefore the current performance of the algorithm paper should be closer to $12.9\%$.

% Improve speaker recognition in noisy conditions
Ming et al \cite{ming2007robust} present in their paper how to improve speaker recognition accuracy on mobile devices in noisy conditions. This approach uses a model training technique based on which missing features may be used to identify noise. The focus of the paper is designing and implementing a biometric mechanism, and is therefore outside the scope of this dissertation project. 

% Voiceprints voice recognition
Another technique in performing speaker recognition involves using voiceprints. These are a set of features extracted from the speaker sample data. Kersta \cite{kersta2005voiceprint} explains the mechanism in more detail. The benefit of having feature extraction based on a voice sample, as opposed to a different voice recognition mechanism, is that voiceprints do not require any secrets. The speaker doesn't have to reproduce a voice sample. This increases the usability of the mechanism in scenarios required by the Pico authenticator. However, a downside to this approach is that it makes replay attacks easier to perform. Any recording of the user is sufficient for an attacker to trick the biometric mechanism.

% Face recognition
%	TODO: need to review the phrasing
A popular paper on face authentication was written by Turk and Pentland \cite{turk1991face}. The biometric authentication process is based on the concept of eigenfaces. Eigenfaces are a name given for the eigenvectors which are used to characterise the features of a face. These features are projected onto the feature space. Using Euclidean distances in the feature space, classification can be performed to correctly identify individuals. An implementation of this mechanism was developed for the Pico unlocking scheme prototype.

% Keystroke analysis
An unconventional authenticating mechanism is presented by Clarke and Furnell \cite{clarke2007authenticating}. They use keystroke analysis in order to make predictions regarding the user of the phone. This mechanism is unobtrusive and authenticates users during normal interactions such as typing a text message or a phone number. It is based on a neural network classifier, reporting an EER of $12.8\%$. Input data used for classification is composed out of timings between successive keystrokes, and the hold time of a pressed key. 






