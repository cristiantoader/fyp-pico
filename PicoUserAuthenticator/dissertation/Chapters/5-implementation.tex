% Chapter Template

\chapter{Implementation Prototype} % Main chapter title
%TODO:
%	- make sure that I emphasize that it is up to the user to allow malware to read data from sensors!

\label{Chapter4} % Change X to a consecutive number; for referencing this chapter elsewhere, use \ref{ChapterX}

\lhead{Chapter 4. \emph{Implementation Prototype}} % Change X to a consecutive number; this is for the header on each page - perhaps a shortened title

Thus far we have developed a new Pico authentication scheme and assessed it using our own token unlocking framework. We then have performed a threat model from an availability, integrity, and confidentiality perspective and have suggested the safest implementation which would be as feasible as possible for the user to adopt.

In this chapter we will described design and implementation details for the prototype of the proposed scheme. The implementation platform will be the Android OS, which uses a Java based SDK for application development.

\section{Authenticator design}
The user authenticator for Android is designed to work as a bound service called UAService. Periodically the service outputs to registered Pico clients the status of the authentication process. Any application may be a client as long as it registers with the service. Furthermore, explicit authentication update requests may be performed by the Pico client.

Since different authentication mechanisms require different update periods, we have chosen each mechanism to be represented by an independent service. This allows for more flexibility such as periodic sampling with different intervals. Another feasible use case for example would be performing voice recognition based on the first few seconds of an outgoing or incoming call. This would require a service that is triggered by a PHONE\_STATE intent.

Each authentication mechanism service is started and managed by the UAService. Communication between the UAService and each authentication mechanism is enabled through intents. Using this communication link, requests can be made from each individual authentication mechanism in order to get the current confidence level. This value is equal to the probability that the owner is present, multiplied by the weight carried by the mechanism. Given that each mechanism runs as an independent service, weight decay may easily be performed using an AlarmManager or simply a function which is called periodically within the authentication thread.

Either periodically UAService gets the confidence level and weight from each mechanism. It then calculates the overall result. If the result is above the threshold requested by the Pico client, a ``Message'' is passed back saying that Pico should unlock. Otherwise a negative result is returned, letting the Pico know it should be locked. 

\section{Implementation details}

\subsection{Main application and services}
The user authenticator for Android is designed to work as a bound service. According to the Android documentation a bound service exposes functionality to other application components and as well as external applications. It is developed as a regular service which implements the onBind() callback method to return an IBinder. The service lives only as long as a component is bound to it. The service implementation class is called UAService.

The UAService is a central node in the application. It is a bound service for any Pico client which wishes to register for events. Furthermore, it binds any authentication mechanism that is available, enabling it for authentication. 

The UAService periodically broadcasts intents to registered clients saying if the Pico should be locked or unlocked. The following interface is exposed to available Pico applications through the ``what'' parameter of the ``Message'' class:
\begin{description}
  \item[MSG\_REGISTER\_CLIENT] \hfill \\
  Used for registering a client. The ``Message'' should have as the ``arg1'' parameter the level of confidence required for unlocking. This value should range from 0 to 100. Any values outside these limits will be truncated within the range.
  
  \item[MSG\_REGISTER\_CLIENT] \hfill \\
  Used for any application to unregister as a listener for this service. No additional parameters required.
  
  \item[MSG\_GET\_STATUS] \hfill \\
  Used by any application when an authentication request is needed. Although the service periodically broadcasts to its registered clients what is the authentication status, explicit requests may also be performed using this ``Message''.
\end{description}

UAService interacts with AuthMech objects in order to communicate with an authentication mechanism. Each object is responsible for interfacing the communication with an authentication mechanism. A valid authentication mechanism service needs to extends the AuthMechService abstract class which defines a standard way of communication with the UAService.

Each AuthMechService is programmed as a bound service. UAService binds these services through AuthMech objects. Each AuthMechService exposes the following message passing interface:
\begin{description}
  \item[AUTH\_MECH\_REGISTER] \hfill \\
  Used for registering the UAService service as a client to the AuthMechService.
  
  \item[AUTH\_MECH\_UNREGISTER] \hfill \\
  Used for unregistering the UAService service as a client to the AuthMechService.
  
  \item[AUTH\_MECH\_GET\_STATUS] \hfill \\
  Used by the UAService in order to request the authentication confidence from the AuthMechService. The value will be returned in the arg1 parameter of the Message passed.
\end{description}

\subsection{Authentication mechanisms}
In order to create a functional prototype, we implemented a couple of mechanisms. The focus of the project is not the quality of the biometric mechanisms involved in the prototype, their sole purpose being to demonstrate a proof of concept. Android devices offer a wide range of sensor data such as GPS, accelerometer, camera, and microphone.

Based on the sensor data offered by Android devices, a wide range of biometric mechanisms can be developed. A non extensive list may include face recognition, voice recognition, iris scanning, keystroke analysis, gait recognition, and many others. The scheme however, requires a clear predefined list of mechanisms offering continuous authentication as well as explicit.

A number of continuous authentication mechanisms may be developed using solely the standard sensors offered by Android devices. The following non-extensive list was achieved, with details regarding what each mechanism means and how it should be implemented:
\begin{description}
  \item[Face recognition] \hfill \\
  This mechanism was also implemented for the purpose of the project, and further details are offered in the following sections. The idea is that based on user behaviour, sampling of the user's face can be performed without any explicit requests. For instance when an user is unlocking the phone it is highly likely that he will be looking at the screen. This creates a good opportunity for the authentication mechanism service to capture an image and determine the confidence level that the unlocker is the actual user.
  
  \item[Voice recognition] \hfill \\
  This mechanism was also implemented for the purpose of the project, and further details are offered in the following sections. Note that voice sampling does not necessarily imply a voice password of any kind. Voice can be analysed from a feature's perspective, regardless of the words being spoken. Voice sampling can be performed at any time. With a frequent enough sampling rate, the owner of the device is likely to be present in most voice recordings. For even better confidence the mechanism should be implemented to start recording when a call is either made or received. On Android this can be achieved by implementing a listener for a PHONE\_STATE intent.
  
  \item[Iris scanning] \hfill \\
  Similar to face recognition, this can be implemented by taking advantage of user behaviour while using the phone. When the phone is unlocked, the user is very likely to face the front camera, allowing for a good face capture. The only problem with this mechanism is the quality of pictures offered by most phones. If the sampling quality is not sufficiently good, meaningful features from the iris may not be extracted, making the confidence level of the mechanism relatively low.
  
  \item[Keystroke analysis] \hfill \\
  The principle of keystroke analysis is based on the patterns in which the user types on his mobile phone. Different features can be extracted here, such as: the amount of time the user takes to type letter sequences, words, or individual letters, words per minute, frequent used words, and many others. Based on these features a confidence level can be generated (not carrying a considerable amount of weight). This is harder to implement using solely the Android SDK. A good starting point would be to have a keyboard application developed for the user, which also communicates with the authentication mechanism. Obviously if the keyboard is disabled by an attacker this should still be considered, especially if the authenticator was originally configured to listen for input.
  
  \item[Gait recognition] \hfill \\
  This mechanism is based on the concept of analysing individual walking patterns. Different people walk in different ways, which even though may not be entirely unique for every individual, would still provide some confidence level regarding the user of the device. In the lack of an existing reliable library, efforts have been made to implement this mechanism, unfortunately unsuccessful. The implementation requires accelerometer data from the device, which needs to be normalised from the sensor's perspective. Android offers activity recognition for walking, driving, or standing still. This is achieved by registering a sensor callback for the TYPE\_STEP\_DETECTOR composite sensor.
  
  \item[Ear shape analysis] \hfill \\
  Studies have shown \cite{} that the shape of the human ear contains enough unique features in order to perform biometric authentication. Taking advantage on user behaviour when using a phone, accurate images can be captured in order to perform such analyses. Within a few seconds from answering the phone, given no peripherals are attached, the user is going to move the phone towards his ear. Based solely on timing and/or accelerometer data, accurate pictures could be taken of the user's ear before the camera gets too close. Images captured by such a mechanism could then be used to calculate an accurate confidence whether the owner is the person who is answering the phone.
  
  \item[Service utilisation] \hfill \\
  This proposed mechanism is not biometric based. It exploits patterns in the Android phone's service and app utilisation. Based on current running applications, services, and the time they were started my create a model where some confidence is given as to whether the owner has changed. This mechanism would only be effective in detecting sudden changes, but may easily be obstructed either by removing the Pico authenticator. Furthermore sudden changes in ownership are not promptly detected which is why the mechanism would have a low weight in the overall scheme. 
  
  \item[Proximity devices] \hfill \\
  A mechanism may be developed which tries to connect with other devices also running the authenticator. The two owners don't necessarily need to know one another for the acknowledgement to be performed. Based on day to day activities, users tend to interact or at least be around a lot of the same people. Weather regular travel schedules, or as a better scenario, working in an office, constantly being in the presence of other known devices should give a confidence as to whether the device is in the presence of the user. This mechanism could only be circumvented by co-workers or friends unlocking the Pico, which is why it should never have sufficient weight to unlock the Pico on its own. In combination with other mechanisms however, it would provide a good sense regarding the owner of the device. It the device is ``in good company'' there is a good chance the owner is also present. This should be enhanced with time data as to when other trusted device are recognized. Furthermore, based on the ID of the devices the owner comes in proximity to, the mechanism may have different weights for different devices. As an example, even though travelling with your family on holiday and most of the devices there are unknown, given that a number of frequent IDs are in the proximity of the device, the mechanism should still consider to some extent that it is in the possession of its owner. This would work similarly with the Picosiblings idea, but each Picosibling is a device running this authentication mechanism which is frequently in the proximity of the owner.
  
  \item[Location data] \hfill \\
  This mechanism is also non biometric. It is similar to ``Proximity devices'' and much easier to implement. Based on Android GPS data, the phone may detect whether it is in an usual location or not. Just as ``Proximity devices'' this mechanism should not carry a high weight in the scheme, especially since it would not provide accurate results in scenarios such as holidays.
  
  \item[Picosiblings]
  The original Picosiblings mechanism may also be used with this scheme. Although not part of the standard set of Android device sensors, if available, a Picosiblings implementation may be included as one of the authentication mechanisms.
\end{description}

Some of the continuous authentication mechanisms may also be used for explicit authentication. Based on the non-extensive list mentioned above, the user may be notified to provide accurate information for the following mechanism: face recognition, voice recognition, iris scanning, keystroke analysis, gait recognition, and ear shape analysis. By notifying the user that he has to provide more accurate authentication data, the mechanisms get a better chance of providing valid results. The decay rate after explicit authentication will be slower in order to maintain the continuous authentication property of the Pico for the duration of the authentication session.

In addition to the mechanisms mentioned above, a number of explicit authentication mechanisms which do not satisfy the continuous authentication property of the Pico may be implemented using the Android SDK. It is important to note that any other mechanisms not included in this list need to satisfy the memory property of the Pico, according to which the user doesn't need to remember any known secret. A non-extensive list of mechanisms includes the following:
\begin{description}
  \item[Fingerprint scanner] \hfill \\
  Devices which may have a fingerprint scanner incorporated, such as the IPhone 5S may use this sensor in order to gather biometric data used for authentication. This mechanism cannot actively be used for continuous authentication due to the fact that the user doesn't come in contact with the sensors on a regular basis. A mechanism can therefore request explicit fingerprint data, which would then be compared with the owner's biometric model, outputting a confidence for the authentication. This confidence will be combined in the calculation of the overall scheme confidence just as any other mechanism, the only difference being in terms of weight and decay rate.
    
  \item[Hand writing recognition] \hfill \\
  The user may be prompted to use the touch screen in order to write a word of his choice. This would guarantee the memoryless property, since the user doesn't need to remember any sort of secret. The handwriting would be analysed with a preconfigured set of handwriting samples in order to determine the confidence level that the owner produced the input.
  
  \item[Lip movement analysis] \hfill \\
  According to the paper \cite{} by TODO, lip movement during speaking may be used to uniquely identify individuals. Lip movement analysis would be performed similarly as described in the paper. The confidence level that the owner produced the input would then be combined in the authentication scheme. This may also be implemented as a continuous authentication mechanism, with with lower success rate expectations due to the way users tend to hold mobile phones, which usually doesn't expose the mouth to the camera.
\end{description}

\subsubsection{Dummy mechanism}
In order to perform tests for different confidence levels, a dummy authentication mechanism was implemented using the AuthDummyService class. It extends the AuthMechService abstract class, which makes it an independent service just to maintain the application model consistent. 

The service contains a data access object (DAO) which in this case only produces random confidence levels within a given range. A thread running within the service makes periodic requests to the DAO in order to mimic an authentication mechanism which periodically samples for data. The service is updated based on the produced value. 

When the UAService wants to update its overall confidence, it makes a AUTH\_MECH\_GET\_STATUS request to the AuthDummyService service, which returns the most recent confidence level multiplied by the current decay factor. The result is combined with the result from the remaining authentication mechanism services.

\subsubsection{Voice recognition}
The voice recognition mechanism is implemented as a VoiceService class extending the AuthMechService abstract class. When the services onCreate() method is called, it starts an authenticator thread which periodically samples data from the device's microphone.

% TODO: cite https://github.com/amaurycrickx/recognito
The library used for voice recognition is called Recognito developed by Amaury Crickx. It is a text independent speaker recognition library developed in Java. It is by no means one of the best voice recognition libraries, but it was best suited for the purpose of this prototype. The library required minimal additional changes. It claims very good results in scenarios with minimal background noise, such as TED talks \cite{} which it was originally tested on by its author.

In order for the application to compile properly, a subset of the rt.jar was required. This is due to ``javax.sound.'' packages included with the library which are not available on Android. Unfortunately ``javax'' is a core library also available in Android, but without any of the sound features. For this reason, although the required packages are included with the application this is purely done to trick the compiler that everything is in place. In reality none of the functionality offered by these packages is used by the application. This is avoided by only using the raw features of the library which require direct sound input without any details regarding sound files and formats.

In order to gather samples compatible with the library and manage them properly, we have created the ``VoiceRecord'' class. This class is responsible of gathering microphone input using a predefined compatible configuration listed in the following listing:
\begin{itemize}
	\item Sample rate: 44100
	\item Channel configuration: AudioFormat.CHANNEL\_IN\_MONO
	\item Audio format: AudioFormat.ENCODING\_PCM\_16BIT
\end{itemize}

The minimum buffer size required by the class is device dependant and pre-calculated in the constructor. The class wraps an AudioRecord object used for gathering microphone data. Due to the nature of the SDK, the recording is saved as a file which then is loaded into memory whenever needed. Although this is not an efficient approach due explicit loads from disk, it serves the purposes of the prototype.

In order to have a better interface to the Recognito library, a DAO class was created. When initialised, the DAO object loads the owner configuration together with the predefined set of background noises. It instantiates a Recognito object and trains it using the loaded data. This is performed using the ``createVocalPrint'' public function made available by the library.

Using the audio data stored as a ``double[]'' array and the sampling rate stored in the ``VoiceRecord'' class, we can then call the ``recognize'' functionality of the library in order to get the closest match to either the owner, or one of the background noises used for training. The library then returns the closest match given its training data, together with the Euclidean distance to that match.

In order to convert the Euclidean distance to a percentage confidence level, an acceptable Euclidean distance threshold is used. Any result above the threshold is considered too high and is truncated to the threshold level. Using the following formula we then convert the value to a confidence level, which is the equivalent of $P(E|M)$, the probability that the evidence belongs to the model.

$$P(E|H) = 1 - \frac{distance}{THRESHOLD}$$

% TODO: make sure this calculation remains consistent
Dividing the distance over the threshold yields a confidence value between 0 and 1, where 1 is a very large distance and hence a bad result. By using one minus this value we invert the meaning, yielding values between 0 and 1 where 1 corresponds to 100\% confidence level. 

Having calculated $P(E|H)$ we then proceed by calculating $P(H|E)$ using the formula mentioned in the design section of this dissertation. Whenever calculating the current confidence level, we use the value of $P(H|E)$ multiplied by the current decay rate, a number which is periodically decreased within the service. Due to the message passing mechanism using intents, this value needs to be an integer and is therefore multiplied by 100. The overall result is stored in the service and updated whenever the decaying weight is modified. When a request is made by UAService, the value is returned using the IBinder message passing mechanism offered by Android.

\subsubsection{Face recognition}
The face recognition mechanism was implemented in the ``FaceService'' class, which extends the AuthMechService abstract class. It is a service running a thread which periodically collects data from the camera. Each sample is analysed using a face recognition library, and a confidence level is outputted for the current sample. Just as the scheme proposes, this confidence level is multiplied by a weight which is a decaying factor.

The library used for face recognition is a port of the Javafaces library \cite{}. This was the closest functional library found that was compatible with the Android API. Javafaces is a library written entirely in Java, but which unfortunately makes use of the ``javax.imageio.'' package which is not available in the standard Android SDK. Since a considerable amount of code needed to be changed, we have created a new library \cite{} based on the original for the Android OS.

I will briefly present the changes made when porting the Javafaces library. The ``BufferedImage'' class had to be replaced by its Android counterpart, Bitmap. All BufferedImage references and initialization had to be changed. The API was modified to support direct Bitmap input in order to add more flexibility and lighten the main code of the authenticator. Original data formats for black and white images were assumed to have a single colour channel representing the grey value. This had to be changed within the code in order to reflect the Bitmap convention where all 3 colour channels are present but have the same value. Additional modification were required such as data type mismatches as well as other smaller issues.

Unfortunately, this library combined with the Android SDK does not provide accurate results. This is due to the fact that the library requires a rectangle bitmap perfectly containing the face of an individual. Unfortunately the Android SDK although offers face detection, it only provides the location of the midway coordinate between the eyes, and the distance between the eyes. Using this data alone, an accurate crop cannot be made. As a solution, yet another library would need to be used in order to detect faces and provide more accurate data regarding their location and boundaries.

Every fixed time interval, a thread running within the ``FaceService'' object samples data from the camera at a fixed time interval. This is performed using an instance of the Camera class. Additional configuration is required based on the orientation of the phone. On the device the prototype was developed \cite{}, when the phone is held normally a 90 degree rotation of the image is required.

By default, the Android API does not easily allow for a Camera picture to be taken without any sort of notification to the user. By default both a shutter sound and a visual preview display should be present. The shutter sound can easily be disabled by simply not implementing any shutter callback function. The preview display however proves to be a bit more difficult. The solution used with this prototype was to exploit Android's option to render the preview image to a ``SurfaceTexture'' object. This satisfies the API's request to have a visual display preview for the camera, while the ``SurfaceTexture'' itself doesn't need to be displayed on screen. Therefore an picture can be taken from a background service without the user being aware of this event.

A DAO class called ``FaceDAO'' was developed in order to interface with the Javafaces library port. The authentication thread running within the ``FaceService'' object periodically captures an image from the camera. The image is then validated using the ``FaceDAO'' object. The value returned from the Javafaces library is the Euclidean distance to the closest registered user, which in our case is the owner of the device. This distance is handled in exactly the same way as the voice recognition mechanism.

Another problem encountered by face recognition mechanism is the size of the data involved in performing the face recognition. With standard pictures, the application runs out of memory and is closed by the Android OS. In order to fix this issue, Bitmaps collected from the camera are scaled to 50% before they are processed by the library.

\subsection{Owner configuration}
In order to configure the biometric authentication mechanisms in a flexible, controlled manner a couple of Android activities were developed. There are used to set the initial owner biometrics based on which the mechanisms will output their confidence levels. These activities use the same DAO classes in order to store the data once it was collected. Due to the size of the data, which is relatively small, the files can be stored in the application's internal memory, making it inaccessible by other applications.

\section{Threat model}
% TODO:
% - clear cache data from android
% TODO: not really relevant since this is just a prototype
% TODO: any cryptographic keys and content is not secure as the application may be decompiled
% TODO: DOS by generating bad sensor data
% TODO: check William Ench et al year
% TODO: referenced android website instead of talk about components
% TODO: more mmap_min_addr stuff. why does it help?

Even though the scheme implementation is a proof of concept, we will continue by analysing different threat models. This will reveal any flaws behind the concept, allowing for a more robust future implementation.

The purpose of the Pico token is to provide a robust authentication mechanism, without the use of any secrets for the owner to remember. Where the Pico unlocking scheme fits, is correctly identifying the owner of the Pico. Attacks may be performed in the form of malware installed on the device while still within the possession of its owner. The main threat however comes from an attacker having physical access to the user's Pico. 

It is important to note that since this is a purely software implementation, physical access may mean either that the attacker is in possession of the phone, or that it may replicate the secretes of the victim's Pico on a separate device. Replicating the Pico secretes would clearly create much more damage for the user from a cost perspective. A total reset of authentication credentials would be necessary for all accounts registered with the Pico device.

% Android specific security
\subsection{Literature review: Android security}
In order to perform a valid threat modelling of the scheme, we need to have a better understanding of the Android security model, and the ``features'' offered by different mechanisms such as Intents and filters based on these intents.

An interesting paper by William Enck et al \cite{enck2009understanding} offers a good description of the Android OS, with a focus on the security aspects of the platform. It is a relatively old paper from 2008 which is the same year of the Android initial release. The initial set of Android open standards was however released sooner, in November 2007, allowing researchers such as William Enck to perform an initial analysis of the system.

Android runs on a port of the Linux kernel. This means that many of the linux security mechanisms such as file permissions, are also part of Android. On top of the kernel lies an App middleware layer which the Java SDK can use in order to extends the functionality of the hand held device.

Each android application is split into multiple components, with no fixed entry point such as a main() routine. Based on its purpose, the SDK defines 4 types of components: activity, service, content provider, and broadcast receivers. More details regarding the purpose of each component can be found on the Android website \cite{liebergeld2013android}. Another important architectural note is that components, and even applications, may communicate between each other using Intents. This type of communication is abbreviated as Inter Component Communication (ICC).

An important note in the paper is that when a Service becomes bound by another component it cannot be terminated by an explicit stop action. This provides an useful guarantee regarding the lifetime of a service. 

Th two types of security enforcements can be split into two categories: ICC and system level. System level security is mainly defined in terms of the user id (UID) and group id (GID) permissions. According to the Android OS, each is allocated an UID and GID. The security guarantees of this mechanism are the same as those of a linux based system. An interesting example would be a vulnerability of the  T-mobile G1 phone browser, which due to this system level security enforcement did not affect any other applications.

The focus of the paper however is ICC security enforcements. These are based on I/O control command of the ``/dev/binder'' special node. Since the node needs to be world readable and writeable, linux cannot mediate ICC. This is performed using labels using a MAC framework enforced by a reference monitor. This defines how apps are allowed to access different components.

Components may be defined as either public or private. This refinement is configured by the ``exported'' field defined in a manifest file. It defines whether or not another application may launch or interact with a component from another application. At the time the paper was written, the ``exported'' field was defaulted to ``true''. However, as shown in a more recent paper \cite{liebergeld2013android} written by Steffen and Mathias in 2013, starting with Linux Android (LA) 4.2  the default of this value was changed to ``false'', therefore conforming to the ``principle of least privilege''.

Components listening for Intents need to have a registered filter in the manifest file. This is both convenient and secure for the developers of the system. If however the developer of an application wishes to restrict access to intent objects, the SDK provides user definable Intent permission labels as well as Service hooks. These provide runtime security checks for the application and prevent data leakage through ICC. Using permission labels, the developer may broadcast events to which only components which registered that permission may access. The same principle applies to service hooks, but in this case a component holding some permissions tries to bind on a service which checks these permissions and exposes different APIs based on the permissions held by the binding component.

Another more recent paper was written by Steffen and Mathias \cite{liebergeld2013android}. The authors focus on deeper issues of the LA, and how they were solved from one update to the other. However, it is shown that OEMs tend not to update the software of their devices once they have shipped, which poses a number of security issues.

The starting point of understanding Android security and how it is bootstrapped is the five step booting process:
\begin{enumerate}
	\item Initial bootloader (IBL) is loaded from ROM.
	\item IBL checks the signature of the bootloader (BL) and loads it into RAM.
	\item BL checks the signature of the linux kernel (LK) and loads it into RAM.
	\item LK initialises all existing hardware and starts the linux ``init'' process \cite{SPECIFY WHAT THIS IS FROM PAPER}.
	\item The init process reads a configuration file and boots the rest of LA.
\end{enumerate}

The android security model described in this paper \cite{liebergeld2013android} is split in two categories: system security, and application security.

The android base system (libraries, app framework, and app runtime) is located in the ``system'' partition. Although this is writeable only by the root user, a number of exploits which allow root access, such as those shown in \cite{liebergeld2013android}, have been possible. Android provides a keychain API used for storing sensitive material such as certificates and credentials. These credentials are encrypted using a master key, which is also stored in AES encrypted format. In this case, security needs to begin somewhere, an assumption has to be made about a state being secure which would allow for multiple security extensions. In this case, the master key is considered to be that point of security. However, given a rooted device, where due to an exploitation a process may be granted root privileges, the master key itself may be retrieved from the system therefore compromising all other credentials.

From the application's perspective, an interesting ``feature'' which may affect the flow of information within LA is the fact that applications from the same author may share private resources. When installing an application the user needs to accept its predefined set of permissions. Due to resource sharing, a situation may present itself when an application which has permissions for the owner's contacts may communicate with an application which has permissions for internet in order to leak confidential data. A developer may therefore construct pairs of legitimate applications in order to mask a data flow attack.

The LA system offers a number of memory corruption mitigations in order to avoid buffer overflow attacks, or return oriented programming attacks. The following list includes mechanisms enabled in time based on the LA version:
\begin{itemize}
	\item Implements mmap\_min\_addr which restricts mmap memory mapping calls. This prevents NULL pointer related attacks.
	\item Implements XN (execute never) bit to mark memory as non-executable. The mechanism prevents attackers from executing remote code.
	\item Address space layout randomisation(ASLR) implemented starting with LA 4.0. This is a first step to preventing return oriented programming attacks. The position of the binary library itself is however static, meaning that after a number of attempts, using trial and error, the attacker may succeed using return oriented programming.
	\item Position independent and randomised linked (PIE) is implemented starting with LA 4.1 in in support of ASLP. This makes position of binary libraries themselves randomised.
	\item Read only relocation and immediate binding space(RELro) was implemented starting with LA 4.1. It solves an ASLR issue where an attacker could modify the global offset table (GOT) used when resolving a function from a dynamically linked library. Before this update an attacker may insert his own code to be executed through the GOT table.
\end{itemize}

A number of security enhancement mechanisms are in place in order to make Android a safer environment for its users. In order to prevent malware within the Android App store (Google Play), a program also known as a ``on device Bouncer''. The purpose of the bouncer is to verify apps prior to installation for any malware signature or patterns. Secure USB debugging was introduced starting with LA 4.4.2, which only allows hosts registered with the device to have USB debugging permissions. This mechanism is circumvented if the user does not have a screen lock.

%TODO: cannot cite F-Secure Mobile Threat Report 2012
As pointed out by Steffen and Mathias, the Android OS is responsible for 96\% of mobile phone malware according to a study from 2012 \cite{}. The authors claim that this is the case due to 4 big issues of the Android concept:
\begin{enumerate}
	\item Security updates are delayed or never deployed. This is due to a number of approvals that an update needs to get throug in order to be pused to devices. This implies an additional cost to the manufacturer(OEM) which does not generate any revenue. The majority of teams working on LA are focusing on current releases, and in some cases there is simply not enough time and resources to merge Google security updates to the OEM repository. Furthermore, the consequences of a failed system update for the user may cause problems as serious as ``brick''-ing the device which is a huge risk for the manufacturer. All these issues contribute to a severe lack in security updates. Therefore, important updates such as RELro are never pushed to LA 4.0, making the device vulnerable.
	
	\item OEMs weakens the security architecture and configuration of LA by introducing custom modifications before they roll out a device.
	
	\item The Android permission model is defective. As pointed out by Steffen and Mathias, according to Kelley et al \cite{kelley2012conundrum}, most users do not understand the permission dialogue when installing an application. Furthermore, even if they could understand the dialogue, most of the time it is ignored in order to be able to use a new exciting app. According to the same study, most applications are over-privileged due to the developers not understanding what each privilege does. Furthermore, as previously pointed out, applications from the same owner may share resources, therefore creating a valid data flow attack path. 
	
	% TODO: cite http://www.techrepublic.com/blog/google-in-the-enterprise/malware-in-the-google-play-store-enemy-inside-the-gates/#.
	\item Google Play has a low barrier for malware. A developer distribution agreement (DDA) and a developer program policy (DPP) need to be greed to and signed by the developer before submitting the application to the Android market. However, Google Play does not check upfront if an application adheres to DDA and DDP. The application is only reviewed if suspect of breaking the agreements. According to \cite{percoco2012adventures} there are ways of circumventing the Bouncer program. An example is treated in an article \cite{} written in Tech Republic.
\end{enumerate}

\subsection{Prototype threat model}
Let us now continue by studying the threat model of the Pico authenticator application. We will consider the security mechanisms presented above as the predefined assumptions made in this model. In order to reduce the threat space we will consider the application is running on a hand held device running Android 4.4.2 with all recent updates.

\subsubsection*{Availability}
Breaking the scheme's availability if the device is in the possession of the attacker is relatively trivial. The application can be uninstalled, or the application data cache can be cleared, therefore removing the owner biometric models for the different mechanisms. Furthermore, in this case the owner is already no longer in possession of their Pico, so basically the Pico is already made unavailable..

Let us continue however and study what can be achieved from a DoS perspective by the attacker from the perspective of the individual user app accounts, which would need to be reset by the owner. In order to gain any sort of access and make credentials reset not possible, or at least have a chance in doing so, the attacker would have to unlock the Pico.

From a malware attack perspective data used by the authentication should not be modifiable. This would guarantee that all mechanisms have their cached biometric data available at all times and may function properly. Due to the Linux permissions mechanism and the fact that each application has its own UID and GID, data stored in internal memory should not be readable or writeable by any other user level application in the system. If however the device is rooted and the owner is mislead into granting root privileges to another application, then the security model would be broken and the data would be exposed. This could lead to deletion which would make the mechanisms not function properly, resulting in a DoS attack.

% TODO: can we fake data from sensors?

\subsubsection*{Integrity}
Just as mentioned in the Availability section, the authenticator should be safe against any data accesses from other applications as long as the application does not have root privileges. This would allow malware to break the integrity property of the data.

From a data flow point of view Intents used for communication within the authenticator as well as with the Pico application are not modifiable. Furthermore Intents are not broadcasted using the implicit Android broadcast mechanism, which makes them impossible to replay or even intercept.

\subsubsection*{Confidentiality}
Considering a circumstance where a root malware process would have access to the authenticator's data stored on disk, this would not lead to a direct compromise of the owner biometric data. All cached files are stored in internal memory in encrypted format. The mechanism used for encryption is RSA, with the private key stored using the Android KeyChain API.

On a rooted device however, the encryption layer provides only another bi-passable layer of security. With root access, an application could retrieve the master key of the KeyChain and use it to retrieve or private key and decode the owner's biometrics.

From a data flow perspective, internally the Pico authenticator uses an self-developed broadcast system. Client processes need to register with the broadcaster, such as the UAService, in order to receive updates. This ensures data confidentiality throughout the system. Furthermore, the authenticator and Pico should be released under the same author. This would allow locking the application from outside Intents as well as interaction with different components. Sandboxing communication is always a desirable property from a confidentiality perspective.

The paper by Adrienne Porter Felt et al \cite{felt2012android} shows that according to their surveys only 17\% of users pay attention to the Android permissions dialogue, and only 3\% understand what each permission represents. A malware application which has granted full permissions gets pass the Bouncer and is installed as an application. Even so, due to the Linux permission model adopted by Android, the confidentiality of the authenticator's data would not be compromised. Instead however, the malware application may collect all relevant on its own from the user, allowing for a powerful replay attack in the future.

\section{Future work}
The application was implemented as a proof of concept. It is developed in order to show that different data may be obtained without the owner's knowledge. Additional improvements can be made in order to increase the confidence level of the authenticator.  Furthermore, due to time constraints and unavailability of free to use biometric libraries, a number of mechanisms were not implemented. The list can easily be extended by simply creating a class which extends the ``AuthMechService'' abstract class.

One way to improve the voice recognition mechanism would be to start sampling data whenever a call is active. This would increase the chances of capturing an accurate sample of the owner's voice. In this context, a better voice recognition library can be used, which supports multiple speakers and/or ignores background noise. If such a library is not available, we can rely on the fact that most of the times people take turns when speaking. For the duration of the call, with a high enough sampling frequency, the individual sampling voice of both participants should be captured. However, it is important to take into account a situation in which the thief is calling the owner on a different phone in order to unlock his Pico.

Immediate improvements can be made to the face recognition mechanism. Just as recommended in the description of the mechanism's implementation, another library which provides more meaningful face coordinates may be used for face detection. Alternatively, and preferably, a different library which performs both face detection an recognition can be integrated with the mechanism.

Another improvement for the face recognition mechanism would be from the data sampling perspective. Instead of capturing images at a fixed interval, pictures should be taken only when the phone unlock event is triggered. While the phone is unlocked it is highly likely that the user will face its front camera.This would provide better chances of processing meaningful data.

\section{Results}
%TODO:
Need to figure out how to present meaningful results for different scenarios, and how the mechanism would work. Use dummy authenticator to generate data.

\section{Related work}
Liang Cai et al makes analyse in their paper \cite{cai2009defending} ways of protecting users from mobile phone sensor sniffing attacks. The authors design a framework used for protecting sensor data from being leaked. From a security perspective it is noted that the user is not to be trusted with granting permissions to different applications. An important point made in this paper is the fact that malware may deny service to legitimate applications such as our authenticator by creating a race condition for acquiring a lock on the sensor. The solution proposed by the authors would be an user notification, allowing for the owner to decide which application acquires the lock. A suggestion to this approach would be to allow for different priority levels, such that malware applications would not acquire the lock in a race condition, or even more, would lose it when a high priority application such as the Pico authenticator would require sensor data.

The paper by Derawi et al \cite{derawi2010unobtrusive} presents the feasibility of implementing gait authentication on Android as an unobtrusive unlocking mechanism. According to the definition offered by the authors ``gait recognition describes a biometric method which allows an automatic verification of the identity of a person by the way he walks''. According to the paper, the Android implementation has an equal error rate (EER) of 20\%. Dedicated devices have an EER of only 12.9\%, and the cause for this is the sampling rate used by the authors. They have used a Google G1 phone with about 40-50 samples per second, which is much inferior to dedicated accelerometers which sample data at 100 samples per second. However, by conducting personal experiments with the Accelerometer of a Google Nexus 5 phone, using the highest sampling setting (SENSOR\_DELAY\_FASTEST) sampling rates go above 100 samples per second. Therefore the performance of devices has increased, making the authentication mechanism more reliable.

Ming et al presents in his paper \cite{ming2007robust} how to improve speaker recognition accuracy on mobile devices in noisy conditions. This approach uses a model training technique based on which missing features may be used to identify noise. The focus of the paper is biometric implementation related and is therefore outside the scope of the project. 

Another technique in performing speaker recognition involves using voiceprints. These are a set of features extracted from the speaker sample data. Kersta explains in their paper \cite{kersta2005voiceprint} the mechanism in more detail. The benefits of having feature extraction based on a voice sample as opposed to a different voice recognition mechanism is the fact that voiceprints do not require knowing any secrets. The speaker doesn't have to reproduce a voice sample. This increases the usability of the mechanism in the scenarios required by the Pico authenticator.

A popular paper on face authentication \cite{turk1991face} was written by Turk and Pentland. The biometric authentication process is based on the concept of eigenfaces. Eigenfaces are a name given for the eigenvectors which are used to characterise the features of a face. These features are projected on to the feature space. Using Euclidean distances in this feature space, a classification can be performed in order to correctly identify individuals. An implementation of this concept was implemented with the prototype of the Pico unlocking scheme.

A more unconventional method for authenticating users was presented by Clarke and Furnell in their paper \cite{clarke2007authenticating} on keystroke analysis. This mechanism is unobtrusive and authenticates users during normal interactions such as typing a text message or a phone number. It is based on a neural network classifier, reporting an EER of 12.8\%. Input data used for classification is composed out of time between successive keystrokes, and hold time of a pressed key. 






