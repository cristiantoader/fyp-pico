% Chapter Template

\chapter{Implementation Prototype} % Main chapter title

\label{Chapter4} % Change X to a consecutive number; for referencing this chapter elsewhere, use \ref{ChapterX}

\lhead{Chapter 4. \emph{Implementation Prototype}} 
% Introduction for implementation
In this chapter we have developed a prototype for the scheme proposed in section \ref{propopsedsol}. We have chosen as an implementation platform the Android Nexus 5 hand held device. The device offers enough sensors to perform biometric and behavioural analysis \footnote{The full range of sensors supported by the Android platform can be found here: http://developer.android.com/guide/topics/sensors/sensors\_overview.html (accessed on 28.05.2014)}. These resources will be used to demonstrate that the scheme can be implemented using similar dedicated hardware that may offer more security features.

We have included a brief overview of the Android development model and the platform's security features in appendix \ref{AppendixA}. This information provides an introduction for understanding the principles used in the design of the prototype.

\section{Implementation overview}
% Introduction with UAService
The Android token unlocking prototype is designed to work as a bound service. It is implemented in the ``UAService'' \footnote{The name of the class stands for User Authentication Service} class. Feedback is provided to clients either after an explicit request or through periodic broadcasts.

% Mention independent services for each mechanism and why
Each authentication mechanism may have a different requirement for sampling and processing data. As an example, voice recognition may gather optimal data during a phone call \footnote{Call events can be intercepted by registering a listener for the PHONE\_STATE event}, while face recognition when the phone screen is unlocked. Therefore, to enable more flexibility in the individual mechanisms' implementation, they are developed as independent services.

% Management of authentication mechanisms
``UAService'' communicates with the authentication mechanisms by binding to their service. On predefined time intervals it requests the confidence level and weight of each mechanism. Using this data it then calculates the overall result according to the design in section \ref{authfeedback}. Feedback is sent back to each registered client for interpretation.

\section{Implementation details}
% Introduction (can expand)
This section presents the implementation of the scheme proposed in section \ref{propopsedsol}. The full source code for the prototype can be downloaded from from github: 

``https://github.com/cristiantoader/fyp-pico''.

\subsection{UAService}
% UAService
The token unlocking mechanism is developed as an Android application. It is started using the ``PicoMainActivity'' class, which acts as a Pico client. The scheme itself is developed to be used as a bound service, implemented in the ``UAService'' component.

% UAService, a safer yet riskier implementation
% TODO: maybe remove this..
A safer implementation would be to create a system service. This requires modifications to the root partition. The process does not resume to simply gaining root privileges and making the modifications. The root directory is mounted as ramdisk, and therefore any direct changes will be reverted once the device is rebooted. In order to make persistent modifications, the user needs to alter the boot image, and flash it on the device. The service needs to be compiled using the Android NDK C compiler. The binary has to be included in the system partition of the boot image in order to be accessible by the ``init'' process during start up. The ``init.rc'' configuration file used by ``init'' also needs to be configured to start the service.

% UAService and lifetime
``PicoMainActivity'' starts ``UAService'' using the ``Context.startService()'' method. Communication is enabled by binding to ``UAService'' using ``Context.bindService()''. This approach protects the lifetime of the authenticator. When ``PicoMainActivity'' gets sent to background and loses control of the screen, `UAService'' is not explicitly unbound. This guarantees that the service will continue running in the background, and should prevent malicious components from stopping it. 

% UAService, central node.
As mentioned, clients need to bind ``UAService'' in order to receive authentication updates. When bound, communication is enabled by exchanging ``Messenger'' objects using the IBinder interface. A ``Messenger'' allows another component to send ``Message'' objects, and defines how they will be handled by the receiver using an ``IncomingHandler''. The ``Messenger'' queues all requests on a single thread, and therefore the application does not require to be thread safe. 

When bound by a client, the ``IncomingHandler'' used by ``UAService''  exposes the following functionality defined by the ``what'' parameter of a received ``Message'': 
\begin{description}
  \item[MSG\_REGISTER\_CLIENT] \hfill \\
  Used for registering a client for periodic broadcasts of the current authentication confidence level. Feedback is provided at a fixed time interval of 1000ms \footnote{An alternative implementation explored in the project was to have each client also register a confidence level using the ``arg1'' parameter of a ``Message''. In this case, the authenticator would only provide each client with a locked/unlocked result. However, this would shift the meaning of client to that of an authentication session, with state managed by the unlocking scheme. A client would therefore have multiple connections, requiring more ICC. Since all ``Messenger'' requests made to ``UAService'' are queued to a single thread, this would slow down the feedback process and possibly lead to a denial of service attack. Therefore we have chosen to reduce the communication overhead, and have each client manage the status of its authentication sessions based on the confidence level provided by the unlocking scheme.}.

  \item[MSG\_UNREGISTER\_CLIENT] \hfill \\
  Used for any application component to unregister as a listener from ``UAService''.
  
  \item[MSG\_GET\_STATUS] \hfill \\
  Used by a client for requesting an explicit authentication status update.
  
\end{description}

% Communication with authentication mechanisms
The ``UAService'' service wraps an ``UserAuthenticator'' object that provides most of the functionality used by the service. The ``UserAuthenticator'' is responsible for collecting data from the authentication services, and computing the final confidence level. The fact that the mechanisms are independent services is hidden from the ``UserAuthenticator'' using ``AuthMech'' objects. Each ``AuthMech'' binds an authentication mechanism, and is responsible of listening for confidence level updates. They provide to the ``UserAuthenticator'' the most recent data available for computing the final result.

% Authentication mechanism side
Each authentication mechanism service extends the ``AuthMechService'' abstract class. This makes them bound services with a standardized ``IncomingHandler'' for communicating with ``AuthMech'' objects. They expose the following ``Message'' passing interface:
\begin{description}
  \item[AUTH\_MECH\_REGISTER] \hfill \\
  Used for registering the ``UAService'' client to the ``AuthMechService''.
  
  \item[AUTH\_MECH\_UNREGISTER] \hfill \\
  Used for unregistering the ``UAService'' client from the ``AuthMechService''.
  \end{description}  
  
\subsection{Authentication mechanisms}
\label{implauthmech}
% Introduction for authentication mechanisms
In order to create a functional prototype of the scheme, we have implemented a number user authentication mechanisms. The result quality of each mechanisms is outside the scope of this project. Their sole purpose is to demonstrate that the design of the scheme is functional, and can be implemented using only a smart phone.

% Mechanism requirements (they are a bit logical rather than engineering)
When developing an authentication mechanism for this scheme, the following abstract requirements need to be satisfied: 
\begin{enumerate}
	\item The result needs to be quantifiable in the form of a percentage ranging from 0 to 100, where 100 means that the mechanism has $100\%$ confidence that the user is the owner of the token.
	\item The mechanism needs to support continuous authentication of the user.
	\item The authentication process needs to be effortless and preferably unobtrusive for the user.
\end{enumerate}
A list of authentication mechanism examples that can be implemented on the Android platform is presented in appendix \ref{AppendixC}.

% AuthMechService functionality
Each mechanism developed for this scheme extends the ``AuthMechService'' abstract class. As mentioned, this class is responsible for defining the mechanism as a bound service. The communication with ``UAService'' is standardized by implementing ``Service.onBind()'' to register the same ``IncomingHandler'' implementation. Furthermore, ``AuthMechService'' defines the decay of a mechanism's weight. This is implemented using a ``Handler'' object that schedules a ``Runnable'' to execute at a predefined time interval. The ``Runnable'' is responsible for decreasing the mechanism's weight and sending the result to the corresponding ``AuthMech''.

% General application design
Although it is not enforced, each authentication mechanism has the same general design. The mechanism implements the ``AuthMechService'' abstract class. Its ``Service.onCreate()'' method starts a threat responsible for collecting data using an access object (DAO), and analysing it using a class that mediates interaction with other components. 

% Euclidean distance conversion
Most biometric libraries provide feedback as an Euclidean distance. To convert it to a percentage confidence level, we define for each mechanism an acceptable threshold. Any result above the threshold is considered too high and is truncated to its value. Using the equation \ref{eq:euclideantoprob} we convert the Euclidean distance to a confidence level. The final result is $P(E|H)$ (the probability that the evidence belongs to the hypothesis) used in equation \ref{eq:final}.

\begin{equation} 
\label{eq:euclideantoprob}
P(E|H) = 1 - \frac{distance}{THRESHOLD}
\end{equation}

% Probability computation estimate
Dividing the distance over the threshold yields a confidence value between 0 and 1, where 1 is a very large distance and hence a bad result. By using one minus this value we invert the meaning. Values will range between 0 and 1, and 1 corresponds to a confidence level of $100\%$. 

% Do Bayesian update
Having known $P(E|H)$ we continue to calculate $P(H|E)$ by using the Bayesian update formula defined in equation \ref{eq:final}. When calculating the final confidence level of the mechanism, we multiply $P(H|E)$ with the current decay rate.

% Listing mechanisms with promises of future details
The following mechanisms have been implemented as part of the prototype: voice recognition, face recognition, location analysis, and a dummy mechanism used for testing. The following sections will provide details regarding their functionality and implementation process.

\subsubsection{Dummy mechanism}
% Introduction: what it does, why it was developed
A dummy authentication mechanism was developed for testing the overall scheme. It produces random confidence levels within a predefined range, which provides a controlled testing environment.

% General implementation detail
The mechanism was developed in the ``DummyService'' class and was designed consistently with the application model. Its ``DummyService.onCreate()'' method creates an authentication thread that periodically generates random values and sends them to ``UAService''. It does not use any DAO, in order not to over complicate its implementation. The random values are generated using a ``DummyAuthMediator'' object. This mimics an authentication mechanism that periodically samples for data.  

\subsubsection{Voice recognition}
% Introduction to mechanism
The voice recognition mechanism is implemented in the ``VoiceService'' class and extends the ``AuthMechService'' abstract class. The ``VoiceService.onCreate()'' method starts a thread that periodically gathers data from the microphone, performs biometric authentication, and produces a confidence level.

% Library used for implementation
The library used for voice recognition is called Recognito\footnote{The library can be downloaded using github from the following link: https://github.com/amaurycrickx/recognito}, and was developed by Amaury Crickx. It use a text independent speaker recognition algorithm developed using Java (SE). Its author claims very good results in scenarios with minimal background noise\footnote{It was tested by the author on TED talks, such as:  https://www.ted.com/talks/browse (visited on 06.01.2014)}.

% Hack to compile with the library
Porting Recognito for Android required no changes. However, in order to package the library, a subset of the ``rt.jar'' Java (SE) library is need for sound file formats. Including the full ``rt.jar'' is not possible due to a package name collision with Android ``javax.*'' system libraries. Therefore, we have included only the ``javax.sound.*'' package using a custom jar. This was purely done to trick the Android Java compiler to build the application. Using ``javax.sound'' features would generate a runtime error. However, we only use Recognito functions which require direct data input, without any knowledge of sound file formats.

% Recording configuration
In order to gather and manage samples compatible with the Recognito library we have created the ``VoiceDAO'' class. Microphone input is gathered using the following predefined configuration:
\begin{itemize}
	\item Sample rate: 44100
	\item Channel configuration: AudioFormat.CHANNEL\_IN\_MONO
	\item Audio format: AudioFormat.ENCODING\_PCM\_16BIT
\end{itemize}

% More random recording stuff..
The minimum buffer size required by `VoiceDAO'' is device dependant and pre-calculated in the constructor. The class wraps an ``AudioRecord'' object used for gathering microphone data. Due limitations of the SDK, the recording is saved as a file and loaded into memory when needed. This makes the implementation not efficient, but it does serve the purpose of the prototype.

% Library Mediator
The ``VoiceAuthMediator'' class was created to mediate calls to the Recognito library. When initialised, it loads the owner configuration, and a predefined set of background noises. It then creates a Recognito object and trains it using the data. This is performed using the ``Recognito.createVocalPrint()'' method.

% Getting confidence level
Every predefined time interval, the ``VoiceService'' authentication thread records data in ``double[]'' format using a ``VoiceDAO'' object. It then calls the ``recognize'' public method of the Recognito object. This returns the Euclidean distance to the closest match, which is either the owner, or one of the background noises used for training. 

% Sending confidence level
 We convert the Euclidean distance from ``VoiceAuthMediator'' to a percentage using equation \ref{eq:euclideantoprob}. The value is stored in the service and the decay process is started. Whenever the weight is modified, a ``Message'' is sent to the corresponding ``AuthMech'' and updates its value. 
 
\subsubsection{Face recognition}
\label{implface}
% Introduction: brief description of the implementation
The face recognition mechanism was implemented in the ``FaceService'' class and extends the ``AuthMechService'' abstract class. When created, the service starts a thread that periodically collects data from the camera. Each sample is analysed using a face recognition library, and a confidence level is outputted for the current sample.

% Javafaces library
For biometric face recognition we use a port of the ``Javafaces'' library \footnote{The ``JavaFaces'' library is maintained at the following address: https://code.google.com/p/javafaces/}. This was the closest functional library found that was compatible with the Android API. Javafaces is written entirely using Java (SE). Unfortunately, it makes use of the ``javax.imageio.'' package which is not available in the standard Android SDK. A considerable amount of code needed to be ported for the Android platform. The new library is publicly available at the following link: https://github.com/cristiantoader/JavafacesLib. It is currently not optimised for public use.

% Javafaces changes for porting to Android
We will briefly present the changes made when porting the ``Javafaces'' library. The ``BufferedImage'' class had to be replaced by its closest Android equivalent, which is ``Bitmap''. All ``BufferedImage'' references in the original project had to be adapted. Furthermore, The API was modified to support direct ``Bitmap'' input in order to add more flexibility and lighten the main code of the authenticator. 

In the original ``Javafaces'' library, data formats for black and white images were assumed to have a single colour channel representing the grey value. This had to be changed to reflect the Bitmap convention that uses all 3 colour channels. Additional modifications were required due to data type mismatches, as well as other related issues.

% Authentication process
%	TODO: include that if no faces are found, the process is not performed
Every predefined time interval, the authentication thread running within the ``FaceService'' object samples data from the camera. This is performed by using a ``CameraUtil'' object. The ``CameraUtil'' class was developed as a mediator to simplify the interface to the Android ``Camera''. For example, it performs additional checks such as the orientation of the phone. 

A DAO class called ``FaceDAO'' was developed to mediate calls to the Android ``Javafaces'' library. Images captured using ``CameraUtil'' are validated using the DAO. The value returned from the ``Javafaces'' library is the Euclidean distance between collected data and the registered owner. This distance is handled in the exact same way as the voice recognition mechanism \ref{implface}.

% Gather picture with no notification: shutter sound
By default, the Android API does not easily allow for a Camera picture to be taken without any sort of notification to the user. Both a shutter sound and a visual preview display should be present. The shutter sound can be disabled easily by not providing a shutter callback function when calling the Camera.takePicture() method. 

% Gather picture with no notification: shutter sound
Disabling the user preview of the camera was more difficult to achieve. The solution used with this prototype was to exploit an Android feature that allows to render the preview in a ``SurfaceTexture'' object. This satisfies the API's requirement to have a visual display preview for the camera, while the ``SurfaceTexture'' itself does not need to be displayed on screen. Therefore a picture can be taken from a background service without any interruption to the user.

% Issue of image too large
Another problem encountered by the face recognition service is data sizes. When the ``Javafaces'' library performed its algorithm, the device was running out of memory. This caused the app to be closed by the Android OS. To fix this issue, Bitmaps collected from the camera are scaled to $50\%$ before they are processed by the library.

% Results
Unfortunately, the library combined with the Android SDK does not provide accurate results. The reason is that it requires as input a bitmap perfectly containing the face of an individual. Unfortunately, although the Android SDK offers face detection, it only provides the location of the midway coordinate between the eyes, and the distance between the eyes. Using this data alone, an accurate crop cannot be made. As a solution, yet another library would need to be used in order to properly detect face regions. This would provide better input data and would increase the precision of the mechanism.

\subsubsection{Location analysis}
% Introduction: short description of the mechanism
The mechanism is based on gathering location data and using it to generate a probability that the owner is present. This is implemented in the ``LocationService'' class that extends the ``AuthMechService'' abstract class. Data is collected periodically by using the ``LocationManager'' provided by the Android API.

% DAO object used for collecting data
A DAO object is used to mediate calls to the Android API and manage the existing owner configuration. It is implemented in the ``LocationDAO'' class. It offers functionality for gathering and saving location updates. It is developed to use the most accurate data provider. The Android API offers the following sources of collecting ``Location'' data:
\begin{itemize}
	\item GPS\_PROVIDER: Collects data from the GPS.
	\item NETWORK\_PROVIDER: Collects data from cell tower and WiFi access points.
	\item PASSIVE\_PROVIDER: Passively collects data from other applications which receive ``Location'' updates.
\end{itemize}

% Describe the algorithm class
External libraries were not used for the authentication process. We have developed a primitive location analysis algorithm in the ``LocationAnalyser'' class. During the configuration stage, which is a process managed by ``LocationActivity'', location data is sampled every 5 minutes and saved in internal storage. After the process has ended, each time a ``Location'' is sent for authentication it is compared with all the locations saved during the configuration process. The final result is the minimum Euclidean distance between the current ``Location'' and any other saved ``Location''. 

% Describe how authenticator thread works
When the service is started by ``UAService'', its ``onCreate()'' method spawns an authentication thread. This thread periodically requests the current location using the DAO. Data is returned in a ``Location'' object and is provided as input to the ``LocationAnalyser''. The result of this operation is an Euclidean distance which gets converted to a percentage using a threshold, just as in the previous mechanisms. The value is stored by the service for future requests from ``UAService''.

% Conclusion
Just as mentioned in section \ref{implauthmech}, the mechanism has a lower confidence level. Although being in a known location provides some confidence that the token was not stolen, it does not offer any guarantees that the device is still with its owner.

\subsection{Owner configuration}
% Owner configuration activities exist
There are a number of Activity components that are used in the configuration of the prototype. Each authentication mechanism has a corresponding Activity that can be started from the main Activity called ``PicoUserAuthenticator''. These are used to register owner biometrics needed by the mechanisms.

% They use DAO objects to store data in internal storage
Each configuration Activity uses the same DAO class as the mechanism Service. The DAO is used for storing the owner data once it was collected. Given that the overall size of the data is relatively small, the files are kept in internal storage. The Linux Android permissions mechanism guarantees its confidentiality and integrity, and therefore further encryption is not necessary.

\subsection{Cryptographic protection}

\section{Conclusion}
% Small overall
We have described the Activity and Service components developed for the prototype, as well as their communication flow. We have ported two biometric libraries and developed a location analysis mechanism. DAO objects facilitate accessing owner configuration files and the interface with auxiliary mechanisms. An overview of the app design can be seen in figure \ref{TODO}.

% TODO: insertgraphics() overall design of the app.

% Limitations and solutions
One of the limitations of the prototype is the lack of explicit authentication mechanisms. Another issue is the precision of the biometric mechanisms, in the lack of better libraries. However, due to the modular design of the application, existing mechanisms can be improved simply by importing a new library and modifying its DAO. The existing set of mechanisms can easily be increased by creating a new class that extends ``AuthMechService'', and implementing the algorithm logic. In order to be managed by ``UAService'', the new mechanism needs to be included in the ``UserAuthenticator.initAvailableDevices()'' method.

% TODO: continue from here
\section{Related work}
% Sensor sniffing
Liang Cai et al \cite{cai2009defending} analyse ways of protecting users from mobile phone sensor sniffing attacks. The authors design a framework used for protecting sensor data from being leaked. From a security perspective the user should not to be trusted with granting permissions to different applications. A solution provided in the paper is for sensors to become locked once they are used. A downside to this is that malware may deny service to legitimate applications (such as our prototype) by creating a race condition for acquiring a sensor lock. This can be solved by using a user notification, allowing for the owner to decide which application acquires the lock. A suggestion to this approach would be to allow for different priority levels, such that malware applications would not acquire the lock in a race condition, or even more, would lose it when a high priority application such as the Pico authenticator would require sensor data.

% Gait recognition
The paper by Derawi et al \cite{derawi2010unobtrusive} presents the feasibility of implementing gait authentication on Android as an unobtrusive unlocking mechanism. According to the definition offered by the authors ``gait recognition describes a biometric method which allows an automatic verification of the identity of a person by the way he walks''. The Android implementation developed by the authors has an equal error rate (EER) of $20\%$. Dedicated devices have an EER of only $12.9\%$, and the main cause for this is the sampling rate available at that time (2010). They have used a Google G1 phone with approximately 40-50 samples per second. This is much inferior to dedicated accelerometers that sample data at 100 samples per second. However, by conducting personal experiments with the accelerometer of a Google Nexus 5 phone, using the highest sampling setting (SENSOR\_DELAY\_FASTEST) the rates go above 100 samples per second. Therefore the current performance of the algorithm paper should be closer to $12.9\%$.

% Improve speaker recognition in noisy conditions
Ming et al \cite{ming2007robust} present in their paper how to improve speaker recognition accuracy on mobile devices in noisy conditions. This approach uses a model training technique based on which missing features may be used to identify noise. The focus of the paper is designing and implementing a biometric mechanism, and is therefore outside the scope of this dissertation project. 

% Voiceprints voice recognition
Another technique in performing speaker recognition involves using voiceprints. These are a set of features extracted from the speaker sample data. Kersta \cite{kersta2005voiceprint} explains the mechanism in more detail. The benefit of having feature extraction based on a voice sample, as opposed to a different voice recognition mechanism, is that voiceprints do not require any secrets. The speaker doesn't have to reproduce a voice sample. This increases the usability of the mechanism in scenarios required by the Pico authenticator. However, a downside to this approach is that it makes replay attacks easier to perform. Any recording of the user is sufficient for an attacker to trick the biometric mechanism.

% Face recognition
%	TODO: need to review the phrasing
A popular paper on face authentication was written by Turk and Pentland \cite{turk1991face}. The biometric authentication process is based on the concept of eigenfaces. Eigenfaces are a name given for the eigenvectors which are used to characterise the features of a face. These features are projected onto the feature space. Using Euclidean distances in the feature space, classification can be performed to correctly identify individuals. An implementation of this mechanism was developed for the Pico unlocking scheme prototype.

% Keystroke analysis
An unconventional authenticating mechanism is presented by Clarke and Furnell \cite{clarke2007authenticating}. They use keystroke analysis in order to make predictions regarding the user of the phone. This mechanism is unobtrusive and authenticates users during normal interactions such as typing a text message or a phone number. It is based on a neural network classifier, reporting an EER of $12.8\%$. Input data used for classification is composed out of timings between successive keystrokes, and the hold time of a pressed key. 






