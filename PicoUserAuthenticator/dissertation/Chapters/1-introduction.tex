
\chapter{Introduction} % Main chapter title
% TODO: can have ECG sensor file:///C:/Users/cristi/Downloads/fyp-papers/sensors-11-06799.pdf A Comprehensive Ubiquitous Healthcare Solution on an Android™ Mobile Device 

\label{Chapter1}

\lhead{Chapter 1. \emph{Introduction}} % Change X to a consecutive number; this is for the header on each page - perhaps a shortened title

% introduction, passwords are widely used but not that great
Passwords are the most widely used electronic authentication mechanism. They rely on reproducing a secret sequence of characters. This originally offered a sufficiently secure authentication mechanism. However, the fundamental concept of remembering a secret makes passwords unsuited for the current technological context. 

% technological problems with passwords
As Robert Morris \cite{morris1979password} emphasises in his paper, there is a constant competition between attackers and security experts. The majority of users try to maximise the usability of passwords by choosing secrets that are easy to remember. This makes the mechanism more vulnerable to brute force attacks (e.g.. dictionary, pre-compiled hashes, rainbow tables \cite{oechslin2003making}). In the past, security experts were able to slow down attacks without any impact to the user. However, with a constant increase in computational power, passwords became easier to breach. 

% technological response
One flaw of passwords is that when chosen freely they tend to be short and predictable. In order to maintain acceptable security guarantees, a number of creation requirements are enforced. Examples include having a minimum length, one or more numeric characters, and one or more special characters. Security experts recommend that each account needs to have an unique password. Furthermore, passwords sometimes require to be changed periodically with something not too similar with the original. 

% the availability downside
As shown by Yan et al \cite{yan2004password}, users choose weak passwords if not given any advice to make them memorable. Theoretically, additional restrictions make the mechanism secure. Having non-intuitive passwords that fully utilise the available character set makes brute force attacks harder to perform. In practice however, users need to memorise numerous, unique, and complex passwords. As shown by Adams \& Sasse \cite{adams1999users} maintaining all restrictions and security advices proves not to be feasible. This often leads to poor practices such as writing the passwords on paper.

% sumarising problem with passwords
The main problem with passwords is the basic principle of users remembering a secret. If the secret is memorable, than an attacker may brute force it with more ease. If it is too complex, then the user may not remember it. Furthermore, since reusing passwords is not safe, and given the memory capacity people have, the mechanism is also not scalable. For these reasons, passwords prove not to be a reliable solution for the future and even present.

% alternative to passwords
A large number of alternatives to passwords are available. However, as shown by Bonneau et al \cite{bonneau2012quest}, the main advantage passwords have over other authentication mechanisms are in terms of deployability and usability. A study by Clarke et al \cite{clarke2002acceptance} shows that although $81\%$ of users agree to an alternative to password based phone unlocking, the majority ignores the existence of available solutions. The main conclusion is that although passwords are not secure, the cost of replacing them and familiarising with a new authentication mechanism is still too inconvenient. 

% Pico to the rescue
The Pico project was designed by Frank Stajano \cite{stajano2011pico} with the purpose of replacing password based mechanisms. Pico is a hardware token that generates and manages user authentication credentials. It has an additional layer of security by only being usable in the presence of its owner. Therefore, a security chain is created where ``who you are'' unlocks ``a secret you have'' which is used for authentication.

% Picosiblings authentication mechanism
The current solution for unlocking Pico is by communicating with small auxiliary devices called Picosiblings \cite{stannard2012good}. They are designed to be embedded in everyday items that users can carry throughout the day (e.g. keys, necklace, rings). Each Picosibling transmits a secret sequence to Pico. When all required secrets are gathered, Pico becomes unlocked and can be used by its owner.

% Downside of Picosiblings authentication mechanism
Picosiblings are a sensible solution to unlocking Pico. However, they are purely based on proximity to the device. As presented in the original Pico paper \cite{stajano2011pico} anyone in possession of both Pico and its Picosiblings can have full access to the owner's accounts for a limited amount of time. This risk is lowered by additional security features. However, the main vulnerability of Picosiblings is that they do not reflect who the user is, but additional things the user has.

% The scope of my project
The purpose of this dissertation is to design and prototype a better token unlocking mechanism for Pico. According to its design, the process should be memoryless, and enable continuous authentication. The token should lock and unlock automatically only in the presence of its owner. The solutions that seem to best fit these requirements are biometric authentication mechanisms. Therefore, we have explored the possibility of combining multiple biometrics and behavioural analysis as part of an unified solution. The output from each mechanism is combined to generate an overall confidence level, reflecting that the owner is still in possession of the Pico.

% Contributions
A number of contributions have been made throughout this dissertation project. The following list presents a summary of these achievements, with further details in the following chapters.
\begin{itemize}
	\item We have created a framework derived from the work by Bonneau et al \cite{bonneau2012quest}. This is used to evaluate a couple of existing token unlocking mechanisms, including Picosiblings. The results are used as a benchmark when evaluating the proposed solution.
	
	\item We have designed a new token unlocking mechanism. The solution may be used in any type of user authentication, but it is presented in the context of unlocking the Pico token. 
	
	\item We have developed an Android prototype. The purpose of the implementation is to prove that the design can be developed using existing hardware. The prototype was not created for performance purposes. However, power analysis as well as timings of different authentication stages were recorded. These should serve as an approximation of the limitations and downsides of the scheme.
	
	\item The scheme is analysed by using the token unlocking evaluation framework. A comparison is made with Picosiblings in order to identify performance differences. We aimed for the proposed scheme to achieve better results in at least some categories of the token unlocking framework.
		
	\item We have analysed and determined the impact of the proposed token unlocking mechanism on the Pico. The analysis is performed using the framework by Bonneau et al \cite{bonneau2012quest}. One of the goals when designing the solution was to make Pico better in terms of at least one property.
	
\end{itemize}	
