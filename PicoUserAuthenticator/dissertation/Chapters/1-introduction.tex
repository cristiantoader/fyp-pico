
\chapter{Introduction} % Main chapter title
% TODO: can have ECG sensor file:///C:/Users/cristi/Downloads/fyp-papers/sensors-11-06799.pdf A Comprehensive Ubiquitous Healthcare Solution on an Android™ Mobile Device 

\label{Chapter1}

\lhead{Chapter 1. \emph{Introduction}} % Change X to a consecutive number; this is for the header on each page - perhaps a shortened title

% introduction, passwords are widely used but not that great
Passwords are currently the most widely used electronic authentication mechanism. They rely on remembering a secret sequence of characters, and providing it as input for the authentication process. This originally offered a sufficiently secure authentication mechanism. However, as shown by Adams \& Sasse \cite{adams1999users}, the fundamental concept of remembering a secret, makes passwords unsuited for the current technological context. 

% technological problems with passwords
As Robert Morris \cite{morris1979password} emphasises in his paper, there is a constant competition between attackers and security experts. The majority of users try to maximise usability of their passwords by choosing secrets which are easy to remember. These however are susceptible to a number of attacks such as brute force, dictionary, pre-compiled hashes, rainbow tables \cite{oechslin2003making}, and others. Security experts were able to slow down the attackers in the past without any impact to the user, but with a constant increase in computational power, passwords became easier to breach. 

% technological response
The main flaw of passwords is that when chosen freely they tend to be short and predictable. In order to maintain acceptable security guarantees, a number of requirements in their creation started to be enforced. Some password mechanisms may require a minimum length, one or more numeric characters, and one or more special characters. Security experts recommend that each account needs to have an unique password. Furthermore, passwords sometimes require to be changed on a regular basis with something not too similar with the original. 

% the availability downside
As shown by Yan et al \cite{yan2004password}, without any additional advice to make the password more memorable, users choose weaker passwords. From a theoretical perspective, additional restrictions would make the mechanism more secure. The users are forced to pick a non-intuitive password, and fully utilise the available character set. This makes dictionary and brute force attacks harder to perform. In practice however, users need to memorise numerous, unique, and complex passwords. As shown by Adams \& Sasse \cite{adams1999users} maintaining all restrictions and security advices proves not to be feasible, leading to poor practices such as writing the passwords down.

% sumarising problem with passwords
The main problem with passwords is the basic principle of users remembering a secret. If the secret is memorable, than an attacker may brute force it with more ease. If it is too complex, then the user may not remember it. Furthermore, since reusing passwords is not safe and given the memory capacity people have, the solution is not scalable. For all these fundamental reasons, passwords prove not to be a reliable solution for the future and even present.

% alternative to passwords
A large number of alternative to passwords are available. However, as shown by Bonneau et al \cite{bonneau2012quest}, the main advantage passwords have over other authentication mechanisms are in terms of deployability and usability. A study by Clarke et al \cite{clarke2002acceptance} shows that although 81\% of users agree to an alternative to password based phone unlocking, the majority ignore the existence of available solutions. The main conclusion we may draw is that although passwords are not secure, the cost of replacing them and familiarising with a new authentication mechanism is still too inconvenient. 

% Pico to the rescue
The Pico project was designed by Frank Stajano \cite{stajano2011pico} with the purpose of replacing password based mechanisms. Pico is a hardware token which generates and manages user authentication credentials. This transforms the problem of knowing a secret into having it. Since anyone in possession of such a hardware token would have access to the owner's accounts, this type of authentication is not very secure. Therefore, Pico adds an additional layer of security by being usable only in the presence of its owner. In a sense a security chain is created where ``who you are'' unlocks ``a secret you have'' which is used for authentication.

% Picosiblings authentication mechanism
In order to identify the presence of its owner, Pico communicates with small devices called Picosiblings \cite{stannard2012good}. These devices are embedded in everyday items that the user carries throughout the day (i.e. keys, necklace, rings). Each Picosibling transmits a secret sequence to the Pico. When all required secrets are gathered, Pico becomes unlocked and may be used by its owner.

% Downside of Picosiblings authentication mechanism
Picosiblings are a sensible solution to unlocking Pico. However, they are purely based on proximity to the device. As suggested in the original Pico paper \cite{stajano2011pico} anyone in possession of both Pico and the Picosiblings would have full access to the owner's accounts for a limited amount of time. Some additional security features are included, such as having a remote online server as a Picosibling. However, the main downside of this approach is the fact that Picosiblings do not reflect who the user is, but rather additional things the user has.

% The scope of my project
The purpose of this project is to design and implement a better token unlocking mechanism for Pico. According to its design, the process should be memoryless, and enable continuous authentication. The token should lock and unlock automatically only in the presence of its owner. The solutions that seem to best fit these requirements are biometric authentication mechanisms. For the purpose of this project we have explored the possibility of combining multiple biometrics and behavioural analysis as part of an unified solution. The output from each mechanism is combined to generate an overall confidence level, reflecting that the owner is still in possession of the Pico.

We will explore and evaluate the original Picosiblings solution as well as other token unlocking schemes. The evaluation will be performed using a framework derived from the work of Bonneau et al \cite{bonneau2012quest}. This will enable a formal analysis of the benefits and downsides of the new authentication scheme in comparison with existing mechanisms. 

\section*{Contribution}
In the process of designing and developing a new token unlocking mechanism, a number of contributions have been made. The following list presents a summary of these achievements, with further details in the following chapters.

\begin{itemize}
	\item We create a framework derived from the work by Bonneau et al \cite{bonneau2012quest}. This is used to evaluate a few existing token unlocking mechanisms, including Picosiblings. The data is then used as a benchmark when evaluating the proposed solution.
	
	\item We design a new token unlocking mechanism. Although the solution may be used in any type of user authentication, it is presented in the context of unlocking the Pico token. The design is analysed using the token unlocking evaluation framework. A comparison is made with the original Picosiblings solution. The aim of the dissertation is for the new scheme to achieve better results in at least some categories of the token unlocking framework.
	
	\item We develop an Android prototype. The implementation is meant to prove that the design is feasible for implementation using existing technologies. The prototype was not developed for performance purposes. However, power analyses as well as timings of different stages of the scheme were recorded to serve as an approximation of the limitations and downsides of the scheme.
	
	\item We analyse and determine the impact of the proposed token unlocking mechanism on the Pico. The analysis is performed based on the original framework by Bonneau et al \cite{bonneau2012quest}. One of the proposed goals when designing the solution was to make Pico better in terms of at least one property.
	
\end{itemize}	
