
\chapter{Introduction} % Main chapter title
% TODO: can have ECG sensor file:///C:/Users/cristi/Downloads/fyp-papers/sensors-11-06799.pdf A Comprehensive Ubiquitous Healthcare Solution on an Android™ Mobile Device 

\label{Chapter1}

\lhead{Chapter 1. \emph{Introduction}} % Change X to a consecutive number; this is for the header on each page - perhaps a shortened title

% introduction, passwords are widely used but not that great
Passwords are the most widely used electronic authentication mechanism. They are a secret sequence of characters used for proving the identity of the user, in order to gain access to a resource. This originally offered a sufficiently secure authentication mechanism. However, their poor scalability makes them unsuitable in the current technological context.

% problem with passwords
The main problem passwords have is the fundamental concept of remembering a secret. According to Yan et al \cite{yan2004password}, users choose weak passwords if not given any advice to make them memorable. This makes the mechanism more vulnerable to brute force attacks (e.g. dictionary, pre-compiled hashes, rainbow tables \cite{oechslin2003making}). As Robert Morris \cite{morris1979password} emphasises in his paper, there is a constant competition between attackers and security experts. With a constant increase in computational power, additional enforcements were needed (i.e. minimum password length, one or more numeric characters, one or more special characters, uniqueness across different accounts) in order to maintain an acceptable security level. As shown by Adams \& Sasse \cite{adams1999users}, this solution proves not to be feasible, and leads users to poor security practices in order to maximise usability.

% Pico to the rescue
The Pico project was designed by Frank Stajano \cite{stajano2011pico} with the purpose of replacing password based mechanisms. Pico is a hardware token that generates and manages user authentication credentials. It has an additional layer of security by only being usable in the presence of its owner. Therefore, a security chain is created where ``who you are'' unlocks ``a secret you have'' which is used for authentication.

% Picosiblings authentication mechanism
The current solution for unlocking Pico is by communicating with small auxiliary devices called Picosiblings \cite{stannard2012good}. They are designed to be embedded in everyday items that users can carry throughout the day (e.g. keys, necklace, rings). Each Picosibling transmits a secret sequence to Pico. When all required secrets are gathered, Pico becomes unlocked and can be used by its owner.

% Downside of Picosiblings authentication mechanism
Picosiblings are a sensible solution to unlocking Pico. However, they are purely based on proximity to the device. As presented in the original Pico paper \cite{stajano2011pico} anyone in possession of both Pico and its Picosiblings can have full access to the owner's accounts for a limited amount of time. This risk is lowered by additional security features. However, the main vulnerability of Picosiblings is that they do not reflect who the user is, but additional things the user has.

% The scope of my project
The purpose of this dissertation is to design and prototype a better token unlocking mechanism for Pico. According to its design, the process should be memoryless, and enable continuous authentication. The token should lock and unlock automatically only in the presence of its owner. The solutions that seem to best fit these requirements are biometric authentication mechanisms. Therefore, we have explored the possibility of combining multiple biometrics and behavioural analysis as part of an unified solution. The output from each mechanism is combined to generate an overall confidence level, reflecting that the owner is still in possession of the Pico.

% Contributions
A number of contributions have been made throughout this dissertation project. The following list presents a summary of these achievements, with further details in the following chapters.
\begin{itemize}
	\item For creating a new unlocking mechanism, we have identified a list of requirements by analysing how Stajano \cite{stajano2011pico} designed the original Pico token.

	\item In order to have an evaluation platform for the solution, we have created an assessment framework derived from the work by Bonneau et al \cite{bonneau2012quest}. This is used to evaluate a couple of existing token unlocking mechanisms, including Picosiblings. The results are used as a benchmark when evaluating the proposed solution.
	
	\item We have designed a new token unlocking mechanism. The solution may be used in any type of user authentication, but it is presented in the context of unlocking the Pico token. 
	
	\item We have developed an Android application prototype. The purpose of the implementation is to check that the design can be developed using existing hardware. 
	
	\item We have analysed the prototype's power consumption as well as timings of different authentication stages. These results should reveal any limitations and downsides of the scheme.
	
	\item The scheme is evaluated using the token unlocking evaluation framework, and the UDS framework developed by Bonneau et al. A comparison is made with Picosiblings in order to identify performance differences. We aimed for the proposed scheme to achieve better results in at least some categories of the token unlocking framework.
	
\end{itemize}	
