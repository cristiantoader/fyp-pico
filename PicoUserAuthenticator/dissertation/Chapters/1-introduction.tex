
\chapter{Introduction} % Main chapter title
% TODO: can have ECG sensor file:///C:/Users/cristi/Downloads/fyp-papers/sensors-11-06799.pdf A Comprehensive Ubiquitous Healthcare Solution on an Android™ Mobile Device 

\label{Chapter1}

\lhead{Chapter 1. \emph{Introduction}} % Change X to a consecutive number; this is for the header on each page - perhaps a shortened title

% introduction, passwords are widely used but not that great
Passwords are currently the most widely used electronic authentication mechanism. They rely on remembering a secret sequence of characters, and providing it as input for the authentication process. This originally provided a sufficiently secure authentication mechanism. However, as shown by Adams \& Sasse \cite{adams1999users}, the fundamental concept of remembering a secret, makes passwords unsuited for the current technological context. 

% technological problems with passwords
As Robert Morris \cite{morris1979password} emphasises in his paper, there is a constant competition between attackers and security experts. The majority of users try to maximise usability of their passwords by choosing secrets which are easy to remember. These however are susceptible to a number of attacks such as brute forcing, dictionary attacks, pre-compiled hashes, rainbow tables, and others. Security experts were able to slow down the attackers, but with an increasing computational power, passwords become easier to breach. 

% technological response
In order to maintain acceptable security guarantees, a number of requirements started to be enforced when choosing a password. Some authentication mechanisms require a minimum number of characters, one or more numeric characters, or one or more special characters. Security experts recommend that each account needs to have an unique password. Furthermore, passwords sometimes require to be changed on a regular basis with something not too similar with the original. 

% TODO: the availability downside!
From a computational perspective the additional restrictions make passwords secure. In practice however, users need to memorise numerous, unique, and complex passwords. Adams \& Sasse \cite{adams1999users} analyse the impact of enforcing password restrictions on the user experience. Users tend to maximise the usability of the password in exchange for security. If not enforced, users tend to ignore recommendations or even worse, they write the passwords down and compromise security in exchange for usability.

% Pico to the rescue
The Pico project \cite{stajano2011pico} by Stajano was designed to make passwords obsolete. It is a hardware token which keeps and generates user authentication credentials. This unburdens the user from the dreaded ``something you know'' transforming it into ``something you have''. This makes the solution scalable with the number of accounts, as well as secure, since the user may no longer choose ``weak'' passwords. 

What differentiates Pico from other token based authentication mechanisms is its locking and unlocking mechanism. Without introducing the need for a PIN or any other known secret, Pico only becomes available in the presence of its owner. This adds security to the threat model when the device is not in the possession of its rightful owner. Currently Pico relies on the concept of Picosiblings \cite{stannard2012good} which are small in the presence of which Pico becomes unlocked.

Although Picosiblings are a sensible solution to unlocking Pico, they purely based on proximity to the device. The purpose of this project is to create a new way of authenticating the user to the Pico device. Pico is defined as a device which unlocks and locks automatically only in the presence of the owner. This is a relatively open concept. How would the device detect presence? The following chapters will reveal the proposed solution.

\section{Contribution}
The project work is strongly related to the Pico project designed by Stajano \cite{stajano2011pico}. The following contributions have been made in order to provide an alternative to the Picosiblings user authentication:

\begin{itemize}
	\item We created a framework for assessing token based authentication mechanisms.
	\item We designed a new way of authenticating the Pico to the user and compare the proposed solution with alternatives.
	\item We developed an Android prototype which functions as a platform which allows further contributions.
	\item We analyse the Pico claims according to the framework developed by Bonneau et al \cite{bonneau2012quest} and determine the impact of the solution on the Pico.
\end{itemize}	

\section{Prerequisites}
Although the focus of the project is creating an alternative concept for the Picosiblings solution, the prototype was developed using the Android Java SDK. Additional knowledge regarding biometric authentication mechanisms and signal processing would also be useful but implementation details are outside the scope of this project.