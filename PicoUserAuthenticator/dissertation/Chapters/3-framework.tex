
\chapter{Assessment framework}

\label{Chapter5}

\lhead{Chapter 5. \emph{Assessment framework}} % Change X to a consecutive number; this is for the header on each page - perhaps a shortened title

The purpose of this chapter is to create an assessment framework for token unlocking mechanisms. This framework will be used to evaluate existing solutions, including the Picosiblings scheme used by Pico. The analysis of the results can then be used to create an alternative solution to Picosiblings. The project aims to achieve better results in some categories, without necessarily completely outperforming it.


%----------------------------------------------------------------------------------------
%	SECTION 1
%----------------------------------------------------------------------------------------

% TODO:
%	- list the framework and description?
%   - (IMPORTANT) can the user disclose it by mistake? FOR FRAMEWORK!
\section{UDS assessment framework}

% Introduction on the existing paper and why it cannot be used as it is
Similar work to what we are trying to achieve in this chapter was performed by Bonneau et al \cite{bonneau2012quest}. The authors create a framework for evaluating web based authentication mechanisms. However, the assessment scheme is not entirely compatible for token unlocking mechanisms. For example, properties such as ``Browser-compatible'' do not apply, while others need to be redefined to fit our context.  The paper however presents a good starting point for our token unlocking evaluation framework. The remainder of this section will present a brief summary of the paper.

% Why did they do the paper.
The motivation behind this paper is to gain insight about the difficulty of replacing passwords. An assessment framework is created, and a number of web authentication mechanisms are evaluated. It is an useful tool in identifying key properties of web based authentication schemes. The framework is intended to provide a benchmark for future proposals.

% How framework is structured.
The framework consists of 25 properties divided into three categories: usability, deployability, and security. For this reason, it is abbreviated by the authors as the ``UDS framework''. An authentication scheme is assessed by evaluating whether it offers or does not offer each property. In the case where a scheme almost offers a property, the authors mark it as quasi-offered. To simplify the framework, properties which are not applicable are marked as offered.

% Results of assessment: passwords.
Since passwords are currently the most widely used authentication mechanism, the results are predictable. Evaluating 35 replacement schemes shows that no scheme completely dominates them. Passwords satisfy all the properties in the deployability category. They score reasonably well in terms of usability, excelling in properties such as: ``nothing-to-carry'', ``efficient-to-use'', ``and easy-recovery-from-loss''. However, from a security perspective passwords don't perform as well. They only offer the ``resilience-to-theft\footnote{Not applicable to passwords}'', ``no-trusted-third-party'', ``requiring-explicit-consent'', and ``unlinkable'' properties. The full evaluation can be found within the paper itself.

% Results of assessment: biometrics.
Biometric mechanisms receive mixed scores on usability. None of them offer the ``infrequent-errors'' property, due to false negative precision. More importantly if biometric data becomes compromised, the possibility of replay attacks makes the authentication mechanism unreliable. They score poorly in deployability partially because they require additional hardware. In terms of security they perform worse than passwords. Replay attacks can be used by an attacker using a pre-recording of the user. Due to the ease with which valid sample data may be gathered by an attacker, they are also not resilient to theft. There is a one to one correlation between the owner and their biometric recording, therefore the ``unlinkable'' property is not offered by these mechanisms. 

% Memory-effortless vs Nothing-to-carry
By analysing the framework results, we see that some authentication schemes, such as security tokens, offer ``memory-effortless'' in exchange for the ``nothing-to-carry'' property.  The only schemes that offer both are biometric mechanisms. This is a consequence of repacing ``something you know'' with ``something you are'' instead of have. For different reasons no mechanism offers both ``memory-effortless'' and be ``resilient-to-theft''.

% Scores and no ranking
When trying to compute an aggregate score using the framework, not all properties should be equal in importance. Different properties should have different weights depending on the purpose of the assessment. For example, if we would try to find the most secure authentication mechanisms, security properties would have a larger weight in the overall evaluation. For this reason, the authors only provide the means for others to make an evaluation based on their needs. No aggregate scores or rankings are provided in the paper. 

% Combining schemes
The authors mention the possibility of combining schemes as part of a two factor authentication. In terms of deployability and usability, the overall scheme offers a property if it is offered by both authentication mechanisms. In terms of security, only one of the two mechanisms needs to offer the property in order for the two factor combination to offer it as well. However, Wimberly \& Liebrock \cite{wimberly2011using} observe that combining passwords with a second authentication mechanism scheme leads to weaker credentials.

% Further details can be found in the paper.
The following section will offer more details on the UDS framework properties which also apply to token unlocking. Further information about the framework are not mentioned in this dissertation for the purpose of brevity.  The full list of properties and the evaluation of a number of mechanisms are described in the original paper by Bonneau et al. 

\section{Token unlocking framework}
% Introduction as to why we use properties from the UDS framework
Unlike web based authentication mechanisms, token unlocking records and processes data locally. For this reason, a subset of the UDS framework properties are also be present in the framework we have developed. Those which do not apply, or would apply to every token unlocking mechanism were removed. Some properties needed to be adapted to the new context of a token, and therefore will have a slightly different meaning.

% List each property, a description, and an example
The following list contains all properties of the UDS framework developed by Bonneau et al \cite{bonneau2012quest} which we adapted and included in the token unlocking framework. A short description is included for the cases when the property is offered or quasi-offered, as well as a small example.
\begin{description}
  
  %
  %	 Accessibility
  %
  \item[Memorywise-effortless] \hfill \\
  Users do not need to remember any type of secret, whether it is a password, physical signature, or drawing. The property was originally quasi-offered in the cases where one secret would be used with multiple accounts, but in the case of tokens this does not apply. As an example the RSA SecurID \footnote{http://www.emc.com/domains/rsa/index.htm?id=1156} is used in conjunction with a password in order to authenticate the user, and therefore does not offer this property.
  
  \item[Nothing-to-carry] \hfill \\
  The unlocking mechanism does not require any additional hardware except for the token. The property is quasi-satisfied in the case of hardware the user would have carried on a normal basis such as a mobile phone. An example of such a mechanism would be the Picosiblings scheme which uses small devices embedded in everyday items. Biometric mechanisms that require additional sensors such as a fingerprint reader do not satisfy this property. 
  
  \item[Easy-to-learn] \hfill \\
  Users who use the unlocking mechanism would be able to learn with ease. The original paper by Bonneau et al \cite{bonneau2012quest}, assess Pico as not easy to learn due to the complexity of the Picosiblings management\footnote{As discussed in the previous chapter \ref{Chapter2}, each Picosibling contains a k-out-of-n secret used to reconstruct the ``Pico Master Key''. The user therefore needs to choose the right combination of Picosiblings in order to unlock the Pico, which may prove difficult}. PINs or passwords however satisfy this property due to the users' familiarity with this type of authentication.
  
  \item[Efficient-to-use] \hfill \\
  The amount of time the user needs to wait for the token to be unlocked is reasonably short. It is important to note that this includes the time required to provide the input for the mechanism. The same applies for setting up the token unlocking mechanism, but with a larger time scale. In the case of PINs for example the input and processing time are very low, and therefore the scheme offers the property. Mechanisms based on biometrics however may not, depending on the implementation.
  
  \item[Infrequent-errors] \hfill \\
  The rightful owner should generally be able to successfully authenticate to the token. Any sort of delays resulted from the unlocking mechanism such as typos during typing or biometric false negatives may contribute to the mechanism's inability to provide this property. As an example, PINs have a limited input length and character set size which makes infrequent errors unlikely and therefore offer the property. Biometric mechanisms, based on the type and implementation may quasi-offer the property, although they generally do not.
    
  \item[Easy-recovery-from-loss] \hfill \\ 
  The meaning of this property was modified to reflect the context of token unlocking. It is offered if the user may easily recover from either the loss of auxiliary devices, or forgotten credentials used in the authentication process. As an example, forgotten PINs do not offer the property as they generally require ordering a new SIM.
  
  %
  %	 Deployability
  %
  \item[Accessible] \hfill \\
  The mechanism is usable by any user regardless of any disabilities or physical conditions. In the original paper, passwords are offered as an example of a scheme which offers this property. A gait recognition unlocking scheme would not offer this property. 
  
  \item[Negligible-cost-per-user] \hfill \\
  The total cost per user of using the scheme, enquired by both the user and the verifier, is negligible.
  
  \item[Mature] \hfill \\
  A large number of users have successfully used the scheme. Any open source projects involving the mechanism, as well as any participation not involving its creators contribute to this property. For example, passwords are widely used and implemented and therefore offer this property.
  
  \item[Non-proprietary] \hfill \\
  Anyone can implement the token unlocking scheme without having to make any sort of payment such as royalties. The technologies involved in the scheme are publicly known and do not rely on any sort of secret.
  
  %
  %	 Security
  %
  \item[Resilient-to-physical-observation] \hfill \\
  An attacker would not be able to impersonate the owner of the token after observing him authenticate. Based on the number of observations required for the attacker to unlock the token, the scheme may quasi-offer the property. The original paper suggests 10-20 times to be sufficient, although it is just an approximation. Physical observation attacks are not restricted to shoulder surfing, and may include video cameras, keystroke sounds, or thermal imaging of the PIN pad.
  
  \item[Resilient-to-targeted-impersonation] \hfill \\
  An attacker should not be able to impersonate the owner of the token by exploiting knowledge of personal details. This may include birthday, full name, family details, and other sensitive information. The scheme should also be resilient to pre-recordings of biometric information which may then be replayed to the authenticator.
  
  \item[Resilient-to-throttled-guessing] \hfill \\
  The scheme is resilient to attacks with a guessing rate restricted by the mechanism. The process cannot be automated due to the lack of physical access to authentication data. This may be achieved using tamper resistant memory. As an example, PINs offer this property because SIM cards become locked after only three unsuccessful attempts.
  
  \item[Resilient-to-unthrottled-guessing] \hfill \\
  The scheme is resilient to attacks with a guessing rate unrestricted by the mechanism. Even though the guessing process is only restricted by the attacker's computational power, the scheme would still not be bypassed within reasonable time. The original paper claims that if the attacker may process $2^40$ to $2^64$ guesses per account, would only be able to compromise less than $1\%$ of accounts. Since are generally designed to have one owner, the original description will be adapted to a single account. Therefore the property is granted only if an attacker requires more than $2^40$ attempts.
  
  \item[Resilient-to-theft] \hfill \\
  The property applies to schemes which use additional hardware other than the token. If the additional hardware becomes in the possession of an attacker, it is not sufficient to unlock the token. For example, auxiliary biometric devices used in the conjunction with the token offer this property. In this case the token would still not be unlocked using the hardware alone. An additional replay attack would be necessary. Picosiblings however only quasi-offers the property. They do not validate the user, relying only on proximity to the Pico. One of the special shares however allows the owner to lock the Pico remotely.
  

  \item[Unlinkable] \hfill \\
  Using the authentication input with any verifier using the same authentication mechanism\footnote{The authentication mechanism is not necessarily used for token unlocking. Any sort of mechanism which requires user authentication is a valid option.} does not compromise the identity of the token owner. As an example the link between a PIN and its owner is not strong enough to make a clear link between the two. However, biometrics are a prime example of schemes which do not offer this property. \ldots
  
\end{description}

% Introduction for added properties
% 	TODO: this sounds a bit bad.. rephrase?
We have augmented the subset from the original UDS framework with a number of properties relevant which are relevant to Picosiblings, PINs, as well as other token unlocking mechanisms. The following list is part of the project's contributions to the overall evaluation framework.

\begin{description}
  \item[Continuous-authentication] \hfill \\
  The token unlocking scheme re-authenticates the user periodically. The process doesn't need to be hidden from the owner, but it is required to be effortless. The token should remain unlocked and usable in the presence of its owner. The scheme needs to detect when the owner is no longer in possession of the token, and lock the device accordingly. When locked, any open authentication session managed by the security token will be closed. The concept is mentioned by Bonneau et al \cite{bonneau2012quest}, but not included in the UDS framework. It is discussed in more detail by Stajano \cite{stajano2011pico} as one of the benefits of the Pico project. Using the UDS classification of the original framework, the property belongs to the Security category.
  
  \item[Multi-level-unlocking] \hfill \\
  The unlocking scheme provides quantifiable feedback, more than a locked or unlocked state. The mechanism offers the possibility of supporting multiple token security permissions. These would be granted based on the confidence level that the user trying to unlock the token is its owner. For example, a $70\%$ confidence level that the owner is present may allow the user to access an email account, but not make any sort of payments or banking transactions. Passwords only provide a ``yes'' or ``no'' answer and therefore do not offer this property. Biometric mechanisms do offer this property. Their output is either a probability or some sort of distance metric that data was collected from the owner. Different confidence levels could therefore enable different security permissions. Using the UDS classification of the original framework, the property belongs to the Security category.
  
  \item[Availability] \hfill \\
  The owner has the ability of using the scheme regardless of external factors. The ability to authenticate should not be impaired by the authentication context such as traffic noise, different light intensities, or restricted movement space. The property is not related to physical disabilities preventing the user from using the scheme but only on contextual influences on data collection. As an example gait recognition would only function while moving on foot and therefore does not offer the property. A mechanism requiring a PIN on the other hand would work in any circumstance. Using the UDS classification of the original framework, the property belongs to the Usability category.
  
\end{description}

\section{Example evaluation}
% Introduction to why we offer some examples
In order to demonstrate how the framework works we have assessed three token based authentication mechanisms: Picosiblings, PIN, and Face-unlock. Each of the mechanisms represent a different type of authentication method. Picosiblings essentially are a secret the owner has, PINs are a secret the owner knows, and Face-unlock is something the owner is. The following sections will provide as an example of how the framework should be applied. Results in the Picosiblings section will be used in the following chapter as a token authentication benchmark which will be used to compare the proposed unlocking scheme with the existing one.

	\subsection{Picosiblings}
	The current token unlocking mechanism used by Pico is called Picosiblings. In order to unlock the token, k-out-of-n secrets are required. Each of these secrets are held within a Picosibling carried by the user. If the Picosibling is within range of the Pico, these secrets are communicated to the token which can then be used to unlock its master key and implicitly the token.
	
	Since the user doesn't need to remember any secrets, the scheme satisfies the memory-effortless property. The nothing-to-carry property is not satisfied since the Picosiblings are external emitters. The easy-to-learn property doesn't quite apply to the authentication mechanism, given that the user only has to carry the Picosiblings. In the original paper this was marked as not satisfied due to the unfamiliarity of the user with the Pico device, not the authentication mechanism. It only partially satisfies the efficient-to-use, and infrequent-errors properties. Easy recovery from loss is not offered neither for the Pico or the Picosiblings, since additional hardware would need to be ordered. The scheme relies fundamentally on the presence of the Picosiblings, but may be used in any normal scenarios as long as the devices are present. Therefore it satisfies the availability property.
	
	The original paper marks Pico with the Picosiblings scheme as not accessible. We will assume the property transfers to the unlocking mechanism as well. The negligible-cost-per-user property is not satisfied due to presumably expensive embedding of Picosiblings in everyday items. The scheme was not used in any open source projects and is at the stage of a prototype, having very little user testing. For this reason the scheme does not satisfy the mature property. Since this is an academic research project, no royalties need to be paid in order to implement the scheme. Therefore the scheme is non-proprietary.
	
	Since it does not rely on any user input it is resilient-to-physical-observations. In disagreement with the original Pico assessment by the authors, since anyone in the possession of the Picosiblings may unlock the Pico it is not resilient-to-targeted-impersonation. It is however resilient-to-throttled-guessing and resilient-to-unthrottled-guessing due to the communication protocol described by Stajano in his paper \cite{stajano2011pico} as well as the resurrecting ducklings protocol \cite{stajano2000resurrecting}. Due to the use of cryptography is is also resilient-to-internal-observations, and if lost the data is unlinkable to its owner. The scheme was designed to provide continuous-authentication. Since the mechanism offers secrets which keep the master key as either locked or unlocked, it does not satisfy the multi-level-unlocking property.

	\subsection{PIN}
	PIN should be viewed as passwords with a severely limited subset of characters. Additional protection comes from steep security measures when the PIN authentication has failed. As an example, for mobile phones typing 3 wrong PINs would lock your SIM card. A lot of the PIN properties should however be similar with those offered by passwords.
	
	Since they scheme relies on knowing a secret, it does not satisfy the memory-effortless property, in exchange for the nothing-to-carry property. Due to its maturity and similarity with passwords users find it easy-to-learn. It is efficient-to-use since it requires no time to compare two small strings. Due to the input method on a small mobile device it does not have the infrequent-errors property. As a divergence from the original paper, we consider PINs not to satisfy the easy-recovery-from-loss property. This is due to the fact that usually it is required to order a new SIM with the same number in order to ``reset'' a forgotten PIN. Due to the fact that it only requires the ability to be typed, it satisfies the availability property in any conditions.
	
	Just as passwords PINs score all points in deployability. They can be used by anyone and are therefore accessible. The have virtually no cost, satisfying the negligible-cost-per-user property. Being a subset of passwords, the mechanism is considered to be mature and non-proprietary.
	
	From a security perspective however PINs score poorly. They are not resilient-to-physical-observation since anyone can eavesdrop the input of a PIN. The resilient-to-targeted-impersonation property is strongly linked to this, but only works if the attacker inputs the correct PIN. The property is therefore only quasi-satisfied.	Unlike passwords, PINs do have the property of resilient-to-throttled-guessing property due to the security model around them. A brute force attack cannot be performed on a locked device unless some sort of implementation error is present, such as an unprotected file. Furthermore PIN mechanisms are generally designed with different penalties such as locking the SIM or the OS for a time period. Due to poor practices in using PINs, they do not have the resilient-to-unthrottled-guessing property. PINs are not resilient-to-internal-observations. The fact that they are chosen randomly by the user means they are unlinkable. However, they do not offer the continuous authentication property due to the fact that they cannot authenticate the user in a non-evasive way. They only offer a locked/unlocked state, and therefore they do not have the multi-level-unlocking property. 
	
	\subsection{Android Face unlock}
	Android offers a face recognition unlock mechanism for the mobile device. A mobile phone may be viewed as a multi-purpose token, making the face-unlocking mechanism a valid token unlocking scheme. This feature was made available starting with Android 4.0 (Ice Cream Sandwich). This essentially is a biometric unlocking mechanism, which due to the sensor platform offered by Android may be used for token unlocking purposes.
	
	As no secrets are required for biometric mechanisms, the scheme is memory-effortless. It satisfies the nothing-to-carry property as well, since in the case of mobile devices with a camera the scheme does not require additional external hardware. The mechanism is efficient-to-learn, since it only needs the user to look at the camera. Time is required to load face models when first loading the mechanism, but not upon further authentication requests. Therefore the scheme quasi-satisfies the efficient-to-use property. Errors are quite frequent for this mechanism, both from personal testing as well as existing literature \cite{stajano2000resurrecting}. The easy-recovery-from-loss does not apply for biometric mechanisms, especially in the case where no external hardware is required. The availability property is not satisfied due to the fact that based on external factors such as light or obstacles between the camera and the face authentication is not possible.
	
	
	To do: please provide some input for cost-per-user, mature, and non-proprietary. Face recognition is accessible for anyone to use. It has a negligible cost-per-user due to the fact that it is non-proprietary and no charges are usually made with each authentication. The mechanism quasi-satisfies the mature property due to the limited user exposure.
	
	Purely by observing the owner authenticate using face recognition does not provide any advantage to an attacker. The scheme therefore has the resilient-to-physical observations property. Targeted impersonation is an issue with any authentication mechanism. Nothing would stop an attacker from taking a picture of the owner and use it in a replay attack. Therefore the scheme does not offer the resilient-to-targeted-impersonation property. The resilient-to-throttled-guessing and resilient-to-unthrottled-guessing properties do not apply and therefore are considered as satisfied.  The resilient-to-internal-observations property is not satisfied in the case of mobile devices. This is due to the fact that malware may capture the exact same data when the screen is turned on for example, or immediately after a picture was taken. If data used in authentication is exposed, it is directly linkable to the owner, and therefore the unlinkable property is not satisfied. Continuous-authentication would only possibly be offered based on implementation and conditions such as the user facing the camera of the phone. The property is therefore only quasi-satisfied, although arguably not satisfied. Multi-level-unlocking would be possible based on an Euclidean distance metric, but no efforts have been made in this regard. 

\section{Conclusions}
At the end of this peace of work a new framework for the evaluation of token unlocking mechanisms was developed. Existing properties have been identified from the literature and added to the framework together with original work. An initial evaluation was made for existing token unlocking mechanisms which will serve as a benchmark for the proposed solution. From the example evaluations, neither of the three token unlocking mechanisms is dominant over the others.

