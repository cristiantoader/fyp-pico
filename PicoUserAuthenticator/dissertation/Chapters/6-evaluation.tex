% Chapter Template
% TODO: do the evaluation!

\chapter{Evaluation} % Main chapter title

\label{Chapter6} % Change X to a consecutive number; for referencing this chapter elsewhere, use \ref{ChapterX}

\lhead{Chapter 6. \emph{Evaluation}} % Change X to a consecutive number; this is for the header on each page - perhaps a shortened title

% Brief summary of this chapter
This chapter presents an evaluation of the proposed token unlocking mechanism. We start by performing a threat model of the Android prototype. This should reveal any security limitations of the implementation, as well as the overall scheme. We continue by analysing the performance of the prototype and discuss how it can be improved.

Given a well defined implementation (\ref{Chapter5}), we assess the scheme using the token unlocking framework (\ref{tokenframework}), and compare the results with the Picosiblings solution. In order to check for overall improvements, we use the UDS framework to evaluate a Pico token that uses the proposed token unlocking scheme, and compare the results with the original work by Bonneau et al \cite{bonneau2012quest}.

\section{Threat model}
% % Introduction on what we want to achieve
% The purpose of the prototype was to provide a proof of concept that the scheme can be developed using existing hardware. However, we will perform a threat analysis in order to ensure that the implementation is secure. The main reason for this is the possibility of gaining additional insight to the scheme's design limitations. Furthermore, the assessment should reveal attack paths that need to be considered in future implementations this mechanism.

% Types of attacks we will be analysing
The threat analysis is performed from an availability, integrity, and confidentiality perspective. We consider the security mechanisms of the Android platform presented in appendix \ref{AppendixA} as predefined assumptions used in this model. Attack paths are analysed in different scenarios based on whether an attacker has physical access to the token or not. Although we are making a security assessment for a prototype developed on the commercial Android platform, similar issues may arise for a future implementation that uses dedicated hardware.

\subsubsection*{Availability}
% Availability: attacker has token
Breaking the scheme's availability while the device is in the possession of the attacker is relatively trivial. The application can be uninstalled, or the application data cache can be cleared, therefore removing the owner biometric models used by the individual mechanisms. Furthermore, in this scenario the owner is no longer in possession of their Pico, so basically the device is already unavailable.

% Availability: attacker does not have token
Let us analyse what denial of service (DoS) exploits can be achieved by a remote attacker. Removing the owner configuration data from internal storage would make the authenticator unusable. This can only be achieved if the attacker (or a malware application designed by the attacker) manages to get root access on the device. Given the Linux permissions model, there would be no way to protect this data from deletion. However, without root access the application data cannot be accessed or modified.

% Availability: dos via sensor locking
Based on each device platform, multiple apps recording data from a sensor may not be possible. This can enable a DoS attack on the prototype by having malware locking sensors before the authenticator. This would make data collection impossible, and therefore the mechanisms' weights would gradually decrease to 0. The overall confidence level would be lowered, preventing the user from authenticating. After performing experiments, we can confirm this problem for the Google Nexus 5 smart phone when two applications try to record microphone data using different sampling rates \footnote{The apps were trying to record microphone data using the AudioRecord class; application one was using a sampling rate of 44100 and application two 22050.}.

% Availability: starting multiple connections
The current prototype is susceptible to a DoS attack caused by too many clients registered with the authenticator. Given that no permission is required to register to the {\tt UAService} component, an unlimited number of connections can be made. Therefore, broadcasting the authentication status to each client may cause considerable delays. Furthermore, in order to unburden the developer from developing thread-safe code, ICC is performed on a single thread. This means that by spamming {\tt UAService} with requests, the attacker can achieve a DoS attack for legitimate Pico clients. This can be fixed if we only allow access to application developed by the same author as the authenticator app.

\subsubsection*{Integrity}
% Integrity: root data modifications
The prototype stores data in internal storage, and cannot be accessed by other applications due to the Linux permissions mechanism. However, as mentioned in the previous section, if the attacker gains root privileges it may modify any data on the device. The attack can be performed regardless of physical access.

% Integrity: Binder communication ok
From a data flow point of view, ICC is performed using the {\tt /dev/binder} node driver. According to the official Android source code \footnote{For convenience, a link to the binder driver is found here: https://android.googlesource.com/kernel/common.git/+/android-3.0/drivers/staging/android/binder.c (visited on 06.02.2014).}, although the node is readable and writeable by any application, communication is performed using IOCTL calls. Data is transferred from one component to the other without the possibility to intercept or modify. Given that Android ICC is secure, data either from the sensors or from app components cannot be tampered.

\subsubsection*{Confidentiality}
% Confidentiality: same owner
Android apps may share private resources only if developed by the same author. This is determined by verifying the signature of the app, which is performed using a private key specific to each developer. Therefore, an attack where owner configuration data is leaked due to private resource sharing would only be possible if the attacker manages to acquire the private key that was used for signing the authenticator app. We will consider this to be a scenario outside the scope of the project.

% Confidentiality
Another case where owner authentication data can be accessed is having malware run with root privileges. This would allow an attacker the rights to read any application's data. However, the owner's biometric files would not be compromised, as they are kept encrypted using the Android Keychain API. Although the data can be read from internal storage, it cannot be interpreted in a meaningful way. Starting with Android 4.3, the Keychain API has hardware support, making the encryption keys non-extractable. 

% Keychain exposure
On Android versions earlier than 4.3, the following confidentiality attack path can be performed. Given root access, the attacker may retrieve the AES master key used by the keychain manager to store credentials. Using this key they can then retrieve the authenticator's application key used for encrypting owner configuration files. By retrieving this final key, the attacker may decode sensitive data (i.e. biometric data) and leak it outside of the system, therefore compromising confidentiality.

% Solution to encrypted data problem
A solution to this problem is not keeping the key used for decryption on the device. It should be generated on the app's first run, and communicated securely back to a credentials server. Whenever the authenticator app starts, it would request the key remotely via a secure connection, use it to decrypt owner authentication files, and discard it without saving.

% Data flow confidentiality
From a data flow perspective, ICC should offer full confidentiality. As previously mentioned, the {\tt /dev/binder} device node used for ICC is managed by a driver which listens for {\tt ioctl} requests. Data cannot be compromised as it is transferred from one component to the other. Only {\tt UAService} is an {\tt exported} component that may be accessed by other apps. It only provides final authentication feedback from the mechanisms, and only exposes a limited API through the IBinder interface object used for {\tt Message} passing.

% Sensor data collection
The communication between the authenticator and Pico can be secured, preventing other applications to register for updates. This can be achieved using runtime permission label checks. Both Pico and its authenticator would need to define these permission in their manifest files. Additional checks would need to be added for ICC in {\tt UAService}.

% Sensor sniffing
Liang Cai et al \cite{cai2009defending} presents the problem of sensor sniffing. A malware application may collect all relevant data on its own from the user, using the same functionality as the prototype. This would allow for a powerful replay attack in the future. Adrienne Porter Felt et al \cite{felt2012android} show that when installing an app only $17\%$ of users pay attention to the Android permissions dialogue, and only $3\%$ understand what each permission represents.

% TODO: change this title
\subsubsection*{Design model attacks}
\label{secdesignattacks}
% Short introduction
Pico needs to be unlocked only in the presence of its owner. In order to do so, the token unlocking scheme needs to gather valid sample data. Let us consider a few scenarios and assess any design issues.
 
% Owner at desk with no data
The most unfavourable scenario is to have the owner silently sitting at work with their smart phone on the table. Voice and face recognition mechanisms would not gather any valid samples. Location data can be collected, offering some confidence that the owner is present. If authentication is required for a high security transaction, such as logging in to an online banking account, the overall score outputted by the scheme would not be high enough to grant access. The scheme would therefore make an explicit authentication request, offering the user the possibility to generate valid data.

% No owner and no data
The main problem with the scheme is not denying service to the owner, but falsely granting it. Given the same situation as before, let us assume the owner forgets the smart phone on their desk and leaves the office for a break. Just as before, only location data can be collected, providing some confidence that the owner is present. Ideally as the owner leaves, most authentication sessions managed by the Pico should be closed. This should happen once the confidence level becomes too low, and the explicit authentication mechanisms are ignored.

% Compromise solution
From the authenticator's perspective there is no difference between the two scenarios presented above. The first suggests that non biometric mechanisms should provide sufficient confidence to provide authentication when the owner ``goes silent''. The second scenario requires the opposite; when the owner can no longer provide biometric data, they are likely no longer with the token and Pico should lock. 

A compromise solution is needed for the two scenarios. Non-biometric mechanisms need to provide a confidence level that is almost sufficient to grant access to most medium-level security accounts. Periodically, explicit authentication requests will be made by the mechanism in order to provide a sufficiently high score. Given the decaying weights, the confidence level will gradually drop until another explicit authentication is required. The weights and decay rates need to be configured in such a way that the time interval between two explicit authentication requests is acceptable for the user, without compromising security. We suggest the time interval of 1 minute, but a user study would be more appropriate to determine this value.

% Alternative solution: heartbeat
An alternative solution to the problem presented above is to have an auxiliary biometric sensor that the owner would carry at all times. A good example is a heartbeat monitor. This can be embedded in an every day item such as a watch. The heartbeat authentication mechanism combined with the existing location analysis mechanism should provide a sufficiently high confidence level to unlock Pico for any medium-security authentication session.

\section{Functional evaluation}
\label{functionaleval}
We have performed a series of tests in order to assess the usability and performance of the scheme's prototype. 

Battery power is a scarce resource for hand-held devices. The amount of time they can function without recharging directly determines their availability. The token unlocking scheme we have proposed requires periodic sampling of sensor data. This can be a power consuming task, if not managed appropriately. Given that the prototype's individual mechanisms use a configurable sampling rate, we continue by analysing the power consumption for different time intervals. The analysis was performed using the Trepn profiling tool developed by Qualcomm\footnote{The tool's official website can be found at the following address: https://developer.qualcomm.com/mobile-development/increase-app-performance/trepn-profiler}. Results for the application's average CPU and Power usage are posted in table \ref{tab:powerprofile}.

\begin{table}
    \begin{tabular}{l|l|l|l}
    Test number & Sampling rate (s) & Average CPU (\%) & Average Power (mW) \\ \hline
    1           & 5                 & 10.61            & 313.17             \\
    2           & 10                & 4.68             & 279.48             \\
    3           & 15                & 4.04             & 192.61             \\
    4           & 20                & 3.27             & 81.57              \\
    5           & 25                & 2.62             & 111.93             \\
    6           & 30                & 2.34             & 72.86              \\
    \end{tabular}
	
	\caption{Application profiling results using Trepn}
	\label{tab:powerprofile}
\end{table}

Results from table \ref{tab:powerprofile} show a high CPU and Power usage for a small sampling rate of 5 seconds. This drops considerably as the sampling time interval increases. In order to have the prototype constantly functional, sampling rates need to be configured in conjunction with the decay process and initial weights of the mechanisms. This is a multivariate optimisation problem, where we are trying to minimise the power consumption and error rate of mechanism.

In order to assess the time performance of the prototype, we have recorded key stages for each mechanism in table \ref{tab:timeprofile}.
\begin{table}
    \begin{tabular}{l|l|l}
		Mechanism         & Initialisation (s) 	& Authentication (s) \\ \hline
		Voice recognition & 2.586	            & 0.897          \\
		Face recognition  & 18.134        		& 1.951          \\
		Location analysis & 0.192          		& 0.012          \\
    \end{tabular}
	\caption{Mechanisms' timing performance results.}
	\label{tab:timeprofile}
\end{table}

Results show a high initialisation time for the face recognition mechanism. This has a direct impact on the token's usability, as the owner has to wait during the first use of the unlocking mechanism. However, authentication time is relatively small with a maximum of 2 seconds. With the current periodic sampling implementation, the delay can be hidden from the user by adjusting the sampling rate of each mechanism. However, results from table \ref{tab:timeprofile} can be relevant for implementations that do not sample data at fixed time intervals.

The precision of each authentication mechanism used in the scheme is outside the scope of this project. However, we have performed a series of informal tests showing that data validation is performed accurately. The face recognition mechanism correctly identifies when a face is not present in the camera image, and background noise is correctly detected by the voice recognition mechanism.

\section{Token unlocking framework evaluation}
We will continue by evaluating the proposed scheme with the token unlocking framework defined in section \ref{tokenframework}.

% Usability
The scheme is ``memorywise-effortless'' because it doesn't require any secrets to provide authentication. The sensors used for authentication are embedded in the token, therefore offering ``nothing-to-carry''. It is also ``easy-to-learn'', as user authentication is performed non-obtrusively. As shown in the evaluation from table \ref{tab:timeprofile}, although user authentication is performed in a timely manner, setting up the scheme may take some time. Therefore, the ``efficient-to-use'' property is only quasi-offered.  The scheme only quasi-offers ``infrequent errors'' because of the underlying biometric and behavioural authentication. Any differences in biometric features that may occur can be resolved by re-configuring the authenticator. The prototype does not have a well defined secure process for this task. In the lack of additional details we mark the scheme to only quasi-offer ``easy-recovery-from-loss''. As briefly shown in section \ref{secdesignattacks}, even in unfavourable scenarios the scheme may still provide authentication to the token. The ``availability'' property is therefore satisfied.

% Deployability
Given that multiple continuous authentication mechanisms are combined, the scheme offers ``accessibility'' to any user, regardless of disabilities. Since it is implemented as an Android app, it has a ``negligible-cost-per-user'' both for the owner and the developer. It is not ``mature'' since it has only been prototyped. The ``non-proprietary'' property is offered, as long as the individual mechanisms are developed using free to use algorithms and libraries.

% Security
From a security perspective the scheme is ``resilient-to-physical-observations''. If an attacker would have valid pre-recordings of the owner for all biometric mechanisms, they would still need to perform these in a location considered safe by the authenticator. Furthermore, the replays would have to be performed periodically in order to keep the authentication session alive. Therefore, due to the difficulty to perform a replay attack, the scheme is considered to be ``resilient-to-targeted-impersonation''. The ``resilient-to-throttled-guessing'', ``resilient-to-unthrottled-guessing'', and ``resilient-to-theft'' properties do not apply. The scheme is not ``unlinkable'' because biometric data is unique for each individual. In order to satisfy the Pico requirements, all mechanism involved in the token unlocking process support ``continuous-authentication''. The authenticator provides as final feedback a confidence level, which allows for ``multi-level-unlocking''. Intentional disclosure of authentication credentials would pose the same difficulties as a replay attack, and therefore the scheme offers ``non-disclosability''.

% Summary of results
The results are summarised in table \ref{table:tokenresults}. Column properties are highlighted to facilitate the comparison with the Picosiblings solution \footnote{The colours have the following meanings based on the result: green - offered, red - not offered, and yellow - quasi offered.}.

\begin{table}
    \begin{tabular}{l|l|l|l}
    Property                            & Picosiblings  						& Proposed scheme \\ \hline
    Memorywise-effortless               & \cellcolor{green!25}Offered       	& \cellcolor{green!25}Offered          	\\
    Nothing-to-carry                    & \cellcolor{yellow!25}Quasi-offered   	& \cellcolor{green!25}Offered          	\\
    Easy-to-learn                       & \cellcolor{red!25}Not-offered   		& \cellcolor{green!25}Offered          	\\
    Efficient-to-use                    & \cellcolor{yellow!25}Quasi-offered 	& \cellcolor{yellow!25}Quasi-offered   	\\
    Infrequent-errors                   & \cellcolor{yellow!25}Quasi-offered 	& \cellcolor{yellow!25}Quasi-offered   	\\
    Easy-recovery-from-loss             & \cellcolor{red!25}Not-offered   		& \cellcolor{yellow!25}Quasi-offered   	\\
    Availability                        & \cellcolor{green!25}Offered       	& \cellcolor{green!25}Offered      	   	\\ \hline
	
    Accessible                          & \cellcolor{green!25}Offered       	& \cellcolor{green!25}Offered          	\\
    Negligible-cost-per-user            & \cellcolor{red!25}Not-offered   		& \cellcolor{green!25}Offered          	\\
    Mature                              & \cellcolor{red!25}Not-offered   		& \cellcolor{red!25}Not-offered    		\\
    Non-proprietary                     & \cellcolor{green!25}Offered       	& \cellcolor{green!25}Offered      		\\ \hline
	
    Resilient-to-physical-observations  & \cellcolor{green!25}Offered       	& \cellcolor{green!25}Offered          	\\
    Resilient-to-targeted-impersonation & \cellcolor{green!25}Offered       	& \cellcolor{green!25}Offered      		\\
    Resilient-to-throttled-guessing     & \cellcolor{green!25}Offered       	& \cellcolor{green!25}Offered          	\\
    Resilient-to-unthrottled-guessing   & \cellcolor{green!25}Offered       	& \cellcolor{green!25}Offered          	\\
    Resilient-to-theft                  & \cellcolor{yellow!25}Quasi-offered   	& \cellcolor{green!25}Offered          	\\
    Unlinkable                          & \cellcolor{green!25}Offered       	& \cellcolor{red!25}Not-offered      	\\
    Continuous-authentication           & \cellcolor{green!25}Offered       	& \cellcolor{green!25}Offered      		\\
    Multi-level-unlocking               & \cellcolor{red!25}Not-offered   		& \cellcolor{green!25}Offered      		\\
    Non-disclosability                  & \cellcolor{red!25}Not-offered   		& \cellcolor{green!25}Offered    		\\
    \end{tabular}

	\caption{Token unlocking framework results compared with Picosiblings.}
	\label{table:tokenresults}

\end{table}

% Conclusion based on comparison
The proposed solution does not completely dominate Picosiblings. This is only because the scheme is not ``unlinkable''. It performs better by offering ``nothing-to-carry'' and quasi-offering ``easy-recovery-from-loss''. The prototype also has a ``negligible-cost-per-user'', which is something Picosiblings do not aim to achieve. In terms of security it is also better by offering the ``resilient-to-theft'', ``multi-level-unlocking'', and ``non-disclosability properties. 

In conclusion, we achieve our proposed goal of providing a solution that is better than Picosiblings in at least one property.

\section{UDS framework evaluation}
We now perform the reassessment of a Pico that uses our proposed token unlocking mechanism. The evaluation is performed using the UDS framework developed by Bonneau et al \cite{bonneau2012quest}. We will compare the result with the original Pico assessment in order to check for improvements. A summary is presented in table \ref{table:udsresults}.

\begin{table}
    \begin{tabular}{c|l|l|l}
    ~             & Property                                & Picosiblings  						& Proposed scheme \\ \hline
    Usability     & Memorywise-effortless                   & \cellcolor{green!25}Offered       	& \cellcolor{green!25}Offered         \\
    ~             & Scalable-for-users                      & \cellcolor{green!25}Offered       	& \cellcolor{green!25}Offered         \\
    ~             & Nothing-to-carry                        & \cellcolor{red!25}Not-offered   		& \cellcolor{red!25}Not-offered     \\
    ~             & Physically-effortless                   & \cellcolor{green!25}Offered       	& \cellcolor{green!25}Offered         \\
    ~             & Easy-to-learn                           & \cellcolor{red!25}Not-offered   		& \cellcolor{green!25}Offered         \\
    ~             & Efficient-to-use                        & \cellcolor{yellow!25}Quasi-offered 	& \cellcolor{yellow!25}Quasi-offered   \\
    ~             & Infrequent-errors                       & \cellcolor{yellow!25}Quasi-offered 	& \cellcolor{yellow!25}Quasi-offered   \\
    ~             & Easy-recovery-from-loss                 & \cellcolor{red!25}Not-offered   		& \cellcolor{red!25}Not-offered     \\ \hline
    Deployability & Accessible                              & \cellcolor{red!25}Not-offered   		& \cellcolor{red!25}Not-offered         \\
    ~             & Negligible-cost-per-user                & \cellcolor{red!25}Not-offered   		& \cellcolor{red!25}Not-offered     \\
    ~             & Server-compatible                       & \cellcolor{red!25}Not-offered   		& \cellcolor{red!25}Not-offered     \\
    ~             & Browser-compatible                      & \cellcolor{red!25}Not-offered   		& \cellcolor{red!25}Not-offered     \\
    ~             & Mature                                  & \cellcolor{red!25}Not-offered   		& \cellcolor{red!25}Not-offered     \\
    ~             & Non-proprietary                         & \cellcolor{green!25}Offered       	& \cellcolor{green!25}Offered         \\ \hline
    Security      & Resilient-to-physical-observations      & \cellcolor{green!25}Offered       	& \cellcolor{green!25}Offered         \\
    ~             & Resilient-to-targeted-impersonation     & \cellcolor{green!25}Offered       	& \cellcolor{green!25}Offered         \\
    ~             & Resilient-to-throttled-guessing         & \cellcolor{green!25}Offered       	& \cellcolor{green!25}Offered         \\
    ~             & Resilient-to-unthrottled-guessing       & \cellcolor{green!25}Offered       	& \cellcolor{green!25}Offered         \\
    ~             & Resilient-to-internal-observaions       & \cellcolor{green!25}Offered       	& \cellcolor{green!25}Offered         \\
    ~             & Resilient-to-leaks-from-other-verifiers & \cellcolor{green!25}Offered       	& \cellcolor{green!25}Offered         \\
    ~             & Resilient-to-phising                    & \cellcolor{green!25}Offered       	& \cellcolor{green!25}Offered         \\
    ~             & Resilient-to-theft                      & \cellcolor{yellow!25}Quasi-offered 	& \cellcolor{green!25}Offered         \\
    ~             & No-trusted-third-party                  & \cellcolor{green!25}Offered       	& \cellcolor{green!25}Offered         \\
    ~             & Requiring-explicit-consent              & \cellcolor{green!25}Offered       	& \cellcolor{green!25}Offered         \\
    ~             & Unlinkable                              & \cellcolor{green!25}Offered       	& \cellcolor{red!25}Not-offered     \\
    \end{tabular}
	
	\caption{UDS framework assessment.}
	\label{table:udsresults}
\end{table}

% How Pico is improved
The UDS framework assessment shows similar results to the token unlocking framework. By using the new scheme, Pico achieves an overall better score. It now offers ``easy-to-learn'', as it no longer requires Picosibling secret share management. In the lack of a cost analysis, we will consider that even with the new scheme the ``negligible-cost-per-user'' property is not offered. By relying on more than auxiliary devices, a Pico that uses the proposed scheme does offer ``resilient-to-theft''.

% The downsides and conclusion
The only property where Picosiblings outperforms the scheme presented in this dissertation is ``unlinkable''. Unfortunately, this trade-off cannot be fixed	as the mechanisms combined in the scheme need to rely on biometrics and behavioural analysis, which are unique for each individual. 

In conclusion, by using the new token unlocking mechanism, overall Pico would improve 2 properties in exchange for one.

