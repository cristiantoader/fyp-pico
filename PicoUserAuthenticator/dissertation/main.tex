%%%%%%%%%%%%%%%%%%%%%%%%%%%%%%%%%%%%%%%%%
% Masters/Doctoral Thesis 
% LaTeX Template
% Version 1.42 (19/1/14)
%
% This template has been downloaded from:
% http://www.latextemplates.com
%
% Original authors:
% Steven Gunn 
% http://users.ecs.soton.ac.uk/srg/softwaretools/document/templates/
% and
% Sunil Patel
% http://www.sunilpatel.co.uk/thesis-template/
%
% License:
% CC BY-NC-SA 3.0 (http://creativecommons.org/licenses/by-nc-sa/3.0/)
%
% Note:
% Make sure to edit document variables in the Thesis.cls file
%
%%%%%%%%%%%%%%%%%%%%%%%%%%%%%%%%%%%%%%%%%

%----------------------------------------------------------------------------------------
%	PACKAGES AND OTHER DOCUMENT CONFIGURATIONS
%----------------------------------------------------------------------------------------

\documentclass[11pt, a4paper, oneside]{Thesis} % Paper size, default font size and one-sided paper

\graphicspath{{Pictures/}} % Specifies the directory where pictures are stored

\usepackage{url}
\usepackage[table]{xcolor}
\usepackage[square, numbers, comma, sort&compress]{natbib} % Use the natbib reference package - read up on this to edit the reference style; if you want text (e.g. Smith et al., 2012) for the in-text references (instead of numbers), remove 'numbers' 
\hypersetup{urlcolor=blue, colorlinks=true} % Colors hyperlinks in blue - change to black if annoying
\title{\ttitle} % Defines the thesis title - don't touch this

\begin{document}

\frontmatter % Use roman page numbering style (i, ii, iii, iv...) for the pre-content pages

\setstretch{1.3} % Line spacing of 1.3
\setlength{\skip\footins}{14pt}

% Define the page headers using the FancyHdr package and set up for one-sided printing
\fancyhead{} % Clears all page headers and footers
\rhead{\thepage} % Sets the right side header to show the page number
\lhead{} % Clears the left side page header

\pagestyle{fancy} % Finally, use the "fancy" page style to implement the FancyHdr headers

\newcommand{\HRule}{\rule{\linewidth}{0.5mm}} % New command to make the lines in the title page

% PDF meta-data
\hypersetup{pdftitle="User Authentication for Pico"}
\hypersetup{pdfsubject="User Authentication for Pico"}
\hypersetup{pdfauthor="Cristian M. Toader"}
\hypersetup{pdfkeywords=\keywordnames}

%----------------------------------------------------------------------------------------
%	TITLE PAGE
%----------------------------------------------------------------------------------------

\begin{titlepage}
\begin{center}

\textsc{\LARGE University of Cambridge}\\[1.5cm] % University name
\textsc{\Large Master's Thesis}\\[0.5cm] % Thesis type

\HRule \\[0.4cm] % Horizontal line
{\huge \bfseries User Authentication for Pico: \\ When to unlock a security token}\\[0.4cm] % Thesis title
\HRule \\[1.5cm] % Horizontal line
 
\begin{minipage}{0.4\textwidth}
\begin{flushleft} \large
\emph{Author:}\\
{Cristian M. Toader\\ Churchill College} % Author name - remove the \href bracket to remove the link
\end{flushleft}
\end{minipage}
\begin{minipage}{0.4\textwidth}
\begin{flushright} \large
\emph{Supervisor:} \\
Dr Frank Stajano
\end{flushright}
\end{minipage}\\[3cm]
 
\large \textit{A thesis submitted in fulfilment of the requirements\\ for the degree of ``Master of Philosophy''}\\[0.3cm] % University requirement text
\textit{in the}\\[0.4cm]
Computer Security Group \\ Computer Laboratory \\[2cm] % Research group name and department name
 
{\large \today}\\[4cm] % Date
%\includegraphics{Logo} % University/department logo - uncomment to place it
 
\vfill
\end{center}

\end{titlepage}

%----------------------------------------------------------------------------------------
%	DECLARATION PAGE
%	Your institution may give you a different text to place here
%----------------------------------------------------------------------------------------

\Declaration{

\addtocontents{toc}{\vspace{1em}} % Add a gap in the Contents, for aesthetics

I, Cristian Toader of Churchill College, being a candidate for the M.Phil. in Advanced Computer Science, hereby declare that this report and the work described in it are my own work, unaided except as may be specified below, and that the report does not contain material that has already been used to any substantial extent for a comparable purpose.

% TODO:
Total word count: 14943 (from page 1, excluding the appendix and bibliography).

Signed:\\
\rule[1em]{25em}{0.5pt} % This prints a line for the signature
 
Date:\\
\rule[1em]{25em}{0.5pt} % This prints a line to write the date
}

\clearpage % Start a new page

%----------------------------------------------------------------------------------------
%	QUOTATION PAGE
%----------------------------------------------------------------------------------------
%
%\pagestyle{empty} % No headers or footers for the following pages
%
%\null\vfill % Add some space to move the quote down the page a bit
%
%\textit{``Thanks to my solid academic training, today I can write hundreds of words on virtually any topic without possessing a shred of information, which %is how I got a good job in journalism."}
%
%\begin{flushright}
%Dave Barry
%\end{flushright}
%
%\vfill\vfill\vfill\vfill\vfill\vfill\null % Add some space at the bottom to position the quote just right
%
%\clearpage % Start a new page

%----------------------------------------------------------------------------------------
%	ABSTRACT PAGE
%   TODO: 
%----------------------------------------------------------------------------------------

\addtotoc{Abstract} % Add the "Abstract" page entry to the Contents

\abstract{\addtocontents{toc}{\vspace{1em}} % Add a gap in the Contents, for aesthetics

The abstract needs to be written at the end.

\clearpage % Start a new page

%----------------------------------------------------------------------------------------
%	ACKNOWLEDGEMENTS
%----------------------------------------------------------------------------------------

\setstretch{1.3} % Reset the line-spacing to 1.3 for body text (if it has changed)

\acknowledgements{\addtocontents{toc}{\vspace{1em}} % Add a gap in the Contents, for aesthetics

The acknowledgements and the people to thank go here, don't forget to include your project advisor\ldots
}
\clearpage % Start a new page

%----------------------------------------------------------------------------------------
%	LIST OF CONTENTS/FIGURES/TABLES PAGES
%----------------------------------------------------------------------------------------

\pagestyle{fancy} % The page style headers have been "empty" all this time, now use the "fancy" headers as defined before to bring them back

\lhead{\emph{Contents}} % Set the left side page header to "Contents"
\tableofcontents % Write out the Table of Contents

\lhead{\emph{List of Figures}} % Set the left side page header to "List of Figures"
\listoffigures % Write out the List of Figures

\lhead{\emph{List of Tables}} % Set the left side page header to "List of Tables"
\listoftables % Write out the List of Tables

%----------------------------------------------------------------------------------------
%	ABBREVIATIONS
%----------------------------------------------------------------------------------------

\clearpage % Start a new page

\setstretch{1.5} % Set the line spacing to 1.5, this makes the following tables easier to read

\lhead{\emph{Abbreviations}} % Set the left side page header to "Abbreviations"
\listofsymbols{ll} % Include a list of Abbreviations (a table of two columns)
{
\textbf{LAH} & \textbf{L}ist \textbf{A}bbreviations \textbf{H}ere \\
%\textbf{Acronym} & \textbf{W}hat (it) \textbf{S}tands \textbf{F}or \\
}

%----------------------------------------------------------------------------------------
%	SYMBOLS
%----------------------------------------------------------------------------------------

%\clearpage % Start a new page
%
%\lhead{\emph{Symbols}} % Set the left side page header to "Symbols"
%
%\listofnomenclature{lll} % Include a list of Symbols (a three column table)
%{
%$a$ & distance & m \\
%$P$ & power & W (Js$^{-1}$) \\
%% Symbol & Name & Unit \\

%& & \\ % Gap to separate the Roman symbols from the Greek
%
%$\omega$ & angular frequency & rads$^{-1}$ \\
%% Symbol & Name & Unit \\
%}
%

%----------------------------------------------------------------------------------------
%	THESIS CONTENT - CHAPTERS
%----------------------------------------------------------------------------------------

\mainmatter % Begin numeric (1,2,3...) page numbering

\pagestyle{fancy} % Return the page headers back to the "fancy" style

% Include the chapters of the thesis as separate files from the Chapters folder
% Uncomment the lines as you write the chapters


\chapter{Introduction} % Main chapter title
% TODO: can have ECG sensor file:///C:/Users/cristi/Downloads/fyp-papers/sensors-11-06799.pdf A Comprehensive Ubiquitous Healthcare Solution on an Android™ Mobile Device 

\label{Chapter1}

\lhead{Chapter 1. \emph{Introduction}} % Change X to a consecutive number; this is for the header on each page - perhaps a shortened title

% introduction, passwords are widely used but not that great
Passwords are the most widely used electronic authentication mechanism. They rely on reproducing a secret sequence of characters. This originally offered a sufficiently secure authentication mechanism. However, the fundamental concept of remembering a secret makes passwords unsuited for the current technological context. 

% technological problems with passwords
As Robert Morris \cite{morris1979password} emphasises in his paper, there is a constant competition between attackers and security experts. The majority of users try to maximise the usability of passwords by choosing secrets that are easy to remember. This makes the mechanism more vulnerable to brute force attacks (e.g.. dictionary, pre-compiled hashes, rainbow tables \cite{oechslin2003making}). In the past, security experts were able to slow down attacks without any impact to the user. However, with a constant increase in computational power, passwords became easier to breach. 

% technological response
One flaw of passwords is that when chosen freely they tend to be short and predictable. In order to maintain acceptable security guarantees, a number of creation requirements are enforced. Examples include having a minimum length, one or more numeric characters, and one or more special characters. Security experts recommend that each account needs to have an unique password. Furthermore, passwords sometimes require to be changed periodically with something not too similar with the original. 

% the availability downside
As shown by Yan et al \cite{yan2004password}, users choose weak passwords if not given any advice to make them memorable. Theoretically, additional restrictions make the mechanism secure. Having non-intuitive passwords that fully utilise the available character set makes brute force attacks harder to perform. In practice however, users need to memorise numerous, unique, and complex passwords. As shown by Adams \& Sasse \cite{adams1999users} maintaining all restrictions and security advices proves not to be feasible. This often leads to poor practices such as writing the passwords on paper.

% sumarising problem with passwords
The main problem with passwords is the basic principle of users remembering a secret. If the secret is memorable, than an attacker may brute force it with more ease. If it is too complex, then the user may not remember it. Furthermore, since reusing passwords is not safe, and given the memory capacity people have, the mechanism is also not scalable. For these reasons, passwords prove not to be a reliable solution for the future and even present.

% alternative to passwords
A large number of alternatives to passwords are available. However, as shown by Bonneau et al \cite{bonneau2012quest}, the main advantage passwords have over other authentication mechanisms are in terms of deployability and usability. A study by Clarke et al \cite{clarke2002acceptance} shows that although $81\%$ of users agree to an alternative to password based phone unlocking, the majority ignores the existence of available solutions. The main conclusion is that although passwords are not secure, the cost of replacing them and familiarising with a new authentication mechanism is still too inconvenient. 

% Pico to the rescue
The Pico project was designed by Frank Stajano \cite{stajano2011pico} with the purpose of replacing password based mechanisms. Pico is a hardware token that generates and manages user authentication credentials. It has an additional layer of security by only being usable in the presence of its owner. Therefore, a security chain is created where ``who you are'' unlocks ``a secret you have'' which is used for authentication.

% Picosiblings authentication mechanism
The current solution for unlocking Pico is by communicating with small auxiliary devices called Picosiblings \cite{stannard2012good}. They are designed to be embedded in everyday items that users can carry throughout the day (e.g. keys, necklace, rings). Each Picosibling transmits a secret sequence to Pico. When all required secrets are gathered, Pico becomes unlocked and can be used by its owner.

% Downside of Picosiblings authentication mechanism
Picosiblings are a sensible solution to unlocking Pico. However, they are purely based on proximity to the device. As presented in the original Pico paper \cite{stajano2011pico} anyone in possession of both Pico and its Picosiblings can have full access to the owner's accounts for a limited amount of time. This risk is lowered by additional security features. However, the main vulnerability of Picosiblings is that they do not reflect who the user is, but additional things the user has.

% The scope of my project
The purpose of this dissertation is to design and prototype a better token unlocking mechanism for Pico. According to its design, the process should be memoryless, and enable continuous authentication. The token should lock and unlock automatically only in the presence of its owner. The solutions that seem to best fit these requirements are biometric authentication mechanisms. Therefore, we have explored the possibility of combining multiple biometrics and behavioural analysis as part of an unified solution. The output from each mechanism is combined to generate an overall confidence level, reflecting that the owner is still in possession of the Pico.

% Contributions
A number of contributions have been made throughout this dissertation project. The following list presents a summary of these achievements, with further details in the following chapters.
\begin{itemize}
	\item We have created a framework derived from the work by Bonneau et al \cite{bonneau2012quest}. This is used to evaluate a couple of existing token unlocking mechanisms, including Picosiblings. The results are used as a benchmark when evaluating the proposed solution.
	
	\item We have designed a new token unlocking mechanism. The solution may be used in any type of user authentication, but it is presented in the context of unlocking the Pico token. 
	
	\item We have developed an Android prototype. The purpose of the implementation is to prove that the design can be developed using existing hardware. The prototype was not created for performance purposes. However, power analysis as well as timings of different authentication stages were recorded. These should serve as an approximation of the limitations and downsides of the scheme.
	
	\item The scheme is analysed by using the token unlocking evaluation framework. A comparison is made with Picosiblings in order to identify performance differences. We aimed for the proposed scheme to achieve better results in at least some categories of the token unlocking framework.
		
	\item We have analysed and determined the impact of the proposed token unlocking mechanism on the Pico. The analysis is performed using the framework by Bonneau et al \cite{bonneau2012quest}. One of the goals when designing the solution was to make Pico better in terms of at least one property.
	
\end{itemize}	


\chapter{Pico: no more passwords!} % Main chapter title

\label{Chapter2}

\lhead{Chapter 2. \emph{Pico}} % Change X to a consecutive number; this is for the header on each page - perhaps a shortened title

The scope of this dissertation project is to design and implement a new unlocking mechanism for the Pico token, as designed by Frank Stajano \cite{stajano2011pico}. A better understanding of the Pico design is therefore necessary. This chapter aims to go into brief detail as to what Pico is, how it works, and what its properties are.

% introduction
Pico is an user authentication hardware token, designed with the purpose of fully replacing passwords. Although other replacement mechanisms exist, they are generally focused on web based authentication. The scope of the solution described by Stajano addresses all instances of password authentication, both web based as well as offline.

% motivation to replace passwords: increased computational power
The motivation behind this project is the fact that passwords are no longer viable in the current technological context. Computing power has grown, making simple passwords easy to break. Longer and more complex passwords are now required. As shown by Adams \& Sasse \cite{adams1999users}, this has a negative impact on the users, which have limited memorising capability.

% motivation to replace passwords: increased number of password accounts
Another reason why passwords are no longer viable is the fact that they are not a scalable solution. Security experts recommend that passwords should be reused for multiple accounts. However, a large number of computer based services require password authentication. In order to respect security recommendations, users would be forced to remember dozens of unique, complex passwords. A study by Florencio et al \cite{florencio2007large} performed over half a million users confirms the negative impact of scalability on password quality. Furthermore, passwords are often forgotten or reused across accounts. 

% fundamental design to improve on passwords
When designing the Pico password replacement mechanism, Stajano decides to have a fresh start. He describes that an alternative for passwords needs to be at least memoryless and scalable, without reducing security. In the case of token based authentication, the solution also needs to be loss and theft resistant. The Pico token was therefore designed to satisfy these fundamental properties, as well as other benefits emphasised in a paper by Bonneau et al \cite{bonneau2012quest} as well as the the original work by Stajano.

% keeping credentials safe
As a token authentication mechanism, Pico transforms ``something you know'' into ``something you have''. It offers support for thousands of credentials which are kept encrypted on the Pico device. The encryption key is also known as the ``Pico Master Key''. If the Pico is not in the possession of its owner it becomes locked. In this state, the ``Pico Master Key'' is unavailable and the user cannot authenticate to any app\footnote{For the purpose of brevity, any mechanism requiring user authentication will be called an ``app'' just as in the original paper by Stajano.}.

% creates and manages credentials
Credentials are generated and managed automatically whenever the owner interacts with an app. Therefore, the responsibility of generating a strong and unique credential, as well as memorising it, is shifted from the user to the Pico. No additional effort such as searching or typing credentials is required.

% continuous authentication
Another important feature offered by Pico is continuous authentication. Traditional password mechanisms provide authentication for an entire session. The user is responsible of managing and closing the session when it is no longer needed. Instead, Pico offers the possibility of periodic re-authentication of its owner using short range radio. If either the Pico or the owner are no longer present, the authentication session is closed. 

% physical design
From a physical perspective, Pico is a small portable dedicated device. Its owner should be carrying it at all times, just as they would with a car key. It contains the following hardware components:
\begin{itemize}
	\item Main button used for authenticating the owner to the app. This is the equivalent of typing the password.
	\item Pairing button used for registering a new account with an app.
	\item Small display used for notifications.
	\item Short range bidirectional radio interface used as a primary communication channel with the app.
	\item Camera used for receiving additional data from the app via 2-dimensional visual codes. This serves as a secondary communication channel.
\end{itemize}

% physical design: what is stored and how
As mentioned before, the Pico main memory is encrypted using the Pico Master Key. It contains thousands of slots used for storing unique credentials used in the authentication process. Each credential consists of public-private key information generated during account creation in a key exchange protocol. The public key belongs to the corresponding app, while the private key was generated when creating the account. 

% account creation
During account creation Pico scans a 2D visual code generated by an app. The image encodes a hash of the apps certificate including the app name and public key. Pico starts the protocol with the app using the radio channel, and the app provides a public key used for communication. The key is validated using the hash from the visual code, and the protocol continues. Pico then initiates a challenge for the app to prove that it is in possession of the corresponding private key, and provides a temporary public key. This protects the identity of the owner, by only showing their public key after the app is authenticated. Only then Pico generates a key pair, sends the public key to the app and stores the key pair.

% account authentication
The account authentication process starts when the user presses the main button and scans the app 2D code. The hash of the app's name and public key are extracted from the 2D image. This information is used to find the corresponding credentials. An ephemeral public key encrypted with the app's public key is sent via the radio channel. The app is authenticated by using this key to require the corresponding (user id, credential) pair. Only after the app is authenticated Pico uses the public key generated during the registration process and authenticates itself to the app.

% Unlocking pico
The locking process is an important aspect of Pico which we have not yet fully described. The token should become unlocked only in the presence of its owner. Currently this is achieved using bidirectional radio communication with small devices called Picosiblings \cite{stannard2012good}. These are meant to be embedded in everyday items that the owner carries around, such as earrings, rings, keys, chains, etc.

% Reconstructing master key.
The Pico authentication credentials are encrypted using the Pico Master Key. The key is not available on the Pico and can only be reconstructed using k-out-of-n secret sharing, as described by Shamir \cite{shamir1979share}. Except for two shares which will be discussed later, each k-out-of-n share is held by a Picosibling. 

Using an initialisation protocol based on the resurrecting duckling \cite{stajano2000resurrecting}, each Picosibling is securely paired with Pico. After the initialisation process, Pico sends periodic ping requests to which all registered Picosibling are expected to respond. During each successful ping, the Picosibling sends its k-out-of-n share back to Pico. If enough secrets are provided the ``Pico Master Key'' is reconstructed and Pico becomes unlocked.

Internally, Pico keeps a slot array for each paired Picosibling. Each slot contains a countdown value, and the key share provided by the Picosibling. The countdown value is decreased periodically. When it expires, the share becomes deleted. Similarly, if k shares are not acquired before a predefined time-out period, all shares are removed.

Except for the Picosiblings, two additional special shares with a larger time-out period are described by the paper:
\begin{itemize}
	\item Biometric measurement used for authenticating the owner to the Pico.
	\item Remote server network connection used for locking the Pico remotely.
\end{itemize}

The possibility of using a smart phone as a Pico is briefly considered in the paper. This would have the advantage of not requiring any additional devices from the user. Modern smart phones provide all the necessary hardware required by Pico. However, this would be a security trade-off in exchange for usability. Mobile phones are an ecosystem for malware, and they present uncertainty regarding the privacy of encrypted data. This option may still be used as a cheaper alternative to prototype and test, which is something we will make use of in this project.








 

\chapter{Assessment framework}

\label{Chapter5}

\lhead{Chapter 5. \emph{Assessment framework}}

The purpose of this chapter is to create an assessment framework for token unlocking mechanisms. This framework will be used to evaluate existing solutions, including the Picosiblings scheme used by Pico. The analysis of the results can then be used to create an alternative unlocking mechanism for Pico. The project aims to achieve better results in some categories, without necessarily completely outperforming it.

% Original UDS assessment framework
%	TODO: add Pico evaluation using the framework.
\section{UDS assessment framework}

% Introduction on the existing paper and why it cannot be used as it is
Similar work to what we are trying to achieve in this chapter was performed by Bonneau et al \cite{bonneau2012quest}. The authors create a framework for evaluating web based authentication mechanisms. However, the assessment scheme is not entirely compatible for token unlocking schemes. For example, properties such as ``Browser-compatible'' do not apply, while others need to be redefined to fit our context.  The paper however presents a good starting point for our token unlocking evaluation framework. The remainder of this section will present a brief summary of the paper.

% Why did they do the paper.
The motivation behind this paper is to gain insight about the difficulty of replacing passwords. An assessment framework is created, and a number of web authentication mechanisms are evaluated. It is an useful tool in identifying key properties of web based authentication schemes. The framework is intended to provide a benchmark for future proposals.

% How framework is structured.
The framework consists of 25 properties divided into three categories: usability, deployability, and security. For this reason, it is abbreviated by the authors as the ``UDS framework''. An authentication scheme is evaluated by assessing whether each property is offered or not. In the case where a scheme almost offers a property, the authors mark it as quasi-offered. To simplify the framework, properties which are not applicable are marked as offered.

% Results of assessment: passwords.
Since passwords are currently the most widely used authentication mechanism, the results are predictable. Evaluating 35 replacement schemes shows that no scheme completely dominates them. Passwords satisfy all the properties in the deployability category. They score reasonably well in terms of usability, excelling in properties such as: ``nothing-to-carry'', ``efficient-to-use'', ``and easy-recovery-from-loss''. However, from a security perspective passwords don't perform well. They only offer the ``resilience-to-theft\footnote{Not applicable to passwords}'', ``no-trusted-third-party'', ``requiring-explicit-consent'', and ``unlinkable'' properties. The full evaluation can be found within the paper itself.

% Results of assessment: biometrics.
Biometric mechanisms receive mixed scores on usability. None of them offer the ``infrequent-errors'' property, due to false negative precision. More importantly if biometric data becomes compromised, the possibility of replay attacks makes the authentication mechanism unreliable. They score poorly in deployability partially because they require additional hardware. In terms of security they perform worse than passwords. Replay attacks can be performed by an attacker using a pre-recording data of the user, making them not ``resilient-to-targeted-impersonation'' and not ``resilient-to-theft''. There is a one to one correlation between the owner and their biometric recording, therefore the ``unlinkable'' property is not offered by these mechanisms. 

% Memory-effortless vs Nothing-to-carry
By analysing the framework results, we see that some authentication schemes, such as security tokens, offer ``memory-effortless'' in exchange for the ``nothing-to-carry'' property.  The only schemes that offer both are biometric mechanisms. This is a consequence of replacing ``something you know'' with ``something you are'' instead of have. For different reasons no mechanism offers both ``memory-effortless'' while being ``resilient-to-theft''.

% Scores and no ranking
When trying to compute an aggregate score using the framework, not all properties should be equal in importance. Different properties should have different weights depending on the purpose of the assessment. For example, if we would try to find the most secure authentication mechanism, security properties would have a larger weight in the overall evaluation. For this reason, the authors only provide the means for others to make an evaluation based on their needs. No aggregate scores or rankings are provided in the paper. 

% Combining schemes
The authors mention the possibility of combining schemes as part of a two factor authentication. In terms of deployability and usability, the overall scheme offers a property if it is offered by both authentication mechanisms. In terms of security, only one of the two mechanisms needs to offer the property in order for the two factor combination to offer it as well. However, Wimberly \& Liebrock \cite{wimberly2011using} observe that combining passwords with a second authentication mechanism scheme leads to weaker credentials and implicitly less security.

% Further details can be found in the paper.
The following section will offer more details on the UDS framework properties which also apply to token unlocking. Further information about the framework are not mentioned in this dissertation for the purpose of brevity.  The full list of properties, their description, and the evaluation of a number of mechanisms are provided in the original paper by Bonneau et al. 

\section{Token unlocking framework}
% Introduction as to why we use properties from the UDS framework
Unlike web based authentication mechanisms, token unlocking schemes record and process data locally. For this reason, a subset of the UDS framework properties are also present in the framework we have developed. Those which do not apply, or would apply to every token unlocking mechanism were removed. Some properties needed to be adapted to the new context of a token, and therefore will have a slightly different meaning.

% List each property, a description, and an example
The following list contains all properties of the UDS framework developed by Bonneau et al \cite{bonneau2012quest} which we adapted and included in the token unlocking framework. A short description is included for the cases when the property is offered or quasi-offered, as well as a small example.
\begin{description}
  
  %
  %	 Usability
  %
  
  % TODO: example not good as PIN unlocks account not the token!
  \item[Memorywise-effortless] \hfill \\
  Users do not need to remember any type of secret. This includes passwords, physical signatures, or drawings. The property was originally quasi-offered if one secret would be used with multiple accounts, but in the case of security tokens this does not apply. As an example the RSA SecurID \footnote{http://www.emc.com/domains/rsa/index.htm?id=1156} is used in conjunction with a password in order to authenticate the user, and therefore does not offer this property.
  
  \item[Nothing-to-carry] \hfill \\
  The unlocking mechanism does not require any additional hardware except for the token. The property is quasi-satisfied in the case of hardware the user would have carried on a normal basis such as a mobile phone. An example of a mechanism that quasi-offers the property is the Picosiblings scheme which uses small devices embedded in everyday items. Biometric mechanisms that require additional sensors such as a fingerprint reader do not satisfy this property. 
  
  \item[Easy-to-learn] \hfill \\
  Users who use the unlocking mechanism would be able to learn it with ease. The original paper by Bonneau et al \cite{bonneau2012quest}, assess Pico as not ``easy-to-learn'' due to the complexity of the Picosiblings management\footnote{As discussed in the previous chapter \ref{Chapter2}, each Picosibling contains a k-out-of-n secret used to reconstruct the ``Pico Master Key''. The user therefore needs to choose the right combination of Picosiblings in order to unlock the Pico, which may prove difficult}. PINs or passwords however satisfy this property due to the users' familiarity with this type of authentication.
  
  \item[Efficient-to-use] \hfill \\
  The amount of time the user needs to wait for the token to be unlocked is reasonably short. This includes the time required to provide the input for the mechanism. The same applies for setting up the token unlocking mechanism, but with a larger time scale. In the case of PINs for example the input and processing time are very low, and therefore the scheme offers the property. Mechanisms based on biometrics however may not, depending on the implementation.
  
  \item[Infrequent-errors] \hfill \\
  The rightful owner should generally be able to successfully authenticate to the token. Any sort of delays resulted from the unlocking mechanism such as typos during typing or biometric false negatives may contribute to the mechanism's inability to provide this property. As an example, PINs have a limited input length and character set size which makes infrequent errors unlikely and therefore offer the property. Biometric mechanisms, based on the type and implementation may quasi-offer the property, although they generally do not.
    
  \item[Easy-recovery-from-loss] \hfill \\ 
  The meaning of this property was modified to reflect the context of token unlocking. It is offered if the user may easily recover from the loss of authentication credentials. Depending on the scheme, this may include the loss of auxiliary devices, forgotten credentials, difference in biometric features. As an example, forgotten PINs offer the property as they generally require a simple reset using an online service.
  
  %
  %	 Deployability
  %
  \item[Accessible] \hfill \\
  The mechanism is usable by any user regardless of any disabilities or physical conditions. In the original paper, passwords are offered as an example of a scheme which offers this property. A gait recognition unlocking scheme would not offer this property. 
  
  \item[Negligible-cost-per-user] \hfill \\
  The total cost per user of using the scheme, enquired by both the user and the verifier, is negligible.
  
  \item[Mature] \hfill \\
  A large number of users have successfully used the scheme. Any open source projects involving the mechanism, as well as any participation not involving its creators contribute to this property. For example, passwords are widely used and implemented and therefore offer the property.
  
  \item[Non-proprietary] \hfill \\
  Anyone can implement the token unlocking scheme without having to make any payment such as royalties. The technologies involved in the scheme are publicly known and do not rely on any sort of secret.
  
  %
  %	 Security
  %
  \item[Resilient-to-physical-observation] \hfill \\
  An attacker would not be able to impersonate the owner of the token after observing him authenticate. Based on the number of observations required for the attacker to unlock the token, the scheme may quasi-offer the property. The original paper suggests 10-20 times to be sufficient, although it is just an approximation. Physical observation attacks are not restricted to shoulder surfing, and may include video cameras, keystroke sounds, or thermal imaging of the PIN pad.
  
  \item[Resilient-to-targeted-impersonation] \hfill \\
  An attacker should not be able to impersonate the owner of the token by exploiting knowledge of personal details. This may include birthday, full name, family details, and other sensitive information. The scheme should also be resilient to pre-recordings of biometric information which may then be replayed to the authenticator.
  
  \item[Resilient-to-throttled-guessing] \hfill \\
  The scheme is resilient to attacks with a guessing rate restricted by the mechanism. The process cannot be automated due to the lack of physical access to authentication data. This may be achieved using tamper resistant memory. As an example, PINs offer this property because SIM cards become locked after only three unsuccessful attempts.
  
  \item[Resilient-to-unthrottled-guessing] \hfill \\
  The scheme is resilient to attacks with a guessing rate unrestricted by the mechanism. Even though the guessing process is only restricted by the attacker's computational power, the scheme would still not be bypassed within reasonable time. The original paper suggests that if the attacker may process $2^{40}$ to $2^{64}$ guesses per account, they would only be able to compromise less than $1\%$ of accounts. Since tokens are generally designed to have one owner, the original description will be adapted to a single account. Therefore the property is granted only if an attacker requires more than $2^{40}$ attempts.
  
  \item[Resilient-to-theft] \hfill \\
  The property applies to schemes which use additional hardware other than the token. If the additional hardware becomes in the possession of an attacker, it is not sufficient to unlock the token. For example, auxiliary biometric devices used in the conjunction with the token offer this property. In this case the token would still not be unlocked using the hardware alone. Picosiblings however only quasi-offer the property. Although they generally rely on proximity to the Pico, the two special shares allow the owner to lock the token remotely.
  
  \item[Unlinkable] \hfill \\
  Using the authentication input with any verifier using the same authentication mechanism\footnote{The authentication mechanism is not necessarily used for token unlocking. Any sort of mechanism which requires user authentication is a valid option.} does not compromise the identity of the token owner. As an example the link between a PIN and its owner is not strong enough to make a clear link between the two. However, biometrics are a prime example of schemes which do not offer this property.
  
\end{description}

% Introduction for added properties
% 	TODO: this sounds a bit bad.. rephrase?
We have augmented the subset from the original UDS framework with a number of properties relevant which are relevant to Picosiblings, PINs, as well as other token unlocking mechanisms. The following list is part of the project's contributions to the overall evaluation framework.

\begin{description}
  \item[Continuous-authentication] \hfill \\
  The token unlocking scheme re-authenticates the user periodically. The process doesn't need to be hidden from the owner, but it is required to be effortless. The token should remain unlocked and usable in the presence of its owner. The scheme needs to detect when the owner is no longer in possession of the token, and lock the device accordingly. When locked, any open authentication session managed by the security token will be closed. The concept is mentioned by Bonneau et al \cite{bonneau2012quest}, but not included in the UDS framework. It is discussed in more detail by Stajano \cite{stajano2011pico} as one of the benefits of the Pico project. Using the UDS classification of the original framework, the property belongs to the Security category.
  
  \item[Multi-level-unlocking] \hfill \\
  The unlocking scheme provides quantifiable feedback, not just a locked or unlocked state. The mechanism offers the possibility of supporting multiple token security permissions. These would be granted based on the confidence level that the user trying to unlock the token is its owner. For example, a $70\%$ confidence level that the owner is present may allow the user to access an email account, but not make any sort of payments or banking transactions. Passwords only provide a ``yes'' or ``no'' answer and therefore do not offer this property. Biometric mechanisms can offer this property. Their output is either a probability or some sort of distance metric that data was collected from the owner. Different confidence levels could therefore enable different security permissions. Using the UDS classification of the original framework, the property belongs to the Security category.
  
  % TODO: maybe offer a different name
  \item[Non-disclosability] \hfill \\
  The owner may not disclose authentication details neither intentionally or unintentionally. This is a broader version of the ``resilient-to-phishing'' and ``Resilient-to-physical-observation'' properties from the original UDS framework. However, the focus here is that the token may only be used by its owner. This is an important property in enterprise situations where the security token should not be shared. Passwords and other schemes based on secrets do not offer this property as the owner could share it with another user without any difficulty. Biometric mechanisms however cannot be easily disclosed. Based on the UDS classification the property belongs to the Security category.
  
  \item[Availability] \hfill \\
  The owner has the ability of using the scheme regardless of external factors. The ability to authenticate should not be impaired by the authentication context such as traffic noise, different light intensities, or restricted movement space. The property is not related to physical disabilities preventing the user from using the scheme but only on contextual influences on data collection. As an example gait recognition would only function while moving on foot and therefore does not offer the property. A mechanism requiring a PIN on the other hand would work in any circumstance. Using the UDS classification of the original framework, the property belongs to the Usability category.
  
\end{description}

\section{Example evaluation}
% Introduction to why we offer some examples
We will demonstrate how the framework should be used by assessing three token based authentication mechanisms: Picosiblings, PIN, and Face-unlock. Each scheme represents a different type of authentication method. Picosiblings essentially are a secret the owner has, PINs are a secret the owner knows, and Face-unlock reflects who the owner is. Results in the Picosiblings section will be used in the following chapter as a benchmark for comparison with our proposed token unlocking scheme.

	%	
	% Picosiblings evaluation
	%=======================================================
	%
	\subsection{Picosiblings}
	% What are Picosiblings in a few words
	Unlocking the Pico token requires k-out-of-n secrets used to reconstruct the Pico Master Key. Each Picosibling contains a secret that is transmitted to Pico using a secure connection via a radio channel. Given enough secrets the master key is reconstructed, and Pico becomes unlocked.
	
	% Picosiblings: Usability
	% 	TODO: 
	% 		- why doesn't it offer recovery-from-loss? expand!
	%
	The scheme doesn't require from its owner any known secret and therefore is ``memorywise-effortless''. Since it relies on devices embedded in everyday items the ``nothing-to-carry'' property is quasi-satisfied. The original paper by Bonneau marks Pico as not ``easy-to-learn'' due to Picosiblings management, which is a characteristic of the unlocking mechanism. It is quasi-``efficient-to-use'' and has quasi-``infrequent-errors'' until proven otherwise. It does not offer the ``easy-recovery-from-loss'' property. The unlocking mechanism relies only on radio communication with the token. This makes it invariable to external factors therefore offering the ``availability'' property.
	
	% Picosiblings: Deployability
	The original paper marks Pico as not ``accessible'' due to the coordinated use of camera, display, and buttons. However, the Picosiblings are ``accessible'' because they are embedded in everyday accessories that any user can wear. Pico doesn't aim to satisfy the ``negligible-cost-per-user'' property, and since no realistic Picosiblings cost estimate exists we will consider the property is not offered. The scheme is at the stage of a prototype, with no external open source contributions, and little user testing. For these reasons it is not considered to be ``mature''. Frank Stajano states in his paper \cite{stajano2011pico} that the design of Pico and Picosiblings are not patented, and no royalties are due. The only requirement for implementing the design is to cite the paper, which makes the unlocking mechanism ``non-proprietary''.
	
	% Picosiblings: Security
	% 	TODO:
	%		- ask about Picosiblings protocol and fill in the reason for resilient-to-targeted-impersonation
	%
	Since the scheme does not rely on any user input it is ``resilient-to-physical-observations''. Based on the description of Picosiblings given by Stajano \cite{stajano2011pico} the scheme offers the ``resilient-to-targeted-impersonation'', ``resilience-to-throttled-guessing'', and ``resilient-to-unthrottled-guessing'' properties. Any attacker which comes in possession of the Picosiblings may unlock the Pico. However due to the auxiliary shared secrets\footnote{Picosiblings also relies on two special shares. One is unlocked using biometric authentication, and the other is provided by an external server. Using these shares would only grant the thief a limited time window before the token is either locked remotely or the shares expire.} the scheme is quasi ``resilient-to-theft''. Each Picosibling only works with one verifier (its master Pico), and therefore offers the ``unlinkable'' property. The scheme was designed to provide ``continuous-authentication''. Because of the k-out-of-n master key reconstruction mechanism, Picosiblings only have the locked or unlocked states and therefore do not offer ``multi-level-unlocking''. The scheme does not satisfy the ``non-disclosability'' property. The owner is free to disclose authentication credentials simply by giving their Picosiblings.
	
	%
	%	PIN
	%=======================================================
	%
	\subsection{PIN}
	% Introduction to PINs and resemblance to passwords
	PINs are token authentication mechanisms similar to passwords. The difference between the two is that they use a smaller set of input characters. Additional protection comes from steep security measures when the authentication has failed. As an example, typing 3 wrong PINs on a mobile phone would lock your SIM card. A lot of the PIN properties should however be similar with those offered by passwords.
	
	% Usability: PINs
	The scheme relies on knowing a secret, which is not ``memorywise-effortless''. It does however offer the ``nothing-to-carry'' property. Because of its similarity with passwords users find it ``easy-to-learn''. The small character set allows for fast user input and validation making PINs ``efficient-to-use''. Mistakes however may still occasionally occur, and due to the lack of visual feedback\footnote{If existent, visual feedback for PINs generally consists of `*' characters.} the scheme only quasi-offers ``infrequent-errors''. PINs are generally easily reset by the manufacturer using online services, granting them ``easy-recovery-from-loss''\footnote{An example of this is the RSA SecurID. An example reset procedure is described at the following link: http://uk.emc.com/collateral/15-min-guide/h12278-am8-help-desk-administrator-guide.pdf}. The scheme offers the ``availability'' property, as the authentication process cannot be impaired by external factors.
	
	% Deployability: PINs
	Just as passwords PINs score all points in deployability. They can be used regardless of disabilities, making them ``accessible''. The have virtually no cost, satisfying the ``negligible-cost-per-user property''. Being a subset of passwords, the mechanism is considered to be ``mature'' and ``non-proprietary''.
	
	% Security: PINs
	% 	TODO: maybe rephrase a bit..
	From a security perspective PINs score poorly. They are not ``resilient-to-physical-observation''. Anyone can eavesdrop the input of a PIN either by shoulder surfing or recording with a camera. Similarly to passwords, PINs are often written down in plain sight. However, in the lack of relevant studies\footnote{Just as Bonneau et al suggest, a relevant study would assess acquaintances' ability to guess the PIN of a subject.} we will mark the scheme to quasi-offer the ``resilient-to-targeted-impersonation'' property. The restricted character set makes PINs adopt harsher security policies when provided invalid input. They are generally locked after three bad attempts, making them ``resilient-to-throttled-guessing''. The ``resilient-to-unthrottled-guessing'' property is implementation dependent. However, security tokens are dedicated devices that generally have tamper resistant memory, making unthrottled guessing not possible. Any hardware PINs may require does not compromise the mechanism, therefore offering ``resilient-to-theft''. Users have the freedom of choosing any PIN. Even in situations when reused with multiple tokens, credentials are generally salted and therefore ``unlinkable''. The scheme does not offer ``continuous-authentication'' due to explicit requests. They can only provide locked or unlocked feedback, and therefore do not offer ``multi-level-unlocking''. The owner may disclose their PIN at any time, making the ``non-disclosability'' property unsatisfied. 
	
	%
	%	Android face unlock
	%=======================================================
	%
	\subsection{Face unlock}
	Although not currently used as a security token unlocking mechanism, face recognition is a viable biometric authentication scheme. It can be ported for a token such as Pico, which is designed to have a camera. With a variety of possible implementations, for accessibility reasons we will analyse the Android face unlock mechanism.
	
	% Face unlock: usability
	Face unlock is ``memorywise-effortless'', as any other biometric scheme. It offers the ``nothing-to-carry property'', the camera being embedded as part of the token. The mechanism is ``easy-to-learn'', since it only needs the user to look at the camera. The authentication process is performed almost instantly, making the scheme ``efficient-to-use''. The scheme is dependent on camera positioning, obstructing objects (i.e. glasses, earrings), and face mimic. In conjunction with the UDS framework assessment of biometrics in general, the scheme does not offer ``infrequent-errors''. If the scheme no longer functions as a result of change in facial traits, Android has a backup unlocking mechanism. This may also be used to disable or recalibrate the scheme, therefore offering ``easy-recovery-from-loss''. The ``availability'' property is not satisfied due to the dependence on external factors such as light or obstacles.
	
	% Face unlock: deployability
	% 	TODO: is it proprietary?
	Android face recognition is ``accessible'' for anyone regardless of disabilities. It offers the ``negligible-cost-per-user'' property, given that the hardware was already present in devices without face recognition features. Due to limited user exposure it is only quasi-``mature''. The scheme relies on proprietary software and therefore is not ``non-proprietary''.
	
	% Face unlock: security
	Observing the owner authenticate using the scheme does not provide any advantage to an attacker. The scheme therefore offers the ``resilient-to-physical-observations'' property. Targeted impersonation is an issue with any biometric mechanism. The scheme is vulnerable to replay attacks (i.e. a picture of the owner's face) and does not offer the ``resilient-to-targeted-impersonation property''. The ''resilient-to-throttled-guessing`` and ``resilient-to-unthrottled-guessing'' properties do not apply. Given the Android implementation, neither does ``resilient-to-theft''. The same authentication data is used with any verifier, and therefore the ``linkable'' property is not offered. The scheme is implemented without ``continuous-authentication'' or ``multi-level-unlocking'' although both can be supported by biometric mechanisms. Given the possibility of deliberately providing data for a replay attack, the scheme only quasi-offers the ``non-disclosability'' property.

	% TODO: can add fingerprint unlock - IPhone
	
\section{Conclusions}
% Conclusion  paragraph
We have developed a token unlocking evaluation framework. The result is strongly related to similar work by Bonneau et al \cite{bonneau2012quest} which was summarised at the beginning of the chapter. Some properties needed to be adapted to fit the context of a security token. We have also contributed with 4 original properties. 

The framework was applied for three sample token unlocking mechanisms. A summary of the results is posted in table \ref{table:results}. Each property is highlighted with an appropriate colour in order to allow for quicker analysis. These will serve as a benchmark for the proposed solution. 

\begin{table}
    \begin{tabular}{l|l|l|l}
    Property                            & PIN           					& Picosiblings  					& Face recognition \\ \hline
    Memorywise-effortless               & \cellcolor{red!25}Not-offered   	& \cellcolor{green!25}Offered       & \cellcolor{green!25}Offered          \\
    Nothing-to-carry                    & \cellcolor{green!25}Offered       & \cellcolor{yellow!25}Quasi-offered   & \cellcolor{green!25}Offered          \\
    Easy-to-learn                       & \cellcolor{green!25}Offered       & \cellcolor{red!25}Not-offered   & \cellcolor{green!25}Offered          \\
    Efficient-to-use                    & \cellcolor{green!25}Offered       & \cellcolor{yellow!25}Quasi-offered & \cellcolor{green!25}Offered          \\
    Infrequent-errors                   & \cellcolor{yellow!25}Quasi-offered & \cellcolor{yellow!25}Quasi-offered & \cellcolor{red!25}Not-offered      \\
    Easy-recovery-from-loss             & \cellcolor{green!25}Offered       & \cellcolor{red!25}Not-offered   & \cellcolor{green!25}Offered          \\
    Availability                        & \cellcolor{green!25}Offered       & \cellcolor{green!25}Offered       & \cellcolor{red!25}Not-offered      \\ \hline
    Accessible                          & \cellcolor{green!25}Offered       & \cellcolor{green!25}Offered       & \cellcolor{green!25}Offered          \\
    Negligible-cost-per-user            & \cellcolor{green!25}Offered       & \cellcolor{red!25}Not-offered   & \cellcolor{green!25}Offered          \\
    Mature                              & \cellcolor{green!25}Offered       & \cellcolor{red!25}Not-offered   & \cellcolor{yellow!25}Quasi-offered    \\
    Non-proprietary                     & \cellcolor{green!25}Offered       & \cellcolor{green!25}Offered       & \cellcolor{red!25}Not-offered      \\ \hline
    Resilient-to-physical-observations  & \cellcolor{red!25}Not-offered   & \cellcolor{green!25}Offered       & \cellcolor{green!25}Offered          \\
    Resilient-to-targeted-impersonation & \cellcolor{yellow!25}Quasi-offered & \cellcolor{green!25}Offered       & \cellcolor{red!25}Not-offered      \\
    Resilient-to-throttled-guessing     & \cellcolor{green!25}Offered       & \cellcolor{green!25}Offered       & \cellcolor{green!25}Offered          \\
    Resilient-to-unthrottled-guessing   & \cellcolor{green!25}Offered       & \cellcolor{green!25}Offered       & \cellcolor{green!25}Offered          \\
    Resilient-to-theft                  & \cellcolor{green!25}Offered       & \cellcolor{yellow!25}Quasi-offered   & \cellcolor{green!25}Offered          \\
    Unlinkable                          & \cellcolor{green!25}Offered       & \cellcolor{green!25}Offered       & \cellcolor{red!25}Not-offered      \\
    Continuous-authentication           & \cellcolor{red!25}Not-offered   & \cellcolor{green!25}Offered       & \cellcolor{red!25}Not-offered      \\
    Multi-level-unlocking               & \cellcolor{red!25}Not-offered   & \cellcolor{red!25}Not-offered   & \cellcolor{red!25}Not-offered      \\
    Non-disclosability                  & \cellcolor{red!25}Not-offered   & \cellcolor{red!25}Not-offered   & \cellcolor{yellow!25}Quasi-offered    \\
    \end{tabular}

	\caption{Token unlocking framework sample assessment.}
	\label{table:results}

\end{table}

As the table shows, none of the example schemes completely dominates the others. They receive mixed scores in terms of availability and security. PINs dominate in terms of deployability, receiving a perfect score. 
% Chapter Template

\chapter{Design} % Main chapter title

\label{Chapter4} % Change X to a consecutive number; for referencing this chapter elsewhere, use \ref{ChapterX}

\lhead{Chapter 4. \emph{Design}} % Change X to a consecutive number; this is for the header on each page - perhaps a shortened title

\section{Design requirements}
% Motivation for design
The framework evaluation of Picosiblings provides insight as to how the scheme can be improved. We identify as a key downside that it does not guarantee the identity of the owner. This information is mainly inferred from the number of Picosibling shares in the proximity of the Pico. However, anyone may be in possession of the shares, therefore being temporarily granted full authentication privileges. This is reflected in the evaluation by failing to fully offer `resilient-to-theft'' and ``non-disclosability''. Another improvement can be made by introducing ``multi-level-unlocking'', allowing for multiple levels of authentication depending on the confidence in the owner's presence.

% Pico properties that need to be maintained
The Pico design proposed by Stajano \cite{stajano2011pico} claims two properties that need to be supported by the token unlocking mechanism: memory effortless authentication, and continuous authentication\footnote{Continuous authentication is defined by the ability to re-authenticate the user without the need for any physical effort.}. 

% TODO: maybe offer more details or what the requirements are
A requirement when designing the new Pico unlocking mechanism is to fully satisfy the the properties presented in this section.

\section{Proposed solution}
\label{propopsedsol}
% combine multiple authentication mechanisms
The idea explored in this dissertation is to simultaneously use multiple memoryless continuous authentication mechanisms. Each mechanism needs to provide a quantifiable confidence level that will be used in calculating a combined score. This satisfies the Pico design requirements. By combining mechanisms we achieve a higher confidence of correctly identifying the owner. Furthermore, given that each individual mechanism supports continuous authentication, using them simultaneously does not create any inconvenience for the owner.

\subsection*{Multi-level unlocking model}
% multi-level unlocking model
The Pico token should no longer enter a general locked or unlocked state. Its most important secret, the ``Pico Master Key'' should be kept in tamper resistant memory, and be accessible at all times. Using the overall score computed by the proposed mechanism, Pico should offer granular user authentication. Each app needs to be associated with a confidence level defined during the registration process. If the overall confidence of the token exceeds the app's confidence level, then it becomes ``unlocked'' for that specific app. All authentication sessions between Pico and apps need to be managed independently based on this model.

% examples  of authentication mechanisms
The scheme should achieve continuous authentication, while correctly identifying the owner of the token. Therefore, we have decided that authentication mechanisms combined in the scheme need to be based either on biometrics or behavioural analysis. Biometric features that can be used include iris, face, voice, and gait. Behavioural sources of data can be obtained from frequent GPS location, travel paths, wireless network connections, and others.

% how it is different than simple biometrics
The solution offered in this project is different from simply stating that Pico is using biometric data as an unlocking mechanism. The novelty in the design is based on how data is combined in order to compute the overall confidence level. 

\subsection*{Decaying weights}
% each mechanism has a weight
Each mechanism of the scheme is assigned a predefined initial weight based on the level of trust it offers in identifying the owner. This doesn't necessarily need to be related to the precision of the mechanism, but it would be a good indicator for choosing the value.

% decaying weights
Data samples captured for authentication are not always meaningful. For example, accelerometer values for gait recognition are only usable when the user is walking. Depending on how the sensors are integrated with the Pico, camera input for face recognition may not always capture a valid image. The confidence of each mechanism should therefore decrease in time from the last valid authentication sample. This introduces another original feature of this scheme, which is having a decaying weight. Each mechanism starts with a predefined initial value that is decreases in time until a valid user data sample is recorded. 

% example of decaying confidence
% 	TODO: can drop this if needed!
Let us take for example a voice recognition mechanism which samples data every minute. The current weight of the mechanism is 0 so its output is completely ignored. The next sample is recorded, and the voice recognition mechanism outputs a confidence of $70\%$ that the owner is present. After the successful recording, the mechanism weight is updated to its predefined starting value of 30. For the next 10 minutes the owner will be silently reading a book. Since the mechanism only identifies background noise, the weight value of 30 decreases in time. This will induce a smaller impact of the mechanism on the overall score. Each mechanism weight can decrease down to 0, at which point the mechanism is ignored. Computing the overall score will be explained in more detail later in the chapter.

\subsection*{Explicit authentication}
% Explicit authentication mechanisms
We need to consider the case where the owner wants to use Pico to authenticate to a high security app, given a low confidence level from the authenticator. As an example, the Pico owner wants to access their bank account after sitting silent in a dark room for the past hour. Let us say the app requires a confidence level of $95\%$. Due to the lack of valid authentication data, the authenticator only outputs a $20\%$ overall confidence that the owner is present. To solve this problem we have introduced the concept of explicit authentication mechanisms. When the confidence score drops below the threshold required by an app, the user is given the chance to provide valid data samples to one or more mechanisms through an explicit request.

% Combining explicit authentication mechanisms
Combining explicit and continuous authentication can be performed consistently with the current design. Whenever explicit authentication is required, the only difference is that the owner becomes aware of the authentication process. Given that prior to the explicit authentication request the unlocking mechanism didn't produce a high enough confidence, it is assumed that this will also happen prior to that. Therefore, explicit authentication requests need to have a slower decay rate. This will enable the continuous authentication process.

\subsection*{Authentication result}
\label{authfeedback}
% Bayesian update
Each mechanism calculates the probability that the data sample belongs to the owner of the token. After each recording, this probability is updated using Bayes' Law. The process is also known as a Bayesian update, and is descried in the following equation:

\begin{equation} 
\label{eq:bayes1}
P(H|E) = \frac{P(H) * P(E|H)}{P(E)}
\end{equation}

In the equation above:
\begin{itemize} 
	\item E: Stands for evidence and in this case represents the data sample.
	\item H: Stands for hypothesis. In this case we refer to the hypothesis that the owner is present.
	\item $P(H|E)$: Represents the probability of hypothesis $H$ after observing evidence $E$. This is the final probability we are trying to compute after each sample. It is also known as the posterior probability. 
	\item $P(H)$: Represents the probability of hypothesis $H$ before observing evidence $E$. This is also known as the prior probability and is the probability computed at the previous step.
	\item $P(E|H)$: Represents the probability that the current evidence belongs to hypothesis $H$. It is the probability outputted by the mechanism given the sample data.
	\item $P(E)$: This is the model evidence, and has a constant value for all hypotheses.
\end{itemize}

Although $P(E)$ is constant we need its value in order to calculate $P(H|E)$. We can compute it by using the ``Law of total probability'':

\begin{equation} 
\label{eq:lotp}
P(E) = \sum_{n}^{}P(E|H_n) * P(H_n)
\end{equation}

Using equation \ref{eq:lotp}, Bayes' Law \ref{eq:bayes1} can be written as:
\begin{equation} 
\label{eq:bayes2}
P(H|E) = \frac{P(H) * P(E|H)}{\sum_{n}^{}P(E|H_n) * P(H_n)}
\end{equation}

Our model contains only two hypotheses\footnote{Arguably there is a third case where the data sample is not a valid recording of an user. This case is ignored and no probability is computed. The only result in this scenario is the resuming of the decay process in the weight of the mechanism.}: the recording of the data either belongs to the owner, or not. We can therefore consider $P(H)$ to be the probability that the data belongs to the owner and $P(\neg H)$ the probability that the data belongs to someone else. This means the value of $P(\neg H)$ is $1 - P(H)$ and $P(E|\neg H) = 1 - P(E|H)$ Introducing this in equation \ref{eq:bayes2}, the rule for updating the mechanism's probability becomes:

\begin{equation} 
\label{eq:final}
P(H|E) = \frac{P(H) * P(E|H)}{P(H) * P(E|H) + P(\neg H) * P(E|\neg H)}
\end{equation}

Equation \ref{eq:final} represents the final probability that the owner is present given the sampled data. All the variables in this equation are known, for reasons explained above.

% Overall confidence
% 	TODO: update this to have wii and wid (decayed and initial), as well as above when describing the decay rate
We have defined how individual scores are calculated, and that each mechanism has a decaying weight. Using this data we can calculate the overall score of the scheme. This is performed by using the following modified weighted sum:

\begin{equation} 
\label{eq:overall}
P_{Total} = \frac{\sum_{i=1}^{n}(w_{id} * P_i(H|E_i))}{\sum_{i=1}^{n}w_i}
\end{equation}

In equation \ref{eq:overall}, $w_{id}$ represents the decayed weight of mechanism $i$, and $w_i$ is its original weight. We have chosen this model because in a scenario where the token has no sample data to collect, all mechanisms would decrease their weights simultaneously. Using a simple weighted sum, this would misleadingly provide a high overall result, even though all decayed weights would be low.  

\section{Related work}
Clarke et al \cite{clarke2005authentication} present statistics confirming the need for an unlocking scheme different from PINs. They conduct a couple of surveys trying to evaluate the reliability of a PIN in unlocking a mobile phone. The paper reveals a high number of bad practices involved in PIN authentication: reusing the PIN with other authenticators, forgetting the PIN, sharing the PIN with someone else, $45\%$ of owners never change the default factory code, $42\%$ only change it once after buying the device.

A promising result showed in the paper is that $83\%$ of users are willing to accept a biometric authentication mechanism to unlock their devices. The following biometric mechanisms were included in the study: fingerprint analysis, voice recognition, iris recognition, hand recognition, keystroke analysis \cite{clarke2003using}, and face recognition. The paper also shows that $61\%$ of users would accept an unobtrusive biometric continuous authentication mechanism. Using multiple biometrics for continuous authentication is mentioned briefly, but each mechanism is used individually based on what the user is doing. As an example, when the user walks he is authenticated using gait recognition, and while he is speaking on the phone, voice recognition.

In a different paper, Clarke et al \cite{clarke2002acceptance} study PIN alternatives for mobile phone unlocking. The authors conduct a survey with interesting results. A remarkable $11\%$ of participants were not aware of PIN authentication. An average of $81\%$ of participants agree that PINs should be replaced with a mechanism that provides better security. Although they report the need and desire for a different type of phone unlocking, many of them do not use currently available alternatives.

Gregory Williamson \cite{williamson2006enhanced} writes in his PhD dissertation  about the need for an enhanced security authentication mechanism for on-line banking. He proposes a multi-factor authentication model, and presents two interesting options: the traditional one where both mechanisms are required in the multi-factor model (blanket authentication), and one where the second authentication mechanism is only requested from the user if the transaction appears to be risky (risk mode authentication). A risky situation is defined as either an important transaction like withdrawing money, or a transaction made under unusual circumstances such as using an unknown device. 

Risk mode authentication is similar in concept with the explicit authentication request used in our proposed token unlocking scheme. Furthermore, Williamson shows that $75\%$ of users agree with having biometric authentication as a second factor authentication for passwords. This shows promising results in adopting our scheme for token unlocking purposes.

Elena Vildjiounaite et al describe in their paper \cite{vildjiounaite2007increasing} a similar authentication mechanism based on combining biometric authentication data on mobile phone devices. The authors explore an alternative to PINs based on a two stage ``risk mode authentication''. The first stage combines biometric data in order to achieve continuous authentication. This is achieved by training a cascade classifier to a target false acceptance rate (FAR)\footnote{The false acceptance rate is the equivalent of false positive precision. It is the probability of incorrectly granting authentication privileges to an user}. Data from mechanisms is merged using a weighted sum fusion rule. Mechanism weights are chosen based on their error rates. The second stage is only enabled if the cascade classifier does not identify the owner as being present. In low noise scenarios, continuous authentication is achieved without the need for an explicit challenge $80\%$ of the time. In noisy situations (city and car noise), the percentage drops ranging from $40$ to $60\%$. The cascade classifier was trained with a FAR of $1\%$, with results showing a false rejection rate (FRR)\footnote{The false rejection rate is the probability of incorrectly denying access to the rightful owner.} of only $3$ to $7\%$.

The paper by Elena Vildjiounaite et al \cite{vildjiounaite2007increasing} is similar with the solution proposed in this dissertation. It also combines multiple authentication mechanisms, each being assigned different weights. Differences between the two are the fact that weights are maintained static over time. The overall sum is computed differently, and there is no mention of Bayesian updates or probabilities. Furthermore, the authors use a classifier instead of producing a confidence level, which cannot be used for granting different levels of security. The results presented by this paper are however encouraging, showing that continuous authentication presents good results using multiple biometric authentication mechanisms.

\section{Conclusions}
% Conclusion
We have designed a new Pico unlocking mechanism that supports Pico's claims for continuous and memory effortless authentication. The scheme is guaranteed to improve on the existing Picosiblings solution at least by offering a better way of correctly identifying its owner.  

% TODO: sounds a bit bad..
An evaluation of the scheme is not yet offered because mechanisms such as ``Negligible-cost-per-user'' are implementation dependent. The next chapter will present a prototype solution. This offers a better definition of the scheme, that can be evaluated using the token unlocking assessment framework. The results will be compared with the current Picosiblings implementation allowing for further analysis and conclusions.


 
% Chapter Template

\chapter{Implementation Prototype} % Main chapter title
%TODO:
%	- make sure that I emphasize that it is up to the user to allow malware to read data from sensors!

\label{Chapter4} % Change X to a consecutive number; for referencing this chapter elsewhere, use \ref{ChapterX}

\lhead{Chapter 4. \emph{Implementation Prototype}} 
% Introduction for implementation
In this chapter we will develop a prototype for the scheme proposed in section \ref{propopsedsol}. Just as presented by Stajano \cite{stajano2011pico}, smart phones offer a cheap alternative to prototype and test. Therefore, we have chosen as an implementation platform the Android Nexus 5 hand held device. It offers enough sensors to perform biometric and behavioural analysis\footnote{The full range of sensors supported by the Android platform can be found here: http://developer.android.com/guide/topics/sensors/sensors\_overview.html (accessed on 28.05.2014)}. These resources will be used to demonstrate that the scheme can be implemented using similar dedicated hardware that may offer more security features.

% Android
\subsection{Android development and security}
% Introduction to android security and dev model
To gain a better understanding of different design decisions and limitations of our implementation, we will present a brief literature review of the Android development platform. Mechanisms and components will be described with an emphasis on security. The prototype developed for this dissertation is a proof of concept. However, we still aim to understand and make proper use of available security mechanisms. 

% First paper: introduction
William Enck et al \cite{enck2009understanding} offer a good introduction to Android application development. They focus on the security aspects of the development platform. It is a relatively old paper (2008), from the same year of the Android initial release. However, the fundamental design principles and security concepts that are discussed did not change considerably. The platform's open standards were made public in November 2007. This allowed researchers such as the authors of this paper to perform a pre-release analysis of the system.

% First paper: OS short description
Android uses as a core operating system a port of the Linux kernel. This introduces to the platform some of the Linux security mechanisms (i.e. file permissions, access control policies). On top of the kernel there is an application middleware layer composed out of the Java Dalvik virtual machine, core Java application libraries, as well as libraries which offer support for storage, sensors, display, and other device features. Applications are supported by the middleware and developed using the Android Java SDK.

% First paper: Android components
The Android development model is based on building an application from multiple components. Based on their purpose, the SDK defines four types: activity, service, content provider, and broadcast receiver. For the purpose of brevity we will not discuss each individual component\footnote{More details on the role of each component can be found on the Android website: http://developer.android.com/guide/components/fundamentals.html}. To allow meaningful interaction, Inter Component Communication (ICC) is enabled using special objects called Intents.

% First paper: binding services
The application we are developing needs to perform most of its processing in the background. It does not require any explicit user interaction. According to the Android model, this should be achieved using Services. To enable convenient component interaction, services may become bound engaging in a client-server communication. An important note made in the paper is that while a Service is bound, it cannot be terminated by an explicit stop action. This provides an useful guarantee regarding its lifetime, which we will use in the prototype.

% First paper: security enforcements, system level
The paper discusses two types of Android security enforcements: ICC, and system level. System level security is based on the Linux permission model. When installed, each app is allocated an UID and GID. This allows internal storage access control restrictions, keeping application data sandboxed from other apps.

% First paper: security enforcements, ICC
% 	TODO: too vague, can expand and cite something
ICC security is the main focus of the paper. Intent communication is based on commands sent to the ``/dev/binder'' device node. The node needs to be world readable and writeable by any application. Therefore, Android cannot mediate ICC using the Linux permissions model. Security relies on a Mandatory Access Control (MAC) framework enforced by a reference monitor. This mechanism validates requests sent to the ``/dev/binder'' node. 

% Manifest file
During development, each application needs to define a manifest file\footnote{Full details regarding the manifest file can be found on the Android website: http://developer.android.com/guide/topics/manifest/manifest-intro.html}. Some of the security configurations defined in this file are: declared components and their capabilities, permissions required by the app, and permissions other apps need to have in order to interact with app components. These entries are used as labels for the MAC framework. 

% First paper, types of components: public/private.
Using the app manifest file, each component can be defined as either public or private. This refinement is configured by the ``exported'' field. It defines whether or not another application may launch or interact with one of its components. When this paper was written, the ``exported'' field was defaulted to ``true''. However, as shown by Steffen and Mathias \cite{liebergeld2013android}  in 2013, starting with Android 4.2  the default of this value was changed to ``false'', and now conforms to the ``principle of least privilege''.

% First paper, Intent filters
Components listening for Intents need to have an intent-filter registered in the application manifest file. This allows them to export only a limited set of intents to other applications. Further restrictions to Intent objects are offered by the SDK using permission labels. This mechanism provides runtime security checks for the application. It is an additional prevention mechanism for data leaks through ICC. An application may broadcast an event throughout the system. By using permission labels, only apps that have the respective permission may process the event. Furthermore, Services may check for permissions when they are bound by another component. This allows them to expose different APIs depending on the binder.

% Paper two!
Steffen and Mathias \cite{liebergeld2013android} focus on deeper issues of the Android platform. They show how problems are solved from one Android version to the other. Unfortunately, OEMs tend not to update the software of their devices once they have shipped, which creates a high security risk.

The starting point of understanding Android security is learning how it is bootstrapped during the five step booting process:
\begin{enumerate}
	\item Initial bootloader (IBL) is loaded from ROM.
	\item IBL checks the signature of the bootloader (BL) and loads it into RAM.
	\item BL checks the signature of the linux kernel (LK) and loads it into RAM.
	\item LK initialises all existing hardware and starts the linux ``init'' process.
	\item The init process reads a configuration file and boots the rest of LA.
\end{enumerate}

The android security model is split by the paper in two categories: system security, and application security.

% keychain encryption and security
Android provides a keychain API used for storing sensitive material such as certificates and other credentials. These are encrypted using a master key, which is stored using AES encryption. Security needs to begin somewhere. An assumption has to be made about a state being secure from which multiple security extensions can be made. In this case, the master key is considered to be that point of security. However, given a rooted device, the master key itself can be retrieved from the system and therefore compromising all other credentials. The Android base system (libraries, app framework, and app runtime) is located in the ``system'' partition. Although it is writeable only by the root user, as mentioned before, exploits which grant this privilege exist. 

% Same author, shared privileges
From the user's perspective, an interesting ``feature'' which may affect the flow of information within Android is the fact that applications from the same author may share private resources. When installing an app the user needs to accept its predefined set of permissions. Due to resource sharing, a situation may present itself where an application that has permissions for the owner's contacts may communicate with an application that has permissions for internet in order to leak confidential data. A developer may therefore construct pairs of legitimate applications in order to mask a data flow attack.

% Android low level security
The Android OS offers a number of memory corruption mitigations in order to avoid buffer overflow attacks, or return oriented programming. The following list 
presents these low level security mechanisms:
\begin{itemize}
	\item Implements mmap\_min\_addr which restricts mmap memory mapping calls. This prevents NULL pointer related attacks.
	\item Implements XN (execute never) bit to mark memory as non-executable. The mechanism prevents attackers from executing remote code passed as data.
	\item Address space layout randomisation(ASLR) was implemented starting with Android 4.0. This is a first step to preventing return oriented programming attacks. The memory location of the binary library itself is however static. After a number of attempts using trial and error, the attacker may succeed using return oriented programming.
	\item Position independent and randomised linker (PIE) is implemented starting with Android 4.1 to support ASLP. This makes the memory location of binary libraries to be randomised.
	\item Read only relocation and immediate binding space (RELro) was implemented starting with Android 4.1. It solves an ASLR issue where an attacker could modify the global offset table (GOT) used when resolving a function from a dynamically linked library. Before this update an attacker may insert his own code to be executed using the GOT table.
\end{itemize}

% On device bouncer
A number of application security mechanisms are in place to make Android a safer environment for its users. A device program also known as the ``Bouncer'' prevents malware to be distributed from the Android App store (Google Play). The purpose of the bouncer is to verify apps prior to installation by checking for malware signatures and patterns. 

% Secure USB debugging
Secure USB debugging was introduced starting with Android 4.4.2. This only allows hosts registered with the device to have USB debugging permissions. The mechanism is circumvented if the user does not have a screen lock.

% The 4 big issues with android and malware
According to the paper, the Android OS is responsible for $96\%$ of mobile phone malware. The authors claim that this is the case due to 4 big issues of the Android platform:
\begin{enumerate}
	\item Security updates are delayed or never deployed. This is due to a number of approvals that an update needs to receive prior to deployment. This introduces an additional cost to the manufacturer (OEM), that does not generate any revenue. The majority of teams working on the Android platform are focusing on current releases. In most cases there are simply not enough resources to merge Google security updates to the OEM repository. Furthermore, the consequences of a failed OS update may cause ``bricking'' of the device, which is a huge risk for the manufacturer. All these issues lead to very few security updates. Therefore, important features such as RELro are never deployed, making older Android releases vulnerable.
	
	\item OEMs weaken the security of Android by introducing custom modifications before they roll out a device.
	
	\item The Android permission model is defective. According to Kelley et al \cite{kelley2012conundrum}, most users do not understand the permission dialogue when installing an application. Furthermore, even if they could understand the dialogue, most of the time it is ignored in order to use the exciting new app. According to the same study, most applications are over-privileged. This is due to developers not understanding what each privilege grants. Furthermore, as previously pointed out, apps developed by the same owner may share resources and implicitly privileges.
	
	\item Google Play has a low barrier for malware. A developer distribution agreement (DDA) and a developer program policy (DPP) need to be agreed to and signed by the developer before submitting the application to the Android market. However, Google Play does not check upfront if an application adheres to DDA and DDP. The application is only reviewed if it becomes suspect of breaking the agreements. Furthermore, according to \cite{percoco2012adventures} there are ways of circumventing the Bouncer program\footnote{An example of such an application is presented in an article written in Tech Republic: http://www.techrepublic.com/blog/google-in-the-enterprise/malware-in-the-google-play-store-enemy-inside-the-gates/\# (visited on 29.05.2014).}. 
\end{enumerate}

% Conclusion of the section, just a summary of what we presented
We have briefly presented the Android development model, existing mechanisms, and the security of the platform. Given this information we may proceed to present the prototype implementation of the token unlocking scheme. 

\section{Authenticator design}
% Introduction with UAService
The Android user authenticator prototype is designed to work as a bound service implemented in the UAService\footnote{The name of the class stands for User Authentication Service} class. The service collects data from each mechanism and computes the final authentication confidence level. The result is provided to clients either after an explicit request or through periodic broadcasts.

% Mention independent services for each mechanism and why
Each authentication mechanism may have a different requirement for sampling and processing data. As an example, voice recognition may gather optimal data during a phone call\footnote{Call events can be intercepted by registering a listener for the PHONE\_STATE event}, while face recognition when the phone screen is unlocked. Therefore, to enable more flexibility in the individual mechanisms' implementation, we have chosen each to be developed as an independent service.

% Management of authentication mechanisms
UAService communicates with the authentication mechanisms by binding their service. This allows message passing using a ServiceConnection object. On predefined time intervals UAService acquires the confidence level and weight of each mechanism. Using this data it then calculates the overall result according to the design in section \ref{authfeedback}. Feedback is sent back to each registered client for interpretation.

\section{Implementation details}
% Introduction (can expand)
This section presents the implementation of the scheme proposed in section \ref{propopsedsol}. The full source code of the prototype can be downloaded from from github: 

``https://github.com/cristiantoader/fyp-pico''.

% TODO: rename section (something with activity and services, or idk)
\subsection{Main application and services}
% UAService
The Android token unlocking scheme is implemented as a bound service in the UAService class. According to the Android documentation a bound service is the server in the client-server interface. It enables other components to send requests and receive responses by binding to it. It is developed as a regular service that implements the ``onBind()'' callback method to return an IBinder. 

% Keep UAService alive via startService().
According to the Android development API guide\footnote{http://developer.android.com/guide/components/bound-services.html} there are two independent scenarios describing the lifetime of a bound service, depending on how the service was started:
\begin{enumerate}
	\item If the service was not previously running, and a ``bindService()'' command is issued by a component, the service is kept alive for as long as clients are still bound. A client becomes unbound by calling ``unbindService()''.
	
	\item If the service is started using ``onStartCommand()'' it can only be stopped if it has no bound clients and an explicit request is made either via ``stopSelf()'' or ``stopService''. Unlike the previous case, its lifetime persists even with no bound components.
\end{enumerate}

% UAService and lifetime
The prototype we have developed takes into account the lifetime of bound services. The ``PicoMainActivity'' class calls ``startService()'' and ``bindService()'' to the ``UAService'' component. When ``PicoMainActivity'' gets sent to background and loses control of the screen, the service is not explicitly unbound. This should prevent other components from killing ``UAService''. 

% UAService, a safer yet riskier implementation
% 	TODO: rephrase the ending a bit
A safer alternative would be to create a root service. This requires modifications to the system partition. The process does not resume to simply gaining root privileges and making the modifications. The root directory is mounted as ramdisk, and therefore any direct changes will be reverted once the device is rebooted. In order to make persistent modifications, the user needs to alter the boot image and re-flash it on the device. The service needs to be compiled using the Android NDK C compiler. The binary has to be included in the system partition of the boot image in order to be accessible by the ``init'' process during start up. The ``init.rc'' configuration file used by ``init'' also needs to be configured to start the service.

% UAService, central node.
% 	TODO: I think I can drop this paragraph if the word count is exceeded
``UAService'' is a central node in the application. It gathers data from the authentication mechanisms, computes the overall confidence data, and sends feedback to the clients. 

In order to receive authentication updates, clients need to bind ``UAService''. Communication is enabled using the ``Messenger'' interface, which is the simplest way to perform Inter Process Communication (IPC). The ``Messenger'' queues all requests on a single thread, and therefore the application does not require to be thread safe. The following commands are exposed to clients such as Pico through the ``what'' parameter of the ``Message'' class:
\begin{description}
  \item[MSG\_REGISTER\_CLIENT] \hfill \\
  Used for registering a client for periodic broadcasts\footnote{An alternative implementation explored in the project was to have each client also register a confidence level using the ``arg1'' parameter of a ``Message''. In this case, the authenticator would only provide each client with a locked/unlocked result. However, this would shift the meaning of client to that of an authentication session, with state managed by the unlocking scheme. A client would therefore have multiple connections, requiring more IPC. Since all ``Messenger'' requests made to ``UAService'' are queued to a single thread, this would slow down the feedback process and possibly even lead to a denial of service attack. Therefore we have chosen to reduce the communication overhead, and have each client manage the status of its authentication sessions based on the confidence level provided by the unlocking scheme.} of the current unlocking confidence level. Feedback is provided at a fixed time interval of 1000ms.

  \item[MSG\_UNREGISTER\_CLIENT] \hfill \\
  Used for any application to unregister as a listener of this service. No additional parameters required.
  
  \item[MSG\_GET\_STATUS] \hfill \\
  Used by any application when an authentication request is needed. Although the service periodically broadcasts the authentication status to its clients, explicit requests may also be performed using this type of ``Message''.
\end{description}

% Introduction tu authentication mechanism services, UAService side
The communication between ``UAService'' and an authentication mechanism service is intermediated by an ``AuthMech'' object. Each ``AuthMech'' is responsible for interfacing the communication with its corresponding service. 

% Authentication mechanism side
From the authentication mechanism's perspective, each service needs to extend the ``AuthMechService'' abstract class. This standardises the communication with the ``AuthMech'' objects, and implicitly ``UAService''. This software engineering approach facilitates adding additional mechanisms with minimal changes to the original code.

Each ``AuthMechService'' is implemented as a bound service. When binded by ``UAService'' through ``AuthMech'' they expose the following message passing interface:
\begin{description}
  \item[AUTH\_MECH\_REGISTER] \hfill \\
  Used for registering the ``UAService'' client to the ``AuthMechService''.
  
  \item[AUTH\_MECH\_UNREGISTER] \hfill \\
  Used for unregistering the ``UAService'' client from the ``AuthMechService''.
  
  % TODO: might want to change this design!
  \item[AUTH\_MECH\_GET\_STATUS] \hfill \\
  Used by the ``UAService'' to request the authentication feedback from the ``AuthMechService''. The Message response has in ``arg1'' the authentication confidence multiplied by the decayed weight, and in ``arg2'' the original weight of the mechanism.
  \end{description}

\subsection{Authentication mechanisms}
\label{implauthmech}
% Introduction for authentication mechanisms
In order to create a functional prototype of the scheme, we have implemented a number user authentication mechanisms. We will not focus on the result quality of the biometric and behavioural mechanisms. Their sole purpose is to demonstrate that the design of the scheme is functional, and can be implemented using only a smart phone.

% Mechanism requirements (they are a bit logical rather than engineering)
When developing an authentication mechanism for the scheme, the following design requirements need to be satisfied: 
\begin{enumerate}
	\item The result needs to be quantifiable in the form of a percentage ranging from 0 to 100, where 100 means that the mechanism has $100\%$ confidence that the user is the owner of the token.
	\item The mechanism needs to support continuous authentication of the user.
	\item The authentication process needs to be effortless and preferably unobtrusive for the user.
\end{enumerate}

% Presenting examples of what we can port on android
% 	TODO: rephrase, make larger a bit
Android provides an extensive sensor API that can support the scheme. This can be used to develop a number of continuous authentication mechanisms.  We have listed the following non-exhaustive set of examples:
\begin{description}
  \item[Face recognition] \hfill \\
  The mechanism is based on capturing an image of the user's face and performing face recognition. Sampling valid face images can be performed without explicit requests by predicting user behaviour. We will use as an example an user that owns a phone with a front-facing camera. When the owner is unlocking the phone, there is a high probability that they will be looking towards the screen. This provides a good opportunity for the face recognition service to capture a valid sample. Using the Android API, this can be achieved by registering a ``BroadcastReceiver'' to listen for the one of the following events: ACTION\_SCREEN\_ON, ACTION\_SCREEN\_OFF, or ACTION\_USER\_PRESENT. The mechanism may continue to perform face recognition based on collected data and a previously recorded sample of the owner. A simple face recognition mechanism was also implemented as part of the prototype.
  
  % TODO: replay attacks are easy if using only features
  \item[Voice recognition] \hfill \\
  A voice recognition mechanism can record data either periodically, or based on Android events. It may then perform voice recognition and provide a confidence level of the owner being present. Voice sampling does not necessarily imply a voice password. An analysis can be performed using feature extraction. This facilitates the sampling process, which may be performed at any time. With a frequent sampling period, the owner of the device is likely to be recorded while speaking, which would provide a valid data sample. For even better confidence the mechanism can be implemented to start recording when a call is either made or received. On Android this can be achieved by listening for a PHONE\_STATE event. A simple voice recognition mechanism was implemented as part of the prototype.
  
  \item[Iris scanning] \hfill \\
  Similar to face recognition, this can be implemented by taking advantage of user behaviour while using the phone. When the phone is unlocked, the user is very likely to face the front camera, allowing for a good capture. The only problem with this mechanism is the quality of pictures offered by most phones. If the sampling quality is not sufficiently good, meaningful features from the iris may not be extracted. This would make the confidence level of the mechanism relatively low, but may change in the future as devices become increasingly performant.
  
  \item[Keystroke analysis] \hfill \\
  This mechanism was inspired from a paper by Clarke et al \cite{clarke2007authenticating}. The principle of keystroke analysis is based on the patterns in which the user types on his mobile phone. Different features can be extracted here, such as: letter sequence timings, words per minute, letters per minute, frequent used words, and others. Using this data a confidence level can be generated. 
  
  This mechanism is harder to implement using solely the Android SDK. A good starting point would be to have a keyboard app developed for the user that also communicates with the authentication mechanism. If the keyboard is disabled by an attacker this should be considered, especially if the authenticator was originally configured to listen for input.
  
  \item[Gait recognition] \hfill \\
  This mechanism is based on analysing individual walking patterns. According to data presented by Derawi et al \cite{derawi2010unobtrusive}, error rates\footnote{The performance indicator used in biometric analysis is the Equal Error Rate (EER).} may vary between $5\%$ to $20\%$.  Android offers native recognition support for walking, driving, or standing still. Applications can register a sensor callback for the TYPE\_STEP\_DETECTOR composite sensor. Whenever such an event is detected, data can be recorded from the accelerometer and validated using an algorithm similar to the one described by Derawi et al \cite{derawi2010unobtrusive}.
  
  \item[Ear shape analysis] \hfill \\
  Research shows (i.e. Burge et al \cite{burge1996ear}, Mu et al \cite{mu2005shape}) that the shape of the human ear contains enough unique features to perform biometric authentication. Taking advantage of user behaviour, valid data can be captured and analysed using a smart phone. We suggest that a picture is taken a few seconds after a phone call event is detected. If no peripherals are attached, the user is likely to move the device towards the ear. Images captured by such a mechanism could then be used to calculate an accurate confidence level of the user's identity. This method was not tested, so therefore we cannot ensure whether the auto-focus of the camera is sufficiently fast to obtain a valid image.
  
  \item[Proximity devices] \hfill \\
  This is an original idea based on providing a confidence level depending on the presence of known devices. The mechanism should connect with other devices that are also running the authenticator. The two owners don't necessarily need to know one another for the acknowledgement to be performed. Whether regular travel schedules, or working in an office, users are constantly being in the presence of other known devices. This should provide a confidence as to whether the device is in the presence of its owner. 
  
  The authentication works by seeking connections with other devices. Whenever a device is identified, its ID is recorded. The mechanism needs to keep track of the number of times it has connected with another device. Some connections may be established for the first time, and should not bring any confidence. Other connections, such as the Pico of a co-worker, would probably have a high number of connections, and therefore the mechanism should output a higher confidence level in its presence. This mechanism is similar to the Picosiblings solution, but with no k-out-of-n secrets. Each Pico is essentially a Picosibling for another Pico, with each device having a different weight based on familiarity.

  As an example, when travelling with your family on holiday most of the devices there are unknown. However, given that a number of frequent IDs are in the proximity of the authenticator, the mechanism should still consider to some extent that it is in the possession of its owner. 
  
  The mechanism can be circumvented in the scenario where co-workers or friends try to unlock the Pico. Due to this downside, it should never have sufficient weight to unlock the token on its own. However, in combination with other mechanisms it would provide a good approximation of whether it is in the possession of its owner. If the device is in good company there is a good chance the owner is also present. 
  
  \item[Location data] \hfill \\
  This mechanism is similar to ``Proximity devices'' and much easier to implement. Based on Android GPS and network location data, the phone may detect whether it is in an usual location or not. Just as ``Proximity devices'' this should not carry a high weight in the scheme, especially since it would not provide accurate results in scenarios such as holidays.
  
   \item[Service utilisation] \hfill \\
  This mechanism exploits patterns in the Android phone's service and app utilisation. Based on current running applications, services, and the time they were started we may create a model where some confidence is given regarding the ownership of the device. This mechanism would only be effective in detecting sudden changes. It would have a low weight in the overall scheme due to its lack in precision. 
  
  \item[Picosiblings]
  The original Picosiblings mechanism may also be used with this scheme. Although not part of the standard set of Android device sensors, if available, a Picosiblings implementation may be included as one of the authentication mechanisms.
\end{description}

% continuous mechanisms for explicit authentication
A number of continuous authentication mechanisms may also be used for explicit authentication. The user can be notified to provide accurate information for the following mechanism: face recognition, voice recognition, iris scanning, keystroke analysis, gait recognition, and ear shape analysis. This creates the opportunity for a valid data sample to be collected.

% explicit authentication mechanisms
A number of explicit authentication mechanisms which do not satisfy the continuous authentication property of Pico may be implemented for the Android platform. It is important to note that additional mechanisms not included in this list need to satisfy the memorywise-effortless property of the token unlocking framework (\ref{tokenframework}). We suggest the following mechanisms for implementation:
\begin{description}
  \item[Fingerprint scanner] \hfill \\
  Devices that incorporate a fingerprint scanner (such as the IPhone 5S) can use the sensor as an explicit authentication mechanism. It cannot be used for continuous authentication because the user doesn't come in contact with the sensors on a regular basis. A mechanism can therefore request explicit fingerprint data, which would then be compared with the owner's biometric model, outputting a confidence for the authentication. The result will be combined in the overall scheme just as any other mechanism. The the only difference will be in terms of weight and decay rate.
    
  \item[Hand writing recognition] \hfill \\
  The user may be prompted to use the touch screen in order to write a word of his choice. This would guarantee the memorywise-effortless property because the user doesn't need to remember any secret. The handwriting would be analysed with a preconfigured set of handwriting samples in order to compute the confidence level that the owner produced the input.
  
  \item[Lip movement analysis] \hfill \\
  According to Faraj and Bigun \cite{faraj2006motion}, analysing lip movement while speaking can be used for authentication. The user would be prompted to provide a data sample such as reading a word provided by the authenticator. Using lip movement authentication, a quantifiable confidence level would be produced. This mechanism can also be implemented as a continuous authentication mechanism. However, data sampling would likely have a low success rate as users tend not to have their mouth within the camera's field of view.
\end{description}

% Introduction for developed mechanisms
In order to have a functional prototype of the scheme, we have developed a number of authentication mechanisms. The following sections will confirm that the Android platform offers sufficient functionality for supporting the scheme. Furthermore, this should stand as proof that the design can be implemented on a dedicated Pico device with a similar set of sensors.

% Listing mechanisms with promises of future details
The following mechanisms have been implemented as part of the prototype: voice recognition, face recognition, location analysis, and a dummy mechanism used for testing. The following sections will provide details regarding their functionality and implementation process.

\subsubsection{Dummy mechanism}
%
% TODO: weight decay!
%============================
% Introduction: what it does, why it was developed
A dummy authentication mechanism was developed for testing the overall scheme. It produces random confidence levels within a predefined range, which provides a good controlled environment for testing purposes.

% General implementation detail
The mechanism was developed consistently with the application model. It is implemented in the ``AuthDummyService'' class, extending the ``AuthMechService'' abstract class. This makes it an independent bound service with a predefined communication interface.

% DAO
All authentication mechanism services have a data access object (DAO) responsible for interfacing with imported libraries and managing authentication data. In this case the DAO only produces random confidence levels within a given range. A thread belonging to the service makes periodic requests to the DAO. This mimics an authentication mechanism that periodically samples for data. The service confidence is updated based on the produced value. 

% sending data back to UAService
When ``UAService'' needs to update its overall confidence, it makes an AUTH\_MECH\_GET\_STATUS request to the service. The reply contains the most recent confidence level multiplied by the current decay factor, and the original weight of the mechanism. 

\subsubsection{Voice recognition}
% Introduction to mechanism
The voice recognition mechanism is implemented in the ``VoiceService'' class. It extends the ``AuthMechService'' abstract class that defines its communication interface. When the service is created, the ``onCreate()'' method is called automatically by the Android platform. The method was developed to start a thread that periodically gathers data from the microphone, performs biometric authentication, and produces a confidence level.

% TODO: decay weight

% Library used for implementation
% 	TODO: library repeats a bit
The library used for voice recognition is called Recognito\footnote{The library can be downloaded using github from the following link: https://github.com/amaurycrickx/recognito} and was developed by Amaury Crickx. It is a text independent speaker recognition library developed in Java. We do not claim that it is the best voice recognition library, but it was best suited for the purpose of this prototype. Porting the library for Android required minimal changes. It claims very good results in scenarios with minimal background noise\footnote{It was tested by the author on TED talks, such as:  https://www.ted.com/talks/browse (visited on 06.01.2014)}.

% Hack to compile with the library
In order for the application to compile the Recognito library, a subset of the rt.jar Java (SE) library was required. This is due to ``javax.sound.'' packages included in Recognito that are not available on Android. Trying to import and use the ``javax.sound.'' package is not possible due to the name collision with the ``javax.'' system library available on Android. Therefore, we had to include ``rt.jar'' as part of the application, but without actually using it. This was purely done to trick the Android Java compiler to package the application. Using ``javax.sound'' features would generate a runtime error. This was avoided by only using Recognito functions which require direct sound input, without any knowledge of sound file formats.

% Recording configuration
In order to gather and manage samples compatible with the Recognito library we have created the ``VoiceRecord'' class. This is responsible of gathering microphone input using the following predefined configuration:
\begin{itemize}
	\item Sample rate: 44100
	\item Channel configuration: AudioFormat.CHANNEL\_IN\_MONO
	\item Audio format: AudioFormat.ENCODING\_PCM\_16BIT
\end{itemize}

% More random recording stuff..
The minimum buffer size required by the ``VoiceRecord'' class is device dependant and pre-calculated in the constructor. The class wraps an ``AudioRecord'' object used for gathering microphone data. Due limitations of the SDK, the recording is saved as a file and loaded into memory when needed. This makes the implementation not efficient, but it does serve the purpose of the prototype.

% DAO
A DAO class is created to facilitate the interface to the Recognito library. When initialised, it loads the owner configuration, and a predefined set of background noises. It then creates a Recognito object and trains it using the data. This is done using the library's ``createVocalPrint'' public method.

% Getting confidence level
Every predefined time interval, the ``VoiceService'' authentication thread records data in ``double[]'' format using the ``VoiceRecord'' class. It then calls the ``recognize'' public method of the Recognito object. This returns the Euclidean distance to the closest match, which is either the owner, or one of the background noises used for training. 

% Getting closest match
To convert the Euclidean distance to a percentage confidence level, we define an acceptable threshold. Any result above the threshold is considered too high and is truncated to its value. Using the following formula we convert the Euclidean distance to a confidence level. The final result is $P(E|H)$ (the probability that the evidence belongs to the hypothesis) used in equation \ref{eq:final}.

$$P(E|H) = 1 - \frac{distance}{THRESHOLD}$$

% Probability computation estimate
Dividing the distance over the threshold yields a confidence value between 0 and 1, where 1 is a very large distance and hence a bad result. By using one minus this value we invert the meaning. Values will range between 0 and 1, and 1 corresponds to a confidence level of $100\%$. 

% Do Bayesian update
Having known $P(E|H)$ we continue to calculate $P(H|E)$ by using the Bayesian update formula defined in equation \ref{eq:final}. When calculating the final confidence level of the mechanism, we multiply $P(H|E)$ with the current decay rate. Due to the message passing API, this value needs to be an integer and is therefore multiplied by 100. The overall result is stored in the service and updated whenever the decaying weight is modified. When a request is made by UAService, the value is returned using the IBinder message passing mechanism.

\subsubsection{Face recognition}
\label{implface}
% Introduction: brief description of the implementation
The face recognition mechanism was implemented in the ``FaceService'' class, which extends the ``AuthMechService'' abstract class. When created, the service starts a thread that periodically collects data from the camera. Each sample is analysed using a face recognition library, and a confidence level is outputted for the current sample. The result is multiplied by the mechanism's weight which is a decaying factor.

% Javafaces library
For biometric face recognition we use a port of the ``Javafaces'' library \footnote{The ``JavaFaces'' library is maintained at the following address: https://code.google.com/p/javafaces/}. This was the closest functional library found that was compatible with the Android API. Javafaces is written entirely using Java (SE). Unfortunately, it makes use of the ``javax.imageio.'' package which is not available in the standard Android SDK. Therefore, a considerable amount of code needed to be ported for the Android platform. The new library is publicly available at the following link:https://github.com/cristiantoader/JavafacesLib. It is currently not optimised for public use.

% Javafaces changes for porting to Android
We will briefly present the changes made when porting the ``Javafaces'' library. The ``BufferedImage'' class had to be replaced by its closest Android equivalent, which is ``Bitmap''. All ``BufferedImage'' references in the original project had to be adapted. Furthermore, The API was modified to support direct ``Bitmap'' input in order to add more flexibility and lighten the main code of the authenticator. 

In the original ``Javafaces'' library, data formats for black and white images were assumed to have a single colour channel representing the grey value. This had to be changed to reflect the Bitmap convention, where all 3 colour channels are present but have the same value. Additional modifications were required due to data type mismatches, as well as other related issues.

% Authentication process
%	TODO: include that if no faces are found, the process is not performed
Every predefined time interval, the authentication thread running within the ``FaceService'' object samples data from the camera. This is performed by using a ``CameraUtil'' object. The ``CameraUtil'' class was developed as a mediator to simplify the interface to the Android ``Camera''. For example, it performs additional checks such as the orientation of the phone. 

A DAO class called ``FaceDAO'' was developed to mediate calls to the Android ``Javafaces'' library. Images captured using ``CameraUtil'' are validated using the DAO. The value returned from the ``Javafaces'' library is the Euclidean distance between collected data and the registered owner. This distance is handled in the exact same way as the voice recognition mechanism \ref{implface}.

% Gather picture with no notification: shutter sound
By default, the Android API does not easily allow for a Camera picture to be taken without any sort of notification to the user. Both a shutter sound and a visual preview display should be present. The shutter sound can be disabled easily by not providing a shutter callback function when calling the Camera.takePicture() method. 

% Gather picture with no notification: shutter sound
Disabling the user preview of the camera was more difficult to achieve. The solution used with this prototype was to exploit an Android feature that allows to render the preview in a ``SurfaceTexture'' object. This satisfies the API's requirement to have a visual display preview for the camera, while the ``SurfaceTexture'' itself does not need to be displayed on screen. Therefore a picture can be taken from a background service without any interruption to the user.

% Issue of image too large
Another problem encountered by the face recognition service is data sizes. When the ``Javafaces'' library performed its algorithm, the device was running out of memory. This caused the app to be closed by the Android OS. To fix this issue, Bitmaps collected from the camera are scaled to $50\%$ before they are processed by the library.

% Results
Unfortunately, the library combined with the Android SDK does not provide accurate results. The reason is that it requires as input a bitmap perfectly containing the face of an individual. Unfortunately, although the Android SDK offers face detection, it only provides the location of the midway coordinate between the eyes, and the distance between the eyes. Using this data alone, an accurate crop cannot be made. As a solution, yet another library would need to be used in order to properly detect face regions. This would provide better input data and would increase the precision of the mechanism.

\subsubsection{Location analysis}
% Introduction: short description of the mechanism
The mechanism is based on gathering location data and using it to generate a probability that the owner is present. This is implemented in the ``LocationService'' class that extends the ``AuthMechService'' abstract class. Data is collected periodically by using the ``LocationManager'' provided by the Android API.

% DAO object used for collecting data
A DAO object is used to mediate calls to the Android API and manage the existing owner configuration. It is implemented in the ``LocationDAO'' class. It offers functionality for gathering and saving location updates. It is developed to use the most accurate data provider. The Android API offers the following sources of collecting ``Location'' data:
\begin{itemize}
	\item GPS\_PROVIDER: Collects data from the GPS.
	\item NETWORK\_PROVIDER: Collects data from cell tower and WiFi access points.
	\item PASSIVE\_PROVIDER: Passively collects data from other applications which receive ``Location'' updates.
\end{itemize}

% Describe the algorithm class
External libraries were not used for the authentication process. We have developed a primitive location analysis algorithm in the ``LocationAnalyser'' class. During the configuration stage, which is a process managed by ``LocationActivity'', location data is sampled every 5 minutes and saved in internal storage. After the process has ended, each time a ``Location'' is sent for authentication it is compared with all the locations saved during the configuration process. The final result is the minimum Euclidean distance between the current ``Location'' and any other saved ``Location''. 

% Describe how authenticator thread works
When the service is started by ``UAService'', its ``onCreate()'' method spawns an authentication thread. This thread periodically requests the current location using the DAO. Data is returned in a ``Location'' object and is provided as input to the ``LocationAnalyser''. The result of this operation is an Euclidean distance which gets converted to a percentage using a threshold, just as in the previous mechanisms. The value is stored by the service for future requests from ``UAService''.

% Conclusion
Just as mentioned in section \ref{implauthmech}, the mechanism has a lower confidence level. Although being in a known location provides some confidence that the token was not stolen, it does not offer any guarantees that the device is still with its owner.

\subsection{Owner configuration}
% Owner configuration activities exist
There are a number of Activity components that are used in the configuration of the prototype. Each authentication mechanism has a corresponding Activity that can be started from the main Activity called ``PicoUserAuthenticator''. These are used to register owner biometrics needed by the mechanisms.

% They use DAO objects to store data in internal storage
Each configuration Activity uses the same DAO class as the mechanism Service. The DAO is used for storing the owner data once it was collected. Given that the overall size of the data is relatively small, the files are kept in internal storage. The Linux Android permissions mechanism guarantees its confidentiality and integrity, and therefore further encryption is not necessary.

\section{Conclusion}
% Small overall
We have described the Activity and Service components developed for the prototype, as well as their communication flow. We have ported two biometric libraries and developed a location analysis mechanism. DAO objects facilitate accessing owner configuration files and the interface with auxiliary mechanisms. An overview of the app design can be seen in figure \ref{TODO}.

% TODO: insertgraphics() overall design of the app.

% Limitations and solutions
One of the limitations of the prototype is the lack of explicit authentication mechanisms. Another issue is the precision of the biometric mechanisms, in the lack of better libraries. However, due to the modular design of the application, existing mechanisms can be improved simply by importing a new library and modifying its DAO. The existing set of mechanisms can easily be increased by creating a new class that extends ``AuthMechService'', and implementing the algorithm logic. In order to be managed by ``UAService'', the new mechanism needs to be included in the ``UserAuthenticator.initAvailableDevices()'' method.

% TODO: continue from here
\section{Related work}
% Sensor sniffing
Liang Cai et al \cite{cai2009defending} analyse ways of protecting users from mobile phone sensor sniffing attacks. The authors design a framework used for protecting sensor data from being leaked. From a security perspective the user should not to be trusted with granting permissions to different applications. An important point made in this paper is that malware may deny service to legitimate applications (such as our prototype) by creating a race condition for acquiring a sensor lock. The solution proposed by the authors would be an user notification, allowing for the owner to decide which application acquires the lock. A suggestion to this approach would be to allow for different priority levels, such that malware applications would not acquire the lock in a race condition, or even more, would lose it when a high priority application such as the Pico authenticator would require sensor data.

% Gait recognition
The paper by Derawi et al \cite{derawi2010unobtrusive} presents the feasibility of implementing gait authentication on Android as an unobtrusive unlocking mechanism. According to the definition offered by the authors ``gait recognition describes a biometric method which allows an automatic verification of the identity of a person by the way he walks''. The Android implementation developed by the authors has an equal error rate (EER) of $20\%$. Dedicated devices have an EER of only $12.9\%$, and the main cause for this is the sampling rate available at that time (2010). They have used a Google G1 phone with approximately 40-50 samples per second. This is much inferior to dedicated accelerometers that sample data at 100 samples per second. However, by conducting personal experiments with the accelerometer of a Google Nexus 5 phone, using the highest sampling setting (SENSOR\_DELAY\_FASTEST) the rates go above 100 samples per second. Therefore the current performance of the algorithm paper should be closer to $12.9\%$.

% Improve speaker recognition in noisy conditions
Ming et al \cite{ming2007robust} present in their paper how to improve speaker recognition accuracy on mobile devices in noisy conditions. This approach uses a model training technique based on which missing features may be used to identify noise. The focus of the paper is designing and implementing a biometric mechanism, and is therefore outside the scope of this dissertation project. 

% Voiceprints voice recognition
Another technique in performing speaker recognition involves using voiceprints. These are a set of features extracted from the speaker sample data. Kersta \cite{kersta2005voiceprint} explains the mechanism in more detail. The benefit of having feature extraction based on a voice sample, as opposed to a different voice recognition mechanism, is that voiceprints do not require any secrets. The speaker doesn't have to reproduce a voice sample. This increases the usability of the mechanism in scenarios required by the Pico authenticator. However, a downside to this approach is that it makes replay attacks easier to perform. Any recording of the user is sufficient for an attacker to trick the biometric mechanism.

% Face recognition
%	TODO: need to review the phrasing
A popular paper on face authentication was written by Turk and Pentland \cite{turk1991face}. The biometric authentication process is based on the concept of eigenfaces. Eigenfaces are a name given for the eigenvectors which are used to characterise the features of a face. These features are projected onto the feature space. Using Euclidean distances in the feature space, classification can be performed to correctly identify individuals. An implementation of this mechanism was developed for the Pico unlocking scheme prototype.

% Keystroke analysis
An unconventional authenticating mechanism is presented by Clarke and Furnell \cite{clarke2007authenticating}. They use keystroke analysis in order to make predictions regarding the user of the phone. This mechanism is unobtrusive and authenticates users during normal interactions such as typing a text message or a phone number. It is based on a neural network classifier, reporting an EER of $12.8\%$. Input data used for classification is composed out of timings between successive keystrokes, and the hold time of a pressed key. 






 
% Chapter Template
% TODO: do the evaluation!

\chapter{Evaluation} % Main chapter title

\label{Chapter6} % Change X to a consecutive number; for referencing this chapter elsewhere, use \ref{ChapterX}

\lhead{Chapter 6. \emph{Evaluation}} % Change X to a consecutive number; this is for the header on each page - perhaps a shortened title

% Brief summary of this chapter
This chapter presents an evaluation of the proposed token unlocking mechanism. We start by performing a threat model of the Android prototype. This should reveal any security limitations of the implementation, as well as the overall scheme. We continue by analysing the performance of the prototype and discuss how it can be improved.

Given a well defined implementation (\ref{Chapter5}), we assess the scheme using the token unlocking framework (\ref{tokenframework}), and compare the results with the Picosiblings solution. In order to check for overall improvements, we use the UDS framework to evaluate a Pico token that uses the proposed token unlocking scheme, and compare the results with the original work by Bonneau et al \cite{bonneau2012quest}.

\section{Threat model}
% % Introduction on what we want to achieve
% The purpose of the prototype was to provide a proof of concept that the scheme can be developed using existing hardware. However, we will perform a threat analysis in order to ensure that the implementation is secure. The main reason for this is the possibility of gaining additional insight to the scheme's design limitations. Furthermore, the assessment should reveal attack paths that need to be considered in future implementations this mechanism.

% Types of attacks we will be analysing
The threat analysis is performed from an availability, integrity, and confidentiality perspective. We consider the security mechanisms of the Android platform presented in appendix \ref{AppendixA} as predefined assumptions used in this model. Attack paths are analysed in different scenarios based on whether an attacker has physical access to the token or not. Although we are making a security assessment for a prototype developed on the commercial Android platform, similar issues may arise for a future implementation that uses dedicated hardware.

\subsubsection*{Availability}
% Availability: attacker has token
Breaking the scheme's availability while the device is in the possession of the attacker is relatively trivial. The application can be uninstalled, or the application data cache can be cleared, therefore removing the owner biometric models used by the individual mechanisms. Furthermore, in this scenario the owner is no longer in possession of their Pico, so basically the device is already unavailable.

% Availability: attacker does not have token
Let us continue and study what denial of service (DoS) exploits can be achieved by a remote attacker. Removing the owner configuration data from internal storage would make the authenticator unusable. This can only be achieved if the attacker (or a malware application designed by the attacker) manages to get root access on the device. Given the Linux permissions model, there would be no way to protect this data from deletion. However, without root access the application data cannot be accessed or modified.

% Availability: dos via sensor locking
Based on each device platform, multiple apps recording data from a sensor may not be possible. This can enable a DoS attack on the prototype by having malware locking sensors before the authenticator. This would make data collection impossible, and therefore the mechanisms' weights would gradually decrease to 0. The overall confidence level would be lowered, preventing the user from authenticating. After performing experiments, we can confirm this problem for the Google Nexus 5 smart phone when two applications try to record microphone data using different sampling rates \footnote{The apps were trying to record microphone data using the AudioRecord class; application one was using a sampling rate of 44100 and application two 22050.}.

% Availability: starting multiple connections
The current prototype is susceptible to a DoS attack caused by too many clients registered with the authenticator. Given that no permission is required to register to the ``UAService'' component, an unlimited number of connections can be made. Therefore, broadcasting the authentication status to each client may cause considerable delays. Furthermore, in order to unburden the developer from developing thread-safe code, ICC is performed on a single thread. This means that by spamming ``UAService'' with requests, the attacker can achieve a DoS attack for legitimate Pico clients. This can be fixed if we only allow access to application developed by the same author as the authenticator app.

\subsubsection*{Integrity}
% Integrity: root data modifications
The prototype stores data in internal storage, and cannot be accessed by other applications due to the Linux permissions mechanism. However, just as mentioned in the previous section, if the attacker gains root privileges it may modify any data on the device. The attack can be performed regardless of physical access.

% Integrity: Binder communication ok
From a data flow point of view, ICC is performed using the ``/dev/binder'' node driver. According to the official Android source code \footnote{For convenience, a link to the binder driver is found here: https://android.googlesource.com/kernel/common.git/+/android-3.0/drivers/staging/android/binder.c (visited on 06.02.2014).}, although the node is readable and writeable by any application, communication is performed using IOCTL calls. Data is transferred from one component to the other without the possibility to intercept or modify. Given that Android ICC is secure, data either from the sensors or from app components cannot be tampered.

\subsubsection*{Confidentiality}
% Confidentiality: same owner
Android apps may share private resources only if developed by the same author. This is determined by verifying the signature of the app, which is performed using a private key specific to each developer. Therefore, an attack where owner configuration data is leaked due to private resource sharing would only be possible if the attacker manages to acquire the private key that was used for signing the authenticator app. We will consider this to be a scenario outside the scope of the project.

% Confidentiality
Another case where owner authentication data can be accessed is having malware run with root privileges. This would allow an attacker the rights to read any application's data. However, the owner's biometric files would not be compromised, as they are kept encrypted using the Android Keychain API. Although the data can be read from internal storage, it cannot be interpreted in a meaningful way. Starting with Android 4.3, the Keychain API has hardware support, making the encryption keys non-extractable. 

% Keychain exposure
On Android versions earlier than 4.3, the following confidentiality attack path can be performed. Given root access, the attacker may retrieve the AES master key used by the keychain manager to store credentials. Using this key they can then retrieve the authenticator's application key used for encrypting owner configuration files. By retrieving this final key, the attacker may decode sensitive data (i.e. biometric data) and leak it outside of the system, therefore compromising confidentiality.

% Solution to encrypted data problem
A solution to this problem is not keeping the key used for decryption on the device. It should be generated on the app's first run, and communicated securely back to a credentials server. Whenever the authenticator app starts, it would request the key remotely via a secure connection, use it to decrypt owner authentication files, and discard it without saving.

% Data flow confidentiality
From a data flow perspective, ICC should offer full confidentiality. As previously mentioned, the ``/dev/binder'' device node used for ICC is managed by a driver which listens for ``ioctl'' requests. Data cannot be compromised as it is transferred from one component to the other. Only ``UAService is an ``exported'' component that may be accessed by other apps. It only provides final authentication feedback from the mechanisms, and only exposes a limited API through the IBinder interface object used for ``Message'' passing.

% Sensor data collection
The communication between the authenticator and Pico can be secured, preventing other applications to register for updates. This can be achieved using runtime permission label checks. Both Pico and its authenticator would need to define these permission in their manifest files. Additional checks would need to be added for ICC in ``UAService''.

% Sensor sniffing
Liang Cai et al \cite{cai2009defending} presents the problem of sensor sniffing. A malware application may collect all relevant data on its own from the user, using the same functionality as the prototype. This would allow for a powerful replay attack in the future. Adrienne Porter Felt et al \cite{felt2012android} show that when installing an app only $17\%$ of users pay attention to the Android permissions dialogue, and only $3\%$ understand what each permission represents.

% TODO: change this title
\subsubsection*{Design model attacks}
\label{secdesignattacks}
% Short introduction
Pico needs to be unlocked only in the presence of its owner. In order to do so, the token unlocking scheme needs to gather valid sample data. Let us consider a few scenarios and assess any design issues.
 
% Owner at desk with no data
The most unfavourable scenario is to have the owner silently sitting at work with their smart phone on the table. Voice and face recognition mechanisms would not gather any valid samples. Location data can be collected, offering some confidence that the owner is present. If authentication is required for a high security transaction, such as logging in to an online banking account, the overall score outputted by the scheme would not be high enough to grant access. The scheme would therefore make an explicit authentication request, offering the user the possibility to generate valid data.

% No owner and no data
The main problem with the scheme is not denying service to the owner, but falsely granting it. Given the same situation as before, let us assume the owner forgets the smart phone on their desk and leaves the office for a break. Just as before, only location data can be collected, providing some confidence that the owner is present. Ideally as the owner leaves, most authentication sessions managed by the Pico should be closed. This should happen once the confidence level becomes too low, and the explicit authentication mechanisms are ignored.

% Compromise solution
From the authenticator's perspective there is no difference between the two scenarios presented above. The first suggests that non biometric mechanisms should provide sufficient confidence to provide authentication when the owner ``goes silent''. The second scenario requires the opposite; when the owner can no longer provide biometric data, they are likely no longer with the token and Pico should lock. 

A compromise solution is needed for the two scenarios. Non-biometric mechanisms need to provide a confidence level that is almost sufficient to grant access to most medium-level security accounts. Periodically, explicit authentication requests will be made by the mechanism in order to provide a sufficiently high score. Given the decaying weights, the confidence level will gradually drop until another explicit authentication is required. The weights and decay rates need to be configured in such a way that the time interval between two explicit authentication requests is acceptable for the user, without compromising security. We suggest the time interval of 1 minute, but an user study would be more appropriate to determine this value.

% Alternative solution: heartbeat
An alternative solution to the problem presented above is to have an auxiliary biometric sensor that the owner would carry at all times. A good example is a heartbeat monitor. This can be embedded in an every day item such as a watch. The heartbeat authentication mechanism combined with the existing location analysis mechanism should provide a sufficiently high confidence level to unlock Pico for any medium-security authentication session.

% TODO!!
\section{Functional evaluation}
\label{functionaleval}
We have performed a series of tests in order to assess the usability of the scheme's prototype. 

Battery power is a scarce resource for hand-held devices. The amount of time they can function without recharging directly determines their availability. The token unlocking scheme we have proposed requires periodic sampling of sensor data. This can be a power consuming task, if not managed appropriately. Given that the prototype's individual mechanisms currently use a configurable sampling rate, we continue by analysing the power consumption changes with various intervals.

The power consumption analysis was performed using the Trepn profiling tool developed by Qualcomm\ref{The tool's official website can be found at the following address: https://developer.qualcomm.com/mobile-development/increase-app-performance/trepn-profiler}. The results are posted in table \ref{tab:powerprofile}. We analyse for different sampling rates the scheme's resource consumption while collecting and processing data. Using the profiler we gather the application's average CPU and power consumption. The tool only provides overall battery usage, which includes other Android components. Therefore, when calculating the average power consumed by the application, we deduct the base power consumption of $30.33 mW$ that was obtained by analysing the system without running the prototype. Although this introduces a margin of error, overall results should serve as a valid approximation.

\begin{table}
    \begin{tabular}{l|l|l|l}
    Test number & Sampling rate (s) & Average CPU (\%) & Average Power (mW) \\ \hline
    1           & 5                 & 10.61            & 313.17             \\
    2           & 10                & 4.68             & 279.48             \\
    3           & 15                & 4.04             & 192.61             \\
    4           & 20                & 3.27             & 81.57              \\
    5           & 25                & 2.62             & 111.93             \\
    6           & 30                & 2.34             & 72.86              \\
    \end{tabular}
	
	\caption{Application profiling results using Trepn}
	\label{tab:powerprofile}
\end{table}

Results from table \ref{tab:powerprofile} show a relatively high CPU and Power usage for a small sampling rate of 5 seconds. This drops considerably in the following tests. In order to have the prototype constantly functional, sampling rates need to be configured in conjunction with the decay process and initial weights of the mechanisms. This is a multivariate optimisation problem, where we are trying to minimise power consumption and the error rate of the device.

In order to assess the time performance scheme's authentication mechanisms, we have timed different sections in table \ref{}. It is important to note that this does not directly impact user experience. In the lack of fast data collection and authentication, previous results are used for unlocking.
\begin{table}
    \begin{tabular}{lll}
		Mechanism         & Initialisation & Authentication \\
		Voice recognition & ~              & ~              \\
		Face recognition  & ~              & ~              \\
		Location analysis & ~              & ~              \\
    \end{tabular}
	\caption{Application profiling results using Trepn}
	\label{tab:timeprofile}
\end{table}

% TODO: review this! it is very likely poorly worded
\section{Token unlocking framework evaluation}
We will continue by evaluating the proposed scheme with the token unlocking framework defined in section \ref{tokenframework}. 

% Usability
The scheme is ``memorywise-effortless'' because it doesn't require any secrets to provide authentication. The sensors used for authentication are embedded in the token, therefore offering ``nothing-to-carry''. It is also ``easy-to-learn'', as user authentication is performed non-obtrusively. As shown in section \ref{functionaleval}, although user authentication is performed in a timely manner, setting up the scheme may take some time. Therefore, the ``efficient-to-use'' property is only quasi-offered.  The scheme only quasi-offers ``infrequent errors'' because of the underlying biometric and behavioural authentication. Any differences in biometric features that may occur can be resolved by re-configuring the authenticator. The prototype does not have a well defined secure process for this task. In the lack of additional details we mark the scheme to only quasi-offer ``easy-recovery-from-loss''. As briefly shown in section \ref{secdesignattacks}, even in unfavourable scenarios the scheme may still provide authentication to the token. The ``availability'' property is therefore satisfied.

% Deployability
Given that multiple continuous authentication mechanisms are combined, the scheme offers ``accessibility'' to any user, regardless of disabilities. Since it is implemented as an Android app, it has a ``negligible-cost-per-user'' both for the owner and the developer. It is not ``mature'' since it has only been prototyped. The ``non-proprietary'' property is offered, as long as the individual mechanisms are developed using free to use algorithms and libraries.

% Security
From a security perspective the scheme is ``resilient-to-physical-observations''. If an attacker would have valid pre-recordings of the owner for all biometric mechanisms, they would still need to perform these in a location considered safe by the authenticator. Furthermore, the replays would have to be performed periodically in order to keep the authentication session alive. Therefore, due to the difficulty to perform a replay attack, the scheme is considered to be ``resilient-to-targeted-impersonation''. The ``resilient-to-throttled-guessing'', ``resilient-to-unthrottled-guessing'', and ``resilient-to-theft'' properties do not apply. The scheme is not ``unlinkable'' because biometric data is unique for each individual. In order to satisfy the Pico requirements, all mechanism involved in the token unlocking process support ``continuous-authentication''. The authenticator provides as final feedback a confidence level, which allows for ``multi-level-unlocking''. Intentional disclosure of authentication credentials would pose the same difficulties as a replay attack, and therefore the scheme offers ``non-disclosability''.

% Summary of results
The results are summarised in table \ref{table:tokenresults}. Column properties are highlighted to facilitate the comparison with the Picosiblings solution \footnote{The colours have the following meanings based on the result: green - offered, red - not offered, and yellow - quasi offered.}.

\begin{table}
    \begin{tabular}{l|l|l|l}
    Property                            & Picosiblings  						& Proposed scheme \\ \hline
    Memorywise-effortless               & \cellcolor{green!25}Offered       	& \cellcolor{green!25}Offered          	\\
    Nothing-to-carry                    & \cellcolor{yellow!25}Quasi-offered   	& \cellcolor{green!25}Offered          	\\
    Easy-to-learn                       & \cellcolor{red!25}Not-offered   		& \cellcolor{green!25}Offered          	\\
    Efficient-to-use                    & \cellcolor{yellow!25}Quasi-offered 	& \cellcolor{yellow!25}Quasi-offered   	\\
    Infrequent-errors                   & \cellcolor{yellow!25}Quasi-offered 	& \cellcolor{yellow!25}Quasi-offered   	\\
    Easy-recovery-from-loss             & \cellcolor{red!25}Not-offered   		& \cellcolor{yellow!25}Quasi-offered   	\\
    Availability                        & \cellcolor{green!25}Offered       	& \cellcolor{green!25}Offered      	   	\\ \hline
	
    Accessible                          & \cellcolor{green!25}Offered       	& \cellcolor{green!25}Offered          	\\
    Negligible-cost-per-user            & \cellcolor{red!25}Not-offered   		& \cellcolor{green!25}Offered          	\\
    Mature                              & \cellcolor{red!25}Not-offered   		& \cellcolor{red!25}Not-offered    		\\
    Non-proprietary                     & \cellcolor{green!25}Offered       	& \cellcolor{green!25}Offered      		\\ \hline
	
    Resilient-to-physical-observations  & \cellcolor{green!25}Offered       	& \cellcolor{green!25}Offered          	\\
    Resilient-to-targeted-impersonation & \cellcolor{green!25}Offered       	& \cellcolor{green!25}Offered      		\\
    Resilient-to-throttled-guessing     & \cellcolor{green!25}Offered       	& \cellcolor{green!25}Offered          	\\
    Resilient-to-unthrottled-guessing   & \cellcolor{green!25}Offered       	& \cellcolor{green!25}Offered          	\\
    Resilient-to-theft                  & \cellcolor{yellow!25}Quasi-offered   	& \cellcolor{green!25}Offered          	\\
    Unlinkable                          & \cellcolor{green!25}Offered       	& \cellcolor{red!25}Not-offered      	\\
    Continuous-authentication           & \cellcolor{green!25}Offered       	& \cellcolor{green!25}Offered      		\\
    Multi-level-unlocking               & \cellcolor{red!25}Not-offered   		& \cellcolor{green!25}Offered      		\\
    Non-disclosability                  & \cellcolor{red!25}Not-offered   		& \cellcolor{green!25}Offered    		\\
    \end{tabular}

	\caption{Token unlocking framework results compared with Picosiblings.}
	\label{table:tokenresults}

\end{table}

% Conclusion based on comparison
The proposed solution does not completely dominate Picosiblings. This is only because the scheme is not ``unlinkable''. It performs better by offering ``nothing-to-carry'' and quasi-offering ``easy-recovery-from-loss''. The prototype also has a ``negligible-cost-per-user'', which is something Picosiblings do not aim to achieve. In terms of security it is also better by offering the ``resilient-to-theft'', ``multi-level-unlocking'', and ``non-disclosability properties. 

In conclusion, we achieve our proposed goal of providing a solution that is better than Picosiblings in at least one property.

\section{UDS framework evaluation}
We now perform the reassessment of a Pico that uses our proposed token unlocking mechanism. The evaluation is performed using the UDS framework developed by Bonneau et al \cite{bonneau2012quest}. We will compare the result with the original Pico assessment in order to check for improvements. A summary is presented in table \ref{table:udsresults}.

\begin{table}
    \begin{tabular}{c|l|l|l}
    ~             & Property                                & Picosiblings  						& Proposed scheme \\ \hline
    Usability     & Memorywise-effortless                   & \cellcolor{green!25}Offered       	& \cellcolor{green!25}Offered         \\
    ~             & Scalable-for-users                      & \cellcolor{green!25}Offered       	& \cellcolor{green!25}Offered         \\
    ~             & Nothing-to-carry                        & \cellcolor{red!25}Not-offered   		& \cellcolor{red!25}Not-offered     \\
    ~             & Physically-effortless                   & \cellcolor{green!25}Offered       	& \cellcolor{green!25}Offered         \\
    ~             & Easy-to-learn                           & \cellcolor{red!25}Not-offered   		& \cellcolor{green!25}Offered         \\
    ~             & Efficient-to-use                        & \cellcolor{yellow!25}Quasi-offered 	& \cellcolor{yellow!25}Quasi-offered   \\
    ~             & Infrequent-errors                       & \cellcolor{yellow!25}Quasi-offered 	& \cellcolor{yellow!25}Quasi-offered   \\
    ~             & Easy-recovery-from-loss                 & \cellcolor{red!25}Not-offered   		& \cellcolor{red!25}Not-offered     \\ \hline
    Deployability & Accessible                              & \cellcolor{red!25}Not-offered   		& \cellcolor{red!25}Not-offered         \\
    ~             & Negligible-cost-per-user                & \cellcolor{red!25}Not-offered   		& \cellcolor{red!25}Not-offered     \\
    ~             & Server-compatible                       & \cellcolor{red!25}Not-offered   		& \cellcolor{red!25}Not-offered     \\
    ~             & Browser-compatible                      & \cellcolor{red!25}Not-offered   		& \cellcolor{red!25}Not-offered     \\
    ~             & Mature                                  & \cellcolor{red!25}Not-offered   		& \cellcolor{red!25}Not-offered     \\
    ~             & Non-proprietary                         & \cellcolor{green!25}Offered       	& \cellcolor{green!25}Offered         \\ \hline
    Security      & Resilient-to-physical-observations      & \cellcolor{green!25}Offered       	& \cellcolor{green!25}Offered         \\
    ~             & Resilient-to-targeted-impersonation     & \cellcolor{green!25}Offered       	& \cellcolor{green!25}Offered         \\
    ~             & Resilient-to-throttled-guessing         & \cellcolor{green!25}Offered       	& \cellcolor{green!25}Offered         \\
    ~             & Resilient-to-unthrottled-guessing       & \cellcolor{green!25}Offered       	& \cellcolor{green!25}Offered         \\
    ~             & Resilient-to-internal-observaions       & \cellcolor{green!25}Offered       	& \cellcolor{green!25}Offered         \\
    ~             & Resilient-to-leaks-from-other-verifiers & \cellcolor{green!25}Offered       	& \cellcolor{green!25}Offered         \\
    ~             & Resilient-to-phising                    & \cellcolor{green!25}Offered       	& \cellcolor{green!25}Offered         \\
    ~             & Resilient-to-theft                      & \cellcolor{yellow!25}Quasi-offered 	& \cellcolor{green!25}Offered         \\
    ~             & No-trusted-third-party                  & \cellcolor{green!25}Offered       	& \cellcolor{green!25}Offered         \\
    ~             & Requiring-explicit-consent              & \cellcolor{green!25}Offered       	& \cellcolor{green!25}Offered         \\
    ~             & Unlinkable                              & \cellcolor{green!25}Offered       	& \cellcolor{red!25}Not-offered     \\
    \end{tabular}
	
	\caption{UDS framework assessment.}
	\label{table:udsresults}
\end{table}

% How Pico is improved
The UDS framework assessment shows similar results to the token unlocking framework. By using the new scheme, Pico achieves an overall better score. It now offers ``easy-to-learn'', as it no longer requires Picosibling secret share management. In the lack of a cost analysis, we will consider that even with the new scheme the ``negligible-cost-per-user'' property is not offered. By relying on more than auxiliary devices, a Pico that uses the proposed scheme does offer ``resilient-to-theft''.

% The downsides and conclusion
The only property where Picosiblings outperforms the scheme presented in this dissertation is ``unlinkable''. Unfortunately, this trade-off cannot be fixed	as the mechanisms combined in the scheme need to rely on biometrics and behavioural analysis, which are unique for each individual. 

In conclusion, by using the new token unlocking mechanism, overall Pico would improve 2 properties in exchange for one.

\section{Future work}
% Introduction to future work
The token unlocking mechanism proposed in this project offers a new perspective to Pico unlocking. The assessment presented in this chapter shows that it offers a reliable alternative to Picosiblings. However, improvements can be made both to its design and implementation.

% Experiment with weights and decay factors
The Android prototype was developed as a proof of concept. Further experiments need to be performed using different weights and decay functions. An user study is required in order to determine the acceptable time interval between consecutive explicit authentication requests, and the implementation needs to be adapted accordingly. Furthermore, explicit authentication mechanisms need to be developed for the Android prototype.

% Add aditional mechanisms and improve accuracy of current.
The set of individual mechanisms used with the scheme's prototype can be improved. Better biometric libraries should be either developed or imported in order to increase the accuracy of the implementation. Furthermore, additional mechanisms should be developed for the platform. A number of viable suggestions are made in appendix \ref{AppendixC}.

% Take data samples based on external events
With the current prototype, the voice and face recognition mechanisms sample data at fixed time intervals. This should be change by taking advantage of user behaviour and Android events. Examples for this were given in section \ref{impleoverview}. 

The face recognition mechanism can be improved either by introducing another library that performs face detection, or by using a different face recognition library that offers both features. Cryptographic support needs to be added for this mechanism.  It can be performed through additional modifications of the ``Javafaces'' library that would allow it to use raw data during the training process.

% Safer implementation
A safer prototype would be to develop a root system service using the Android NDK C compiler. The binary has to be included in the system partition of the boot image in order to be accessible by the ``init'' process during start up. The ``init.rc'' configuration file used by ``init'' also needs to be configured to start the service. This implementation requires modifications to the ``/system'' partition. The process does not resume to simply gaining root privileges. The root directory ``/.'' is mounted as ramdisk, and therefore any modifications will be reverted once the device is rebooted. In order to make persistent changes, the user needs to modify the boot image, and re-flash it on the device.

\section{Conclusions}
The purpose of this dissertation was to create a new scheme for unlocking the Pico token. We have adapted the UDS framework developed by Bonneau et al \cite{bonneau2012quest} to create a token unlocking framework. Both frameworks were used in evaluating the solution proposed in this dissertation. Results have shown that although the new scheme does not completely outperform Picosiblings, they offer a larger number of benefits. The new token unlocking mechanism was prototyped on an Android smart phone, proving that the design can be implemented using existing hardware.


%=========================================================================================================================
% END OF PAPER
%=========================================================================================================================
 
% Chapter Template

\chapter{Conclusion} % Main chapter title

\label{Chapter7} % Change X to a consecutive number; for referencing this chapter elsewhere, use \ref{ChapterX}

\lhead{Chapter 7. \emph{Conclusion}} % Change X to a consecutive number; this is for the header on each page - perhaps a shortened title

The purpose of this dissertation was to provide an alternative unlocking scheme for the Pico token \cite{stajano2011pico}. By analysing its design in chapter \ref{Chapter2}, we have concluded that an alternative mechanism requires to be memorywise effortless, and provide support for continuous authentication.

We have identified and briefly presented the web authentication UDS assessment framework developed by Bonneau et al \cite{bonneau2012quest}. This work provided an initial evaluation of Pico using Picosiblings. Analysing the paper revealed issues of the Pico token, some of which directly related to its unlocking mechanism (e.g. not easy to learn).

To have a better way of evaluating a token unlocking scheme, in section \ref{tokenframework} we have created a derivation of the original framework by Bonneau et al. Some properties were removed, and others were changed in order to fit the context of token unlocking mechanisms. Furthermore, we have extended the framework by adding four original properties.

Having a list of requirements, and a way of assessing the new solution, we have designed a scheme based on combining biometric and behavioural analysis mechanisms. Each mechanism generates a probability that the owner is in possession of the token. These probabilities are combined using a modified weighted sum, in order to generate an overall confidence level. Another original contribution is that each mechanism has an initial weight that decays in time from one valid data sample to the other. 

% android prototype
Given a well defined design of the new scheme, we have successfully developed a prototype using a Google Nexus 5 device that runs an Android 4.4.2 operating system. This has proven that the solution can be implemented using existing hardware. Furthermore, the prototype offered useful insight for future implementations of the scheme. We have concluded that an efficient implementation requires authentication mechanisms to run as independent processes. We have presented a basic application design, and showed how different components should interact. As presented, data should be validated prior to analysis, and in the lack of a valid sample, the weight decay process should continue. An unexpected problem was that although the Android platform offers a wide range of sensors for the implementation of multiple mechanisms (details in appendix \ref{AppendixC}), the lack of open source biometric libraries has lead to a low precision of the overall scheme.

% evaluation of scheme
The proposed unlocking mechanism was evaluated using the UDS framework developed by Bonneau et al, and the token unlocking framework developed in section \ref{tokenframework}. The results of the analysis have shown that the new proposed scheme cannot completely outperform Picosiblings due to the ``unlinkable'' property. Otherwise, an overall improvement is achieved in the number of offered properties. In addition, the new unlocking mechanism offers the possibility of having granular unlocking, where the Pico token could offer individual locked and unlocked states based on the current confidence level and the security level required by the account.

A threat model of the prototype has shown a number of attacks that may be performed on the Android application. Important insight was provided when studying design model attacks in section \ref{secdesignattacks}. This has shown that in scenarios where no valid data can be collected, a compromise needs to be made regarding the time interval between successful explicit authentication requests. Together with the power consumption analysis from section \ref{functionaleval}, this essentially becomes a multivariate optimisation problem where we need to minimise power consumption and user inconvenience, while maximising accuracy.

\section*{Future work}
As shown in the previous section, the new token unlocking scheme offers Pico an overall improvement. However, additional work is required in order to improve details of the design, as well as the prototype. Additional details are presented in this section. 

% Experiment with weights and decay factors
The Android prototype was developed as a proof of concept. Further experiments need to be performed using different weights and decay functions. A user study is required in order to determine the acceptable time interval between consecutive explicit authentication requests, and the implementation needs to be adapted accordingly. When performing this analysis, the power consumption results from section \ref{functionaleval} need to be considered, in order to improve the lifetime of the device.

% Add aditional mechanisms and improve accuracy of current.
The set of individual mechanisms used with the scheme's prototype can be improved. Explicit authentication mechanisms are not currently supported and need to be implemented. Better biometric libraries should be either developed or imported in order to increase accuracy. Furthermore, additional mechanisms should be added for the platform. A number of viable suggestions are made in appendix \ref{AppendixC}.

% Take data samples based on external events
With the current prototype, the voice and face recognition mechanisms sample data at fixed time intervals. This should be change by taking advantage of user behaviour and Android events. Examples for this were given in section \ref{impleoverview}. 

The face recognition mechanism can be improved either by introducing another library that performs face detection, or by using a different face recognition library that offers both features. Cryptographic support needs to be added for this mechanism. It can be performed through additional modifications of the {\tt Javafaces} library, that would allow it to use raw binary data during the training process.

% Safer implementation
A safer prototype would be to develop a root system service using the Android NDK C compiler. The binary has to be included in the system partition of the boot image in order to be accessible by the {\tt init} process during start up. The {\tt init.rc} configuration file used by {\tt init} also needs to be configured to start the service. This implementation requires modifications to the {\tt /system} partition. The process does not limit to simply gaining root privileges. The root {\tt /} directory is mounted as ramdisk, and therefore any modifications will be reverted once the device is rebooted. In order to make persistent changes, the boot image needs to be modified and re-flashed on the device.
 

%----------------------------------------------------------------------------------------
%	THESIS CONTENT - APPENDICES
%----------------------------------------------------------------------------------------

\addtocontents{toc}{\vspace{2em}} % Add a gap in the Contents, for aesthetics

\appendix % Cue to tell LaTeX that the following 'chapters' are Appendices

% Include the appendices of the thesis as separate files from the Appendices folder
% Uncomment the lines as you write the Appendices

% Appendix A

\chapter{Android development and security} % Main appendix title

\label{AppendixB} % For referencing this appendix elsewhere, use \ref{AppendixA}

\lhead{Appendix B. \emph{Android development and security}} % This is for the header on each page - perhaps a shortened title

% Introduction to android security and dev model
To gain a better understanding of different design decisions and limitations of our implementation, we will present a brief literature review of the Android development platform. Mechanisms and components will be described with an emphasis on security.

% First paper: introduction
William Enck et al \cite{enck2009understanding} offer a good introduction to Android application development. They focus on the security aspects of the development platform. It is a relatively old paper (2008), from the same year of the Android initial release. However, the fundamental design principles and security concepts that are discussed did not change considerably. The platform's open standards were made public in November 2007. This allowed researchers such as the authors of this paper to perform a pre-release analysis of the system.

% First paper: OS short description
Android uses as a core operating system a port of the Linux kernel. This introduces to the platform some of the Linux security mechanisms (i.e. file permissions, access control policies). On top of the kernel there is an application middleware layer composed out of the Java Dalvik virtual machine, core Java application libraries, as well as libraries which offer support for storage, sensors, display, and other device features. Applications are supported by the middleware and developed using the Android Java SDK.

% First paper: Android components
The Android development model is based on building an application from multiple components. Depending on their purpose, the SDK defines four types: activity, service, content provider, and broadcast receiver. For the purpose of brevity we will not discuss each individual component\footnote{More details on the role of each component can be found on the Android website: http://developer.android.com/guide/components/fundamentals.html}. To allow meaningful interaction, Inter Component Communication (ICC) is enabled using special objects called Intents.

% First paper: binding services
The application we are developing needs to perform most of its processing in the background. It does not require any explicit user interaction. According to the Android model, this should be achieved using Services. To enable convenient component interaction, services can be bound engaging in a client-server communication. An important note made in the paper is that while a Service is bound, it cannot be terminated by an explicit stop action. According to the Android development API guide \footnote{http://developer.android.com/guide/components/bound-services.html} there are two independent scenarios describing the lifetime of a bound service:
\begin{enumerate}
	\item If the service was not previously running, and a ``bindService()'' command is issued by a component, the service is kept alive for as long as clients are still bound. A client becomes unbound by calling ``unbindService()''.
	
	\item If the service is started using ``onStartCommand()'' it can only be stopped if it has no bound clients and an explicit request is made either via ``stopSelf()'' or ``stopService''. Unlike the previous case, its lifetime persists even with no bound components.
\end{enumerate}


% First paper: security enforcements, system level
The paper discusses two types of Android security enforcements: ICC, and system level. System level security is based on the Linux permission model. When installed, each app is allocated an UID and GID. This allows internal storage access control restrictions, keeping application data sandboxed from other apps.

% First paper: security enforcements, ICC
ICC security is the main focus of the paper. Intent communication is based on commands sent to the ``/dev/binder'' device node. The node needs to be world readable and writeable by any application. Therefore, Android cannot mediate ICC using the Linux permissions model. Security relies on a Mandatory Access Control (MAC) framework enforced by a reference monitor. This protection is implemented by the driver responsible for processing IOCTL calls for the ``/dev/binder'' node. 

% Manifest file
During development, each application needs to define a manifest file\footnote{Full details regarding the manifest file can be found on the Android website: http://developer.android.com/guide/topics/manifest/manifest-intro.html}. Some of the security configurations defined in this file are: declared components and their capabilities, permissions required by the app, and permissions other apps need to have in order to interact with app components. These entries are used as labels for the MAC framework. 

% First paper, types of components: public/private.
Using the app manifest file, each component can be defined as either public or private. This refinement is configured by the ``exported'' field. It defines whether or not another application may launch or interact with one of its components. When this paper was written, the ``exported'' field was defaulted to ``true''. However, as shown by Steffen and Mathias \cite{liebergeld2013android}  in 2013, starting with Android 4.2  the default of this value was changed to ``false'', and now conforms to the ``principle of least privilege''.

% First paper, Intent filters
Components listening for Intents need to have an intent-filter registered in the application manifest file. This allows them to export only a limited set of intents to other applications. Further restrictions to Intent objects are offered by the SDK using permission labels. This mechanism provides runtime security checks for the application. It is an additional prevention mechanism for data leaks through ICC. An application may broadcast an event throughout the system. By using permission labels, only apps that have the respective permission may process the event. Furthermore, Services may check for permissions when they are bound by another component. This allows them to expose different APIs depending on the binder.

% Paper two!
Steffen and Mathias \cite{liebergeld2013android} focus on deeper issues of the Android platform. They show how problems are solved from one Android version to the other. Unfortunately, OEMs tend not to update the software of their devices once they have shipped, which creates a high security risk.

The starting point of understanding Android security is learning how it is bootstrapped during the five step booting process:
\begin{enumerate}
	\item Initial bootloader (IBL) is loaded from ROM.
	\item IBL checks the signature of the bootloader (BL) and loads it into RAM.
	\item BL checks the signature of the linux kernel (LK) and loads it into RAM.
	\item LK initialises all existing hardware and starts the linux ``init'' process.
	\item The init process reads a configuration file and boots the rest of LA.
\end{enumerate}

The android security model is split by the paper in two categories: system security, and application security.

% keychain encryption and security
Android provides a keychain API used for storing sensitive material such as certificates and other credentials. These are encrypted using a master key, which is stored using AES encryption. Security needs to begin somewhere. An assumption has to be made about a state being secure from which multiple security extensions can be made. In this case, the master key is considered to be that point of security. However, given a rooted device, the master key itself can be retrieved from the system and therefore compromising all other credentials. The Android base system (libraries, app framework, and app runtime) is located in the ``system'' partition. Although it is writeable only by the root user, as mentioned before, exploits which grant this privilege exist. 

% Same author, shared privileges
From the user's perspective, an interesting ``feature'' which may affect the flow of information within Android is the fact that applications from the same author may share private resources. When installing an app the user needs to accept its predefined set of permissions. Due to resource sharing, a situation may present itself where an application that has permissions for the owner's contacts may communicate with an application that has permissions for internet in order to leak confidential data. A developer may therefore construct pairs of legitimate applications in order to mask a data flow attack.

% Android low level security
The Android OS offers a number of memory corruption mitigations in order to avoid buffer overflow attacks, or return oriented programming. The following list 
presents these low level security mechanisms:
\begin{itemize}
	\item Implements mmap\_min\_addr which restricts mmap memory mapping calls. This prevents NULL pointer related attacks.
	\item Implements XN (execute never) bit to mark memory as non-executable. The mechanism prevents attackers from executing remote code passed as data.
	\item Address space layout randomisation(ASLR) was implemented starting with Android 4.0. This is a first step to preventing return oriented programming attacks. The memory location of the binary library itself is however static. After a number of attempts using trial and error, the attacker may succeed using return oriented programming.
	\item Position independent and randomised linker (PIE) is implemented starting with Android 4.1 to support ASLP. This makes the memory location of binary libraries to be randomised.
	\item Read only relocation and immediate binding space (RELro) was implemented starting with Android 4.1. It solves an ASLR issue where an attacker could modify the global offset table (GOT) used when resolving a function from a dynamically linked library. Before this update an attacker may insert his own code to be executed using the GOT table.
\end{itemize}

% On device bouncer
A number of application security mechanisms are in place to make Android a safer environment for its users. A device program also known as the ``Bouncer'' prevents malware to be distributed from the Android App store (Google Play). The purpose of the bouncer is to verify apps prior to installation by checking for malware signatures and patterns. 

% Secure USB debugging
Secure USB debugging was introduced starting with Android 4.4.2. This only allows hosts registered with the device to have USB debugging permissions. The mechanism is circumvented if the user does not have a screen lock.

% The 4 big issues with android and malware
According to the paper, the Android OS is responsible for $96\%$ of mobile phone malware. The authors claim that this is the case due to 4 big issues of the Android platform:
\begin{enumerate}
	\item Security updates are delayed or never deployed. This is due to a number of approvals that an update needs to receive prior to deployment. This introduces an additional cost to the manufacturer (OEM), that does not generate any revenue. The majority of teams working on the Android platform are focusing on current releases. In most cases there are simply not enough resources to merge Google security updates to the OEM repository. Furthermore, the consequences of a failed OS update may cause ``bricking'' of the device, which is a huge risk for the manufacturer. All these issues lead to very few security updates. Therefore, important features such as RELro are never deployed, making older Android releases vulnerable.
	
	\item OEMs weaken the security of Android by introducing custom modifications before they roll out a device.
	
	\item The Android permission model is defective. According to Kelley et al \cite{kelley2012conundrum}, most users do not understand the permission dialogue when installing an application. Furthermore, even if they could understand the dialogue, most of the time it is ignored in order to use the exciting new app. According to the same study, most applications are over-privileged. This is due to developers not understanding what each privilege grants. Furthermore, as previously pointed out, apps developed by the same owner may share resources and implicitly privileges.
	
	\item Google Play has a low barrier for malware. A developer distribution agreement (DDA) and a developer program policy (DPP) need to be agreed to and signed by the developer before submitting the application to the Android market. However, Google Play does not check upfront if an application adheres to DDA and DDP. The application is only reviewed if it becomes suspect of breaking the agreements. Furthermore, according to \cite{percoco2012adventures} there are ways of circumventing the Bouncer program\footnote{An example of such an application is presented in an article written in Tech Republic: http://www.techrepublic.com/blog/google-in-the-enterprise/malware-in-the-google-play-store-enemy-inside-the-gates/\# (visited on 29.05.2014).}. 
\end{enumerate}

% Conclusion of the section, just a summary of what we presented
We have briefly presented the Android development model, existing mechanisms, and the security of the platform. This information should be sufficient to understand the principles involved in the design of the prototype developed for this dissertation project.


% Appendix Template

\chapter{Token Unlocking Framework evaluation examples} % Main appendix title

\label{AppendixA} % Change X to a consecutive letter; for referencing this appendix elsewhere, use \ref{AppendixX}

\lhead{Appendix A. \emph{Token evaluation examples}} % Change X to a consecutive letter; this is for the header on each page - perhaps a shortened title

The following sections present examples of how the token unlocking framework should be used. We will be assessing PINs, and biometric face unlock. Together with the Picosiblings evaluation in section \ref{picosiblingseval}, each scheme represents a different type of authentication method. Picosiblings essentially are a secret the owner has, PINs are a secret the owner knows, and Face-unlock reflects who the owner is. 

%
%	PIN
%=======================================================
%
\section{PIN}
% Introduction to PINs and resemblance to passwords
PINs are token authentication mechanisms similar to passwords. The difference between the two is that they use a smaller set of input characters. Additional protection comes from steep security measures when the authentication challenge has failed. As an example, typing 3 wrong PINs on a mobile phone would lock the owner's SIM card. A lot of the PIN properties should however be similar with those offered by passwords.
	
% Usability: PINs
The scheme relies on knowing a secret, which is not ``memorywise-effortless''. It does however offer the ``nothing-to-carry'' property. Because of its similarity with passwords users find it ``easy-to-learn''. The small character set allows for fast user input and validation making PINs ``efficient-to-use''. Mistakes however may still occasionally occur, and due to the lack of visual feedback \footnote{If existent, visual feedback for PINs generally consists of `*' characters.} the scheme only quasi-offers ``infrequent-errors''. PINs are generally easily reset by the manufacturer using online services, therefore having ``easy-recovery-from-loss'' \footnote{An example of this is the RSA SecurID. An example reset procedure is described at the following link: http://uk.emc.com/collateral/15-min-guide/h12278-am8-help-desk-administrator-guide.pdf}. The scheme offers the ``availability'' property, as the authentication process cannot be impaired by external factors.
	
% Deployability: PINs
Just as passwords PINs score all points in deployability. They can be used regardless of disabilities, making them ``accessible''. They have virtually no cost, satisfying the ``negligible-cost-per-user'' property. Being a subset of passwords, we consider the mechanism to be ``mature'' and ``non-proprietary''.
	
% Security: PINs
From a security perspective PINs score poorly. They are not ``resilient-to-physical-observation''. Anyone can eavesdrop the input of a PIN either by shoulder surfing or recording with a camera. Just as passwords, PINs are often written down in plain sight. However, in the lack of relevant studies\footnote{Just as Bonneau et al suggest \cite{bonneau2012quest}, a relevant study would assess acquaintances' ability to guess the PIN of a subject.} we will mark the scheme to quasi-offer ``resilient-to-targeted-impersonation''. The restricted character set makes PINs adopt harsher security policies when provided invalid input. They are generally locked after three bad attempts, making them ``resilient-to-throttled-guessing''. The ``resilient-to-unthrottled-guessing'' property is implementation dependent. However, security tokens are dedicated devices that generally have tamper resistant memory, making unthrottled guessing not possible. Any hardware PINs may require does not compromise the mechanism, therefore offering ``resilient-to-theft''. Users have the freedom of choosing any PIN. Even in situations when reused with multiple tokens, credentials are generally salted and therefore ``unlinkable''. The scheme does not offer ``continuous-authentication'' because the process is not effortless for the user. They can only provide locked or unlocked feedback, and therefore do not offer ``multi-level-unlocking''. The owner may disclose their PIN at any time, making the ``non-disclosability'' property unsatisfied. 
	
%
%	Android face unlock
%=======================================================
%
\section{Face unlock}
Although not currently used as a security token unlocking mechanism, face recognition is a viable biometric authentication scheme. It can be ported for a token such as Pico, which is designed to have a camera. With a variety of possible implementations, for accessibility reasons we will analyse the Android face unlocking mechanism.
	
% Face unlock: usability
Face unlock is ``memorywise-effortless'', as any other biometric scheme. It offers the ``nothing-to-carry property'', the camera being embedded as part of the token. The mechanism is ``easy-to-learn'', since it only needs the user to look at the camera. The authentication process is performed almost instantly, making the scheme ``efficient-to-use''. The scheme is dependent on camera positioning, obstructing objects (e.g. glasses, earrings), and face mimic. In conjunction with the UDS framework assessment of biometrics in general, the scheme does not offer ``infrequent-errors''. If the scheme no longer functions as a result of change in facial traits, Android has a backup unlocking mechanism. This may also be used to disable or recalibrate the scheme, therefore offering ``easy-recovery-from-loss''. The ``availability'' property is not satisfied due to the dependence on external factors such as light or obstacles.
	
% Face unlock: deployability
Android face recognition is ``accessible'' for anyone regardless of disabilities. It offers the ``negligible-cost-per-user'' property, given that the hardware was already present in devices without face recognition features. Due to limited user exposure it is only quasi-``mature''. On Android, the scheme is implemented as not ``non-proprietary''.
	
% Face unlock: security
Observing the owner authenticate does not provide any advantage to an attacker. It therefore offers the ``resilient-to-physical-observations'' property. Targeted impersonation is an issue with any biometric mechanism. The scheme is vulnerable to replay attacks (i.e. a picture of the owner's face) and therefore does not offer ``resilient-to-targeted-impersonation''. The ''resilient-to-throttled-guessing`` and ``resilient-to-unthrottled-guessing'' properties do not apply. Given the Android implementation, neither does ``resilient-to-theft''. The same authentication data is used with any verifier, and therefore the ``unlinkable'' property is not offered. The scheme is implemented without ``continuous-authentication'' or ``multi-level-unlocking'' although both can be supported by biometric mechanisms. Given the possibility of deliberately providing data for a replay attack, the scheme only quasi-offers the ``non-disclosability'' property.

%
%	TODO: can add fingerprint unlock - IPhone
%=======================================================
%
% Appendix Template

\chapter{Examples of supported Android authentication mechanisms} % Main appendix title

\label{AppendixC} % Change X to a consecutive letter; for referencing this appendix elsewhere, use \ref{AppendixX}

\lhead{Appendix C. \emph{Example authentication mechanisms}}

% Presenting examples of what we can port on android
% 	TODO: rephrase, make larger a bit
Android provides an extensive sensor API that can support the token unlocking scheme proposed in section \ref{propopsedsol}. This can be used to develop a number of continuous authentication mechanisms.  We have listed the following non-exhaustive set of examples:
\begin{description}
  \item[Face recognition] \hfill \\
  The mechanism is based on capturing an image of the user's face and performing face recognition. Sampling valid face images can be performed without explicit requests by predicting user behaviour. We will use as an example an user that owns a phone with a front-facing camera. When the owner is unlocking the phone, there is a high probability that they will be looking towards the screen. This provides a good opportunity for the face recognition service to capture a valid sample. Using the Android API, this can be achieved by registering a ``BroadcastReceiver'' to listen for the one of the following events: ACTION\_SCREEN\_ON, ACTION\_SCREEN\_OFF, or ACTION\_USER\_PRESENT. The mechanism may continue to perform face recognition based on collected data and a previously recorded sample of the owner. A simple face recognition mechanism was also implemented as part of the prototype.
  
  % TODO: replay attacks are easy if using only features
  \item[Voice recognition] \hfill \\
  A voice recognition mechanism can record data either periodically, or based on Android events. It may then perform voice recognition and provide a confidence level of the owner being present. Voice sampling does not necessarily imply a voice password. An analysis can be performed using feature extraction. This facilitates the sampling process, which may be performed at any time. With a frequent sampling period, the owner of the device is likely to be recorded while speaking, which would provide a valid data sample. For even better confidence the mechanism can be implemented to start recording when a call is either made or received. On Android this can be achieved by listening for a PHONE\_STATE event. A simple voice recognition mechanism was implemented as part of the prototype.
  
  \item[Iris scanning] \hfill \\
  Similar to face recognition, this can be implemented by taking advantage of user behaviour while using the phone. When the phone is unlocked, the user is very likely to face the front camera, allowing for a good capture. The only problem with this mechanism is the quality of pictures offered by most phones. If the sampling quality is not sufficiently good, meaningful features from the iris may not be extracted. This would make the confidence level of the mechanism relatively low, but may change in the future as devices become increasingly performant.
  
  \item[Keystroke analysis] \hfill \\
  This mechanism was inspired from a paper by Clarke et al \cite{clarke2007authenticating}. The principle of keystroke analysis is based on the patterns in which the user types on his mobile phone. Different features can be extracted here, such as: letter sequence timings, words per minute, letters per minute, frequent used words, and others. Using this data a confidence level can be generated. 
  
  This mechanism is harder to implement using solely the Android SDK. A good starting point would be to have a keyboard app developed for the user that also communicates with the authentication mechanism. If the keyboard is disabled by an attacker this should be considered, especially if the authenticator was originally configured to listen for input.
  
  \item[Gait recognition] \hfill \\
  This mechanism is based on analysing individual walking patterns. According to data presented by Derawi et al \cite{derawi2010unobtrusive}, error rates\footnote{The performance indicator used in biometric analysis is the Equal Error Rate (EER).} may vary between $5\%$ to $20\%$.  Android offers native recognition support for walking, driving, or standing still. Applications can register a sensor callback for the TYPE\_STEP\_DETECTOR composite sensor. Whenever such an event is detected, data can be recorded from the accelerometer and validated using an algorithm similar to the one described by Derawi et al \cite{derawi2010unobtrusive}.
  
  \item[Ear shape analysis] \hfill \\
  Research shows (i.e. Burge et al \cite{burge1996ear}, Mu et al \cite{mu2005shape}) that the shape of the human ear contains enough unique features to perform biometric authentication. Taking advantage of user behaviour, valid data can be captured and analysed using a smart phone. We suggest that a picture is taken a few seconds after a phone call event is detected. If no peripherals are attached, the user is likely to move the device towards the ear. Images captured by such a mechanism could then be used to calculate an accurate confidence level of the user's identity. This method was not tested, so therefore we cannot ensure whether the auto-focus of the camera is sufficiently fast to obtain a valid image.
  
  \item[Proximity devices] \hfill \\
  This is an original idea based on providing a confidence level depending on the presence of known devices. The mechanism should connect with other devices that are also running the authenticator. The two owners don't necessarily need to know one another for the acknowledgement to be performed. Whether regular travel schedules, or working in an office, users are constantly being in the presence of other known devices. This should provide a confidence as to whether the device is in the presence of its owner. 
  
  The authentication works by seeking connections with other devices. Whenever a device is identified, its ID is recorded. The mechanism needs to keep track of the number of times it has connected with another device. Some connections may be established for the first time, and should not bring any confidence. Other connections, such as the Pico of a co-worker, would probably have a high number of connections, and therefore the mechanism should output a higher confidence level in its presence. This mechanism is similar to the Picosiblings solution, but with no k-out-of-n secrets. Each Pico is essentially a Picosibling for another Pico, with each device having a different weight based on familiarity.

  As an example, when travelling with your family on holiday most of the devices there are unknown. However, given that a number of frequent IDs are in the proximity of the authenticator, the mechanism should still consider to some extent that it is in the possession of its owner. 
  
  The mechanism can be circumvented in the scenario where co-workers or friends try to unlock the Pico. Due to this downside, it should never have sufficient weight to unlock the token on its own. However, in combination with other mechanisms it would provide a good approximation of whether it is in the possession of its owner. If the device is in good company there is a good chance the owner is also present. 
  
  \item[Location data] \hfill \\
  This mechanism is similar to ``Proximity devices'' and much easier to implement. Based on Android GPS and network location data, the phone may detect whether it is in an usual location or not. Just as ``Proximity devices'' this should not carry a high weight in the scheme, especially since it would not provide accurate results in scenarios such as holidays.
  
   \item[Service utilisation] \hfill \\
  This mechanism exploits patterns in the Android phone's service and app utilisation. Based on current running applications, services, and the time they were started we may create a model where some confidence is given regarding the ownership of the device. This mechanism would only be effective in detecting sudden changes. It would have a low weight in the overall scheme due to its lack in precision. 
  
  \item[Picosiblings]
  The original Picosiblings mechanism may also be used with this scheme. Although not part of the standard set of Android device sensors, if available, a Picosiblings implementation may be included as one of the authentication mechanisms.
\end{description}

% continuous mechanisms for explicit authentication
A number of continuous authentication mechanisms may also be used for explicit authentication. The user can be notified to provide accurate information for the following mechanisms: face recognition, voice recognition, iris scanning, keystroke analysis, gait recognition, and ear shape analysis. This creates the opportunity for a valid data sample to be collected.

% explicit authentication mechanisms
A number of explicit authentication mechanisms which do not satisfy the continuous authentication property of Pico may be implemented for the Android platform. It is important to note that additional mechanisms not included in this list need to satisfy the memorywise-effortless property of the token unlocking framework (\ref{tokenframework}). We suggest the following mechanisms for implementation:
\begin{description}
  \item[Fingerprint scanner] \hfill \\
  Devices that incorporate a fingerprint scanner (such as the IPhone 5S) can use the sensor as an explicit authentication mechanism. It cannot be used for continuous authentication because the user doesn't come in contact with the sensors on a regular basis. A mechanism can therefore request explicit fingerprint data, which would then be compared with the owner's biometric model, outputting a confidence for the authentication. The result will be combined in the overall scheme just as any other mechanism. The the only difference will be in terms of weight and decay rate.
    
  \item[Hand writing recognition] \hfill \\
  The user may be prompted to use the touch screen in order to write a word of his choice. This would guarantee the memorywise-effortless property because the user doesn't need to remember any secret. The handwriting would be analysed with a preconfigured set of handwriting samples in order to compute the confidence level that the owner produced the input.
  
  \item[Lip movement analysis] \hfill \\
  According to Faraj and Bigun \cite{faraj2006motion}, analysing lip movement while speaking can be used for authentication. The user would be prompted to provide a data sample such as reading a word provided by the authenticator. Using lip movement authentication, a quantifiable confidence level would be produced. This mechanism can also be implemented as a continuous authentication mechanism. However, data sampling would likely have a low success rate as users tend not to have their mouth within the camera's field of view.
\end{description}

\addtocontents{toc}{\vspace{2em}} % Add a gap in the Contents, for aesthetics

\backmatter

%----------------------------------------------------------------------------------------
%	BIBLIOGRAPHY
%----------------------------------------------------------------------------------------

\label{Bibliography}

\lhead{\emph{Bibliography}} % Change the page header to say "Bibliography"

\bibliographystyle{unsrtnat} % Use the "unsrtnat" BibTeX style for formatting the Bibliography

\bibliography{Bibliography} % The references (bibliography) information are stored in the file named "Bibliography.bib"

\end{document}  