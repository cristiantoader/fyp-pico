%%%%%%%%%%%%%%%%%%%%%%%%%%%%%%%%%%%%%%%%%
% Masters/Doctoral Thesis 
% LaTeX Template
% Version 1.42 (19/1/14)
%
% This template has been downloaded from:
% http://www.latextemplates.com
%
% Original authors:
% Steven Gunn 
% http://users.ecs.soton.ac.uk/srg/softwaretools/document/templates/
% and
% Sunil Patel
% http://www.sunilpatel.co.uk/thesis-template/
%
% License:
% CC BY-NC-SA 3.0 (http://creativecommons.org/licenses/by-nc-sa/3.0/)
%
% Note:
% Make sure to edit document variables in the Thesis.cls file
%
%%%%%%%%%%%%%%%%%%%%%%%%%%%%%%%%%%%%%%%%%

%----------------------------------------------------------------------------------------
%	PACKAGES AND OTHER DOCUMENT CONFIGURATIONS
%----------------------------------------------------------------------------------------

\documentclass[11pt, a4paper, oneside]{Thesis} % Paper size, default font size and one-sided paper
%TC:group tabular 1 1

\graphicspath{{Pictures/}} % Specifies the directory where pictures are stored

\usepackage{url}
\usepackage[table]{xcolor}
\usepackage[square, numbers, comma, sort&compress]{natbib} % Use the natbib reference package - read up on this to edit the reference style; if you want text (e.g. Smith et al., 2012) for the in-text references (instead of numbers), remove 'numbers' 
\hypersetup{urlcolor=blue, colorlinks=true} % Colors hyperlinks in blue - change to black if annoying
\title{\ttitle} % Defines the thesis title - don't touch this

\begin{document}

\frontmatter % Use roman page numbering style (i, ii, iii, iv...) for the pre-content pages

\setstretch{1.3} % Line spacing of 1.3
\setlength{\skip\footins}{14pt}

% Define the page headers using the FancyHdr package and set up for one-sided printing
\fancyhead{} % Clears all page headers and footers
\rhead{\thepage} % Sets the right side header to show the page number
\lhead{} % Clears the left side page header

\pagestyle{fancy} % Finally, use the "fancy" page style to implement the FancyHdr headers

\newcommand{\HRule}{\rule{\linewidth}{0.5mm}} % New command to make the lines in the title page

% PDF meta-data
\hypersetup{pdftitle="User Authentication for Pico"}
\hypersetup{pdfsubject="User Authentication for Pico"}
\hypersetup{pdfauthor="Cristian M. Toader"}
\hypersetup{pdfkeywords=\keywordnames}

%----------------------------------------------------------------------------------------
%	TITLE PAGE
%----------------------------------------------------------------------------------------

\begin{titlepage}
\begin{center}

\textsc{\LARGE University of Cambridge}\\[1.5cm] % University name
\textsc{\Large Master's Thesis}\\[0.5cm] % Thesis type

\HRule \\[0.4cm] % Horizontal line
{\huge \bfseries User Authentication for Pico: \\ When to unlock a security token}\\[0.4cm] % Thesis title
\HRule \\[1.5cm] % Horizontal line
 
\begin{minipage}{0.4\textwidth}
\begin{flushleft} \large
\emph{Author:}\\
{Cristian M. Toader\\ Churchill College} % Author name - remove the \href bracket to remove the link
\end{flushleft}
\end{minipage}
\begin{minipage}{0.4\textwidth}
\begin{flushright} \large
\emph{Supervisor:} \\
Dr Frank Stajano
\end{flushright}
\end{minipage}\\[3cm]
 
\large \textit{A dissertation submitted for the degree of\\``Master of Philosophy'' in Advanced Computer Science}\\[0.3cm] % University requirement text
\textit{(Option B - Research project)}\\[0.4cm]
Computer Security Group \\ Computer Laboratory \\[2cm] 
 
{\large \today}\\[4cm] % Date
%\includegraphics{Logo} % University/department logo - uncomment to place it
 
\vfill
\end{center}

\end{titlepage}

%----------------------------------------------------------------------------------------
%	DECLARATION PAGE
%	Your institution may give you a different text to place here
%----------------------------------------------------------------------------------------

\Declaration{

\addtocontents{toc}{\vspace{1em}} % Add a gap in the Contents, for aesthetics

I, Cristian Toader of Churchill College, being a candidate for the M.Phil. in Advanced Computer Science, hereby declare that this report and the work described in it are my own work, unaided except as may be specified below, and that the report does not contain material that has already been used to any substantial extent for a comparable purpose.

Total word count: 14931 (starting from page 1 excluding appendix sections and bibliography) \footnote{Word count was computed using the command: {\tt texcount -opt=tcoptions.txt Chapters/*.tex}.}.

Signed:\\
\rule[1em]{25em}{0.5pt} % This prints a line for the signature
 
Date:\\
\rule[1em]{25em}{0.5pt} % This prints a line to write the date
}

\clearpage % Start a new page

%----------------------------------------------------------------------------------------
%	QUOTATION PAGE
%----------------------------------------------------------------------------------------
%
%\pagestyle{empty} % No headers or footers for the following pages
%
%\null\vfill % Add some space to move the quote down the page a bit
%
%\textit{``Thanks to my solid academic training, today I can write hundreds of words on virtually any topic without possessing a shred of information, which %is how I got a good job in journalism."}
%
%\begin{flushright}
%Dave Barry
%\end{flushright}
%
%\vfill\vfill\vfill\vfill\vfill\vfill\null % Add some space at the bottom to position the quote just right
%
%\clearpage % Start a new page

%----------------------------------------------------------------------------------------
%	ABSTRACT PAGE
%----------------------------------------------------------------------------------------

\addtotoc{Abstract} % Add the "Abstract" page entry to the Contents

\abstract{\addtocontents{toc}{\vspace{1em}} % Add a gap in the Contents, for aesthetics
Passwords are currently the most widely used authentication mechanism. However, as shown in the literature numerous times, they have become obsolete for the current technological context. The Pico project designed by Stajano \cite{stajano2011pico} was created with the purpose of completely replacing passwords. It is a security token used for generating credentials and providing authentication for the user. It has an additional layer of security, by only being usable in the presence of its owner. This is currently achieved using auxiliary devices called Picosiblings. The scope of this dissertation was to create a new token unlocking mechanism that would offer Pico a different perspective to detecting the presence of its owner. 

We have started by analysing the Pico design, and identifying any requirements for the new unlocking mechanism. In order to have a reliable way of assessing the solution, we have selected an assessment framework developed by Bonneau et al \cite{bonneau2012quest}. Furthermore, we have created a token unlocking assessment framework adapted from a subset of the work by Bonneau et al, to which we have added additional properties.

Having identified the set of requirements, as well the evaluation criteria, we have considered that a solution based on biometrics would be most appropriate. We have designed a token unlocking scheme that combines multiple biometric and behavioural analysis mechanisms in order to generate an overall confidence level. The solution was prototyped using an Android 4.4.2 smart-phone device, therefore proving that it can be developed using existing hardware. Additional threat modelling, and power analyses offer additional insight regarding the limitations of the scheme.

Both the token unlocking and Bonneau et al \cite{bonneau2012quest} frameworks were used for assessing the proposed solution. Comparing the results with those of Picosiblings shows that we do not completely outperform the original solution, but we do offer an overall better result.

\clearpage % Start a new page

%----------------------------------------------------------------------------------------
%	ACKNOWLEDGEMENTS
%----------------------------------------------------------------------------------------

\setstretch{1.3} % Reset the line-spacing to 1.3 for body text (if it has changed)

\acknowledgements{\addtocontents{toc}{\vspace{1em}} % Add a gap in the Contents, for aesthetics

In writing this dissertation, I would like to thank my supervisor, Dr Frank Stajano. I am especially grateful for providing me with with such an interesting research topic, and including me as part of the Pico team. I appreciate all the advice and the learning environment that he has created during our weekly discussions. Furthermore, his patience and useful suggestions while writing this dissertation are invaluable.

Without the help of the University of Cambridge Computer Laboratory, I would not have been able to finish this dissertation. They have provided me with all the support necessary while conducting the research part of the project. I am especially grateful for the creative discussions and suggestions provided by the team of Pico research assistants during our regular meetings. I would also like to thank Laurent Simon, for invaluable advice while developing the Android prototype.

I appreciate the hard work and dedication of the open source community. In particular I would like to thank the developers of the Javafaces and Recognito libraries, without which the prototype for this dissertation could not have been completed.

Finally, I would like to thank Stefan Saftescu and Leila Malika for proofreading and suggestions made for a draft of my dissertation. As a non-native speaker, their advice has considerably improved the quality of my project.

}
\clearpage % Start a new page

%----------------------------------------------------------------------------------------
%	LIST OF CONTENTS/FIGURES/TABLES PAGES
%----------------------------------------------------------------------------------------

\pagestyle{fancy} % The page style headers have been "empty" all this time, now use the "fancy" headers as defined before to bring them back

\lhead{\emph{Contents}} % Set the left side page header to "Contents"
\tableofcontents % Write out the Table of Contents

\lhead{\emph{List of Figures}} % Set the left side page header to "List of Figures"
\listoffigures % Write out the List of Figures

\lhead{\emph{List of Tables}} % Set the left side page header to "List of Tables"
\listoftables % Write out the List of Tables

%----------------------------------------------------------------------------------------
%	ABBREVIATIONS
%----------------------------------------------------------------------------------------

%\clearpage % Start a new page

%\setstretch{1.5} % Set the line spacing to 1.5, this makes the following tables easier to read

%\lhead{\emph{Abbreviations}} % Set the left side page header to "Abbreviations"
%\listofsymbols{ll} % Include a list of Abbreviations (a table of two columns)
%{
%\textbf{LAH} & \textbf{L}ist \textbf{A}bbreviations \textbf{H}ere \\
%\textbf{Acronym} & \textbf{W}hat (it) \textbf{S}tands \textbf{F}or \\
%}

%----------------------------------------------------------------------------------------
%	SYMBOLS
%----------------------------------------------------------------------------------------

%\clearpage % Start a new page
%
%\lhead{\emph{Symbols}} % Set the left side page header to "Symbols"
%
%\listofnomenclature{lll} % Include a list of Symbols (a three column table)
%{
%$a$ & distance & m \\
%$P$ & power & W (Js$^{-1}$) \\
%% Symbol & Name & Unit \\

%& & \\ % Gap to separate the Roman symbols from the Greek
%
%$\omega$ & angular frequency & rads$^{-1}$ \\
%% Symbol & Name & Unit \\
%}
%

%----------------------------------------------------------------------------------------
%	THESIS CONTENT - CHAPTERS
%----------------------------------------------------------------------------------------

\mainmatter % Begin numeric (1,2,3...) page numbering

\pagestyle{fancy} % Return the page headers back to the "fancy" style

% Include the chapters of the thesis as separate files from the Chapters folder
% Uncomment the lines as you write the chapters


\chapter{Introduction} % Main chapter title
% TODO: can have ECG sensor file:///C:/Users/cristi/Downloads/fyp-papers/sensors-11-06799.pdf A Comprehensive Ubiquitous Healthcare Solution on an Android™ Mobile Device 

\label{Chapter1}

\lhead{Chapter 1. \emph{Introduction}} % Change X to a consecutive number; this is for the header on each page - perhaps a shortened title

% introduction, passwords are widely used but not that great
Passwords are the most widely used electronic authentication mechanism. They are a secret sequence of characters used for proving the identity of the user, in order to gain access to a resource. This originally offered a sufficiently secure authentication mechanism. However, their poor scalability makes them unsuitable in the current technological context.

% problem with passwords
The main problem passwords have is the fundamental concept of remembering a secret. According to Yan et al \cite{yan2004password}, users choose weak passwords if not given any advice to make them memorable. This makes the mechanism more vulnerable to brute force attacks (e.g. dictionary, pre-compiled hashes, rainbow tables \cite{oechslin2003making}). As Robert Morris \cite{morris1979password} emphasises in his paper, there is a constant competition between attackers and security experts. With a constant increase in computational power, additional enforcements were needed (i.e. minimum password length, one or more numeric characters, one or more special characters, uniqueness across different accounts) in order to maintain an acceptable security level. As shown by Adams \& Sasse \cite{adams1999users}, this solution proves not to be feasible, and leads users to poor security practices in order to maximise usability.

% Pico to the rescue
The Pico project was designed by Frank Stajano \cite{stajano2011pico} with the purpose of replacing password based mechanisms. Pico is a hardware token that generates and manages user authentication credentials. It has an additional layer of security by only being usable in the presence of its owner. Therefore, a security chain is created where ``who you are'' unlocks ``a secret you have'' which is used for authentication.

% Picosiblings authentication mechanism
The current solution for unlocking Pico is by communicating with small auxiliary devices called Picosiblings \cite{stannard2012good}. They are designed to be embedded in everyday items that users can carry throughout the day (e.g. keys, necklace, rings). Each Picosibling transmits a secret sequence to Pico. When all required secrets are gathered, Pico becomes unlocked and can be used by its owner.

% Downside of Picosiblings authentication mechanism
Picosiblings are a sensible solution to unlocking Pico. However, they are purely based on proximity to the device. As presented in the original Pico paper \cite{stajano2011pico} anyone in possession of both Pico and its Picosiblings can have full access to the owner's accounts for a limited amount of time. This risk is lowered by additional security features. However, the main vulnerability of Picosiblings is that they do not reflect who the user is, but additional things the user has.

% The scope of my project
The purpose of this dissertation is to design and prototype a better token unlocking mechanism for Pico. According to its design, the process should be memoryless, and enable continuous authentication. The token should lock and unlock automatically only in the presence of its owner. The solutions that seem to best fit these requirements are biometric authentication mechanisms. Therefore, we have explored the possibility of combining multiple biometrics and behavioural analysis as part of an unified solution. The output from each mechanism is combined to generate an overall confidence level, reflecting that the owner is still in possession of the Pico.

% Contributions
A number of contributions have been made throughout this dissertation project. The following list presents a summary of these achievements, with further details in the following chapters.
\begin{itemize}
	\item For creating a new unlocking mechanism, we have identified a list of requirements by analysing how Stajano \cite{stajano2011pico} designed the original Pico token.

	\item In order to have an evaluation platform for the solution, we have created an assessment framework derived from the work by Bonneau et al \cite{bonneau2012quest}. This is used to evaluate a couple of existing token unlocking mechanisms, including Picosiblings. The results are used as a benchmark when evaluating the proposed solution.
	
	\item We have designed a new token unlocking mechanism. The solution may be used in any type of user authentication, but it is presented in the context of unlocking the Pico token. 
	
	\item We have developed an Android application prototype. The purpose of the implementation is to check that the design can be developed using existing hardware. 
	
	\item We have analysed the prototype's power consumption as well as timings of different authentication stages. These results should reveal any limitations and downsides of the scheme.
	
	\item The scheme is evaluated using the token unlocking evaluation framework, and the UDS framework developed by Bonneau et al. A comparison is made with Picosiblings in order to identify performance differences. We aimed for the proposed scheme to achieve better results in at least some categories of the token unlocking framework.
	
\end{itemize}	


\chapter{Pico: no more passwords!} % Main chapter title

\label{Chapter2}

\lhead{Chapter 2. \emph{Pico}} % Change X to a consecutive number; this is for the header on each page - perhaps a shortened title

% Introduction: why we do this
This dissertation project aims to design and implement a new Pico unlocking mechanism. A better understanding of the Pico design \cite{stajano2011pico} is therefore necessary. This chapter aims to go into brief detail as to what Pico is, how it works, and what are its properties.

% introduction
Pico is an user authentication hardware token, designed with the purpose of fully replacing passwords. Although other alternative mechanisms exist, they are generally focused on web based authentication. The solution described by Stajano addresses all instances of password authentication, both web based as well as local.

% motivation to replace passwords: increased computational power
The motivation behind Pico is the fact that passwords are no longer viable in the current technological context. Computing power has grown, making simple passwords easy to break. Longer and more complex passwords are now required. However, as shown by Adams \& Sasse \cite{adams1999users}, this has a negative effect on the users, which have a limited memorising capability.

% motivation to replace passwords: increased number of password accounts
Another reason why passwords are no longer viable is that they are not a scalable solution. Security experts recommend that passwords should not be reused for multiple accounts. However, a large number of computer based services require this type of authentication. In order to respect security recommendations, users would be forced to remember dozens of unique, complex passwords. A study by Florencio et al \cite{florencio2007large} confirms the negative impact of scalability on password quality.

% fundamental design to improve on passwords
When designing Pico, Stajano decides to have a fresh start. He describes that an alternative for passwords needs to be at least memoryless and scalable, without compromising security. In the case of token based authentication, the solution also needs to be loss and theft resistant. Pico was therefore designed to satisfy these fundamental properties. It provides a number of additional benefits also described in the work by Bonneau et al \cite{bonneau2012quest}.

% keeping credentials safe
As a token authentication mechanism, Pico transforms ``something you know'' into ``something you have''. It offers support for thousands of credentials that are kept encrypted on the Pico device. The encryption key is also known as the ``Pico Master Key''. If the Pico is not in the possession of its owner it becomes locked. In this state, the ``Pico Master Key'' is unavailable and the user cannot authenticate to any app\footnote{For the purpose of brevity, any mechanism requiring user authentication will be called an ``app''. This naming convention was used in the original paper by Stajano.}.

% creates and manages credentials
Credentials are generated and managed automatically whenever the owner interacts with an app. Therefore, the responsibility of generating a strong and unique credential, as well as memorising it, is shifted from the user to the Pico. No additional effort such as searching or typing is required.

% continuous authentication
Another important feature offered by Pico is continuous authentication. Traditional password mechanisms provide authentication for an entire session. The user is responsible of managing and closing the session when it is no longer needed. Instead, Pico offers the possibility of periodic re-authentication of its owner using short range radio communication. If either the Pico or the owner are no longer present, the authentication session is closed. 

% physical design
From a physical perspective, Pico is a small portable dedicated device. Its owner should be carrying it at all times, just as they would with a car key. It contains the following hardware components:
\begin{itemize}
	\item Main button used for authenticating the owner to the app. This is the equivalent of typing the password.
	\item Pairing button used for registering a new account with an app.
	\item Small display used for notifications.
	\item Short range bidirectional radio interface used as a primary communication channel with the app.
	\item Camera used for receiving additional data from the app via 2-dimensional visual codes. This serves as a secondary communication channel.
\end{itemize}

% physical design: what is stored and how
As mentioned before, the Pico main memory is encrypted using the ``Pico Master Key''. The token contains thousands of slots used for storing unique credentials used in during authentication. Each credential consists of public-private key information generated during account creation in a key exchange protocol. The public key belongs to the corresponding app, while the private key was generated when registering the account. 

% account creation
During account creation Pico scans a 2D visual code generated by the app. The image encodes a hash of the app's certificate that includes its name and public key. Pico starts the registration protocol through the radio channel, and the app responds with a public key used for communication. The key is validated using the hash from the visual code, and the protocol continues. Pico then initiates a challenge for the app to prove that it is in possession of the corresponding private key. It also provides to the app a temporary public key used for communication. Once the app is authenticated, Pico generates a key pair and sends the account public key to the app. To complete the registration it stores the generated private key and the app's public key.

% account authentication
The account authentication process starts when the user presses the main button and scans the app 2D code. The hash of the app's name and public key are extracted from the 2D image. This information is used to find the corresponding credentials. An ephemeral public key encrypted with the app's public key is sent via the radio channel. The app is authenticated by using this key to require the corresponding (user id, credential) pair. Only after the app is authenticated Pico uses the public key generated during the registration process and authenticates itself to the app.

% Unlocking pico
The locking process is an important aspect of Pico that was not yet fully described. Currently this is achieved by using bidirectional radio communication with small devices called Picosiblings \cite{stannard2012good}. These are meant to be embedded in everyday items that the owner carries around, such as earrings, keys, chains, and rings.

% Reconstructing master key.
The Pico authentication credentials are encrypted using the ``Pico Master Key''. The key is not available on the token and can only be reconstructed using k-out-of-n secret sharing, as described by Shamir \cite{shamir1979share}. Except for two shares which will be discussed later, each k-out-of-n secret is held by a Picosibling. 

Using an initialisation process based on the ``resurrecting duckling'' \cite{stajano2000resurrecting} policy, each Picosibling is securely paired with Pico. The token sends periodic ping requests to which all initialised Picosiblings are expected to respond. During a successful ping, each Picosibling sends its k-out-of-n share to Pico. Given enough secrets, the ``Pico Master Key'' is reconstructed and Pico becomes unlocked.

Internally, Pico keeps a slot array for each paired Picosibling. Each slot contains a countdown value, and the key share provided by the Picosibling. When the countdown value expires, the share becomes deleted. Similarly, if k shares are not acquired before a predefined time-out period, all shares are removed.

Except for the Picosiblings, two additional special shares with a larger time-out period are described by the paper:
\begin{itemize}
	\item Biometric measurement used for authenticating the owner to Pico.
	\item A server network connection used for locking Pico remotely.
\end{itemize}

% Smart phones as a Pico
The possibility of using a smart phone as a Pico is briefly considered in the paper. This would have the advantage of not requiring any additional devices from the user. Modern smart phones provide all the necessary hardware required by Pico. However, this would be a security trade-off in exchange for usability. Mobile phones are an ecosystem for malware, and they present uncertainty regarding the privacy of encrypted data. This option may still be used as a cheap alternative to prototype and test.








 

\chapter{Assessment framework}

\label{Chapter3}

\lhead{Chapter 3. \emph{Assessment framework}} % Change X to a consecutive number; this is for the header on each page - perhaps a shortened title

The purpose of this chapter is to create a framework for assessing token based authentication mechanisms. Using this framework we can then compare existing solutions. Having the framework as a performance compass we can then continue by designing an alternative to the Picosiblings unlocking mechanism.

%----------------------------------------------------------------------------------------
%	SECTION 1
%----------------------------------------------------------------------------------------
\section{Related work}
% TODO:
%	- list the framework and description?

In the paper ``The Quest to Replace Passwords: A Framework for Comparative Evaluation of Web Authentication Schemes'' \cite{bonneau2012quest} the authors develop a framework for evaluating web based authentication mechanisms. The purpose of the framework is to identify authentication schemes which outperform passwords. The framework is intended to provide a benchmark for future web authentication proposals.

The framework focuses of three classes of properties which are abbreviated as UDS: usability, deployability, and security. Each class contains a set of properties, totalling a number of 25 benefits. A mechanism may either offer, quasi-offer, or not offer a property. Properties which are not applicable to a mechanism are marked as ``offered'' to simplify the framework.

Using the framework to evaluate 35 password replacement schemes shows that no scheme is dominant over passwords. According to the evaluation, passwords score perfectly in deployability. They score reasonably in terms of usability, excelling in properties such as: nothing-to-carry, efficient-to-use, and easy-recovery-from-loss. In terms of security however, passwords don't perform as well, only receiving points in resilience-to-theft (not applicable), no-trusted-third-party, requiring-explicit-consent, and unlinkable. The full list of properties and their description can be found within the paper itself.

Biometric mechanisms receive mixed scores on usability. None of them have the infrequent-errors property which is a precision problem related to false negatives. More importantly if the biometric data is exposed by malware for instance, the authentication mechanism may not be used by the user any more. They score poorly in deployability due to the additional hardware required. In terms of security they perform worse than passwords. Replay attacks can be used by an attacker using a recording of some sort in order to trick the sensor. They are not resilient to theft, since they require an additional device. The fact that they uniquely link the owner to the recording means that the owner may be linked back to the data, therefore not granting the ``unlinkable'' property. 

The paper notes that the memory-effortless property versus nothing-to-carry is only achieved by biometric schemes. None of the mechanisms manage to fully achieve memory-effortless and be resilient-to-theft. This is due to the fact that most mechanisms replace something you know with something you have.

The authors do not produce aggregate scores or rankings. This is due to the fact that not all properties are equal in importance, but different properties would have different weights depending on the scheme's application domain. 

Combining schemes is mentioned as a two factor arrangement. This would result in a mechanism which in terms of usability and deployability would only have the properties which are granted by both schemes. In security however it would have the properties of both mechanisms. As shown in the paper, according to \cite{wimberly2011using} the presence of a two factor authentication would lead the user to creating weaker passwords.

Further details about the framework itself are not mentioned in this dissertation for the purpose of brevity. The framework and explanations for each property can be found in the original paper \cite{bonneau2012quest} by Bonneau et al.

\section{Token unlocking framework}
Web based authentication mechanisms are initiated locally and performed remotely. In contrast, token based authentication is initiated and performed locally, leaving no room for man in the middle attacks or any other 3rd party participation. For this reason, a subset of the properties described in paper \cite{bonneau2012quest} by Bonneau et al should also be present in the framework we have developed. Properties from the original framework by Bonneau et al which do not apply, or would be satisfied by any token unlocking mechanism were removed in this work.

The following list shows what properties of the framework developed by Bonneau et al are relevant to token based authentication mechanisms:
\begin{description}
  \item[Memory-effortless] \hfill \\
  Different types of tokens would have different results. For instance the RSA SecurID \cite{} token doesn't require any authentication, while the FIDO (Fast IDentity Online) Alliance \cite{} may use a PIN requiring a known secret.
  
  \item[Nothing-to-carry] \hfill \\
  Different token unlocking mechanisms may require additional hardware. Such may be the case for some biometric schemes.
  
  \item[Easy-to-learn] \hfill \\
  Token authentication mechanisms may have different learning curves. As an example a CAP reader is fairly easy to use, while a Pico device may prove more difficult for the inexperienced.
  
  \item[Efficient-to-use] \hfill \\
  Time required by the token user authentication mechanism may differ from one type of authentication to the other. The time required for registering a new user or unlocking the token for its owner should be reasonably short.
  
  \item[Infrequent-errors] \hfill \\
  The token unlocking mechanism may reject true positives. If the number of false negatives is reasonably low, then the mechanism has this property.
  
  \item[Easy-recovery-from-loss] \hfill \\
  The user's ability to get another token which uses the same authentication mechanism. Tokens which unlock using biometrics for instance, if not properly secured may lead to the user's inability to use that mechanism again.
  
  \item[Accessible] \hfill \\
  The ability for all users to use the authentication mechanism. As an example, PINs may be entered by any user regardless of disabilities, on the other hand other biometric mechanisms may not be available.
  
  \item[Negligible-cost-per-user] \hfill \\
  The total cost enquired by the user in order to use the authentication mechanism.
  
  \item[Mature] \hfill \\
  It refers to the number of users that have successfully used the mechanism, open source projects based on the mechanism, and any other usage by a third party which did not participate in the development of the scheme.
  
  \item[Non-proprietary] \hfill \\
  Anyone can implement the token unlocking scheme without having to pay royalties to anyone else.
  
  \item[Resilient-to-physical-observation] \hfill \\
  An attacker should not be able to impersonate the user after observing them authenticate.
  
  \item[Resilient-to-targeted-impersonation] \hfill \\
  An attacker should not be able to impersonate the user using knowledge about the user, or previous recordings of his biometrics.
  
  \item[Resilient-to-throttled-guessing] \hfill \\
  The resilience to an attacker automating a guessing process in order to brute force the unlock of the token.
  
  \item[Resilient-to-unthrottled-guessing] \hfill \\
  An attacker which only physical access to the token cannot guess the required unlocking resource.
  
  \item[Resilient-to-internal-observation] \hfill \\
  An attacker cannot tamper with the token in order to intercept user input. Furthermore it is impossible for the attacker to gather the input from within the token's storage.
  
  \item[Unlinkable] \hfill \\
  The unlocking mechanism does not generate data which if leaked would compromise the identity of the user. \ldots
  
\end{description}

The properties described above are derived from the original framework presented by \cite{bonneau2012quest}. Additional details relevant to each property, including when a property is only quasi (partially) satisfied may be found in the original work by Bonneau et al. Some properties, such as Nothing-to-carry or Server-compatible, do not apply for token unlocking schemes and therefore are not included.

% TODO: add additional properties which are original
Although the framework by Bonneau et al provides a good base set of properties, a few others are needed in order to fully characterise token unlocking mechanisms. The following list is part of the project's contributions to the overall evaluation framework.

\begin{description}
  \item[Continuous-authentication] \hfill \\
  The concept, although mentioned, is omitted from the framework developed by Bonneau et al \cite{bonneau2012quest}. It stressed a bit more as part of the benefits of Pico by Stajano \cite{stajano2011pico}. This is a property belonging to the security category of the original framework. The property is satisfied if, once authenticated, the user remains authenticated to the token for as long as he is in its presence. This is similar to an authentication session with the added property that the session remains active for as long as the user requires it. The property should be satisfied by mechanisms which may re-perform authentication in a non explicit way, leaving the user unaware of the underlying process.
  
  \item[Multi-level-unlocking] \hfill \\
  This is a security category property. If satisfied, the unlocking mechanism may allow for multiple types of unlocking based on the user confidence or identity. This is something that may be characteristic for mechanisms which involve biometrics or accounts with multiple security levels.
  
  \item[Availability] \hfill \\
  This is an usability property. If satisfied, the user has the ability to use the unlocking mechanism in any circumstance. This is not related to any disabilities preventing the user from using certain mechanisms. The properly is strictly related to whether the scheme can be used in any circumstances, and may provide feedback regardless of light, noise, or other external factors. As an example gait recognition would only function while moving on foot, or a recognizer that restricts access based on pulse would not satisfy this requirement. A mechanism requiring only a PIN on the other hand would work in any circumstance.
  
\end{description}

\section{Example evaluation}
In order to demonstrate how the framework works we have assessed three token based authentication mechanisms: Picosiblings, PIN, and Face-unlock. Each of the mechanisms represent a different type of authentication method. Picosiblings essentially are a secret the owner has, PINs are a secret the owner knows, and Face-unlock is something the owner is. The following sections will provide as an example of how the framework should be applied. Results in the Picosiblings section will be used in the following chapter as a token authentication benchmark which will be used to compare the proposed unlocking scheme with the existing one.

	\subsection{Picosiblings}
	The current token unlocking mechanism used by Pico is called Picosiblings. In order to unlock the token, k-out-of-n secrets are required. Each of these secrets are held within a Picosibling carried by the user. If the Picosibling is within range of the Pico, these secrets are communicated to the token which can then be used to unlock its master key and implicitly the token.
	
	Since the user doesn't need to remember any secrets, the scheme satisfies the memory-effortless property. The nothing-to-carry property is not satisfied since the Picosiblings are external emitters. As the original paper suggests, the scheme is not easy to learn due to user's unfamiliarity with the mechanism. It is only partially satisfies the efficient-to-use, and infrequent-errors properties. Easy recovery from loss is not offered neither for the Pico or the Picosiblings, since additional hardware would need to be ordered. The scheme relies fundamentally on the presence of the Picosiblings, but may be used in any normal scenarios as long as the devices are present. Therefore it satisfies the availability property.
	
	The original paper marks Pico with the Picosiblings scheme as not accessible. We will assume the property transfers to the unlocking mechanism as well. The negligible-cost-per-user property is not satisfied due to presumably expensive embedding of Picosiblings in everyday items. The scheme was not used in any open source projects and is at the stage of a prototype, having very little user testing. For this reason the scheme does not satisfy the mature property. Since this is an academic research project, no royalties need to be paid in order to implement the scheme. Therefore the scheme is non-proprietary.
	
	Since it does not rely on any user input it is resilient-to-physical-observations. In disagreement with the original Pico assessment by the authors, since anyone in the possession of the Picosiblings may unlock the Pico it is not resilient-to-targeted-impersonation. It is however resilient-to-throttled-guessing and resilient-to-unthrottled-guessing due to the communication protocol described by Stajano in his paper \cite{stajano2011pico} as well as the resurrecting ducklings protocol \cite{stajano2000resurrecting}. Due to the use of cryptography is is also resilient-to-internal-observations, and if lost the data is unlinkable to its owner. The scheme was designed to provide continuous-authentication. Since the mechanism offers secrets which keep the master key as either locked or unlocked, it does not satisfy the multi-level-unlocking property.

	\subsection{PIN}
	PIN should be viewed as passwords with a severely limited subset of characters. Additional protection comes from steep security measures when the PIN authentication has failed. As an example, for mobile phones typing 3 wrong PINs would lock your SIM card. A lot of the PIN properties should however be similar with those offered by passwords.
	
	Since they scheme relies on knowing a secret, it does not satisfy the memory-effortless property, in exchange for the nothing-to-carry property. Due to its maturity and similarity with passwords users find it easy-to-learn. It is efficient-to-use since it requires no time to compare two small strings. Due to the input method on a small mobile device it does not have the infrequent-errors property. As a divergence from the original paper, we consider PINs not to satisfy the easy-recovery-from-loss property. This is due to the fact that usually it is required to order a new SIM with the same number in order to ``reset'' a forgotten PIN. Due to the fact that it only requires the ability to be typed, it satisfies the availability property in any conditions.
	
	Just as passwords PINs score all points in deployability. They can be used by anyone and are therefore accessible. The have virtually no cost, satisfying the negligible-cost-per-user property. Being a subset of passwords, the mechanism is considered to be mature and non-proprietary.
	
	From a security perspective however PINs score poorly. They are not resilient-to-physical-observation since anyone can eavesdrop the input of a PIN. The resilient-to-targeted-impersonation property is strongly linked to this, but only works if the attacker inputs the correct PIN. The property is therefore only quasi-satisfied.	Unlike passwords, PINs do have the property of resilient-to-throttled-guessing property due to the security model around them. A brute force attack cannot be performed on a locked device unless some sort of implementation error is present, such as an unprotected file. Furthermore PIN mechanisms are generally designed with different penalties such as locking the SIM or the OS for a time period. Due to poor practices in using PINs, they do not have the resilient-to-unthrottled-guessing property. PINs are not resilient-to-internal-observations. The fact that they are chosen randomly by the user means they are unlinkable. However, they do not offer the continuous authentication property due to the fact that they cannot authenticate the user in a non-evasive way. They only offer a locked/unlocked state, and therefore they do not have the multi-level-unlocking property. 
	
	\subsection{Android Face unlock}
	Android offers a face recognition unlock mechanism for the mobile device. A mobile phone may be viewed as a multi-purpose token, making the face-unlocking mechanism a valid token unlocking scheme. This feature was made available starting with Android 4.0 (Ice Cream Sandwich). This essentially is a biometric unlocking mechanism, which due to the sensor platform offered by Android may be used for token unlocking purposes.
	
	As no secrets are required for biometric mechanisms, the scheme is memory-effortless. It satisfies the nothing-to-carry property as well, since in the case of mobile devices with a camera the scheme does not require additional external hardware. The mechanism is efficient-to-learn, since it only needs the user to look at the camera. Time is required to load face models when first loading the mechanism, but not upon further authentication requests. Therefore the scheme quasi-satisfies the efficient-to-use property. Errors are quite frequent for this mechanism, both from personal testing as well as existing literature \cite{stajano2000resurrecting}. The easy-recovery-from-loss does not apply for biometric mechanisms, especially in the case where no external hardware is required. The availability property is not satisfied due to the fact that based on external factors such as light or obstacles between the camera and the face authentication is not possible.
	
	
	To do: please provide some input for cost-per-user, mature, and non-proprietary. Face recognition is accessible for anyone to use. It has a negligible cost-per-user due to the fact that it is non-proprietary and no charges are usually made with each authentication. The mechanism quasi-satisfies the mature property due to the limited user exposure.
	
	Purely by observing the owner authenticate using face recognition does not provide any advantage to an attacker. The scheme therefore has the resilient-to-physical observations property. Targeted impersonation is an issue with any authentication mechanism. Nothing would stop an attacker from taking a picture of the owner and use it in a replay attack. Therefore the scheme does not offer the resilient-to-targeted-impersonation property. The resilient-to-throttled-guessing and resilient-to-unthrottled-guessing properties do not apply and therefore are considered as satisfied.  The resilient-to-internal-observations property is not satisfied in the case of mobile devices. This is due to the fact that malware may capture the exact same data when the screen is turned on for example, or immediately after a picture was taken. If data used in authentication is exposed, it is directly linkable to the owner, and therefore the unlinkable property is not satisfied. Continuous-authentication would only possibly be offered based on implementation and conditions such as the user facing the camera of the phone. The property is therefore only quasi-satisfied, although arguably not satisfied. Multi-level-unlocking would be possible based on an Euclidean distance metric, but no efforts have been made in this regard. 

\section{Conclusions}
At the end of this peace of work a new framework for the evaluation of token unlocking mechanisms was developed. Existing properties have been identified from the literature and added to the framework together with original work. An initial evaluation was made for existing token unlocking mechanisms which will serve as a benchmark for the proposed solution. From the example evaluations, neither of the three token unlocking mechanisms is dominant over the others.


% Chapter Template

\chapter{Design} % Main chapter title

\label{Chapter3} % Change X to a consecutive number; for referencing this chapter elsewhere, use \ref{ChapterX}

\lhead{Chapter 3. \emph{Design}} % Change X to a consecutive number; this is for the header on each page - perhaps a shortened title

\section{Proposed design}
% Motivation for design
The framework evaluation of Picosiblings provides insight as to how the scheme can be improved. We identify as a key downside that it does not guarantee the identity of the owner. This information is mainly inferred from the number of Picosibling shares the user has. However, anyone may be in the possession of the shares, therefore being granted full temporary authentication privileges. This is reflected in the evaluation by failing to fully offer the `resilient-to-theft'' and ``non-disclosability'' properties. A further improvement can be made by introducing ``multi-level-unlocking'', allowing for multiple levels of authentication depending on the confidence in the owner's presence.

% Pico properties that need to be maintained
The Pico design proposed by Stajano \cite{stajano2011pico} claims two properties that also need to be supported by the token unlocking mechanism: the authentication process is memory effortless; and the unlocking scheme needs to support continuous authentication\footnote{Continuous authentication is defined by the ability to re-authenticate the user without the need for any physical effort.}. These features need to be satisfied when designing the new token unlocking mechanism.

% combine multiple authentication mechanisms
The idea explored in this dissertation project is to simultaneously use multiple continuous authentication mechanisms. Each mechanism needs to provide a quantifiable confidence level which will be used in calculating a combined score. This satisfies the memoryless and continuous authentication properties required by Pico. By combining mechanisms we achieve a higher confidence of correctly identifying the owner. Furthermore, given that each individual mechanism supports continuous authentication, using them simultaneously does not create any inconvenience for the owner.

% multi-level unlocking model
The Pico token should no longer enter a general locked or unlocked state. Its most important secret, the ``Pico Master Key'' should be kept in tamper resistant memory, and be accessible at all times. Using the overall score computed by the proposed mechanism, Pico should offer granular user authentication. Each user account needs to be associated with a confidence level defined by the app during the registration process. If the overall confidence level of the unlocking mechanism exceeds the app's confidence level, then the token becomes ``unlocked'' for that specific app. All authentication sessions between Pico and apps need to be managed independently based on this model.

% examples  of authentication mechanisms
The scheme should achieve continuous authentication, while correctly identifying the owner of the token. For this reason we have decided that authentication mechanisms combined in the scheme need to be based either on biometrics or behavioural analysis. Biometric features that can be used with this scheme include iris, face, voice, and gait. Behavioural sources of data can be obtained from frequent GPS location, travel paths, wireless network connections, and others.

% how it is different than simple biometrics
The solution offered in this project is different from simply stating that Pico is using biometric data as an unlocking mechanism. The novelty in the design is based on how data is combined in order to compute the overall confidence level. All mechanisms are assigned a different initial weight based on the level of trust it offers in identifying the owner. This doesn't necessarily need to be related to the precision of the mechanism, but it would be a good indicator for choosing a value.

% decaying weight
Data samples captured for the owner authentication process are not always meaningful. For example, accelerometer values for gait recognition are only usable when the user is travelling on foot. Depending on how the sensors are integrated with the Pico, camera input for face recognition may not always capture a valid image. The confidence of each mechanism should therefore decrease in time from the last valid authentication sample. This introduces another original feature of this scheme, which is having a decaying weight. Each mechanism starts with a predefined initial value, reflecting the weight of the mechanism in the overall unlocking process. This value is decreases in time until a valid user data sample is provided to the mechanism for authentication. 

% example of decaying confidence
Let us take for example a voice recognition mechanism which samples data every minute. The current weight of the mechanism is 0 so its output is completely ignored. The next sample is recorded, and the voice recognition mechanisms outputs a confidence of $70\%$ that the owner is present. After the successful recording, the mechanism weight is updated to its predefined starting value of 30. For the next 10 minutes the owner will be silently reading a book. Since the mechanism only identifies background noise, the weight value of 30 decreases in time. This will induce a smaller impact of the voice recognition mechanism on the overall score. The confidence of each mechanism can decrease up to 0, at which point the mechanism is ignored. Computing the overall score will be explained in more detail later in the chapter.

% Bayesian update
Each mechanism outputs a value, which is the probability that the sample data belongs to the owner of the token. Upon each recording, this probability is updated using Bayes' Law. This process is also known as a Bayesian update. The equation is described below:

\begin{equation} 
\label{eq:bayes1}
P(H|E) = \frac{P(H) * P(E|H)}{P(E)}
\end{equation}

In the equation above:
\begin{itemize} 
	\item E: Stands for evidence and in this case represents the data sample.
	\item H: Stands for hypothesis. In this case we refer to the hypothesis that the owner is present.
	\item $P(H|E)$: Represents the probability of hypothesis $H$ after observing evidence $E$. This is the final probability we are trying to compute after each sample. It is also known as the posterior probability. 
	\item $P(H)$: Represents the probability of hypothesis $H$ before observing evidence $E$. This is also known as the prior probability and is the probability computed at the previous step.
	\item $P(E|H)$: Represents the probability that the current evidence belongs to hypothesis $H$. It is the probability outputted by the biometric mechanism given the sample data.
	\item $P(E)$: This is the model evidence, and has a constant value for all hypothesis.
\end{itemize}

Although $P(E)$ is constant we need its value in order to calculate $P(H|E)$. We can compute it using the ``Law of total probability'', which is the following:

\begin{equation} 
\label{eq:lotp}
P(E) = \sum_{n}^{}P(E|H_n) * P(H_n)
\end{equation}

Using equation \ref{eq:lotp} the Bayes' Law equation \ref{eq:bayes1} becomes:
\begin{equation} 
\label{eq:bayes2}
P(H|E) = \frac{P(H) * P(E|H)}{\sum_{n}^{}P(E|H_n) * P(H_n)}
\end{equation}

Our model however, contains only two hypothesis\footnote{Arguably there is a third case where the data sample is not a valid recording of an user. This is ignored and no probability is computed. The only result in this case would be a decay in the weight of the mechanism.}: the recording of the data either belongs to the owner, or not. We can therefore consider $P(H)$ to be the hypothesis that the data belongs to the owner and $P(~H)$ that the data belongs to someone else. Obviously the value of $P(\neg H)$ is $1 - P(H)$ and $P(E|\neg H) = 1 - P(E|H)$ Introducing these values in equation \ref{eq:bayes2}, the rule for updating the mechanism's probability becomes:

\begin{equation} 
\label{eq:final}
P(H|E) = \frac{P(H) * P(E|H)}{P(H) * P(E|H) + P(\neg H) * P(E|\neg H)}
\end{equation}

Equation \ref{eq:final} represents the final probability that the owner is present given the sampled data. All the variables in this equation are known, for reasons explained above.

% Overall confidence
% 	TODO: update this to have wii and wid (decayed and initial), as well as above when describing the decay rate
Thus far we have defined how individual scores are calculated, and that each mechanism has a decaying weight. Using this data we can calculate the overall score of the scheme. This is performed quite trivially by using a weighted sum. Equation \ref{eq:overall} describes the process. 

\begin{equation} 
\label{eq:overall}
P_{Total} = \frac{\sum_{i=1}^{n}(w_ii * P_i(H|E_i))}{\sum_{i=1}^{n}w_id}
\end{equation}




% TODO: CONTINUE HERE
The result is sent to Pico in order to adjust the state of current authentication sessions. If score required by the app is lower than the overall score provided by the scheme, the user is granted authentication access. Given the continuous authentication property, the Pico token will continue to ask its authenticator whether the confidence level is still satisfied. Based on the decay rate of the weights and the input data available of the authenticator's mechanisms this will constantly be recalculated.

% Explicit authentication mechanisms
At some point the confidence level required by Pico might be too high for the authenticator to grant access. As an example the owner will want to access it's bank account after being silent in a dark room for the past hour. Let us say this would require a confidence level of $95\%$, while the authenticator may only output a $20\%$ confidence that the user is still present. Given the circumstances, an explicit authentication mechanism may be required from the user in order to increase the current confidence level. 

% Combining explicit authentication mechanisms
Combining explicit authentication with the current design can be performed consistently with the continuous authentication mechanisms. Whenever an explicit authentication is required, the only difference will be the fact that the user becomes aware of the authentication process. They are prompted to pass an authentication challenge (i.e. facial recognition, voice recognition). This would guarantee valid input for the authenticator which may then proceed to compute an accurate score.

\section{Framework evaluation}
We will continue by evaluating the new proposed scheme with the token unlocking framework defined in the previous chapter. 

\begin{description}
  \item[Memory-effortless: Satisfied] \hfill \\
  None of the authentication mechanisms require any sort of known secret. Authentication is granted based on biometrics and behavioural analysis.
  
  \item[Nothing-to-carry: Quasi-satisfied] \hfill \\
  This property is only quasi-satisfied due to the fact that it relies on the implementation of the design. Ideally all authentication data should be gathered from an unified device containing the Pico. Alternatively however, the scheme can be implemented using individual sensors which the owner would have to carry, which is why the property is not fully granted.
  
  \item[Easy-to-learn: Satisfied] \hfill \\
  In order to satisfy Pico's property of continuous authentication, all mechanisms part of the scheme I developed also need to have this property. Therefore the authentication process is non-transparent to the user, and therefore there is nothing to learn.
  
  \item[Efficient-to-use: Satisfied] \hfill \\
  The authentication data is collected either at fixed time intervals, or is fired during special events. The authentication process however, does not fully depend on recent data. A response may be generated without any recent authentication data. Therefore the time spent by the mechanism to generate a response is immediate.
  
  \item[Infrequent-errors: Quasi-satisfied] \hfill \\
  Given that the scheme depends on biometric mechanisms, the quality of the errors is as good as the underlying biometrics. If the scheme cannot generate a high enough confidence an explicit biometric challenge will be issued for the user to satisfy. Since the original biometric mechanisms do not have this property, to some extent neither will the scheme I have designed. However, the scheme is combining multiple biometrics results with different score weights based on importance and accuracy. This is much more likely to be accurate, which is why I will mark this as Quasi-satisfied. For a more accurate response, the design needs testing with a high quality prototype. 
  
  \item[Easy-recovery-from-loss: Not-satisfied] \hfill \\
  Token based mechanisms in general do not have this property due to the inconvenience of replacing the token. In our case, the property is also not satisfied. The user would have to re-acquire a new token and reconfigure the owner's biometric data. Furthermore based on the mechanism, such as location settings or gait recognition, the token is likely to require an adaptation period.
  
  \item[Availability: Satisfied] \hfill \\
  Some mechanisms are not always available even though enabled, especially due to the continuous authentication property. As an example gait recognition while sitting in an office. However, the scheme may use a multitude of mechanisms with the unlikeness that all of them are unavailable. For instance location history may predict with a certain confidence that the owner still in possession of the token. This propery is aided by the explicit authentication mechanism which requires explicit input from the user.
  
  \item[Accessible: Satisfied] \hfill \\
  Due to the fact that the scheme is based on multiple biometrics and location settings, I consider this property to be Satisfied or as a very least Quasi-satisfied. The scheme functions based on available biometrics, without having any predefined solutions. It is highly unlikely that the owner cannot generate any of the available biometric inputs, especially for some such as ``face recognition''.
  
  \item[Negligible-cost-per-user: Quasi-satisfied] \hfill \\
  This property depends on the way in which the scheme is implemented. If the implementation is based on high quality sensors embedded in items of clothing and such, then the property is not satisfied. If the implementation reuses sensors that the user already possesses, the the property is fully satisfied as the cost is 0. An example of such an implementation would be an Android application/service possibly using the future Google Glass hardware.
  
  \item[Mature: Not satisfied] \hfill \\
  This property is not satisfied as the project is at the level of a work in progress prototype. The design is quite fresh and was not implemented by any third party. Neither was is reviewed by the open source community or has had any user feedback.
  
  \item[Non-proprietary: Satisfied] \hfill \\
  Anyone can implement the scheme without any restrictions such as royalty checks or any other sort of payment to anyone else.
  
  \item[Resilient-to-physical-observation: Satisfied] \hfill \\
  Since the mechanism is based on biometric data, simple observations from an attacker cannot lead to compromising the user's authentication to the token. The attacker would have no way of reproducing the input through simple observation.
  
  % TODO: think about explicit authentication to keep this quasi
  \item[Resilient-to-targeted-impersonation: Quasi-satisfied] \hfill \\
  Saying that the scheme Quasi-satisfies this property is a bit generous. Each of the mechanisms is vulnerable to a replay attack. An attacker may record one of the user's biometric and replay it as a token input. However, given that the token uses multiple mechanisms, some of which being location based, this is a highly unlikely occurrence. The only vulnerable point would be the explicit authentication mechanisms, which carry a lot of weight.
  
  % TODO: find citation for this
  \item[Resilient-to-throttled-guessing: Satisfied] \hfill \\
  The amount of throttled guessing required for the user to break one of the biometric mechanisms is far too large for this to actually be a threat.
  
  \item[Resilient-to-unthrottled-guessing: Satisfied] \hfill \\
  Given that the Resilient-to-throttled-guessing property is satisfied, this property is also satisfied.
  
  % TODO: talk more about this
  \item[Resilient-to-internal-observation: Satisfied] \hfill \\
  This property does not apply to this scheme. 
  
  \item[Unlinkable: Not-satisfied] \hfill \\
  Just as any of the biometric mechanisms, this property is not satisfied by the mechanism. The authentication data maps uniquely to the owner of the token.
  
  \item[Continuous-authentication: Satisfied] \hfill \\
  The mechanism was designed with continuous authentication in mind. Data is collected periodically with a confidence weight decaying over time. This allows for the token to be used at any time based on current existing data. The only exception breaking the model would be the explicit authentication mechanisms, but these could only be triggered at the beginning of an authentication process using the token.
  
  \item[Multi-level-unlocking: Satisfied] \hfill \\
  This property is fully satisfied by the authentication mechanism. It allows the token to grant access to different authentication accounts based on the precomputed level of confidence that the owner is present. 
  
\end{description}

Let us continue by comparing the results of our proposed scheme with the original Picosiblings solution. The results are summarised in the following table. In the ``Proposed scheme'' column, properties which are highlighted in order to facilitate the comparison with the Picosiblings solution. The colour green means that the proposed scheme is better, red worse, and no colour means that both properties have the same value.

\begin{table}
    \begin{tabular}{l|l|l}
    Property                            & Picosiblings    & Proposed scheme \\ \hline
    Memory-effortless                   & Satisfied       & Satisfied       \\
    Nothing-to-carry                    & Not-satisfied   & \cellcolor{green!25} Quasi-satisfied \\
    Easy-to-learn                       & Satisfied   	  & Satisfied       \\
    Efficient-to-use                    & Quasi-satisfied & \cellcolor{green!25} Satisfied       \\
    Infrequent-errors                   & Quasi-satisfied & Quasi-satisfied \\
    Easy-recovery-from-loss             & Not-satisfied   & Not-satisfied   \\
    Availability                        & Satisfied       & Satisfied       \\ \hline
    Accessible                          & Not-satisfied   & \cellcolor{green!25} Satisfied       \\
    Negligible-cost-per-user            & Not-satisfied   & \cellcolor{green!25} Quasi-satisfied \\
    Mature                              & Not-satisfied   & Not-satisfied   \\
    Non-proprietary                     & Satisfied       & Satisfied       \\ \hline
    Resilient-to-physical-observation   & Satisfied       & Satisfied       \\
    Resilient-to-targeted-impersonation & Satisfied       & Satisfied       \\
    Resilient-to-throttled-guessing     & Satisfied       & Satisfied       \\
    Resilient-to-unthrottled-guessing   & Satisfied       & Satisfied       \\
    Resilient-to-internal-observation   & Satisfied       & Satisfied       \\
    Unlinkable                          & Satisfied       & \cellcolor{red!25} Not-satisfied   \\
    Continuous-authentication           & Satisfied       & Satisfied       \\
    Multi-level-unlocking               & Not-satisfied   & \cellcolor{green!25} Satisfied       \\
    \end{tabular}
\end{table}

As the table shows, the proposed solution does not completely dominate the Picosiblings solution, and this is only because of the ``Unlinkable'' property. Given that our solution is fundamentally based on biometric data, this property could never be achieved. However, our solution performs better than Picosiblings in 5 other properties. Important points of improvement are accessibility, which makes the proposed scheme viable for a larger number of people. The Multi-level-unlocking property is another good improvement, allowing for an enhanced security model.



\section{Conceptual design threat Model}
% availability (dos)
%   - communication
%   - cpu
%   - battery
%   - practicality of cryptography on small devices

% integrity
% confidentiality

% are sensor readings what they say or are they manipulated

An accurate threat model on the proposed unlock mechanism must start by analysing the set of assumptions made about the mechanism. From there we can identify available threats and how the scheme can be exploited in order to unlock the Pico without owner permission. Throughout the threat model we will explain how relaxing the initial set of assumptions may change the security outcome. Each model is analysed from an Availability, Integrity, and Confidentiality.

It is important to note that confidentiality is an important category in this evaluation. This is because the device will store sensitive biometric data which is directly linkable to the user. Losing this data, especially in plain-text, would disable the user from ever using the biometric device for which the data was leaked. This is due to the fact that the leaked data could always be replayed, successfully tricking the biometric mechanism.

In each subsection, the model will obviously only introduce issues with the mechanism. Therefore when reading a subsection, the issues are not only those currently presented, but also those from previous subsections that lead up to that point.

\subsection{Dedicated device with dedicated sensors}
We will start from the assumption that the unlock mechanism is integrated on the same device with the Pico. The device is assumed to be dedicated and runs no other software. Furthermore, the set of available sensors will also be integrated within the device. Alternatively there may also be peripheral sensors, with no way for an attacker to tamper with the communication to the authenticator. 

	\subsubsection*{Availability}
	From an availability point of view, an outside attacker cannot create a denial of service scenario. Interactions with the device are performed physically, so therefore the device cannot be made unavailable while in the possession of its owner. If the Pico would temporarily lose ownership, from a software perspective it would lock up due to mismatching biometric and location data, but would become available again in the presence of the owner. 
	
	Only hardware modification would affect data availability. Simply disconnecting the sensor would not affect the scheme's ability to generate viable results due to the fact that multiple biometrics are used. However an attacker could modify a sensor to output wrong data, tricking it into saying the user is never the owner. This would create a successful denial of service attack path where a few sensors output that the owner is never present. 
	
	\subsubsection*{Integrity}
	Communication paths are not accessible from the outside and therefore cannot be tampered with in order to modify data. Furthermore the device is not running any other software and is therefore safe from any malware attacks. 
	
	Only physical tampering with the device would change data integrity. Modifying one of the sensor's and changing its output to some random data would be undetectable by the mechanism. 
	
	% TODO: resurected ducklings how to?
	\subsection*{Confidentiality}
	No software access as well as no communication with the outside (i.e. wired communication) means that data is safe as long as the device is with its owner.
	
	If the device were to be lost, 
	Storage data should be kept encrypted, similar to the way Ironkey \cite{} protects its data. Unfortunately an attack path may already be identified which is due to the fact that using this model the decryption key needs to be stored on the device. An attacker which has hardware access could therefore extract the key and decode the data. The original Picosiblings solution circumvented this approach by keeping 

\subsection{Dedicated device with shared sensors}
We will relax the original set of assumptions by saying that the communication path with the sensors is no longer secure. Furthermore the sensors may be shared with other owners, via a wireless communication link for example. Another feasible scenario is that although sensors are located on the same device as the Pico, the Pico application is fully compartmentalised from the outside world. 

What we are trying to stress with this scenario is that the sensors are no longer part of a trusted secure box, but are outside and communication with them, as well as their input may no longer be secure.

	\subsubsection*{Availability}
	% keep all sensors locked
	Since the sensors are no longer dedicated, other users may access sensor data. Depending on the hardware and software platform supporting the sensors, this may lead to a denial of service attack on the scheme. For example, if the sensors may only have one owner at a time, an attacker may request data from all sensors keeping them locked from the biometric authentication mechanisms. If the system is built in such a way, then there is nothing the scheme could do to prevent this other than keep the sensors constantly locked for itself. However since the model is built on the concept of shared sensors, this might not be a feasible solution.
	
	% intercept communication and replace sensor data
	Furthermore, communication paths are no longer dedicated. Weather the communication channel is radio or pure software, this introduces a new attack path. A ``man in the middle'' type of attack may be performed where information data from the sensors is dropped and replaced with bad data. This would create a scenario similar to the one in the previous section, but without the need for physically modifying the sensors.
	
	\subsubsection*{Integrity}
	% no software compromise
	Having shared communication paths with the sensors means that data integrity may be compromised from outside. This goal would be achieved in the previous model only by physically modifying the sensors. Furthermore if the sensors are on the same device as the Pico, malware may modify output data leading to unsuccessful mechanism authentication.
	
	Since Pico and the authenticating mechanism are fully compartmentalised from the outside, their communication is still secure. This compartmentalisation however needs to include all types of storage and communication.

	\subsubsection*{Confidentiality}
	Unfortunately having shared sensors introduces quite a big confidentiality issue. Given that the sensor data required for authentication is shared, nothing would stop an attacker from collecting just as the Pico unlocking mechanism would. This data could then be replayed to the authenticator in order to unlock the Pico. 
	
	This is quite a critical issue. An example of feasible attack pattern would be. A peace of malware analyses when the sensors are locked, and makes assumptions as to when the Pico authenticator is locking them. Based on these assumptions the malware then captures sensor data immediately after the lock was released therefore capturing a possibly valid sample of data. 
	
	A more elaborate peace of malware could detect patterns such as time intervals or events that trigger sensor locking. Knowing these patterns it could therefore lock the sensors and gather data just before the Pico authenticator would, and then trick the authenticator by sending it a replay or possibly modified data.
	
	Yet another scenario in these circumstances would be to send the Pico authenticator constant bad data and anticipate the trigger of an explicit authentication request to the user. By locking the sensors at that key time the peace of malware could acquire a high quality data sample. Since most of the mechanisms used by the scheme are biometrics, that data sample would represent permanent damage to the user, as an authentication mechanism using that type of biometric could be replayed in any circumstance. 
	
	Since the Pico unlocking mechanism is fully compartmentalised, access its storage is secure and therefore any stored credentials are fully protected.

\subsection{Insecure communication with Pico}
This is a special case model which assumes that Pico and the authenticator we have developed are communicating over an insecure channel. The only element we need to consider is the communication between the two participants.

	\subsubsection*{Availability}
	To do.
	
	\subsubsection*{Integrity}
	To do.
	
	\subsubsection*{Confidentiality}
	To do.

	
\subsection{Shared device with shared components}
We will relax the model even more in order to better fit reality constraints when implementing the mechanism. In this model, Pico and its authentication mechanism reside in a computing model with shared storage resources. The security of Pico and its authenticator may only be as good as the underlying OS. In order to have a meaningful use-case scenario.

	\subsubsection*{Availability}
	% TODO: consider case where data is deleted from disk
	To do.
	
	\subsubsection*{Integrity}
	To do.
	
	\subsubsection*{Confidentiality}
	To do.

\subsection{Proposed secure implementation}
% http://www.trustonic.com/technology/trustzone-and-tee
% TODO: this is based on having both pico and its authenticator running in trustzone, sensor locking when capturing data, releasing when data is no longer meaningful. Secure objects for biometric data
% TODO: consider peripherals in TZ

A secure proposed implementation is viable using an Android telephone running a TrustZone enabled ARM processor available in ARMv6KZ \cite{} and later models. This device would essentially be divided into two ``worlds'': the normal world running the untrusted Android OS, and the trusted world running a small operating system written for TrustZone. Both operating systems are booted at power up. In addition the TrustZone OS loads a public/private key pair which is inaccessible from Android.  

Ideally Pico would be implemented with its authenticator within TrustZone. This would essentially guarantee complete separation from a memory perspective leaving any sort of malware attack impossible via memory. 

Persistent memory is however required in order to store data for each individual biometric mechanism used in the authentication scheme. Unfortunately this type of memory is not protected by the TrustZone OS and constitutes a way for a third party to attack the scheme. However, we could use the TrustZone OS key pair in order to encrypt biometric data on disk. Even though this data is available from Android it would be fully confidential. If properly stored within Android, the OS may even protect its integrity from outside attacks.

Let us consider however that the Android OS has been completely compromised by the attacker and is therefore ``hostile''. Under these circumstances data confidentiality can still be fully guaranteed. The TrustZone public key could still be used in order to encrypt the biometric data before writing it to disk. Attacks from a memory perspective may only be performed by modifying data stored on disk. This may only lead to a denial of service for the owner, but not a confidentiality breach.

Let us briefly discuss any issues using the availability-integrity-confidentiality framework.
	\subsubsection*{Availability}
	Only plausible attacks are denial of service through deleting biometric cache files from disk. This would require constant reconfiguration for the Pico scheme, making the Pico unavailable.
	
	\subsubsection*{Integrity}
	Data integrity may only be altered from cache files on disk.
	
	\subsubsection*{Confidentiality}
	No known attacks on data confidentiality other than capturing sensor data just as the authenticator would. However this would be possible with or without the Pico being present.

\section{Related work}
Clarke et al write in their paper \cite{clarke2005authentication} a few interesting concepts strongly related to the design proposed in this dissertation. They conduct a couple of surveys trying to assess the reliability of a PIN as an authentication mechanism for a mobile phone. In a study involving 297 participants, they assess the use of mobile phone devices in day to day life, existing authentication mechanisms, and the users' attitude towards further security options. The paper reveals a number of bad practices with PIN authentication such as $45\%$ never changing the default factory code, $42\%$ only changing it once after buying the device, weakness due to reusing the PIN in other authentications, forgetting the pin, and sharing the PIN with someone else.

The paper \cite{clarke2005authentication} however shows that $83\%$ of users are willing to accept some sort of biometric authentication mechanism in order to unlock their devices. The mechanisms included in the study ranked by popularity by an IBG study \cite{} are: fingerprint analysis, voice recognition, iris recognition, hand recognition, keystroke analysis \cite{clarke2003using}, and face recognition. The paper also talks about continuous authentication, showing that $61\%$ of users would accept a non intrusive biometric continuous authentication mechanism. Combining multiple biometric for continuous authentication is mentioned briefly, but from the perspective of having each active sequentially based on the current user task, which is a divergence point from what we are trying to achieve in this dissertation.

A similar paper \cite{clarke2002acceptance} written by Clarke et al studies the need for mobile phone authentication mechanisms alternative to the PIN. The authors conduct a survey with interesting results. A remarkable $11\%$ of participants were not even aware of the PIN authentication method used for unlocking a mobile phone. An average of $81\%$ of participants agree that different mechanisms should be used, which provide more security. Subscribes have reported both the need and desire for using alternative authentication mechanisms, but at the same time many of them do not use available alternatives available today. More details regarding the study can be found in the original paper \cite{clarke2002acceptance}.

% TODO: could maybe do more on this paper, check notes.
Gregory Williamson writes in his PhD dissertation \cite{williamson2006enhanced} about the need for an enhanced security authentication mechanism for on-line banking. He proposes a multi-factor authentication model, and presents two interesting options: the traditional one where both authentications are required in the multi-factor model (blanket authentication), and one where the second authentication mechanism is only requested from the user if the transactions appears to be risky (risk mode authentication). A risky situation is defined as either an important transaction such as withdrawing money, or a transaction made under unusual circumstances such as from an unknown device. 

A similar approach to the risk mode authentication presented by Williamson \cite{williamson2006enhanced} is proposed in this project. Our scheme yields a confidence level which may or may not be sufficient to unlock the Pico based on the current active transactions. Similarly, if the confidence level is not high enough, an explicit authentication mechanism will prompt the user for input. As the dissertation by Williamson shows, $75\%$ of users questioned in his study agree with having biometric authentication as a secondary mechanism. This shows promising results in adopting our scheme for token unlocking purposes.

Elena Vildjiounaite et al describe in their paper \cite{vildjiounaite2007increasing} a similar mechanism of combining biometric authentication data on mobile phone devices. The authors identify the security downside of granting authentication for a long time after a single verification challenge, which is the case for password based systems. They explore an alternative based a two stage risk mode authentication \cite{williamson2006enhanced}. The first stage combines biometric data in order to achieve continuous authentication. This is achieved by training a cascade classifier to a target false acceptance rate (FAR) using as data a weighted sum fusion rule. Mechanism weights are chosen based on total error rates. The second stage is only enabled if the cascade classifier does not identify the owner as being present. In low noise scenarios 80\% of the time continuous authentication is achieved without the need for an explicit challenge. In noisy situations (city and car noise), 40 to 60\% of authentication is obtained in a unobtrusive way. The cascade classifier was trained with a FAR of 1\%, with results showing a false rejection rate (FRR) of only 3 - 7\%.

The paper by Elena Vildjiounaite et al \cite{vildjiounaite2007increasing} is similar in with the solution proposed in this dissertation through the fact that it also combines multiple authentication mechanisms, each being assigned different weights. Differences between the two are in the fact that weights are maintained static in time. The weights of the sums are computed differently, and there is no mention regarding bayesian updates or probabilities. Furthermore, the authors use a classifier instead of producing a confidence level which may be used for granting different levels of security. The results presented by this paper are however encouraging, showing that continuous authentication is feasible using multiple authentication mechanisms. 
% Chapter Template

\chapter{Implementation Prototype} % Main chapter title

\label{Chapter5} % Change X to a consecutive number; for referencing this chapter elsewhere, use \ref{ChapterX}

\lhead{Chapter 5. \emph{Implementation Prototype}} % Change X to a consecutive number; this is for the header on each page - perhaps a shortened title

Thus far we have developed a new Pico authentication scheme and assessed it using our own token unlocking framework. We then have performed a threat model from an availability, integrity, and confidentiality perspective and have suggested the safest implementation which would be as feasible as possible for the user to adopt.

In this chapter we will described design and implementation details for the prototype of the proposed scheme. The implementation platform will be the Android OS, which uses a Java based SDK for application development.

\section{Authenticator design}
The user authenticator for Android is designed to work as a bound service called UAService. Periodically the service outputs to registered Pico clients the status of the authentication process. Any application may be a client as long as it registers with the service. Furthermore, explicit authentication update requests may be performed by the Pico client.

Since different authentication mechanisms require different update periods, we have chosen each mechanism to be represented by an independent service. This allows for more flexibility such as periodic sampling with different intervals. Another feasible use case for example would be performing voice recognition based on the first few seconds of an outgoing or incoming call. This would require a service that is triggered by a PHONE\_STATE intent.

Each authentication mechanism service is started and managed by the UAService. Communication between the UAService and each authentication mechanism is enabled through intents. Using this communication link, requests can be made from each individual authentication mechanism in order to get the current confidence level. This value is equal to the probability that the owner is present, multiplied by the weight carried by the mechanism. Given that each mechanism runs as an independent service, weight decay may easily be performed using an AlarmManager or simply a function which is called periodically within the authentication thread.

Either periodically UAService gets the confidence level and weight from each mechanism. It then calculates the overall result. If the result is above the threshold requested by the Pico client, a ``Message'' is passed back saying that Pico should unlock. Otherwise a negative result is returned, letting the Pico know it should be locked. 

\section{Implementation details}

\subsection{Main application and services}
The user authenticator for Android is designed to work as a bound service. According to the Android documentation a bound service exposes functionality to other application components and as well as external applications. It is developed as a regular service which implements the onBind() callback method to return an IBinder. The service lives only as long as a component is bound to it. The service implementation class is called UAService.

The UAService is a central node in the application. It is a bound service for any Pico client which wishes to register for events. Furthermore, it binds any authentication mechanism that is available, enabling it for authentication. 

The UAService periodically broadcasts intents to registered clients saying if the Pico should be locked or unlocked. The following interface is exposed to available Pico applications through the ``what'' parameter of the ``Message'' class:
\begin{description}
  \item[MSG\_REGISTER\_CLIENT] \hfill \\
  Used for registering a client. The ``Message'' should have as the ``arg1'' parameter the level of confidence required for unlocking. This value should range from 0 to 100. Any values outside these limits will be truncated within the range.
  
  \item[MSG\_REGISTER\_CLIENT] \hfill \\
  Used for any application to unregister as a listener for this service. No additional parameters required.
  
  \item[MSG\_GET\_STATUS] \hfill \\
  Used by any application when an authentication request is needed. Although the service periodically broadcasts to its registered clients what is the authentication status, explicit requests may also be performed using this ``Message''.
\end{description}

UAService interacts with AuthMech objects in order to communicate with an authentication mechanism. Each object is responsible for interfacing the communication with an authentication mechanism. A valid authentication mechanism service needs to extends the AuthMechService abstract class which defines a standard way of communication with the UAService.

Each AuthMechService is programmed as a bound service. UAService binds these services through AuthMech objects. Each AuthMechService exposes the following message passing interface:
\begin{description}
  \item[AUTH\_MECH\_REGISTER] \hfill \\
  Used for registering the UAService service as a client to the AuthMechService.
  
  \item[AUTH\_MECH\_UNREGISTER] \hfill \\
  Used for unregistering the UAService service as a client to the AuthMechService.
  
  \item[AUTH\_MECH\_GET\_STATUS] \hfill \\
  Used by the UAService in order to request the authentication confidence from the AuthMechService. The value will be returned in the arg1 parameter of the Message passed.
\end{description}

\subsection{Authentication mechanisms}
In order to create a functional prototype, we implemented a couple of mechanisms. The focus of the project is not the quality of the biometric mechanisms involved in the prototype, their sole purpose being to demonstrate a proof of concept. Android devices offer a wide range of sensor data such as GPS, accelerometer, camera, and microphone.

Based on the sensor data offered by Android devices, a wide range of biometric mechanisms can be developed. A non extensive list may include face recognition, voice recognition, iris scanning, keystroke analysis, gait recognition, and many others. The scheme however, requires a clear predefined list of mechanisms offering continuous authentication as well as explicit.

A number of continuous authentication mechanisms may be developed using solely the standard sensors offered by Android devices. The following non-extensive list was achieved, with details regarding what each mechanism means and how it should be implemented:
\begin{description}
  \item[Face recognition] \hfill \\
  This mechanism was also implemented for the purpose of the project, and further details are offered in the following sections. The idea is that based on user behaviour, sampling of the user's face can be performed without any explicit requests. For instance when an user is unlocking the phone it is highly likely that he will be looking at the screen. This creates a good opportunity for the authentication mechanism service to capture an image and determine the confidence level that the unlocker is the actual user.
  
  \item[Voice recognition] \hfill \\
  This mechanism was also implemented for the purpose of the project, and further details are offered in the following sections. Note that voice sampling does not necessarily imply a voice password of any kind. Voice can be analysed from a feature's perspective, regardless of the words being spoken. Voice sampling can be performed at any time. With a frequent enough sampling rate, the owner of the device is likely to be present in most voice recordings. For even better confidence the mechanism should be implemented to start recording when a call is either made or received. On Android this can be achieved by implementing a listener for a PHONE\_STATE intent.
  
  \item[Iris scanning] \hfill \\
  Similar to face recognition, this can be implemented by taking advantage of user behaviour while using the phone. When the phone is unlocked, the user is very likely to face the front camera, allowing for a good face capture. The only problem with this mechanism is the quality of pictures offered by most phones. If the sampling quality is not sufficiently good, meaningful features from the iris may not be extracted, making the confidence level of the mechanism relatively low.
  
  \item[Keystroke analysis] \hfill \\
  The principle of keystroke analysis is based on the patterns in which the user types on his mobile phone. Different features can be extracted here, such as: the amount of time the user takes to type letter sequences, words, or individual letters, words per minute, frequent used words, and many others. Based on these features a confidence level can be generated (not carrying a considerable amount of weight). This is harder to implement using solely the Android SDK. A good starting point would be to have a keyboard application developed for the user, which also communicates with the authentication mechanism. Obviously if the keyboard is disabled by an attacker this should still be considered, especially if the authenticator was originally configured to listen for input.
  
  \item[Gait recognition] \hfill \\
  This mechanism is based on the concept of analysing individual walking patterns. Different people walk in different ways, which even though may not be entirely unique for every individual, would still provide some confidence level regarding the user of the device. In the lack of an existing reliable library, efforts have been made to implement this mechanism, unfortunately unsuccessful. The implementation requires accelerometer data from the device, which needs to be normalised from the sensor's perspective. Android offers activity recognition for walking, driving, or standing still. This is achieved by registering a sensor callback for the TYPE\_STEP\_DETECTOR composite sensor.
  
  \item[Ear shape analysis] \hfill \\
  Studies have shown \cite{} that the shape of the human ear contains enough unique features in order to perform biometric authentication. Taking advantage on user behaviour when using a phone, accurate images can be captured in order to perform such analyses. Within a few seconds from answering the phone, given no peripherals are attached, the user is going to move the phone towards his ear. Based solely on timing and/or accelerometer data, accurate pictures could be taken of the user's ear before the camera gets too close. Images captured by such a mechanism could then be used to calculate an accurate confidence whether the owner is the person who is answering the phone.
  
  \item[Service utilisation] \hfill \\
  This proposed mechanism is not biometric based. It exploits patterns in the Android phone's service and app utilisation. Based on current running applications, services, and the time they were started my create a model where some confidence is given as to whether the owner has changed. This mechanism would only be effective in detecting sudden changes, but may easily be obstructed either by removing the Pico authenticator. Furthermore sudden changes in ownership are not promptly detected which is why the mechanism would have a low weight in the overall scheme. 
  
  \item[Proximity devices] \hfill \\
  A mechanism may be developed which tries to connect with other devices also running the authenticator. The two owners don't necessarily need to know one another for the acknowledgement to be performed. Based on day to day activities, users tend to interact or at least be around a lot of the same people. Weather regular travel schedules, or as a better scenario, working in an office, constantly being in the presence of other known devices should give a confidence as to whether the device is in the presence of the user. This mechanism could only be circumvented by co-workers or friends unlocking the Pico, which is why it should never have sufficient weight to unlock the Pico on its own. In combination with other mechanisms however, it would provide a good sense regarding the owner of the device. It the device is ``in good company'' there is a good chance the owner is also present. This should be enhanced with time data as to when other trusted device are recognized. Furthermore, based on the ID of the devices the owner comes in proximity to, the mechanism may have different weights for different devices. As an example, even though travelling with your family on holiday and most of the devices there are unknown, given that a number of frequent IDs are in the proximity of the device, the mechanism should still consider to some extent that it is in the possession of its owner. This would work similarly with the Picosiblings idea, but each Picosibling is a device running this authentication mechanism which is frequently in the proximity of the owner.
  
  \item[Location data] \hfill \\
  This mechanism is also non biometric. It is similar to ``Proximity devices'' and much easier to implement. Based on Android GPS data, the phone may detect whether it is in an usual location or not. Just as ``Proximity devices'' this mechanism should not carry a high weight in the scheme, especially since it would not provide accurate results in scenarios such as holidays.
  
  \item[Picosiblings]
  The original Picosiblings mechanism may also be used with this scheme. Although not part of the standard set of Android device sensors, if available, a Picosiblings implementation may be included as one of the authentication mechanisms.
\end{description}

Some of the continuous authentication mechanisms may also be used for explicit authentication. Based on the non-extensive list mentioned above, the user may be notified to provide accurate information for the following mechanism: face recognition, voice recognition, iris scanning, keystroke analysis, gait recognition, and ear shape analysis. By notifying the user that he has to provide more accurate authentication data, the mechanisms get a better chance of providing valid results. The decay rate after explicit authentication will be slower in order to maintain the continuous authentication property of the Pico for the duration of the authentication session.

In addition to the mechanisms mentioned above, a number of explicit authentication mechanisms which do not satisfy the continuous authentication property of the Pico may be implemented using the Android SDK. It is important to note that any other mechanisms not included in this list need to satisfy the memory property of the Pico, according to which the user doesn't need to remember any known secret. A non-extensive list of mechanisms includes the following:
\begin{description}
  \item[Fingerprint scanner] \hfill \\
  Devices which may have a fingerprint scanner incorporated, such as the IPhone 5S may use this sensor in order to gather biometric data used for authentication. This mechanism cannot actively be used for continuous authentication due to the fact that the user doesn't come in contact with the sensors on a regular basis. A mechanism can therefore request explicit fingerprint data, which would then be compared with the owner's biometric model, outputting a confidence for the authentication. This confidence will be combined in the calculation of the overall scheme confidence just as any other mechanism, the only difference being in terms of weight and decay rate.
    
  \item[Hand writing recognition] \hfill \\
  The user may be prompted to use the touch screen in order to write a word of his choice. This would guarantee the memoryless property, since the user doesn't need to remember any sort of secret. The handwriting would be analysed with a preconfigured set of handwriting samples in order to determine the confidence level that the owner produced the input.
  
  \item[Lip movement analysis] \hfill \\
  According to the paper \cite{} by TODO, lip movement during speaking may be used to uniquely identify individuals. Lip movement analysis would be performed similarly as described in the paper. The confidence level that the owner produced the input would then be combined in the authentication scheme. This may also be implemented as a continuous authentication mechanism, with with lower success rate expectations due to the way users tend to hold mobile phones, which usually doesn't expose the mouth to the camera.
\end{description}

\subsubsection{Dummy mechanism}
In order to perform tests for different confidence levels, a dummy authentication mechanism was implemented using the AuthDummyService class. It extends the AuthMechService abstract class, which makes it an independent service just to maintain the application model consistent. 

The service contains a data access object (DAO) which in this case only produces random confidence levels within a given range. A thread running within the service makes periodic requests to the DAO in order to mimic an authentication mechanism which periodically samples for data. The service is updated based on the produced value. 

When the UAService wants to update its overall confidence, it makes a AUTH\_MECH\_GET\_STATUS request to the AuthDummyService service, which returns the most recent confidence level multiplied by the current decay factor. The result is combined with the result from the remaining authentication mechanism services.

\subsubsection{Voice recognition}
The voice recognition mechanism is implemented as a VoiceService class extending the AuthMechService abstract class. When the services onCreate() method is called, it starts an authenticator thread which periodically samples data from the device's microphone.

% TODO: carry on from here..
The library used for voice recognition is called Recognito.

\subsubsection{Face recognition}

\subsection{Owner configuration}

\section{Threat model}
% TODO:
% - clear cache data from android
% TODO: not really relevant since this is just a prototype


\section{Results}
 
% Chapter Template
% TODO: do the evaluation!

\chapter{Evaluation} % Main chapter title

\label{Chapter6} % Change X to a consecutive number; for referencing this chapter elsewhere, use \ref{ChapterX}

\lhead{Chapter 6. \emph{Evaluation}} % Change X to a consecutive number; this is for the header on each page - perhaps a shortened title

\section{Framework evaluation}
We will continue by evaluating the new proposed scheme with the token unlocking framework defined in the previous chapter. 

\begin{description}
  \item[Memory-effortless: Satisfied] \hfill \\
  None of the authentication mechanisms require any sort of known secret. Authentication is granted based on biometrics and behavioural analysis.
  
  \item[Nothing-to-carry: Quasi-satisfied] \hfill \\
  This property is only quasi-satisfied due to the fact that it relies on the implementation of the design. Ideally all authentication data should be gathered from an unified device containing the Pico. Alternatively however, the scheme can be implemented using individual sensors which the owner would have to carry, which is why the property is not fully granted.
  
  \item[Easy-to-learn: Satisfied] \hfill \\
  In order to satisfy Pico's property of continuous authentication, all mechanisms part of the scheme I developed also need to have this property. Therefore the authentication process is non-transparent to the user, and therefore there is nothing to learn.
  
  \item[Efficient-to-use: Satisfied] \hfill \\
  The authentication data is collected either at fixed time intervals, or is fired during special events. The authentication process however, does not fully depend on recent data. A response may be generated without any recent authentication data. Therefore the time spent by the mechanism to generate a response is immediate.
  
  \item[Infrequent-errors: Quasi-satisfied] \hfill \\
  Given that the scheme depends on biometric mechanisms, the quality of the errors is as good as the underlying biometrics. If the scheme cannot generate a high enough confidence an explicit biometric challenge will be issued for the user to satisfy. Since the original biometric mechanisms do not have this property, to some extent neither will the scheme I have designed. However, the scheme is combining multiple biometrics results with different score weights based on importance and accuracy. This is much more likely to be accurate, which is why I will mark this as Quasi-satisfied. For a more accurate response, the design needs testing with a high quality prototype. 
  
  \item[Easy-recovery-from-loss: Not-satisfied] \hfill \\
  Token based mechanisms in general do not have this property due to the inconvenience of replacing the token. In our case, the property is also not satisfied. The user would have to re-acquire a new token and reconfigure the owner's biometric data. Furthermore based on the mechanism, such as location settings or gait recognition, the token is likely to require an adaptation period.
  
  \item[Availability: Satisfied] \hfill \\
  Some mechanisms are not always available even though enabled, especially due to the continuous authentication property. As an example gait recognition while sitting in an office. However, the scheme may use a multitude of mechanisms with the unlikeness that all of them are unavailable. For instance location history may predict with a certain confidence that the owner still in possession of the token. This propery is aided by the explicit authentication mechanism which requires explicit input from the user.
  
  \item[Accessible: Satisfied] \hfill \\
  Due to the fact that the scheme is based on multiple biometrics and location settings, I consider this property to be Satisfied or as a very least Quasi-satisfied. The scheme functions based on available biometrics, without having any predefined solutions. It is highly unlikely that the owner cannot generate any of the available biometric inputs, especially for some such as ``face recognition''.
  
  \item[Negligible-cost-per-user: Quasi-satisfied] \hfill \\
  This property depends on the way in which the scheme is implemented. If the implementation is based on high quality sensors embedded in items of clothing and such, then the property is not satisfied. If the implementation reuses sensors that the user already possesses, the the property is fully satisfied as the cost is 0. An example of such an implementation would be an Android application/service possibly using the future Google Glass hardware.
  
  \item[Mature: Not satisfied] \hfill \\
  This property is not satisfied as the project is at the level of a work in progress prototype. The design is quite fresh and was not implemented by any third party. Neither was is reviewed by the open source community or has had any user feedback.
  
  \item[Non-proprietary: Satisfied] \hfill \\
  Anyone can implement the scheme without any restrictions such as royalty checks or any other sort of payment to anyone else.
  
  \item[Resilient-to-physical-observation: Satisfied] \hfill \\
  Since the mechanism is based on biometric data, simple observations from an attacker cannot lead to compromising the user's authentication to the token. The attacker would have no way of reproducing the input through simple observation.
  
  % TODO: think about explicit authentication to keep this quasi
  \item[Resilient-to-targeted-impersonation: Quasi-satisfied] \hfill \\
  Saying that the scheme Quasi-satisfies this property is a bit generous. Each of the mechanisms is vulnerable to a replay attack. An attacker may record one of the user's biometric and replay it as a token input. However, given that the token uses multiple mechanisms, some of which being location based, this is a highly unlikely occurrence. The only vulnerable point would be the explicit authentication mechanisms, which carry a lot of weight.
  
  % TODO: find citation for this
  \item[Resilient-to-throttled-guessing: Satisfied] \hfill \\
  The amount of throttled guessing required for the user to break one of the biometric mechanisms is far too large for this to actually be a threat.
  
  \item[Resilient-to-unthrottled-guessing: Satisfied] \hfill \\
  Given that the Resilient-to-throttled-guessing property is satisfied, this property is also satisfied.
  
  % TODO: talk more about this
  \item[Resilient-to-internal-observation: Satisfied] \hfill \\
  This property does not apply to this scheme. 
  
  \item[Unlinkable: Not-satisfied] \hfill \\
  Just as any of the biometric mechanisms, this property is not satisfied by the mechanism. The authentication data maps uniquely to the owner of the token.
  
  \item[Continuous-authentication: Satisfied] \hfill \\
  The mechanism was designed with continuous authentication in mind. Data is collected periodically with a confidence weight decaying over time. This allows for the token to be used at any time based on current existing data. The only exception breaking the model would be the explicit authentication mechanisms, but these could only be triggered at the beginning of an authentication process using the token.
  
  \item[Multi-level-unlocking: Satisfied] \hfill \\
  This property is fully satisfied by the authentication mechanism. It allows the token to grant access to different authentication accounts based on the precomputed level of confidence that the owner is present. 
  
\end{description}

Let us continue by comparing the results of our proposed scheme with the original Picosiblings solution. The results are summarised in the following table. In the ``Proposed scheme'' column, properties which are highlighted in order to facilitate the comparison with the Picosiblings solution. The colour green means that the proposed scheme is better, red worse, and no colour means that both properties have the same value.

\begin{table}
    \begin{tabular}{l|l|l}
    Property                            & Picosiblings    & Proposed scheme \\ \hline
    Memory-effortless                   & Satisfied       & Satisfied       \\
    Nothing-to-carry                    & Not-satisfied   & \cellcolor{green!25} Quasi-satisfied \\
    Easy-to-learn                       & Satisfied   	  & Satisfied       \\
    Efficient-to-use                    & Quasi-satisfied & \cellcolor{green!25} Satisfied       \\
    Infrequent-errors                   & Quasi-satisfied & Quasi-satisfied \\
    Easy-recovery-from-loss             & Not-satisfied   & Not-satisfied   \\
    Availability                        & Satisfied       & Satisfied       \\ \hline
    Accessible                          & Not-satisfied   & \cellcolor{green!25} Satisfied       \\
    Negligible-cost-per-user            & Not-satisfied   & \cellcolor{green!25} Quasi-satisfied \\
    Mature                              & Not-satisfied   & Not-satisfied   \\
    Non-proprietary                     & Satisfied       & Satisfied       \\ \hline
    Resilient-to-physical-observation   & Satisfied       & Satisfied       \\
    Resilient-to-targeted-impersonation & Satisfied       & Satisfied       \\
    Resilient-to-throttled-guessing     & Satisfied       & Satisfied       \\
    Resilient-to-unthrottled-guessing   & Satisfied       & Satisfied       \\
    Resilient-to-internal-observation   & Satisfied       & Satisfied       \\
    Unlinkable                          & Satisfied       & \cellcolor{red!25} Not-satisfied   \\
    Continuous-authentication           & Satisfied       & Satisfied       \\
    Multi-level-unlocking               & Not-satisfied   & \cellcolor{green!25} Satisfied       \\
    \end{tabular}
\end{table}

As the table shows, the proposed solution does not completely dominate the Picosiblings solution, and this is only because of the ``Unlinkable'' property. Given that our solution is fundamentally based on biometric data, this property could never be achieved. However, our solution performs better than Picosiblings in 5 other properties. Important points of improvement are accessibility, which makes the proposed scheme viable for a larger number of people. The Multi-level-unlocking property is another good improvement, allowing for an enhanced security model.



\section{Conceptual design threat Model}
% availability (dos)
%   - communication
%   - cpu
%   - battery
%   - practicality of cryptography on small devices

% integrity
% confidentiality

% are sensor readings what they say or are they manipulated

An accurate threat model on the proposed unlock mechanism must start by analysing the set of assumptions made about the mechanism. From there we can identify available threats and how the scheme can be exploited in order to unlock the Pico without owner permission. Throughout the threat model we will explain how relaxing the initial set of assumptions may change the security outcome. Each model is analysed from an Availability, Integrity, and Confidentiality.

It is important to note that confidentiality is an important category in this evaluation. This is because the device will store sensitive biometric data which is directly linkable to the user. Losing this data, especially in plain-text, would disable the user from ever using the biometric device for which the data was leaked. This is due to the fact that the leaked data could always be replayed, successfully tricking the biometric mechanism.

In each subsection, the model will obviously only introduce issues with the mechanism. Therefore when reading a subsection, the issues are not only those currently presented, but also those from previous subsections that lead up to that point.

\subsection{Dedicated device with dedicated sensors}
We will start from the assumption that the unlock mechanism is integrated on the same device with the Pico. The device is assumed to be dedicated and runs no other software. Furthermore, the set of available sensors will also be integrated within the device. Alternatively there may also be peripheral sensors, with no way for an attacker to tamper with the communication to the authenticator. 

	\subsubsection*{Availability}
	From an availability point of view, an outside attacker cannot create a denial of service scenario. Interactions with the device are performed physically, so therefore the device cannot be made unavailable while in the possession of its owner. If the Pico would temporarily lose ownership, from a software perspective it would lock up due to mismatching biometric and location data, but would become available again in the presence of the owner. 
	
	Only hardware modification would affect data availability. Simply disconnecting the sensor would not affect the scheme's ability to generate viable results due to the fact that multiple biometrics are used. However an attacker could modify a sensor to output wrong data, tricking it into saying the user is never the owner. This would create a successful denial of service attack path where a few sensors output that the owner is never present. 
	
	\subsubsection*{Integrity}
	Communication paths are not accessible from the outside and therefore cannot be tampered with in order to modify data. Furthermore the device is not running any other software and is therefore safe from any malware attacks. 
	
	Only physical tampering with the device would change data integrity. Modifying one of the sensor's and changing its output to some random data would be undetectable by the mechanism. 
	
	% TODO: resurected ducklings how to?
	\subsection*{Confidentiality}
	No software access as well as no communication with the outside (i.e. wired communication) means that data is safe as long as the device is with its owner.
	
	If the device were to be lost, 
	Storage data should be kept encrypted, similar to the way Ironkey \cite{} protects its data. Unfortunately an attack path may already be identified which is due to the fact that using this model the decryption key needs to be stored on the device. An attacker which has hardware access could therefore extract the key and decode the data. The original Picosiblings solution circumvented this approach by keeping 

\subsection{Dedicated device with shared sensors}
We will relax the original set of assumptions by saying that the communication path with the sensors is no longer secure. Furthermore the sensors may be shared with other owners, via a wireless communication link for example. Another feasible scenario is that although sensors are located on the same device as the Pico, the Pico application is fully compartmentalised from the outside world. 

What we are trying to stress with this scenario is that the sensors are no longer part of a trusted secure box, but are outside and communication with them, as well as their input may no longer be secure.

	\subsubsection*{Availability}
	% keep all sensors locked
	Since the sensors are no longer dedicated, other users may access sensor data. Depending on the hardware and software platform supporting the sensors, this may lead to a denial of service attack on the scheme. For example, if the sensors may only have one owner at a time, an attacker may request data from all sensors keeping them locked from the biometric authentication mechanisms. If the system is built in such a way, then there is nothing the scheme could do to prevent this other than keep the sensors constantly locked for itself. However since the model is built on the concept of shared sensors, this might not be a feasible solution.
	
	% intercept communication and replace sensor data
	Furthermore, communication paths are no longer dedicated. Weather the communication channel is radio or pure software, this introduces a new attack path. A ``man in the middle'' type of attack may be performed where information data from the sensors is dropped and replaced with bad data. This would create a scenario similar to the one in the previous section, but without the need for physically modifying the sensors.
	
	\subsubsection*{Integrity}
	% no software compromise
	Having shared communication paths with the sensors means that data integrity may be compromised from outside. This goal would be achieved in the previous model only by physically modifying the sensors. Furthermore if the sensors are on the same device as the Pico, malware may modify output data leading to unsuccessful mechanism authentication.
	
	Since Pico and the authenticating mechanism are fully compartmentalised from the outside, their communication is still secure. This compartmentalisation however needs to include all types of storage and communication.

	\subsubsection*{Confidentiality}
	Unfortunately having shared sensors introduces quite a big confidentiality issue. Given that the sensor data required for authentication is shared, nothing would stop an attacker from collecting just as the Pico unlocking mechanism would. This data could then be replayed to the authenticator in order to unlock the Pico. 
	
	This is quite a critical issue. An example of feasible attack pattern would be. A peace of malware analyses when the sensors are locked, and makes assumptions as to when the Pico authenticator is locking them. Based on these assumptions the malware then captures sensor data immediately after the lock was released therefore capturing a possibly valid sample of data. 
	
	A more elaborate peace of malware could detect patterns such as time intervals or events that trigger sensor locking. Knowing these patterns it could therefore lock the sensors and gather data just before the Pico authenticator would, and then trick the authenticator by sending it a replay or possibly modified data.
	
	Yet another scenario in these circumstances would be to send the Pico authenticator constant bad data and anticipate the trigger of an explicit authentication request to the user. By locking the sensors at that key time the peace of malware could acquire a high quality data sample. Since most of the mechanisms used by the scheme are biometrics, that data sample would represent permanent damage to the user, as an authentication mechanism using that type of biometric could be replayed in any circumstance. 
	
	Since the Pico unlocking mechanism is fully compartmentalised, access its storage is secure and therefore any stored credentials are fully protected.

\subsection{Insecure communication with Pico}
This is a special case model which assumes that Pico and the authenticator we have developed are communicating over an insecure channel. The only element we need to consider is the communication between the two participants.

	\subsubsection*{Availability}
	To do.
	
	\subsubsection*{Integrity}
	To do.
	
	\subsubsection*{Confidentiality}
	To do.

	
\subsection{Shared device with shared components}
We will relax the model even more in order to better fit reality constraints when implementing the mechanism. In this model, Pico and its authentication mechanism reside in a computing model with shared storage resources. The security of Pico and its authenticator may only be as good as the underlying OS. In order to have a meaningful use-case scenario.

	\subsubsection*{Availability}
	% TODO: consider case where data is deleted from disk
	To do.
	
	\subsubsection*{Integrity}
	To do.
	
	\subsubsection*{Confidentiality}
	To do.

\subsection{Proposed secure implementation}
% http://www.trustonic.com/technology/trustzone-and-tee
% TODO: this is based on having both pico and its authenticator running in trustzone, sensor locking when capturing data, releasing when data is no longer meaningful. Secure objects for biometric data
% TODO: consider peripherals in TZ

A secure proposed implementation is viable using an Android telephone running a TrustZone enabled ARM processor available in ARMv6KZ \cite{} and later models. This device would essentially be divided into two ``worlds'': the normal world running the untrusted Android OS, and the trusted world running a small operating system written for TrustZone. Both operating systems are booted at power up. In addition the TrustZone OS loads a public/private key pair which is inaccessible from Android.  

Ideally Pico would be implemented with its authenticator within TrustZone. This would essentially guarantee complete separation from a memory perspective leaving any sort of malware attack impossible via memory. 

Persistent memory is however required in order to store data for each individual biometric mechanism used in the authentication scheme. Unfortunately this type of memory is not protected by the TrustZone OS and constitutes a way for a third party to attack the scheme. However, we could use the TrustZone OS key pair in order to encrypt biometric data on disk. Even though this data is available from Android it would be fully confidential. If properly stored within Android, the OS may even protect its integrity from outside attacks.

Let us consider however that the Android OS has been completely compromised by the attacker and is therefore ``hostile''. Under these circumstances data confidentiality can still be fully guaranteed. The TrustZone public key could still be used in order to encrypt the biometric data before writing it to disk. Attacks from a memory perspective may only be performed by modifying data stored on disk. This may only lead to a denial of service for the owner, but not a confidentiality breach.

Let us briefly discuss any issues using the availability-integrity-confidentiality framework.
	\subsubsection*{Availability}
	Only plausible attacks are denial of service through deleting biometric cache files from disk. This would require constant reconfiguration for the Pico scheme, making the Pico unavailable.
	
	\subsubsection*{Integrity}
	Data integrity may only be altered from cache files on disk.
	
	\subsubsection*{Confidentiality}
	No known attacks on data confidentiality other than capturing sensor data just as the authenticator would. However this would be possible with or without the Pico being present.

% IMPLEMENATION==============================================================


\section{Threat model}
% TODO:
% - clear cache data from android
% TODO: not really relevant since this is just a prototype
% TODO: any cryptographic keys and content is not secure as the application may be decompiled
% TODO: DOS by generating bad sensor data
% TODO: check William Ench et al year
% TODO: referenced android website instead of talk about components
% TODO: more mmap_min_addr stuff. why does it help?

Even though the scheme implementation is a proof of concept, we will continue by analysing different threat models. This will reveal any flaws behind the concept, allowing for a more robust future implementation.

The purpose of the Pico token is to provide a robust authentication mechanism, without the use of any secrets for the owner to remember. Where the Pico unlocking scheme fits, is correctly identifying the owner of the Pico. Attacks may be performed in the form of malware installed on the device while still within the possession of its owner. The main threat however comes from an attacker having physical access to the user's Pico. 

It is important to note that since this is a purely software implementation, physical access may mean either that the attacker is in possession of the phone, or that it may replicate the secretes of the victim's Pico on a separate device. Replicating the Pico secretes would clearly create much more damage for the user from a cost perspective. A total reset of authentication credentials would be necessary for all accounts registered with the Pico device.



\subsection{Prototype threat model}
Let us now continue by studying the threat model of the Pico authenticator application. We will consider the security mechanisms presented above as the predefined assumptions made in this model. In order to reduce the threat space we will consider the application is running on a hand held device running Android 4.4.2 with all recent updates.

\subsubsection*{Availability}
Breaking the scheme's availability if the device is in the possession of the attacker is relatively trivial. The application can be uninstalled, or the application data cache can be cleared, therefore removing the owner biometric models for the different mechanisms. Furthermore, in this case the owner is already no longer in possession of their Pico, so basically the Pico is already made unavailable..

Let us continue however and study what can be achieved from a DoS perspective by the attacker from the perspective of the individual user app accounts, which would need to be reset by the owner. In order to gain any sort of access and make credentials reset not possible, or at least have a chance in doing so, the attacker would have to unlock the Pico.

From a malware attack perspective data used by the authentication should not be modifiable. This would guarantee that all mechanisms have their cached biometric data available at all times and may function properly. Due to the Linux permissions mechanism and the fact that each application has its own UID and GID, data stored in internal memory should not be readable or writeable by any other user level application in the system. If however the device is rooted and the owner is mislead into granting root privileges to another application, then the security model would be broken and the data would be exposed. This could lead to deletion which would make the mechanisms not function properly, resulting in a DoS attack.

% TODO: can we fake data from sensors?

\subsubsection*{Integrity}
Just as mentioned in the Availability section, the authenticator should be safe against any data accesses from other applications as long as the application does not have root privileges. This would allow malware to break the integrity property of the data.

From a data flow point of view Intents used for communication within the authenticator as well as with the Pico application are not modifiable. Furthermore Intents are not broadcasted using the implicit Android broadcast mechanism, which makes them impossible to replay or even intercept.

\subsubsection*{Confidentiality}
Considering a circumstance where a root malware process would have access to the authenticator's data stored on disk, this would not lead to a direct compromise of the owner biometric data. All cached files are stored in internal memory in encrypted format. The mechanism used for encryption is RSA, with the private key stored using the Android KeyChain API.

On a rooted device however, the encryption layer provides only another bi-passable layer of security. With root access, an application could retrieve the master key of the KeyChain and use it to retrieve or private key and decode the owner's biometrics.

From a data flow perspective, internally the Pico authenticator uses an self-developed broadcast system. Client processes need to register with the broadcaster, such as the UAService, in order to receive updates. This ensures data confidentiality throughout the system. Furthermore, the authenticator and Pico should be released under the same author. This would allow locking the application from outside Intents as well as interaction with different components. Sandboxing communication is always a desirable property from a confidentiality perspective.

The paper by Adrienne Porter Felt et al \cite{felt2012android} shows that according to their surveys only 17\% of users pay attention to the Android permissions dialogue, and only 3\% understand what each permission represents. A malware application which has granted full permissions gets pass the Bouncer and is installed as an application. Even so, due to the Linux permission model adopted by Android, the confidentiality of the authenticator's data would not be compromised. Instead however, the malware application may collect all relevant on its own from the user, allowing for a powerful replay attack in the future.

\section{Future work}
The application was implemented as a proof of concept. It is developed in order to show that different data may be obtained without the owner's knowledge. Additional improvements can be made in order to increase the confidence level of the authenticator.  Furthermore, due to time constraints and unavailability of free to use biometric libraries, a number of mechanisms were not implemented. The list can easily be extended by simply creating a class which extends the ``AuthMechService'' abstract class.

One way to improve the voice recognition mechanism would be to start sampling data whenever a call is active. This would increase the chances of capturing an accurate sample of the owner's voice. In this context, a better voice recognition library can be used, which supports multiple speakers and/or ignores background noise. If such a library is not available, we can rely on the fact that most of the times people take turns when speaking. For the duration of the call, with a high enough sampling frequency, the individual sampling voice of both participants should be captured. However, it is important to take into account a situation in which the thief is calling the owner on a different phone in order to unlock his Pico.

Immediate improvements can be made to the face recognition mechanism. Just as recommended in the description of the mechanism's implementation, another library which provides more meaningful face coordinates may be used for face detection. Alternatively, and preferably, a different library which performs both face detection an recognition can be integrated with the mechanism.

Another improvement for the face recognition mechanism would be from the data sampling perspective. Instead of capturing images at a fixed interval, pictures should be taken only when the phone unlock event is triggered. While the phone is unlocked it is highly likely that the user will face its front camera.This would provide better chances of processing meaningful data. 
% Chapter Template

\chapter{Conclusion} % Main chapter title

\label{Chapter7} % Change X to a consecutive number; for referencing this chapter elsewhere, use \ref{ChapterX}

\lhead{Chapter 7. \emph{Conclusion}} % Change X to a consecutive number; this is for the header on each page - perhaps a shortened title

The purpose of this dissertation was to provide an alternative unlocking scheme for the Pico token \cite{stajano2011pico}. We have started by analysing its design in chapter \ref{Chapter2}. Using this we have identified that a new unlocking mechanism requires to be memorywise effortless, and provide support for continuous authentication.

We have identified and briefly presented the web authentication UDS assessment framework developed by Bonneau et al \cite{bonneau2012quest}. This work provided an initial evaluation of Pico using Picosiblings. Analysing the paper revealed issues of the Pico token, some of which directly related to its unlocking mechanism (e.g. not easy to learn).

To have a better way of evaluating a token unlocking scheme, in section \ref{tokenframework} we have created a derivation of the original framework by Bonneau et al. Some properties were removed, and others were changed in order to fit the context of token unlocking mechanisms. Furthermore, we have extended the framework by adding 4 original properties.

Having a list of requirements, and a way of assessing the new solution, we have designed a scheme based on combining biometric and behavioural analysis mechanisms. Each mechanism generates a probability that the owner is in possession of the token. These probabilities are combined using a modified weighted sum, in order to generate an overall confidence level. Another original contribution is that each mechanism has an initial weight that decays in time from one valid data sample to the other. 

% android prototype
Given a well defined design of the new scheme, we have successfully developed a prototype using a Google Nexus 5 device that runs an Android 4.4.2 operating system. This has proven that the solution can be implemented using existing hardware. Furthermore, the prototype offered useful insight for future implementations of the scheme. We have concluded that an efficient implementation requires authentication mechanisms to run as independent processes. We have presented a basic application design, and showed how different components should interact. As presented, data should be validated prior to analysis, and in the lack of a valid sample, the weight decay process should continue. An unexpected problem was that although the Android platform offers a wide range of sensors for the implementation of multiple mechanisms (details in appendix \ref{AppendixC}), the lack of open source biometric libraries has lead to a low precision of the overall scheme.

% evaluation of scheme
The proposed unlocking mechanism was evaluated using the UDS framework developed by Bonneau et al, and the token unlocking framework developed in section \ref{tokenframework}. The results of the analysis have shown that the new proposed scheme cannot completely outperform Picosiblings due to the ``unlinkable'' property. Otherwise, results have shown an overall improvement in the number of offered properties. In addition, the new unlocking mechanism offers the possibility of having a granular unlocking, where the Pico token could offer individual locked and ulocked states based on the current confidence level and the security level required by the account.

A threat model of the prototype has shown a number of attacks that may be performed on the Android application. Important insight was provided when studying design model attacks in section \ref{secdesignattacks}. This has shown that in scenarios where no valid data can be collected, a compromise needs to be made regarding the time interval between successful explicit authentication requests. Together with the power consumption analysis, this essentially becomes an multivariate optimisation problem where we are trying to minimise power consumption and user inconvenience, while maximising accuracy.

\section*{Future work}
As shown in the previous section, the new token unlocking scheme offers Pico an overall improvement. However, additional work is required in order to improve details of the design, as well as the prototype. Additional details are presented in this section. 

% Experiment with weights and decay factors
The Android prototype was developed as a proof of concept. Further experiments need to be performed using different weights and decay functions. A user study is required in order to determine the acceptable time interval between consecutive explicit authentication requests, and the implementation needs to be adapted accordingly. When performing this analysis, the power consumption results from section \ref{} need to be considered, in order to improve the lifetime of the device.

% Add aditional mechanisms and improve accuracy of current.
The set of individual mechanisms used with the scheme's prototype can be improved. Explicit authentication mechanisms are not currently supported and need to be implemented. Better biometric libraries should be either developed or imported in order to increase accuracy. Furthermore, additional mechanisms should be added for the platform. A number of viable suggestions are made in appendix \ref{AppendixC}.

% Take data samples based on external events
With the current prototype, the voice and face recognition mechanisms sample data at fixed time intervals. This should be change by taking advantage of user behaviour and Android events. Examples for this were given in section \ref{impleoverview}. 

The face recognition mechanism can be improved either by introducing another library that performs face detection, or by using a different face recognition library that offers both features. Cryptographic support needs to be added for this mechanism. It can be performed through additional modifications of the {\tt Javafaces} library that would allow it to use raw binary data during the training process.

% Safer implementation
A safer prototype would be to develop a root system service using the Android NDK C compiler. The binary has to be included in the system partition of the boot image in order to be accessible by the {\tt init} process during start up. The {\tt init.rc} configuration file used by {\tt init} also needs to be configured to start the service. This implementation requires modifications to the {\tt /system} partition. The process does not limit to simply gaining root privileges. The root directory {\tt /} is mounted as ramdisk, and therefore any modifications will be reverted once the device is rebooted. In order to make persistent changes, the user needs to modify the boot image, and re-flash it on the device.
 

%----------------------------------------------------------------------------------------
%	THESIS CONTENT - APPENDICES
%----------------------------------------------------------------------------------------

\addtocontents{toc}{\vspace{2em}} % Add a gap in the Contents, for aesthetics

\appendix % Cue to tell LaTeX that the following 'chapters' are Appendices

% Include the appendices of the thesis as separate files from the Appendices folder
% Uncomment the lines as you write the Appendices

% Appendix A

\chapter{Android development and security} % Main appendix title

\label{AppendixA} % For referencing this appendix elsewhere, use \ref{AppendixA}

\lhead{Appendix A. \emph{Android development and security}} % This is for the header on each page - perhaps a shortened title

% Introduction to android security and dev model
To gain a better understanding of different design decisions and limitations of our implementation, we will present a brief literature review of the Android development platform. Mechanisms and components will be described with an emphasis on security. The prototype developed for this dissertation is a proof of concept. However, we still aim to understand and make proper use of available security mechanisms. 

% First paper: introduction
William Enck et al \cite{enck2009understanding} offer a good introduction to Android application development. They focus on the security aspects of the development platform. It is a relatively old paper (2008), from the same year of the Android initial release. However, the fundamental design principles and security concepts that are discussed did not change considerably. The platform's open standards were made public in November 2007. This allowed researchers such as the authors of this paper to perform a pre-release analysis of the system.

% First paper: OS short description
Android uses as a core operating system a port of the Linux kernel. This introduces to the platform some of the Linux security mechanisms (i.e. file permissions, access control policies). On top of the kernel there is an application middleware layer composed out of the Java Dalvik virtual machine, core Java application libraries, as well as libraries which offer support for storage, sensors, display, and other device features. Applications are supported by the middleware and developed using the Android Java SDK.

% First paper: Android components
The Android development model is based on building an application from multiple components. Based on their purpose, the SDK defines four types: activity, service, content provider, and broadcast receiver. For the purpose of brevity we will not discuss each individual component\footnote{More details on the role of each component can be found on the Android website: http://developer.android.com/guide/components/fundamentals.html}. To allow meaningful interaction, Inter Component Communication (ICC) is enabled using special objects called Intents.

% First paper: binding services
The application we are developing needs to perform most of its processing in the background. It does not require any explicit user interaction. According to the Android model, this should be achieved using Services. To enable convenient component interaction, services may become bound engaging in a client-server communication. An important note made in the paper is that while a Service is bound, it cannot be terminated by an explicit stop action. This provides an useful guarantee regarding its lifetime, which we will use in the prototype.

% First paper: security enforcements, system level
The paper discusses two types of Android security enforcements: ICC, and system level. System level security is based on the Linux permission model. When installed, each app is allocated an UID and GID. This allows internal storage access control restrictions, keeping application data sandboxed from other apps.

% First paper: security enforcements, ICC
% 	TODO: too vague, can expand and cite something
ICC security is the main focus of the paper. Intent communication is based on commands sent to the ``/dev/binder'' device node. The node needs to be world readable and writeable by any application. Therefore, Android cannot mediate ICC using the Linux permissions model. Security relies on a Mandatory Access Control (MAC) framework enforced by a reference monitor. This mechanism validates requests sent to the ``/dev/binder'' node. 

% Manifest file
During development, each application needs to define a manifest file\footnote{Full details regarding the manifest file can be found on the Android website: http://developer.android.com/guide/topics/manifest/manifest-intro.html}. Some of the security configurations defined in this file are: declared components and their capabilities, permissions required by the app, and permissions other apps need to have in order to interact with app components. These entries are used as labels for the MAC framework. 

% First paper, types of components: public/private.
Using the app manifest file, each component can be defined as either public or private. This refinement is configured by the ``exported'' field. It defines whether or not another application may launch or interact with one of its components. When this paper was written, the ``exported'' field was defaulted to ``true''. However, as shown by Steffen and Mathias \cite{liebergeld2013android}  in 2013, starting with Android 4.2  the default of this value was changed to ``false'', and now conforms to the ``principle of least privilege''.

% First paper, Intent filters
Components listening for Intents need to have an intent-filter registered in the application manifest file. This allows them to export only a limited set of intents to other applications. Further restrictions to Intent objects are offered by the SDK using permission labels. This mechanism provides runtime security checks for the application. It is an additional prevention mechanism for data leaks through ICC. An application may broadcast an event throughout the system. By using permission labels, only apps that have the respective permission may process the event. Furthermore, Services may check for permissions when they are bound by another component. This allows them to expose different APIs depending on the binder.

% Paper two!
Steffen and Mathias \cite{liebergeld2013android} focus on deeper issues of the Android platform. They show how problems are solved from one Android version to the other. Unfortunately, OEMs tend not to update the software of their devices once they have shipped, which creates a high security risk.

The starting point of understanding Android security is learning how it is bootstrapped during the five step booting process:
\begin{enumerate}
	\item Initial bootloader (IBL) is loaded from ROM.
	\item IBL checks the signature of the bootloader (BL) and loads it into RAM.
	\item BL checks the signature of the linux kernel (LK) and loads it into RAM.
	\item LK initialises all existing hardware and starts the linux ``init'' process.
	\item The init process reads a configuration file and boots the rest of LA.
\end{enumerate}

The android security model is split by the paper in two categories: system security, and application security.

% keychain encryption and security
Android provides a keychain API used for storing sensitive material such as certificates and other credentials. These are encrypted using a master key, which is stored using AES encryption. Security needs to begin somewhere. An assumption has to be made about a state being secure from which multiple security extensions can be made. In this case, the master key is considered to be that point of security. However, given a rooted device, the master key itself can be retrieved from the system and therefore compromising all other credentials. The Android base system (libraries, app framework, and app runtime) is located in the ``system'' partition. Although it is writeable only by the root user, as mentioned before, exploits which grant this privilege exist. 

% Same author, shared privileges
From the user's perspective, an interesting ``feature'' which may affect the flow of information within Android is the fact that applications from the same author may share private resources. When installing an app the user needs to accept its predefined set of permissions. Due to resource sharing, a situation may present itself where an application that has permissions for the owner's contacts may communicate with an application that has permissions for internet in order to leak confidential data. A developer may therefore construct pairs of legitimate applications in order to mask a data flow attack.

% Android low level security
The Android OS offers a number of memory corruption mitigations in order to avoid buffer overflow attacks, or return oriented programming. The following list 
presents these low level security mechanisms:
\begin{itemize}
	\item Implements mmap\_min\_addr which restricts mmap memory mapping calls. This prevents NULL pointer related attacks.
	\item Implements XN (execute never) bit to mark memory as non-executable. The mechanism prevents attackers from executing remote code passed as data.
	\item Address space layout randomisation(ASLR) was implemented starting with Android 4.0. This is a first step to preventing return oriented programming attacks. The memory location of the binary library itself is however static. After a number of attempts using trial and error, the attacker may succeed using return oriented programming.
	\item Position independent and randomised linker (PIE) is implemented starting with Android 4.1 to support ASLP. This makes the memory location of binary libraries to be randomised.
	\item Read only relocation and immediate binding space (RELro) was implemented starting with Android 4.1. It solves an ASLR issue where an attacker could modify the global offset table (GOT) used when resolving a function from a dynamically linked library. Before this update an attacker may insert his own code to be executed using the GOT table.
\end{itemize}

% On device bouncer
A number of application security mechanisms are in place to make Android a safer environment for its users. A device program also known as the ``Bouncer'' prevents malware to be distributed from the Android App store (Google Play). The purpose of the bouncer is to verify apps prior to installation by checking for malware signatures and patterns. 

% Secure USB debugging
Secure USB debugging was introduced starting with Android 4.4.2. This only allows hosts registered with the device to have USB debugging permissions. The mechanism is circumvented if the user does not have a screen lock.

% The 4 big issues with android and malware
According to the paper, the Android OS is responsible for $96\%$ of mobile phone malware. The authors claim that this is the case due to 4 big issues of the Android platform:
\begin{enumerate}
	\item Security updates are delayed or never deployed. This is due to a number of approvals that an update needs to receive prior to deployment. This introduces an additional cost to the manufacturer (OEM), that does not generate any revenue. The majority of teams working on the Android platform are focusing on current releases. In most cases there are simply not enough resources to merge Google security updates to the OEM repository. Furthermore, the consequences of a failed OS update may cause ``bricking'' of the device, which is a huge risk for the manufacturer. All these issues lead to very few security updates. Therefore, important features such as RELro are never deployed, making older Android releases vulnerable.
	
	\item OEMs weaken the security of Android by introducing custom modifications before they roll out a device.
	
	\item The Android permission model is defective. According to Kelley et al \cite{kelley2012conundrum}, most users do not understand the permission dialogue when installing an application. Furthermore, even if they could understand the dialogue, most of the time it is ignored in order to use the exciting new app. According to the same study, most applications are over-privileged. This is due to developers not understanding what each privilege grants. Furthermore, as previously pointed out, apps developed by the same owner may share resources and implicitly privileges.
	
	\item Google Play has a low barrier for malware. A developer distribution agreement (DDA) and a developer program policy (DPP) need to be agreed to and signed by the developer before submitting the application to the Android market. However, Google Play does not check upfront if an application adheres to DDA and DDP. The application is only reviewed if it becomes suspect of breaking the agreements. Furthermore, according to \cite{percoco2012adventures} there are ways of circumventing the Bouncer program\footnote{An example of such an application is presented in an article written in Tech Republic: http://www.techrepublic.com/blog/google-in-the-enterprise/malware-in-the-google-play-store-enemy-inside-the-gates/\# (visited on 29.05.2014).}. 
\end{enumerate}

% Conclusion of the section, just a summary of what we presented
We have briefly presented the Android development model, existing mechanisms, and the security of the platform. This information should be sufficient to understand the principles involved in the design of the prototype developed for this dissertation project.


% TODO: integrate!===================================================================================
According to the Android documentation, a bound service is the server in the client-server interface. It enables other components to send requests and receive responses.


% Keep UAService alive via startService().
According to the Android development API guide \footnote{http://developer.android.com/guide/components/bound-services.html} there are two independent scenarios describing the lifetime of a bound service:
\begin{enumerate}
	\item If the service was not previously running, and a ``bindService()'' command is issued by a component, the service is kept alive for as long as clients are still bound. A client becomes unbound by calling ``unbindService()''.
	
	\item If the service is started using ``onStartCommand()'' it can only be stopped if it has no bound clients and an explicit request is made either via ``stopSelf()'' or ``stopService''. Unlike the previous case, its lifetime persists even with no bound components.
\end{enumerate}

% Appendix Template

\chapter{Token Unlocking Framework evaluation examples} % Main appendix title

\label{AppendixA} % Change X to a consecutive letter; for referencing this appendix elsewhere, use \ref{AppendixX}

\lhead{Appendix A. \emph{Token evaluation examples}} % Change X to a consecutive letter; this is for the header on each page - perhaps a shortened title

The following sections present examples of how the token unlocking framework should be used. We will be assessing PINs, and biometric face unlock. Together with the Picosiblings evaluation in section \ref{picosiblingseval}, each scheme represents a different type of authentication method. Picosiblings essentially are a secret the owner has, PINs are a secret the owner knows, and Face-unlock reflects who the owner is. 

%
%	PIN
%=======================================================
%
\section{PIN}
% Introduction to PINs and resemblance to passwords
PINs are token authentication mechanisms similar to passwords. The difference between the two is that they use a smaller set of input characters. Additional protection comes from steep security measures when the authentication challenge has failed. As an example, typing 3 wrong PINs on a mobile phone would lock the owner's SIM card. A lot of the PIN properties should however be similar with those offered by passwords.
	
% Usability: PINs
The scheme relies on knowing a secret, which is not ``memorywise-effortless''. It does however offer the ``nothing-to-carry'' property. Because of its similarity with passwords users find it ``easy-to-learn''. The small character set allows for fast user input and validation making PINs ``efficient-to-use''. Mistakes however may still occasionally occur, and due to the lack of visual feedback \footnote{If existent, visual feedback for PINs generally consists of `*' characters.} the scheme only quasi-offers ``infrequent-errors''. PINs are generally easily reset by the manufacturer using online services, therefore having ``easy-recovery-from-loss'' \footnote{An example of this is the RSA SecurID. An example reset procedure is described at the following link: http://uk.emc.com/collateral/15-min-guide/h12278-am8-help-desk-administrator-guide.pdf}. The scheme offers the ``availability'' property, as the authentication process cannot be impaired by external factors.
	
% Deployability: PINs
Just as passwords PINs score all points in deployability. They can be used regardless of disabilities, making them ``accessible''. They have virtually no cost, satisfying the ``negligible-cost-per-user'' property. Being a subset of passwords, we consider the mechanism to be ``mature'' and ``non-proprietary''.
	
% Security: PINs
From a security perspective PINs score poorly. They are not ``resilient-to-physical-observation''. Anyone can eavesdrop the input of a PIN either by shoulder surfing or recording with a camera. Just as passwords, PINs are often written down in plain sight. However, in the lack of relevant studies\footnote{Just as Bonneau et al suggest \cite{bonneau2012quest}, a relevant study would assess acquaintances' ability to guess the PIN of a subject.} we will mark the scheme to quasi-offer ``resilient-to-targeted-impersonation''. The restricted character set makes PINs adopt harsher security policies when provided invalid input. They are generally locked after three bad attempts, making them ``resilient-to-throttled-guessing''. The ``resilient-to-unthrottled-guessing'' property is implementation dependent. However, security tokens are dedicated devices that generally have tamper resistant memory, making unthrottled guessing not possible. Any hardware PINs may require does not compromise the mechanism, therefore offering ``resilient-to-theft''. Users have the freedom of choosing any PIN. Even in situations when reused with multiple tokens, credentials are generally salted and therefore ``unlinkable''. The scheme does not offer ``continuous-authentication'' because the process is not effortless for the user. They can only provide locked or unlocked feedback, and therefore do not offer ``multi-level-unlocking''. The owner may disclose their PIN at any time, making the ``non-disclosability'' property unsatisfied. 
	
%
%	Android face unlock
%=======================================================
%
\section{Face unlock}
Although not currently used as a security token unlocking mechanism, face recognition is a viable biometric authentication scheme. It can be ported for a token such as Pico, which is designed to have a camera. With a variety of possible implementations, for accessibility reasons we will analyse the Android face unlocking mechanism.
	
% Face unlock: usability
Face unlock is ``memorywise-effortless'', as any other biometric scheme. It offers the ``nothing-to-carry property'', the camera being embedded as part of the token. The mechanism is ``easy-to-learn'', since it only needs the user to look at the camera. The authentication process is performed almost instantly, making the scheme ``efficient-to-use''. The scheme is dependent on camera positioning, obstructing objects (e.g. glasses, earrings), and face mimic. In conjunction with the UDS framework assessment of biometrics in general, the scheme does not offer ``infrequent-errors''. If the scheme no longer functions as a result of change in facial traits, Android has a backup unlocking mechanism. This may also be used to disable or recalibrate the scheme, therefore offering ``easy-recovery-from-loss''. The ``availability'' property is not satisfied due to the dependence on external factors such as light or obstacles.
	
% Face unlock: deployability
Android face recognition is ``accessible'' for anyone regardless of disabilities. It offers the ``negligible-cost-per-user'' property, given that the hardware was already present in devices without face recognition features. Due to limited user exposure it is only quasi-``mature''. On Android, the scheme is implemented as not ``non-proprietary''.
	
% Face unlock: security
Observing the owner authenticate does not provide any advantage to an attacker. It therefore offers the ``resilient-to-physical-observations'' property. Targeted impersonation is an issue with any biometric mechanism. The scheme is vulnerable to replay attacks (i.e. a picture of the owner's face) and therefore does not offer ``resilient-to-targeted-impersonation''. The ''resilient-to-throttled-guessing`` and ``resilient-to-unthrottled-guessing'' properties do not apply. Given the Android implementation, neither does ``resilient-to-theft''. The same authentication data is used with any verifier, and therefore the ``unlinkable'' property is not offered. The scheme is implemented without ``continuous-authentication'' or ``multi-level-unlocking'' although both can be supported by biometric mechanisms. Given the possibility of deliberately providing data for a replay attack, the scheme only quasi-offers the ``non-disclosability'' property.

%
%	TODO: can add fingerprint unlock - IPhone
%=======================================================
%
% Appendix Template

\chapter{Examples of supported Android authentication mechanisms} % Main appendix title

\label{AppendixC} % Change X to a consecutive letter; for referencing this appendix elsewhere, use \ref{AppendixX}

\lhead{Appendix C. \emph{Example authentication mechanisms}}

% Presenting examples of what we can port on android
% 	TODO: rephrase, make larger a bit
Android provides an extensive sensor API that can support the token unlocking scheme proposed in section \ref{propopsedsol}. This can be used to develop a number of continuous authentication mechanisms.  We have listed the following non-exhaustive set of examples:
\begin{description}
  \item[Face recognition] \hfill \\
  The mechanism is based on capturing an image of the user's face and performing face recognition. Sampling valid face images can be performed without explicit requests by predicting user behaviour. We will use as an example an user that owns a phone with a front-facing camera. When the owner is unlocking the phone, there is a high probability that they will be looking towards the screen. This provides a good opportunity for the face recognition service to capture a valid sample. Using the Android API, this can be achieved by registering a ``BroadcastReceiver'' to listen for the one of the following events: ACTION\_SCREEN\_ON, ACTION\_SCREEN\_OFF, or ACTION\_USER\_PRESENT. The mechanism may continue to perform face recognition based on collected data and a previously recorded sample of the owner. A simple face recognition mechanism was also implemented as part of the prototype.
  
  % TODO: replay attacks are easy if using only features
  \item[Voice recognition] \hfill \\
  A voice recognition mechanism can record data either periodically, or based on Android events. It may then perform voice recognition and provide a confidence level of the owner being present. Voice sampling does not necessarily imply a voice password. An analysis can be performed using feature extraction. This facilitates the sampling process, which may be performed at any time. With a frequent sampling period, the owner of the device is likely to be recorded while speaking, which would provide a valid data sample. For even better confidence the mechanism can be implemented to start recording when a call is either made or received. On Android this can be achieved by listening for a PHONE\_STATE event. A simple voice recognition mechanism was implemented as part of the prototype.
  
  \item[Iris scanning] \hfill \\
  Similar to face recognition, this can be implemented by taking advantage of user behaviour while using the phone. When the phone is unlocked, the user is very likely to face the front camera, allowing for a good capture. The only problem with this mechanism is the quality of pictures offered by most phones. If the sampling quality is not sufficiently good, meaningful features from the iris may not be extracted. This would make the confidence level of the mechanism relatively low, but may change in the future as devices become increasingly performant.
  
  \item[Keystroke analysis] \hfill \\
  This mechanism was inspired from a paper by Clarke et al \cite{clarke2007authenticating}. The principle of keystroke analysis is based on the patterns in which the user types on his mobile phone. Different features can be extracted here, such as: letter sequence timings, words per minute, letters per minute, frequent used words, and others. Using this data a confidence level can be generated. 
  
  This mechanism is harder to implement using solely the Android SDK. A good starting point would be to have a keyboard app developed for the user that also communicates with the authentication mechanism. If the keyboard is disabled by an attacker this should be considered, especially if the authenticator was originally configured to listen for input.
  
  \item[Gait recognition] \hfill \\
  This mechanism is based on analysing individual walking patterns. According to data presented by Derawi et al \cite{derawi2010unobtrusive}, error rates\footnote{The performance indicator used in biometric analysis is the Equal Error Rate (EER).} may vary between $5\%$ to $20\%$.  Android offers native recognition support for walking, driving, or standing still. Applications can register a sensor callback for the TYPE\_STEP\_DETECTOR composite sensor. Whenever such an event is detected, data can be recorded from the accelerometer and validated using an algorithm similar to the one described by Derawi et al \cite{derawi2010unobtrusive}.
  
  \item[Ear shape analysis] \hfill \\
  Research shows (i.e. Burge et al \cite{burge1996ear}, Mu et al \cite{mu2005shape}) that the shape of the human ear contains enough unique features to perform biometric authentication. Taking advantage of user behaviour, valid data can be captured and analysed using a smart phone. We suggest that a picture is taken a few seconds after a phone call event is detected. If no peripherals are attached, the user is likely to move the device towards the ear. Images captured by such a mechanism could then be used to calculate an accurate confidence level of the user's identity. This method was not tested, so therefore we cannot ensure whether the auto-focus of the camera is sufficiently fast to obtain a valid image.
  
  \item[Proximity devices] \hfill \\
  This is an original idea based on providing a confidence level depending on the presence of known devices. The mechanism should connect with other devices that are also running the authenticator. The two owners don't necessarily need to know one another for the acknowledgement to be performed. Whether regular travel schedules, or working in an office, users are constantly being in the presence of other known devices. This should provide a confidence as to whether the device is in the presence of its owner. 
  
  The authentication works by seeking connections with other devices. Whenever a device is identified, its ID is recorded. The mechanism needs to keep track of the number of times it has connected with another device. Some connections may be established for the first time, and should not bring any confidence. Other connections, such as the Pico of a co-worker, would probably have a high number of connections, and therefore the mechanism should output a higher confidence level in its presence. This mechanism is similar to the Picosiblings solution, but with no k-out-of-n secrets. Each Pico is essentially a Picosibling for another Pico, with each device having a different weight based on familiarity.

  As an example, when travelling with your family on holiday most of the devices there are unknown. However, given that a number of frequent IDs are in the proximity of the authenticator, the mechanism should still consider to some extent that it is in the possession of its owner. 
  
  The mechanism can be circumvented in the scenario where co-workers or friends try to unlock the Pico. Due to this downside, it should never have sufficient weight to unlock the token on its own. However, in combination with other mechanisms it would provide a good approximation of whether it is in the possession of its owner. If the device is in good company there is a good chance the owner is also present. 
  
  \item[Location data] \hfill \\
  This mechanism is similar to ``Proximity devices'' and much easier to implement. Based on Android GPS and network location data, the phone may detect whether it is in an usual location or not. Just as ``Proximity devices'' this should not carry a high weight in the scheme, especially since it would not provide accurate results in scenarios such as holidays.
  
   \item[Service utilisation] \hfill \\
  This mechanism exploits patterns in the Android phone's service and app utilisation. Based on current running applications, services, and the time they were started we may create a model where some confidence is given regarding the ownership of the device. This mechanism would only be effective in detecting sudden changes. It would have a low weight in the overall scheme due to its lack in precision. 
  
  \item[Picosiblings]
  The original Picosiblings mechanism may also be used with this scheme. Although not part of the standard set of Android device sensors, if available, a Picosiblings implementation may be included as one of the authentication mechanisms.
\end{description}

% continuous mechanisms for explicit authentication
A number of continuous authentication mechanisms may also be used for explicit authentication. The user can be notified to provide accurate information for the following mechanisms: face recognition, voice recognition, iris scanning, keystroke analysis, gait recognition, and ear shape analysis. This creates the opportunity for a valid data sample to be collected.

% explicit authentication mechanisms
A number of explicit authentication mechanisms which do not satisfy the continuous authentication property of Pico may be implemented for the Android platform. It is important to note that additional mechanisms not included in this list need to satisfy the memorywise-effortless property of the token unlocking framework (\ref{tokenframework}). We suggest the following mechanisms for implementation:
\begin{description}
  \item[Fingerprint scanner] \hfill \\
  Devices that incorporate a fingerprint scanner (such as the IPhone 5S) can use the sensor as an explicit authentication mechanism. It cannot be used for continuous authentication because the user doesn't come in contact with the sensors on a regular basis. A mechanism can therefore request explicit fingerprint data, which would then be compared with the owner's biometric model, outputting a confidence for the authentication. The result will be combined in the overall scheme just as any other mechanism. The the only difference will be in terms of weight and decay rate.
    
  \item[Hand writing recognition] \hfill \\
  The user may be prompted to use the touch screen in order to write a word of his choice. This would guarantee the memorywise-effortless property because the user doesn't need to remember any secret. The handwriting would be analysed with a preconfigured set of handwriting samples in order to compute the confidence level that the owner produced the input.
  
  \item[Lip movement analysis] \hfill \\
  According to Faraj and Bigun \cite{faraj2006motion}, analysing lip movement while speaking can be used for authentication. The user would be prompted to provide a data sample such as reading a word provided by the authenticator. Using lip movement authentication, a quantifiable confidence level would be produced. This mechanism can also be implemented as a continuous authentication mechanism. However, data sampling would likely have a low success rate as users tend not to have their mouth within the camera's field of view.
\end{description}

\addtocontents{toc}{\vspace{2em}} % Add a gap in the Contents, for aesthetics

\backmatter

%----------------------------------------------------------------------------------------
%	BIBLIOGRAPHY
%----------------------------------------------------------------------------------------

\label{Bibliography}

\lhead{\emph{Bibliography}} % Change the page header to say "Bibliography"

\bibliographystyle{unsrtnat} % Use the "unsrtnat" BibTeX style for formatting the Bibliography

\bibliography{Bibliography} % The references (bibliography) information are stored in the file named "Bibliography.bib"

\end{document}  
